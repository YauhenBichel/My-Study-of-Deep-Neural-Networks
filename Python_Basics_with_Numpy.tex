\documentclass[11pt]{article}

    \usepackage[breakable]{tcolorbox}
    \usepackage{parskip} % Stop auto-indenting (to mimic markdown behaviour)
    
    \usepackage{iftex}
    \ifPDFTeX
    	\usepackage[T1]{fontenc}
    	\usepackage{mathpazo}
    \else
    	\usepackage{fontspec}
    \fi

    % Basic figure setup, for now with no caption control since it's done
    % automatically by Pandoc (which extracts ![](path) syntax from Markdown).
    \usepackage{graphicx}
    % Maintain compatibility with old templates. Remove in nbconvert 6.0
    \let\Oldincludegraphics\includegraphics
    % Ensure that by default, figures have no caption (until we provide a
    % proper Figure object with a Caption API and a way to capture that
    % in the conversion process - todo).
    \usepackage{caption}
    \DeclareCaptionFormat{nocaption}{}
    \captionsetup{format=nocaption,aboveskip=0pt,belowskip=0pt}

    \usepackage[Export]{adjustbox} % Used to constrain images to a maximum size
    \adjustboxset{max size={0.9\linewidth}{0.9\paperheight}}
    \usepackage{float}
    \floatplacement{figure}{H} % forces figures to be placed at the correct location
    \usepackage{xcolor} % Allow colors to be defined
    \usepackage{enumerate} % Needed for markdown enumerations to work
    \usepackage{geometry} % Used to adjust the document margins
    \usepackage{amsmath} % Equations
    \usepackage{amssymb} % Equations
    \usepackage{textcomp} % defines textquotesingle
    % Hack from http://tex.stackexchange.com/a/47451/13684:
    \AtBeginDocument{%
        \def\PYZsq{\textquotesingle}% Upright quotes in Pygmentized code
    }
    \usepackage{upquote} % Upright quotes for verbatim code
    \usepackage{eurosym} % defines \euro
    \usepackage[mathletters]{ucs} % Extended unicode (utf-8) support
    \usepackage{fancyvrb} % verbatim replacement that allows latex
    \usepackage{grffile} % extends the file name processing of package graphics 
                         % to support a larger range
    \makeatletter % fix for grffile with XeLaTeX
    \def\Gread@@xetex#1{%
      \IfFileExists{"\Gin@base".bb}%
      {\Gread@eps{\Gin@base.bb}}%
      {\Gread@@xetex@aux#1}%
    }
    \makeatother

    % The hyperref package gives us a pdf with properly built
    % internal navigation ('pdf bookmarks' for the table of contents,
    % internal cross-reference links, web links for URLs, etc.)
    \usepackage{hyperref}
    % The default LaTeX title has an obnoxious amount of whitespace. By default,
    % titling removes some of it. It also provides customization options.
    \usepackage{titling}
    \usepackage{longtable} % longtable support required by pandoc >1.10
    \usepackage{booktabs}  % table support for pandoc > 1.12.2
    \usepackage[inline]{enumitem} % IRkernel/repr support (it uses the enumerate* environment)
    \usepackage[normalem]{ulem} % ulem is needed to support strikethroughs (\sout)
                                % normalem makes italics be italics, not underlines
    \usepackage{mathrsfs}
    

    
    % Colors for the hyperref package
    \definecolor{urlcolor}{rgb}{0,.145,.698}
    \definecolor{linkcolor}{rgb}{.71,0.21,0.01}
    \definecolor{citecolor}{rgb}{.12,.54,.11}

    % ANSI colors
    \definecolor{ansi-black}{HTML}{3E424D}
    \definecolor{ansi-black-intense}{HTML}{282C36}
    \definecolor{ansi-red}{HTML}{E75C58}
    \definecolor{ansi-red-intense}{HTML}{B22B31}
    \definecolor{ansi-green}{HTML}{00A250}
    \definecolor{ansi-green-intense}{HTML}{007427}
    \definecolor{ansi-yellow}{HTML}{DDB62B}
    \definecolor{ansi-yellow-intense}{HTML}{B27D12}
    \definecolor{ansi-blue}{HTML}{208FFB}
    \definecolor{ansi-blue-intense}{HTML}{0065CA}
    \definecolor{ansi-magenta}{HTML}{D160C4}
    \definecolor{ansi-magenta-intense}{HTML}{A03196}
    \definecolor{ansi-cyan}{HTML}{60C6C8}
    \definecolor{ansi-cyan-intense}{HTML}{258F8F}
    \definecolor{ansi-white}{HTML}{C5C1B4}
    \definecolor{ansi-white-intense}{HTML}{A1A6B2}
    \definecolor{ansi-default-inverse-fg}{HTML}{FFFFFF}
    \definecolor{ansi-default-inverse-bg}{HTML}{000000}

    % commands and environments needed by pandoc snippets
    % extracted from the output of `pandoc -s`
    \providecommand{\tightlist}{%
      \setlength{\itemsep}{0pt}\setlength{\parskip}{0pt}}
    \DefineVerbatimEnvironment{Highlighting}{Verbatim}{commandchars=\\\{\}}
    % Add ',fontsize=\small' for more characters per line
    \newenvironment{Shaded}{}{}
    \newcommand{\KeywordTok}[1]{\textcolor[rgb]{0.00,0.44,0.13}{\textbf{{#1}}}}
    \newcommand{\DataTypeTok}[1]{\textcolor[rgb]{0.56,0.13,0.00}{{#1}}}
    \newcommand{\DecValTok}[1]{\textcolor[rgb]{0.25,0.63,0.44}{{#1}}}
    \newcommand{\BaseNTok}[1]{\textcolor[rgb]{0.25,0.63,0.44}{{#1}}}
    \newcommand{\FloatTok}[1]{\textcolor[rgb]{0.25,0.63,0.44}{{#1}}}
    \newcommand{\CharTok}[1]{\textcolor[rgb]{0.25,0.44,0.63}{{#1}}}
    \newcommand{\StringTok}[1]{\textcolor[rgb]{0.25,0.44,0.63}{{#1}}}
    \newcommand{\CommentTok}[1]{\textcolor[rgb]{0.38,0.63,0.69}{\textit{{#1}}}}
    \newcommand{\OtherTok}[1]{\textcolor[rgb]{0.00,0.44,0.13}{{#1}}}
    \newcommand{\AlertTok}[1]{\textcolor[rgb]{1.00,0.00,0.00}{\textbf{{#1}}}}
    \newcommand{\FunctionTok}[1]{\textcolor[rgb]{0.02,0.16,0.49}{{#1}}}
    \newcommand{\RegionMarkerTok}[1]{{#1}}
    \newcommand{\ErrorTok}[1]{\textcolor[rgb]{1.00,0.00,0.00}{\textbf{{#1}}}}
    \newcommand{\NormalTok}[1]{{#1}}
    
    % Additional commands for more recent versions of Pandoc
    \newcommand{\ConstantTok}[1]{\textcolor[rgb]{0.53,0.00,0.00}{{#1}}}
    \newcommand{\SpecialCharTok}[1]{\textcolor[rgb]{0.25,0.44,0.63}{{#1}}}
    \newcommand{\VerbatimStringTok}[1]{\textcolor[rgb]{0.25,0.44,0.63}{{#1}}}
    \newcommand{\SpecialStringTok}[1]{\textcolor[rgb]{0.73,0.40,0.53}{{#1}}}
    \newcommand{\ImportTok}[1]{{#1}}
    \newcommand{\DocumentationTok}[1]{\textcolor[rgb]{0.73,0.13,0.13}{\textit{{#1}}}}
    \newcommand{\AnnotationTok}[1]{\textcolor[rgb]{0.38,0.63,0.69}{\textbf{\textit{{#1}}}}}
    \newcommand{\CommentVarTok}[1]{\textcolor[rgb]{0.38,0.63,0.69}{\textbf{\textit{{#1}}}}}
    \newcommand{\VariableTok}[1]{\textcolor[rgb]{0.10,0.09,0.49}{{#1}}}
    \newcommand{\ControlFlowTok}[1]{\textcolor[rgb]{0.00,0.44,0.13}{\textbf{{#1}}}}
    \newcommand{\OperatorTok}[1]{\textcolor[rgb]{0.40,0.40,0.40}{{#1}}}
    \newcommand{\BuiltInTok}[1]{{#1}}
    \newcommand{\ExtensionTok}[1]{{#1}}
    \newcommand{\PreprocessorTok}[1]{\textcolor[rgb]{0.74,0.48,0.00}{{#1}}}
    \newcommand{\AttributeTok}[1]{\textcolor[rgb]{0.49,0.56,0.16}{{#1}}}
    \newcommand{\InformationTok}[1]{\textcolor[rgb]{0.38,0.63,0.69}{\textbf{\textit{{#1}}}}}
    \newcommand{\WarningTok}[1]{\textcolor[rgb]{0.38,0.63,0.69}{\textbf{\textit{{#1}}}}}
    
    
    % Define a nice break command that doesn't care if a line doesn't already
    % exist.
    \def\br{\hspace*{\fill} \\* }
    % Math Jax compatibility definitions
    \def\gt{>}
    \def\lt{<}
    \let\Oldtex\TeX
    \let\Oldlatex\LaTeX
    \renewcommand{\TeX}{\textrm{\Oldtex}}
    \renewcommand{\LaTeX}{\textrm{\Oldlatex}}
    % Document parameters
    % Document title
    \title{Python\_Basics\_with\_Numpy}
    
    
    
    
    
% Pygments definitions
\makeatletter
\def\PY@reset{\let\PY@it=\relax \let\PY@bf=\relax%
    \let\PY@ul=\relax \let\PY@tc=\relax%
    \let\PY@bc=\relax \let\PY@ff=\relax}
\def\PY@tok#1{\csname PY@tok@#1\endcsname}
\def\PY@toks#1+{\ifx\relax#1\empty\else%
    \PY@tok{#1}\expandafter\PY@toks\fi}
\def\PY@do#1{\PY@bc{\PY@tc{\PY@ul{%
    \PY@it{\PY@bf{\PY@ff{#1}}}}}}}
\def\PY#1#2{\PY@reset\PY@toks#1+\relax+\PY@do{#2}}

\expandafter\def\csname PY@tok@w\endcsname{\def\PY@tc##1{\textcolor[rgb]{0.73,0.73,0.73}{##1}}}
\expandafter\def\csname PY@tok@c\endcsname{\let\PY@it=\textit\def\PY@tc##1{\textcolor[rgb]{0.25,0.50,0.50}{##1}}}
\expandafter\def\csname PY@tok@cp\endcsname{\def\PY@tc##1{\textcolor[rgb]{0.74,0.48,0.00}{##1}}}
\expandafter\def\csname PY@tok@k\endcsname{\let\PY@bf=\textbf\def\PY@tc##1{\textcolor[rgb]{0.00,0.50,0.00}{##1}}}
\expandafter\def\csname PY@tok@kp\endcsname{\def\PY@tc##1{\textcolor[rgb]{0.00,0.50,0.00}{##1}}}
\expandafter\def\csname PY@tok@kt\endcsname{\def\PY@tc##1{\textcolor[rgb]{0.69,0.00,0.25}{##1}}}
\expandafter\def\csname PY@tok@o\endcsname{\def\PY@tc##1{\textcolor[rgb]{0.40,0.40,0.40}{##1}}}
\expandafter\def\csname PY@tok@ow\endcsname{\let\PY@bf=\textbf\def\PY@tc##1{\textcolor[rgb]{0.67,0.13,1.00}{##1}}}
\expandafter\def\csname PY@tok@nb\endcsname{\def\PY@tc##1{\textcolor[rgb]{0.00,0.50,0.00}{##1}}}
\expandafter\def\csname PY@tok@nf\endcsname{\def\PY@tc##1{\textcolor[rgb]{0.00,0.00,1.00}{##1}}}
\expandafter\def\csname PY@tok@nc\endcsname{\let\PY@bf=\textbf\def\PY@tc##1{\textcolor[rgb]{0.00,0.00,1.00}{##1}}}
\expandafter\def\csname PY@tok@nn\endcsname{\let\PY@bf=\textbf\def\PY@tc##1{\textcolor[rgb]{0.00,0.00,1.00}{##1}}}
\expandafter\def\csname PY@tok@ne\endcsname{\let\PY@bf=\textbf\def\PY@tc##1{\textcolor[rgb]{0.82,0.25,0.23}{##1}}}
\expandafter\def\csname PY@tok@nv\endcsname{\def\PY@tc##1{\textcolor[rgb]{0.10,0.09,0.49}{##1}}}
\expandafter\def\csname PY@tok@no\endcsname{\def\PY@tc##1{\textcolor[rgb]{0.53,0.00,0.00}{##1}}}
\expandafter\def\csname PY@tok@nl\endcsname{\def\PY@tc##1{\textcolor[rgb]{0.63,0.63,0.00}{##1}}}
\expandafter\def\csname PY@tok@ni\endcsname{\let\PY@bf=\textbf\def\PY@tc##1{\textcolor[rgb]{0.60,0.60,0.60}{##1}}}
\expandafter\def\csname PY@tok@na\endcsname{\def\PY@tc##1{\textcolor[rgb]{0.49,0.56,0.16}{##1}}}
\expandafter\def\csname PY@tok@nt\endcsname{\let\PY@bf=\textbf\def\PY@tc##1{\textcolor[rgb]{0.00,0.50,0.00}{##1}}}
\expandafter\def\csname PY@tok@nd\endcsname{\def\PY@tc##1{\textcolor[rgb]{0.67,0.13,1.00}{##1}}}
\expandafter\def\csname PY@tok@s\endcsname{\def\PY@tc##1{\textcolor[rgb]{0.73,0.13,0.13}{##1}}}
\expandafter\def\csname PY@tok@sd\endcsname{\let\PY@it=\textit\def\PY@tc##1{\textcolor[rgb]{0.73,0.13,0.13}{##1}}}
\expandafter\def\csname PY@tok@si\endcsname{\let\PY@bf=\textbf\def\PY@tc##1{\textcolor[rgb]{0.73,0.40,0.53}{##1}}}
\expandafter\def\csname PY@tok@se\endcsname{\let\PY@bf=\textbf\def\PY@tc##1{\textcolor[rgb]{0.73,0.40,0.13}{##1}}}
\expandafter\def\csname PY@tok@sr\endcsname{\def\PY@tc##1{\textcolor[rgb]{0.73,0.40,0.53}{##1}}}
\expandafter\def\csname PY@tok@ss\endcsname{\def\PY@tc##1{\textcolor[rgb]{0.10,0.09,0.49}{##1}}}
\expandafter\def\csname PY@tok@sx\endcsname{\def\PY@tc##1{\textcolor[rgb]{0.00,0.50,0.00}{##1}}}
\expandafter\def\csname PY@tok@m\endcsname{\def\PY@tc##1{\textcolor[rgb]{0.40,0.40,0.40}{##1}}}
\expandafter\def\csname PY@tok@gh\endcsname{\let\PY@bf=\textbf\def\PY@tc##1{\textcolor[rgb]{0.00,0.00,0.50}{##1}}}
\expandafter\def\csname PY@tok@gu\endcsname{\let\PY@bf=\textbf\def\PY@tc##1{\textcolor[rgb]{0.50,0.00,0.50}{##1}}}
\expandafter\def\csname PY@tok@gd\endcsname{\def\PY@tc##1{\textcolor[rgb]{0.63,0.00,0.00}{##1}}}
\expandafter\def\csname PY@tok@gi\endcsname{\def\PY@tc##1{\textcolor[rgb]{0.00,0.63,0.00}{##1}}}
\expandafter\def\csname PY@tok@gr\endcsname{\def\PY@tc##1{\textcolor[rgb]{1.00,0.00,0.00}{##1}}}
\expandafter\def\csname PY@tok@ge\endcsname{\let\PY@it=\textit}
\expandafter\def\csname PY@tok@gs\endcsname{\let\PY@bf=\textbf}
\expandafter\def\csname PY@tok@gp\endcsname{\let\PY@bf=\textbf\def\PY@tc##1{\textcolor[rgb]{0.00,0.00,0.50}{##1}}}
\expandafter\def\csname PY@tok@go\endcsname{\def\PY@tc##1{\textcolor[rgb]{0.53,0.53,0.53}{##1}}}
\expandafter\def\csname PY@tok@gt\endcsname{\def\PY@tc##1{\textcolor[rgb]{0.00,0.27,0.87}{##1}}}
\expandafter\def\csname PY@tok@err\endcsname{\def\PY@bc##1{\setlength{\fboxsep}{0pt}\fcolorbox[rgb]{1.00,0.00,0.00}{1,1,1}{\strut ##1}}}
\expandafter\def\csname PY@tok@kc\endcsname{\let\PY@bf=\textbf\def\PY@tc##1{\textcolor[rgb]{0.00,0.50,0.00}{##1}}}
\expandafter\def\csname PY@tok@kd\endcsname{\let\PY@bf=\textbf\def\PY@tc##1{\textcolor[rgb]{0.00,0.50,0.00}{##1}}}
\expandafter\def\csname PY@tok@kn\endcsname{\let\PY@bf=\textbf\def\PY@tc##1{\textcolor[rgb]{0.00,0.50,0.00}{##1}}}
\expandafter\def\csname PY@tok@kr\endcsname{\let\PY@bf=\textbf\def\PY@tc##1{\textcolor[rgb]{0.00,0.50,0.00}{##1}}}
\expandafter\def\csname PY@tok@bp\endcsname{\def\PY@tc##1{\textcolor[rgb]{0.00,0.50,0.00}{##1}}}
\expandafter\def\csname PY@tok@fm\endcsname{\def\PY@tc##1{\textcolor[rgb]{0.00,0.00,1.00}{##1}}}
\expandafter\def\csname PY@tok@vc\endcsname{\def\PY@tc##1{\textcolor[rgb]{0.10,0.09,0.49}{##1}}}
\expandafter\def\csname PY@tok@vg\endcsname{\def\PY@tc##1{\textcolor[rgb]{0.10,0.09,0.49}{##1}}}
\expandafter\def\csname PY@tok@vi\endcsname{\def\PY@tc##1{\textcolor[rgb]{0.10,0.09,0.49}{##1}}}
\expandafter\def\csname PY@tok@vm\endcsname{\def\PY@tc##1{\textcolor[rgb]{0.10,0.09,0.49}{##1}}}
\expandafter\def\csname PY@tok@sa\endcsname{\def\PY@tc##1{\textcolor[rgb]{0.73,0.13,0.13}{##1}}}
\expandafter\def\csname PY@tok@sb\endcsname{\def\PY@tc##1{\textcolor[rgb]{0.73,0.13,0.13}{##1}}}
\expandafter\def\csname PY@tok@sc\endcsname{\def\PY@tc##1{\textcolor[rgb]{0.73,0.13,0.13}{##1}}}
\expandafter\def\csname PY@tok@dl\endcsname{\def\PY@tc##1{\textcolor[rgb]{0.73,0.13,0.13}{##1}}}
\expandafter\def\csname PY@tok@s2\endcsname{\def\PY@tc##1{\textcolor[rgb]{0.73,0.13,0.13}{##1}}}
\expandafter\def\csname PY@tok@sh\endcsname{\def\PY@tc##1{\textcolor[rgb]{0.73,0.13,0.13}{##1}}}
\expandafter\def\csname PY@tok@s1\endcsname{\def\PY@tc##1{\textcolor[rgb]{0.73,0.13,0.13}{##1}}}
\expandafter\def\csname PY@tok@mb\endcsname{\def\PY@tc##1{\textcolor[rgb]{0.40,0.40,0.40}{##1}}}
\expandafter\def\csname PY@tok@mf\endcsname{\def\PY@tc##1{\textcolor[rgb]{0.40,0.40,0.40}{##1}}}
\expandafter\def\csname PY@tok@mh\endcsname{\def\PY@tc##1{\textcolor[rgb]{0.40,0.40,0.40}{##1}}}
\expandafter\def\csname PY@tok@mi\endcsname{\def\PY@tc##1{\textcolor[rgb]{0.40,0.40,0.40}{##1}}}
\expandafter\def\csname PY@tok@il\endcsname{\def\PY@tc##1{\textcolor[rgb]{0.40,0.40,0.40}{##1}}}
\expandafter\def\csname PY@tok@mo\endcsname{\def\PY@tc##1{\textcolor[rgb]{0.40,0.40,0.40}{##1}}}
\expandafter\def\csname PY@tok@ch\endcsname{\let\PY@it=\textit\def\PY@tc##1{\textcolor[rgb]{0.25,0.50,0.50}{##1}}}
\expandafter\def\csname PY@tok@cm\endcsname{\let\PY@it=\textit\def\PY@tc##1{\textcolor[rgb]{0.25,0.50,0.50}{##1}}}
\expandafter\def\csname PY@tok@cpf\endcsname{\let\PY@it=\textit\def\PY@tc##1{\textcolor[rgb]{0.25,0.50,0.50}{##1}}}
\expandafter\def\csname PY@tok@c1\endcsname{\let\PY@it=\textit\def\PY@tc##1{\textcolor[rgb]{0.25,0.50,0.50}{##1}}}
\expandafter\def\csname PY@tok@cs\endcsname{\let\PY@it=\textit\def\PY@tc##1{\textcolor[rgb]{0.25,0.50,0.50}{##1}}}

\def\PYZbs{\char`\\}
\def\PYZus{\char`\_}
\def\PYZob{\char`\{}
\def\PYZcb{\char`\}}
\def\PYZca{\char`\^}
\def\PYZam{\char`\&}
\def\PYZlt{\char`\<}
\def\PYZgt{\char`\>}
\def\PYZsh{\char`\#}
\def\PYZpc{\char`\%}
\def\PYZdl{\char`\$}
\def\PYZhy{\char`\-}
\def\PYZsq{\char`\'}
\def\PYZdq{\char`\"}
\def\PYZti{\char`\~}
% for compatibility with earlier versions
\def\PYZat{@}
\def\PYZlb{[}
\def\PYZrb{]}
\makeatother


    % For linebreaks inside Verbatim environment from package fancyvrb. 
    \makeatletter
        \newbox\Wrappedcontinuationbox 
        \newbox\Wrappedvisiblespacebox 
        \newcommand*\Wrappedvisiblespace {\textcolor{red}{\textvisiblespace}} 
        \newcommand*\Wrappedcontinuationsymbol {\textcolor{red}{\llap{\tiny$\m@th\hookrightarrow$}}} 
        \newcommand*\Wrappedcontinuationindent {3ex } 
        \newcommand*\Wrappedafterbreak {\kern\Wrappedcontinuationindent\copy\Wrappedcontinuationbox} 
        % Take advantage of the already applied Pygments mark-up to insert 
        % potential linebreaks for TeX processing. 
        %        {, <, #, %, $, ' and ": go to next line. 
        %        _, }, ^, &, >, - and ~: stay at end of broken line. 
        % Use of \textquotesingle for straight quote. 
        \newcommand*\Wrappedbreaksatspecials {% 
            \def\PYGZus{\discretionary{\char`\_}{\Wrappedafterbreak}{\char`\_}}% 
            \def\PYGZob{\discretionary{}{\Wrappedafterbreak\char`\{}{\char`\{}}% 
            \def\PYGZcb{\discretionary{\char`\}}{\Wrappedafterbreak}{\char`\}}}% 
            \def\PYGZca{\discretionary{\char`\^}{\Wrappedafterbreak}{\char`\^}}% 
            \def\PYGZam{\discretionary{\char`\&}{\Wrappedafterbreak}{\char`\&}}% 
            \def\PYGZlt{\discretionary{}{\Wrappedafterbreak\char`\<}{\char`\<}}% 
            \def\PYGZgt{\discretionary{\char`\>}{\Wrappedafterbreak}{\char`\>}}% 
            \def\PYGZsh{\discretionary{}{\Wrappedafterbreak\char`\#}{\char`\#}}% 
            \def\PYGZpc{\discretionary{}{\Wrappedafterbreak\char`\%}{\char`\%}}% 
            \def\PYGZdl{\discretionary{}{\Wrappedafterbreak\char`\$}{\char`\$}}% 
            \def\PYGZhy{\discretionary{\char`\-}{\Wrappedafterbreak}{\char`\-}}% 
            \def\PYGZsq{\discretionary{}{\Wrappedafterbreak\textquotesingle}{\textquotesingle}}% 
            \def\PYGZdq{\discretionary{}{\Wrappedafterbreak\char`\"}{\char`\"}}% 
            \def\PYGZti{\discretionary{\char`\~}{\Wrappedafterbreak}{\char`\~}}% 
        } 
        % Some characters . , ; ? ! / are not pygmentized. 
        % This macro makes them "active" and they will insert potential linebreaks 
        \newcommand*\Wrappedbreaksatpunct {% 
            \lccode`\~`\.\lowercase{\def~}{\discretionary{\hbox{\char`\.}}{\Wrappedafterbreak}{\hbox{\char`\.}}}% 
            \lccode`\~`\,\lowercase{\def~}{\discretionary{\hbox{\char`\,}}{\Wrappedafterbreak}{\hbox{\char`\,}}}% 
            \lccode`\~`\;\lowercase{\def~}{\discretionary{\hbox{\char`\;}}{\Wrappedafterbreak}{\hbox{\char`\;}}}% 
            \lccode`\~`\:\lowercase{\def~}{\discretionary{\hbox{\char`\:}}{\Wrappedafterbreak}{\hbox{\char`\:}}}% 
            \lccode`\~`\?\lowercase{\def~}{\discretionary{\hbox{\char`\?}}{\Wrappedafterbreak}{\hbox{\char`\?}}}% 
            \lccode`\~`\!\lowercase{\def~}{\discretionary{\hbox{\char`\!}}{\Wrappedafterbreak}{\hbox{\char`\!}}}% 
            \lccode`\~`\/\lowercase{\def~}{\discretionary{\hbox{\char`\/}}{\Wrappedafterbreak}{\hbox{\char`\/}}}% 
            \catcode`\.\active
            \catcode`\,\active 
            \catcode`\;\active
            \catcode`\:\active
            \catcode`\?\active
            \catcode`\!\active
            \catcode`\/\active 
            \lccode`\~`\~ 	
        }
    \makeatother

    \let\OriginalVerbatim=\Verbatim
    \makeatletter
    \renewcommand{\Verbatim}[1][1]{%
        %\parskip\z@skip
        \sbox\Wrappedcontinuationbox {\Wrappedcontinuationsymbol}%
        \sbox\Wrappedvisiblespacebox {\FV@SetupFont\Wrappedvisiblespace}%
        \def\FancyVerbFormatLine ##1{\hsize\linewidth
            \vtop{\raggedright\hyphenpenalty\z@\exhyphenpenalty\z@
                \doublehyphendemerits\z@\finalhyphendemerits\z@
                \strut ##1\strut}%
        }%
        % If the linebreak is at a space, the latter will be displayed as visible
        % space at end of first line, and a continuation symbol starts next line.
        % Stretch/shrink are however usually zero for typewriter font.
        \def\FV@Space {%
            \nobreak\hskip\z@ plus\fontdimen3\font minus\fontdimen4\font
            \discretionary{\copy\Wrappedvisiblespacebox}{\Wrappedafterbreak}
            {\kern\fontdimen2\font}%
        }%
        
        % Allow breaks at special characters using \PYG... macros.
        \Wrappedbreaksatspecials
        % Breaks at punctuation characters . , ; ? ! and / need catcode=\active 	
        \OriginalVerbatim[#1,codes*=\Wrappedbreaksatpunct]%
    }
    \makeatother

    % Exact colors from NB
    \definecolor{incolor}{HTML}{303F9F}
    \definecolor{outcolor}{HTML}{D84315}
    \definecolor{cellborder}{HTML}{CFCFCF}
    \definecolor{cellbackground}{HTML}{F7F7F7}
    
    % prompt
    \makeatletter
    \newcommand{\boxspacing}{\kern\kvtcb@left@rule\kern\kvtcb@boxsep}
    \makeatother
    \newcommand{\prompt}[4]{
        \ttfamily\llap{{\color{#2}[#3]:\hspace{3pt}#4}}\vspace{-\baselineskip}
    }
    

    
    % Prevent overflowing lines due to hard-to-break entities
    \sloppy 
    % Setup hyperref package
    \hypersetup{
      breaklinks=true,  % so long urls are correctly broken across lines
      colorlinks=true,
      urlcolor=urlcolor,
      linkcolor=linkcolor,
      citecolor=citecolor,
      }
    % Slightly bigger margins than the latex defaults
    
    \geometry{verbose,tmargin=1in,bmargin=1in,lmargin=1in,rmargin=1in}
    
    

\begin{document}
    
    \maketitle
    
    

    
    \hypertarget{python-basics-with-numpy-optional-assignment}{%
\section{Python Basics with Numpy (optional
assignment)}\label{python-basics-with-numpy-optional-assignment}}

Welcome to your first assignment. This exercise gives you a brief
introduction to Python. Even if you've used Python before, this will
help familiarize you with the functions we'll need.

\textbf{Instructions:} - You will be using Python 3. - Avoid using
for-loops and while-loops, unless you are explicitly told to do so. -
After coding your function, run the cell right below it to check if your
result is correct.

\textbf{After this assignment you will:} - Be able to use iPython
Notebooks - Be able to use numpy functions and numpy matrix/vector
operations - Understand the concept of ``broadcasting'' - Be able to
vectorize code

Let's get started!

    \hypertarget{table-of-contents}{%
\subsection{Table of Contents}\label{table-of-contents}}

\begin{itemize}
\tightlist
\item
  Section \ref{0}

  \begin{itemize}
  \tightlist
  \item
    Section \ref{ex-1}
  \end{itemize}
\item
  Section \ref{1}

  \begin{itemize}
  \tightlist
  \item
    Section \ref{1-1}

    \begin{itemize}
    \tightlist
    \item
      Section \ref{ex-2}
    \item
      Section \ref{ex-3}
    \end{itemize}
  \item
    Section \ref{1-2}

    \begin{itemize}
    \tightlist
    \item
      Section \ref{ex-4}
    \end{itemize}
  \item
    Section \ref{1-3}

    \begin{itemize}
    \tightlist
    \item
      Section \ref{ex-5}
    \end{itemize}
  \item
    Section \ref{1-4}

    \begin{itemize}
    \tightlist
    \item
      Section \ref{ex-6}
    \item
      Section \ref{ex-7}
    \end{itemize}
  \end{itemize}
\item
  Section \ref{2}

  \begin{itemize}
  \tightlist
  \item
    Section \ref{2-1}

    \begin{itemize}
    \tightlist
    \item
      Section \ref{ex-8}
    \item
      Section \ref{ex-9}
    \end{itemize}
  \end{itemize}
\end{itemize}

    \#\# About iPython Notebooks \#\#

iPython Notebooks are interactive coding environments embedded in a
webpage. You will be using iPython notebooks in this class. You only
need to write code between the \# your code here comment. After writing
your code, you can run the cell by either pressing ``SHIFT''+``ENTER''
or by clicking on ``Run Cell'' (denoted by a play symbol) in the upper
bar of the notebook.

We will often specify ``(≈ X lines of code)'' in the comments to tell
you about how much code you need to write. It is just a rough estimate,
so don't feel bad if your code is longer or shorter.

\#\#\# Exercise 1 Set test to \texttt{"Hello\ World"} in the cell below
to print ``Hello World'' and run the two cells below.

    \begin{tcolorbox}[breakable, size=fbox, boxrule=1pt, pad at break*=1mm,colback=cellbackground, colframe=cellborder]
\prompt{In}{incolor}{1}{\boxspacing}
\begin{Verbatim}[commandchars=\\\{\}]
\PY{c+c1}{\PYZsh{} (≈ 1 line of code)}
\PY{c+c1}{\PYZsh{} test = }
\PY{c+c1}{\PYZsh{} YOUR CODE STARTS HERE}
\PY{n}{test} \PY{o}{=} \PY{l+s+s2}{\PYZdq{}}\PY{l+s+s2}{Hello World}\PY{l+s+s2}{\PYZdq{}}

\PY{c+c1}{\PYZsh{} YOUR CODE ENDS HERE}
\end{Verbatim}
\end{tcolorbox}

    \begin{tcolorbox}[breakable, size=fbox, boxrule=1pt, pad at break*=1mm,colback=cellbackground, colframe=cellborder]
\prompt{In}{incolor}{2}{\boxspacing}
\begin{Verbatim}[commandchars=\\\{\}]
\PY{n+nb}{print} \PY{p}{(}\PY{l+s+s2}{\PYZdq{}}\PY{l+s+s2}{test: }\PY{l+s+s2}{\PYZdq{}} \PY{o}{+} \PY{n}{test}\PY{p}{)}
\end{Verbatim}
\end{tcolorbox}

    \begin{Verbatim}[commandchars=\\\{\}]
test: Hello World
    \end{Verbatim}

    \textbf{Expected output}: test: Hello World

    What you need to remember :

\begin{itemize}
\tightlist
\item
  Run your cells using SHIFT+ENTER (or ``Run cell'')
\item
  Write code in the designated areas using Python 3 only
\item
  Do not modify the code outside of the designated areas
\end{itemize}

    \#\# 1 - Building basic functions with numpy \#\#

Numpy is the main package for scientific computing in Python. It is
maintained by a large community (www.numpy.org). In this exercise you
will learn several key numpy functions such as \texttt{np.exp},
\texttt{np.log}, and \texttt{np.reshape}. You will need to know how to
use these functions for future assignments.

\#\#\# 1.1 - sigmoid function, np.exp() \#\#\#

Before using \texttt{np.exp()}, you will use \texttt{math.exp()} to
implement the sigmoid function. You will then see why \texttt{np.exp()}
is preferable to \texttt{math.exp()}.

\#\#\# Exercise 2 - basic\_sigmoid Build a function that returns the
sigmoid of a real number x. Use \texttt{math.exp(x)} for the exponential
function.

\textbf{Reminder}: \(sigmoid(x) = \frac{1}{1+e^{-x}}\) is sometimes also
known as the logistic function. It is a non-linear function used not
only in Machine Learning (Logistic Regression), but also in Deep
Learning.

To refer to a function belonging to a specific package you could call it
using \texttt{package\_name.function()}. Run the code below to see an
example with \texttt{math.exp()}.

    \begin{tcolorbox}[breakable, size=fbox, boxrule=1pt, pad at break*=1mm,colback=cellbackground, colframe=cellborder]
\prompt{In}{incolor}{4}{\boxspacing}
\begin{Verbatim}[commandchars=\\\{\}]
\PY{k+kn}{import} \PY{n+nn}{math}
\PY{k+kn}{from} \PY{n+nn}{public\PYZus{}tests} \PY{k+kn}{import} \PY{o}{*}

\PY{c+c1}{\PYZsh{} GRADED FUNCTION: basic\PYZus{}sigmoid}

\PY{k}{def} \PY{n+nf}{basic\PYZus{}sigmoid}\PY{p}{(}\PY{n}{x}\PY{p}{)}\PY{p}{:}
    \PY{l+s+sd}{\PYZdq{}\PYZdq{}\PYZdq{}}
\PY{l+s+sd}{    Compute sigmoid of x.}

\PY{l+s+sd}{    Arguments:}
\PY{l+s+sd}{    x \PYZhy{}\PYZhy{} A scalar}

\PY{l+s+sd}{    Return:}
\PY{l+s+sd}{    s \PYZhy{}\PYZhy{} sigmoid(x)}
\PY{l+s+sd}{    \PYZdq{}\PYZdq{}\PYZdq{}}
    \PY{c+c1}{\PYZsh{} (≈ 1 line of code)}
    \PY{c+c1}{\PYZsh{} s = }
    \PY{c+c1}{\PYZsh{} YOUR CODE STARTS HERE}
    \PY{n}{s} \PY{o}{=} \PY{l+m+mi}{1} \PY{o}{/} \PY{p}{(}\PY{l+m+mi}{1} \PY{o}{+} \PY{n}{math}\PY{o}{.}\PY{n}{exp}\PY{p}{(}\PY{o}{\PYZhy{}}\PY{n}{x}\PY{p}{)}\PY{p}{)}
    
    \PY{c+c1}{\PYZsh{} YOUR CODE ENDS HERE}
    
    \PY{k}{return} \PY{n}{s}
\end{Verbatim}
\end{tcolorbox}

    \begin{tcolorbox}[breakable, size=fbox, boxrule=1pt, pad at break*=1mm,colback=cellbackground, colframe=cellborder]
\prompt{In}{incolor}{5}{\boxspacing}
\begin{Verbatim}[commandchars=\\\{\}]
\PY{n+nb}{print}\PY{p}{(}\PY{l+s+s2}{\PYZdq{}}\PY{l+s+s2}{basic\PYZus{}sigmoid(1) = }\PY{l+s+s2}{\PYZdq{}} \PY{o}{+} \PY{n+nb}{str}\PY{p}{(}\PY{n}{basic\PYZus{}sigmoid}\PY{p}{(}\PY{l+m+mi}{1}\PY{p}{)}\PY{p}{)}\PY{p}{)}

\PY{n}{basic\PYZus{}sigmoid\PYZus{}test}\PY{p}{(}\PY{n}{basic\PYZus{}sigmoid}\PY{p}{)}
\end{Verbatim}
\end{tcolorbox}

    \begin{Verbatim}[commandchars=\\\{\}]
basic\_sigmoid(1) = 0.7310585786300049
\textcolor{ansi-green-intense}{ All tests passed.}
    \end{Verbatim}

    Actually, we rarely use the ``math'' library in deep learning because
the inputs of the functions are real numbers. In deep learning we mostly
use matrices and vectors. This is why numpy is more useful.

    \begin{tcolorbox}[breakable, size=fbox, boxrule=1pt, pad at break*=1mm,colback=cellbackground, colframe=cellborder]
\prompt{In}{incolor}{6}{\boxspacing}
\begin{Verbatim}[commandchars=\\\{\}]
\PY{c+c1}{\PYZsh{}\PYZsh{}\PYZsh{} One reason why we use \PYZdq{}numpy\PYZdq{} instead of \PYZdq{}math\PYZdq{} in Deep Learning \PYZsh{}\PYZsh{}\PYZsh{}}

\PY{n}{x} \PY{o}{=} \PY{p}{[}\PY{l+m+mi}{1}\PY{p}{,} \PY{l+m+mi}{2}\PY{p}{,} \PY{l+m+mi}{3}\PY{p}{]} \PY{c+c1}{\PYZsh{} x becomes a python list object}
\PY{n}{basic\PYZus{}sigmoid}\PY{p}{(}\PY{n}{x}\PY{p}{)} \PY{c+c1}{\PYZsh{} you will see this give an error when you run it, because x is a vector.}
\end{Verbatim}
\end{tcolorbox}

    \begin{Verbatim}[commandchars=\\\{\}]

        ---------------------------------------------------------------------------

        TypeError                                 Traceback (most recent call last)

        <ipython-input-6-f93a844e80fe> in <module>
          2 
          3 x = [1, 2, 3] \# x becomes a python list object
    ----> 4 basic\_sigmoid(x) \# you will see this give an error when you run it, because x is a vector.
    

        <ipython-input-4-6ae268e943b0> in basic\_sigmoid(x)
         17     \# s =
         18     \# YOUR CODE STARTS HERE
    ---> 19     s = 1 / (1 + math.exp(-x))
         20 
         21     \# YOUR CODE ENDS HERE


        TypeError: bad operand type for unary -: 'list'

    \end{Verbatim}

    In fact, if \$ x = (x\_1, x\_2, \ldots, x\_n)\$ is a row vector then
\texttt{np.exp(x)} will apply the exponential function to every element
of x. The output will thus be:
\texttt{np.exp(x)\ =\ (e\^{}\{x\_1\},\ e\^{}\{x\_2\},\ ...,\ e\^{}\{x\_n\})}

    \begin{tcolorbox}[breakable, size=fbox, boxrule=1pt, pad at break*=1mm,colback=cellbackground, colframe=cellborder]
\prompt{In}{incolor}{7}{\boxspacing}
\begin{Verbatim}[commandchars=\\\{\}]
\PY{k+kn}{import} \PY{n+nn}{numpy} \PY{k}{as} \PY{n+nn}{np}

\PY{c+c1}{\PYZsh{} example of np.exp}
\PY{n}{t\PYZus{}x} \PY{o}{=} \PY{n}{np}\PY{o}{.}\PY{n}{array}\PY{p}{(}\PY{p}{[}\PY{l+m+mi}{1}\PY{p}{,} \PY{l+m+mi}{2}\PY{p}{,} \PY{l+m+mi}{3}\PY{p}{]}\PY{p}{)}
\PY{n+nb}{print}\PY{p}{(}\PY{n}{np}\PY{o}{.}\PY{n}{exp}\PY{p}{(}\PY{n}{t\PYZus{}x}\PY{p}{)}\PY{p}{)} \PY{c+c1}{\PYZsh{} result is (exp(1), exp(2), exp(3))}
\end{Verbatim}
\end{tcolorbox}

    \begin{Verbatim}[commandchars=\\\{\}]
[ 2.71828183  7.3890561  20.08553692]
    \end{Verbatim}

    Furthermore, if x is a vector, then a Python operation such as
\(s = x + 3\) or \(s = \frac{1}{x}\) will output s as a vector of the
same size as x.

    \begin{tcolorbox}[breakable, size=fbox, boxrule=1pt, pad at break*=1mm,colback=cellbackground, colframe=cellborder]
\prompt{In}{incolor}{8}{\boxspacing}
\begin{Verbatim}[commandchars=\\\{\}]
\PY{c+c1}{\PYZsh{} example of vector operation}
\PY{n}{t\PYZus{}x} \PY{o}{=} \PY{n}{np}\PY{o}{.}\PY{n}{array}\PY{p}{(}\PY{p}{[}\PY{l+m+mi}{1}\PY{p}{,} \PY{l+m+mi}{2}\PY{p}{,} \PY{l+m+mi}{3}\PY{p}{]}\PY{p}{)}
\PY{n+nb}{print} \PY{p}{(}\PY{n}{t\PYZus{}x} \PY{o}{+} \PY{l+m+mi}{3}\PY{p}{)}
\end{Verbatim}
\end{tcolorbox}

    \begin{Verbatim}[commandchars=\\\{\}]
[4 5 6]
    \end{Verbatim}

    Any time you need more info on a numpy function, we encourage you to
look at
\href{https://docs.scipy.org/doc/numpy-1.10.1/reference/generated/numpy.exp.html}{the
official documentation}.

You can also create a new cell in the notebook and write
\texttt{np.exp?} (for example) to get quick access to the documentation.

\#\#\# Exercise 3 - sigmoid Implement the sigmoid function using numpy.

\textbf{Instructions}: x could now be either a real number, a vector, or
a matrix. The data structures we use in numpy to represent these shapes
(vectors, matrices\ldots) are called numpy arrays. You don't need to
know more for now.
\[ \text{For } x \in \mathbb{R}^n \text{,     } sigmoid(x) = sigmoid\begin{pmatrix}
    x_1  \\
    x_2  \\
    ...  \\
    x_n  \\
\end{pmatrix} = \begin{pmatrix}
    \frac{1}{1+e^{-x_1}}  \\
    \frac{1}{1+e^{-x_2}}  \\
    ...  \\
    \frac{1}{1+e^{-x_n}}  \\
\end{pmatrix}\tag{1} \]

    \begin{tcolorbox}[breakable, size=fbox, boxrule=1pt, pad at break*=1mm,colback=cellbackground, colframe=cellborder]
\prompt{In}{incolor}{11}{\boxspacing}
\begin{Verbatim}[commandchars=\\\{\}]
\PY{c+c1}{\PYZsh{} GRADED FUNCTION: sigmoid}

\PY{k}{def} \PY{n+nf}{sigmoid}\PY{p}{(}\PY{n}{x}\PY{p}{)}\PY{p}{:}
    \PY{l+s+sd}{\PYZdq{}\PYZdq{}\PYZdq{}}
\PY{l+s+sd}{    Compute the sigmoid of x}

\PY{l+s+sd}{    Arguments:}
\PY{l+s+sd}{    x \PYZhy{}\PYZhy{} A scalar or numpy array of any size}

\PY{l+s+sd}{    Return:}
\PY{l+s+sd}{    s \PYZhy{}\PYZhy{} sigmoid(x)}
\PY{l+s+sd}{    \PYZdq{}\PYZdq{}\PYZdq{}}
    
    \PY{c+c1}{\PYZsh{} (≈ 1 line of code)}
    \PY{c+c1}{\PYZsh{} s = }
    \PY{c+c1}{\PYZsh{} YOUR CODE STARTS HERE}
    \PY{n}{s} \PY{o}{=} \PY{l+m+mi}{1} \PY{o}{/} \PY{p}{(}\PY{l+m+mi}{1} \PY{o}{+} \PY{n}{np}\PY{o}{.}\PY{n}{exp}\PY{p}{(}\PY{o}{\PYZhy{}}\PY{n}{x}\PY{p}{)}\PY{p}{)}
    
    \PY{c+c1}{\PYZsh{} YOUR CODE ENDS HERE}
    
    \PY{k}{return} \PY{n}{s}
\end{Verbatim}
\end{tcolorbox}

    \begin{tcolorbox}[breakable, size=fbox, boxrule=1pt, pad at break*=1mm,colback=cellbackground, colframe=cellborder]
\prompt{In}{incolor}{12}{\boxspacing}
\begin{Verbatim}[commandchars=\\\{\}]
\PY{n}{t\PYZus{}x} \PY{o}{=} \PY{n}{np}\PY{o}{.}\PY{n}{array}\PY{p}{(}\PY{p}{[}\PY{l+m+mi}{1}\PY{p}{,} \PY{l+m+mi}{2}\PY{p}{,} \PY{l+m+mi}{3}\PY{p}{]}\PY{p}{)}
\PY{n+nb}{print}\PY{p}{(}\PY{l+s+s2}{\PYZdq{}}\PY{l+s+s2}{sigmoid(t\PYZus{}x) = }\PY{l+s+s2}{\PYZdq{}} \PY{o}{+} \PY{n+nb}{str}\PY{p}{(}\PY{n}{sigmoid}\PY{p}{(}\PY{n}{t\PYZus{}x}\PY{p}{)}\PY{p}{)}\PY{p}{)}

\PY{n}{sigmoid\PYZus{}test}\PY{p}{(}\PY{n}{sigmoid}\PY{p}{)}
\end{Verbatim}
\end{tcolorbox}

    \begin{Verbatim}[commandchars=\\\{\}]
sigmoid(t\_x) = [0.73105858 0.88079708 0.95257413]
\textcolor{ansi-green-intense}{ All tests passed.}
    \end{Verbatim}

    \#\#\# 1.2 - Sigmoid Gradient

As you've seen in lecture, you will need to compute gradients to
optimize loss functions using backpropagation. Let's code your first
gradient function.

\#\#\# Exercise 4 - sigmoid\_derivative Implement the function
sigmoid\_grad() to compute the gradient of the sigmoid function with
respect to its input x. The formula is:

\[sigmoid\_derivative(x) = \sigma'(x) = \sigma(x) (1 - \sigma(x))\tag{2}\]

You often code this function in two steps: 1. Set s to be the sigmoid of
x. You might find your sigmoid(x) function useful. 2. Compute
\(\sigma'(x) = s(1-s)\)

    \begin{tcolorbox}[breakable, size=fbox, boxrule=1pt, pad at break*=1mm,colback=cellbackground, colframe=cellborder]
\prompt{In}{incolor}{13}{\boxspacing}
\begin{Verbatim}[commandchars=\\\{\}]
\PY{c+c1}{\PYZsh{} GRADED FUNCTION: sigmoid\PYZus{}derivative}

\PY{k}{def} \PY{n+nf}{sigmoid\PYZus{}derivative}\PY{p}{(}\PY{n}{x}\PY{p}{)}\PY{p}{:}
    \PY{l+s+sd}{\PYZdq{}\PYZdq{}\PYZdq{}}
\PY{l+s+sd}{    Compute the gradient (also called the slope or derivative) of the sigmoid function with respect to its input x.}
\PY{l+s+sd}{    You can store the output of the sigmoid function into variables and then use it to calculate the gradient.}
\PY{l+s+sd}{    }
\PY{l+s+sd}{    Arguments:}
\PY{l+s+sd}{    x \PYZhy{}\PYZhy{} A scalar or numpy array}

\PY{l+s+sd}{    Return:}
\PY{l+s+sd}{    ds \PYZhy{}\PYZhy{} Your computed gradient.}
\PY{l+s+sd}{    \PYZdq{}\PYZdq{}\PYZdq{}}
    
    \PY{c+c1}{\PYZsh{}(≈ 2 lines of code)}
    \PY{c+c1}{\PYZsh{} s = }
    \PY{c+c1}{\PYZsh{} ds = }
    \PY{c+c1}{\PYZsh{} YOUR CODE STARTS HERE}
    \PY{n}{s} \PY{o}{=} \PY{l+m+mi}{1} \PY{o}{/} \PY{p}{(}\PY{l+m+mi}{1} \PY{o}{+} \PY{n}{np}\PY{o}{.}\PY{n}{exp}\PY{p}{(}\PY{o}{\PYZhy{}}\PY{n}{x}\PY{p}{)}\PY{p}{)}
    \PY{n}{ds} \PY{o}{=} \PY{n}{s}\PY{o}{*}\PY{p}{(}\PY{l+m+mi}{1}\PY{o}{\PYZhy{}}\PY{n}{s}\PY{p}{)}
    
    \PY{c+c1}{\PYZsh{} YOUR CODE ENDS HERE}
    
    \PY{k}{return} \PY{n}{ds}
\end{Verbatim}
\end{tcolorbox}

    \begin{tcolorbox}[breakable, size=fbox, boxrule=1pt, pad at break*=1mm,colback=cellbackground, colframe=cellborder]
\prompt{In}{incolor}{14}{\boxspacing}
\begin{Verbatim}[commandchars=\\\{\}]
\PY{n}{t\PYZus{}x} \PY{o}{=} \PY{n}{np}\PY{o}{.}\PY{n}{array}\PY{p}{(}\PY{p}{[}\PY{l+m+mi}{1}\PY{p}{,} \PY{l+m+mi}{2}\PY{p}{,} \PY{l+m+mi}{3}\PY{p}{]}\PY{p}{)}
\PY{n+nb}{print} \PY{p}{(}\PY{l+s+s2}{\PYZdq{}}\PY{l+s+s2}{sigmoid\PYZus{}derivative(t\PYZus{}x) = }\PY{l+s+s2}{\PYZdq{}} \PY{o}{+} \PY{n+nb}{str}\PY{p}{(}\PY{n}{sigmoid\PYZus{}derivative}\PY{p}{(}\PY{n}{t\PYZus{}x}\PY{p}{)}\PY{p}{)}\PY{p}{)}

\PY{n}{sigmoid\PYZus{}derivative\PYZus{}test}\PY{p}{(}\PY{n}{sigmoid\PYZus{}derivative}\PY{p}{)}
\end{Verbatim}
\end{tcolorbox}

    \begin{Verbatim}[commandchars=\\\{\}]
sigmoid\_derivative(t\_x) = [0.19661193 0.10499359 0.04517666]
\textcolor{ansi-green-intense}{ All tests passed.}
    \end{Verbatim}

    \#\#\# 1.3 - Reshaping arrays \#\#\#

Two common numpy functions used in deep learning are
\href{https://docs.scipy.org/doc/numpy/reference/generated/numpy.ndarray.shape.html}{np.shape}
and
\href{https://docs.scipy.org/doc/numpy/reference/generated/numpy.reshape.html}{np.reshape()}.
- X.shape is used to get the shape (dimension) of a matrix/vector X. -
X.reshape(\ldots) is used to reshape X into some other dimension.

For example, in computer science, an image is represented by a 3D array
of shape \((length, height, depth = 3)\). However, when you read an
image as the input of an algorithm you convert it to a vector of shape
\((length*height*3, 1)\). In other words, you ``unroll'', or reshape,
the 3D array into a 1D vector.

\#\#\# Exercise 5 - image2vector Implement \texttt{image2vector()} that
takes an input of shape (length, height, 3) and returns a vector of
shape (length*height*3, 1). For example, if you would like to reshape an
array v of shape (a, b, c) into a vector of shape (a*b,c) you would do:

\begin{Shaded}
\begin{Highlighting}[]
\NormalTok{v }\OperatorTok{=}\NormalTok{ v.reshape((v.shape[}\DecValTok{0}\NormalTok{] }\OperatorTok{*}\NormalTok{ v.shape[}\DecValTok{1}\NormalTok{], v.shape[}\DecValTok{2}\NormalTok{])) }\CommentTok{\# v.shape[0] = a ; v.shape[1] = b ; v.shape[2] = c}
\end{Highlighting}
\end{Shaded}

\begin{itemize}
\tightlist
\item
  Please don't hardcode the dimensions of image as a constant. Instead
  look up the quantities you need with \texttt{image.shape{[}0{]}}, etc.
\item
  You can use v = v.reshape(-1, 1). Just make sure you understand why it
  works.
\end{itemize}

    \begin{tcolorbox}[breakable, size=fbox, boxrule=1pt, pad at break*=1mm,colback=cellbackground, colframe=cellborder]
\prompt{In}{incolor}{15}{\boxspacing}
\begin{Verbatim}[commandchars=\\\{\}]
\PY{c+c1}{\PYZsh{} GRADED FUNCTION:image2vector}

\PY{k}{def} \PY{n+nf}{image2vector}\PY{p}{(}\PY{n}{image}\PY{p}{)}\PY{p}{:}
    \PY{l+s+sd}{\PYZdq{}\PYZdq{}\PYZdq{}}
\PY{l+s+sd}{    Argument:}
\PY{l+s+sd}{    image \PYZhy{}\PYZhy{} a numpy array of shape (length, height, depth)}
\PY{l+s+sd}{    }
\PY{l+s+sd}{    Returns:}
\PY{l+s+sd}{    v \PYZhy{}\PYZhy{} a vector of shape (length*height*depth, 1)}
\PY{l+s+sd}{    \PYZdq{}\PYZdq{}\PYZdq{}}
    
    \PY{c+c1}{\PYZsh{} (≈ 1 line of code)}
    \PY{c+c1}{\PYZsh{} v =}
    \PY{c+c1}{\PYZsh{} YOUR CODE STARTS HERE}
    \PY{n}{v} \PY{o}{=} \PY{n}{image}\PY{o}{.}\PY{n}{reshape}\PY{p}{(}\PY{n}{image}\PY{o}{.}\PY{n}{shape}\PY{p}{[}\PY{l+m+mi}{0}\PY{p}{]} \PY{o}{*} \PY{n}{image}\PY{o}{.}\PY{n}{shape}\PY{p}{[}\PY{l+m+mi}{1}\PY{p}{]} \PY{o}{*} \PY{n}{image}\PY{o}{.}\PY{n}{shape}\PY{p}{[}\PY{l+m+mi}{2}\PY{p}{]}\PY{p}{,} \PY{l+m+mi}{1}\PY{p}{)}
    \PY{c+c1}{\PYZsh{} YOUR CODE ENDS HERE}
    
    \PY{k}{return} \PY{n}{v}
\end{Verbatim}
\end{tcolorbox}

    \begin{tcolorbox}[breakable, size=fbox, boxrule=1pt, pad at break*=1mm,colback=cellbackground, colframe=cellborder]
\prompt{In}{incolor}{16}{\boxspacing}
\begin{Verbatim}[commandchars=\\\{\}]
\PY{c+c1}{\PYZsh{} This is a 3 by 3 by 2 array, typically images will be (num\PYZus{}px\PYZus{}x, num\PYZus{}px\PYZus{}y,3) where 3 represents the RGB values}
\PY{n}{t\PYZus{}image} \PY{o}{=} \PY{n}{np}\PY{o}{.}\PY{n}{array}\PY{p}{(}\PY{p}{[}\PY{p}{[}\PY{p}{[} \PY{l+m+mf}{0.67826139}\PY{p}{,}  \PY{l+m+mf}{0.29380381}\PY{p}{]}\PY{p}{,}
                     \PY{p}{[} \PY{l+m+mf}{0.90714982}\PY{p}{,}  \PY{l+m+mf}{0.52835647}\PY{p}{]}\PY{p}{,}
                     \PY{p}{[} \PY{l+m+mf}{0.4215251} \PY{p}{,}  \PY{l+m+mf}{0.45017551}\PY{p}{]}\PY{p}{]}\PY{p}{,}

                   \PY{p}{[}\PY{p}{[} \PY{l+m+mf}{0.92814219}\PY{p}{,}  \PY{l+m+mf}{0.96677647}\PY{p}{]}\PY{p}{,}
                    \PY{p}{[} \PY{l+m+mf}{0.85304703}\PY{p}{,}  \PY{l+m+mf}{0.52351845}\PY{p}{]}\PY{p}{,}
                    \PY{p}{[} \PY{l+m+mf}{0.19981397}\PY{p}{,}  \PY{l+m+mf}{0.27417313}\PY{p}{]}\PY{p}{]}\PY{p}{,}

                   \PY{p}{[}\PY{p}{[} \PY{l+m+mf}{0.60659855}\PY{p}{,}  \PY{l+m+mf}{0.00533165}\PY{p}{]}\PY{p}{,}
                    \PY{p}{[} \PY{l+m+mf}{0.10820313}\PY{p}{,}  \PY{l+m+mf}{0.49978937}\PY{p}{]}\PY{p}{,}
                    \PY{p}{[} \PY{l+m+mf}{0.34144279}\PY{p}{,}  \PY{l+m+mf}{0.94630077}\PY{p}{]}\PY{p}{]}\PY{p}{]}\PY{p}{)}

\PY{n+nb}{print} \PY{p}{(}\PY{l+s+s2}{\PYZdq{}}\PY{l+s+s2}{image2vector(image) = }\PY{l+s+s2}{\PYZdq{}} \PY{o}{+} \PY{n+nb}{str}\PY{p}{(}\PY{n}{image2vector}\PY{p}{(}\PY{n}{t\PYZus{}image}\PY{p}{)}\PY{p}{)}\PY{p}{)}

\PY{n}{image2vector\PYZus{}test}\PY{p}{(}\PY{n}{image2vector}\PY{p}{)}
\end{Verbatim}
\end{tcolorbox}

    \begin{Verbatim}[commandchars=\\\{\}]
image2vector(image) = [[0.67826139]
 [0.29380381]
 [0.90714982]
 [0.52835647]
 [0.4215251 ]
 [0.45017551]
 [0.92814219]
 [0.96677647]
 [0.85304703]
 [0.52351845]
 [0.19981397]
 [0.27417313]
 [0.60659855]
 [0.00533165]
 [0.10820313]
 [0.49978937]
 [0.34144279]
 [0.94630077]]
\textcolor{ansi-green-intense}{ All tests passed.}
    \end{Verbatim}

    \#\#\# 1.4 - Normalizing rows

Another common technique we use in Machine Learning and Deep Learning is
to normalize our data. It often leads to a better performance because
gradient descent converges faster after normalization. Here, by
normalization we mean changing x to \$ \frac{x}{\| x\|} \$ (dividing
each row vector of x by its norm).

For example, if \[x = \begin{bmatrix}
        0 & 3 & 4 \\
        2 & 6 & 4 \\
\end{bmatrix}\tag{3}\] then
\[\| x\| = \text{np.linalg.norm(x, axis=1, keepdims=True)} = \begin{bmatrix}
    5 \\
    \sqrt{56} \\
\end{bmatrix}\tag{4} \] and
\[ x\_normalized = \frac{x}{\| x\|} = \begin{bmatrix}
    0 & \frac{3}{5} & \frac{4}{5} \\
    \frac{2}{\sqrt{56}} & \frac{6}{\sqrt{56}} & \frac{4}{\sqrt{56}} \\
\end{bmatrix}\tag{5}\]

Note that you can divide matrices of different sizes and it works fine:
this is called broadcasting and you're going to learn about it in part
5.

With \texttt{keepdims=True} the result will broadcast correctly against
the original x.

\texttt{axis=1} means you are going to get the norm in a row-wise
manner. If you need the norm in a column-wise way, you would need to set
\texttt{axis=0}.

numpy.linalg.norm has another parameter \texttt{ord} where we specify
the type of normalization to be done (in the exercise below you'll do
2-norm). To get familiar with the types of normalization you can visit
\href{https://numpy.org/doc/stable/reference/generated/numpy.linalg.norm.html}{numpy.linalg.norm}

\#\#\# Exercise 6 - normalize\_rows Implement normalizeRows() to
normalize the rows of a matrix. After applying this function to an input
matrix x, each row of x should be a vector of unit length (meaning
length 1).

\textbf{Note}: Don't try to use \texttt{x\ /=\ x\_norm}. For the matrix
division numpy must broadcast the x\_norm, which is not supported by the
operant \texttt{/=}

    \begin{tcolorbox}[breakable, size=fbox, boxrule=1pt, pad at break*=1mm,colback=cellbackground, colframe=cellborder]
\prompt{In}{incolor}{17}{\boxspacing}
\begin{Verbatim}[commandchars=\\\{\}]
\PY{c+c1}{\PYZsh{} GRADED FUNCTION: normalize\PYZus{}rows}

\PY{k}{def} \PY{n+nf}{normalize\PYZus{}rows}\PY{p}{(}\PY{n}{x}\PY{p}{)}\PY{p}{:}
    \PY{l+s+sd}{\PYZdq{}\PYZdq{}\PYZdq{}}
\PY{l+s+sd}{    Implement a function that normalizes each row of the matrix x (to have unit length).}
\PY{l+s+sd}{    }
\PY{l+s+sd}{    Argument:}
\PY{l+s+sd}{    x \PYZhy{}\PYZhy{} A numpy matrix of shape (n, m)}
\PY{l+s+sd}{    }
\PY{l+s+sd}{    Returns:}
\PY{l+s+sd}{    x \PYZhy{}\PYZhy{} The normalized (by row) numpy matrix. You are allowed to modify x.}
\PY{l+s+sd}{    \PYZdq{}\PYZdq{}\PYZdq{}}
    
    \PY{c+c1}{\PYZsh{}(≈ 2 lines of code)}
    \PY{c+c1}{\PYZsh{} Compute x\PYZus{}norm as the norm 2 of x. Use np.linalg.norm(..., ord = 2, axis = ..., keepdims = True)}
    \PY{c+c1}{\PYZsh{} x\PYZus{}norm =}
    \PY{c+c1}{\PYZsh{} Divide x by its norm.}
    \PY{c+c1}{\PYZsh{} x =}
    \PY{c+c1}{\PYZsh{} YOUR CODE STARTS HERE}
    \PY{n}{x\PYZus{}norm} \PY{o}{=} \PY{n}{np}\PY{o}{.}\PY{n}{linalg}\PY{o}{.}\PY{n}{norm}\PY{p}{(}\PY{n}{x}\PY{p}{,} \PY{n+nb}{ord} \PY{o}{=} \PY{l+m+mi}{2}\PY{p}{,} \PY{n}{axis} \PY{o}{=} \PY{l+m+mi}{1}\PY{p}{,} \PY{n}{keepdims} \PY{o}{=} \PY{k+kc}{True}\PY{p}{)}
    \PY{n}{x} \PY{o}{=} \PY{n}{x} \PY{o}{/} \PY{n}{x\PYZus{}norm}
    
    \PY{c+c1}{\PYZsh{} YOUR CODE ENDS HERE}

    \PY{k}{return} \PY{n}{x}
\end{Verbatim}
\end{tcolorbox}

    \begin{tcolorbox}[breakable, size=fbox, boxrule=1pt, pad at break*=1mm,colback=cellbackground, colframe=cellborder]
\prompt{In}{incolor}{18}{\boxspacing}
\begin{Verbatim}[commandchars=\\\{\}]
\PY{n}{x} \PY{o}{=} \PY{n}{np}\PY{o}{.}\PY{n}{array}\PY{p}{(}\PY{p}{[}\PY{p}{[}\PY{l+m+mi}{0}\PY{p}{,} \PY{l+m+mi}{3}\PY{p}{,} \PY{l+m+mi}{4}\PY{p}{]}\PY{p}{,}
              \PY{p}{[}\PY{l+m+mi}{1}\PY{p}{,} \PY{l+m+mi}{6}\PY{p}{,} \PY{l+m+mi}{4}\PY{p}{]}\PY{p}{]}\PY{p}{)}
\PY{n+nb}{print}\PY{p}{(}\PY{l+s+s2}{\PYZdq{}}\PY{l+s+s2}{normalizeRows(x) = }\PY{l+s+s2}{\PYZdq{}} \PY{o}{+} \PY{n+nb}{str}\PY{p}{(}\PY{n}{normalize\PYZus{}rows}\PY{p}{(}\PY{n}{x}\PY{p}{)}\PY{p}{)}\PY{p}{)}

\PY{n}{normalizeRows\PYZus{}test}\PY{p}{(}\PY{n}{normalize\PYZus{}rows}\PY{p}{)}
\end{Verbatim}
\end{tcolorbox}

    \begin{Verbatim}[commandchars=\\\{\}]
normalizeRows(x) = [[0.         0.6        0.8       ]
 [0.13736056 0.82416338 0.54944226]]
\textcolor{ansi-green-intense}{ All tests passed.}
    \end{Verbatim}

    \textbf{Note}: In normalize\_rows(), you can try to print the shapes of
x\_norm and x, and then rerun the assessment. You'll find out that they
have different shapes. This is normal given that x\_norm takes the norm
of each row of x. So x\_norm has the same number of rows but only 1
column. So how did it work when you divided x by x\_norm? This is called
broadcasting and we'll talk about it now!

    \#\#\# Exercise 7 - softmax Implement a softmax function using numpy.
You can think of softmax as a normalizing function used when your
algorithm needs to classify two or more classes. You will learn more
about softmax in the second course of this specialization.

\textbf{Instructions}: -
\(\text{for } x \in \mathbb{R}^{1\times n} \text{,     }\)

\begin{align*}
 softmax(x) &= softmax\left(\begin{bmatrix}
    x_1  &&
    x_2 &&
    ...  &&
    x_n  
\end{bmatrix}\right) \\&= \begin{bmatrix}
    \frac{e^{x_1}}{\sum_{j}e^{x_j}}  &&
    \frac{e^{x_2}}{\sum_{j}e^{x_j}}  &&
    ...  &&
    \frac{e^{x_n}}{\sum_{j}e^{x_j}} 
\end{bmatrix} 
\end{align*}

\begin{itemize}
\tightlist
\item
  \(\text{for a matrix } x \in \mathbb{R}^{m \times n} \text{,  $x_{ij}$ maps to the element in the $i^{th}$ row and $j^{th}$ column of $x$, thus we have: }\)
\end{itemize}

\begin{align*}
softmax(x) &= softmax\begin{bmatrix}
            x_{11} & x_{12} & x_{13} & \dots  & x_{1n} \\
            x_{21} & x_{22} & x_{23} & \dots  & x_{2n} \\
            \vdots & \vdots & \vdots & \ddots & \vdots \\
            x_{m1} & x_{m2} & x_{m3} & \dots  & x_{mn}
            \end{bmatrix} \\ \\&= 
 \begin{bmatrix}
    \frac{e^{x_{11}}}{\sum_{j}e^{x_{1j}}} & \frac{e^{x_{12}}}{\sum_{j}e^{x_{1j}}} & \frac{e^{x_{13}}}{\sum_{j}e^{x_{1j}}} & \dots  & \frac{e^{x_{1n}}}{\sum_{j}e^{x_{1j}}} \\
    \frac{e^{x_{21}}}{\sum_{j}e^{x_{2j}}} & \frac{e^{x_{22}}}{\sum_{j}e^{x_{2j}}} & \frac{e^{x_{23}}}{\sum_{j}e^{x_{2j}}} & \dots  & \frac{e^{x_{2n}}}{\sum_{j}e^{x_{2j}}} \\
    \vdots & \vdots & \vdots & \ddots & \vdots \\
    \frac{e^{x_{m1}}}{\sum_{j}e^{x_{mj}}} & \frac{e^{x_{m2}}}{\sum_{j}e^{x_{mj}}} & \frac{e^{x_{m3}}}{\sum_{j}e^{x_{mj}}} & \dots  & \frac{e^{x_{mn}}}{\sum_{j}e^{x_{mj}}}
\end{bmatrix} \\ \\ &= \begin{pmatrix}
    softmax\text{(first row of x)}  \\
    softmax\text{(second row of x)} \\
    \vdots  \\
    softmax\text{(last row of x)} \\
\end{pmatrix} 
\end{align*}

    \textbf{Notes:} Note that later in the course, you'll see ``m'' used to
represent the ``number of training examples'', and each training example
is in its own column of the matrix. Also, each feature will be in its
own row (each row has data for the same feature).\\
Softmax should be performed for all features of each training example,
so softmax would be performed on the columns (once we switch to that
representation later in this course).

However, in this coding practice, we're just focusing on getting
familiar with Python, so we're using the common math notation
\(m \times n\)\\
where \(m\) is the number of rows and \(n\) is the number of columns.

    \begin{tcolorbox}[breakable, size=fbox, boxrule=1pt, pad at break*=1mm,colback=cellbackground, colframe=cellborder]
\prompt{In}{incolor}{21}{\boxspacing}
\begin{Verbatim}[commandchars=\\\{\}]
\PY{c+c1}{\PYZsh{} GRADED FUNCTION: softmax}

\PY{k}{def} \PY{n+nf}{softmax}\PY{p}{(}\PY{n}{x}\PY{p}{)}\PY{p}{:}
    \PY{l+s+sd}{\PYZdq{}\PYZdq{}\PYZdq{}Calculates the softmax for each row of the input x.}

\PY{l+s+sd}{    Your code should work for a row vector and also for matrices of shape (m,n).}

\PY{l+s+sd}{    Argument:}
\PY{l+s+sd}{    x \PYZhy{}\PYZhy{} A numpy matrix of shape (m,n)}

\PY{l+s+sd}{    Returns:}
\PY{l+s+sd}{    s \PYZhy{}\PYZhy{} A numpy matrix equal to the softmax of x, of shape (m,n)}
\PY{l+s+sd}{    \PYZdq{}\PYZdq{}\PYZdq{}}
    
    \PY{c+c1}{\PYZsh{}(≈ 3 lines of code)}
    \PY{c+c1}{\PYZsh{} Apply exp() element\PYZhy{}wise to x. Use np.exp(...).}
    \PY{c+c1}{\PYZsh{} x\PYZus{}exp = ...}

    \PY{c+c1}{\PYZsh{} Create a vector x\PYZus{}sum that sums each row of x\PYZus{}exp. Use np.sum(..., axis = 1, keepdims = True).}
    \PY{c+c1}{\PYZsh{} x\PYZus{}sum = ...}
    
    \PY{c+c1}{\PYZsh{} Compute softmax(x) by dividing x\PYZus{}exp by x\PYZus{}sum. It should automatically use numpy broadcasting.}
    \PY{c+c1}{\PYZsh{} s = ...}
    
    \PY{c+c1}{\PYZsh{} YOUR CODE STARTS HERE}
    \PY{n}{x\PYZus{}exp} \PY{o}{=} \PY{n}{np}\PY{o}{.}\PY{n}{exp}\PY{p}{(}\PY{n}{x}\PY{p}{)}
    \PY{n}{x\PYZus{}sum} \PY{o}{=} \PY{n}{np}\PY{o}{.}\PY{n}{sum}\PY{p}{(}\PY{n}{x\PYZus{}exp}\PY{p}{,} \PY{n}{axis} \PY{o}{=} \PY{l+m+mi}{1}\PY{p}{,} \PY{n}{keepdims} \PY{o}{=} \PY{k+kc}{True}\PY{p}{)}
    \PY{n}{s} \PY{o}{=} \PY{n}{x\PYZus{}exp} \PY{o}{/} \PY{n}{x\PYZus{}sum}
    
    \PY{c+c1}{\PYZsh{} YOUR CODE ENDS HERE}
    
    \PY{k}{return} \PY{n}{s}
\end{Verbatim}
\end{tcolorbox}

    \begin{tcolorbox}[breakable, size=fbox, boxrule=1pt, pad at break*=1mm,colback=cellbackground, colframe=cellborder]
\prompt{In}{incolor}{22}{\boxspacing}
\begin{Verbatim}[commandchars=\\\{\}]
\PY{n}{t\PYZus{}x} \PY{o}{=} \PY{n}{np}\PY{o}{.}\PY{n}{array}\PY{p}{(}\PY{p}{[}\PY{p}{[}\PY{l+m+mi}{9}\PY{p}{,} \PY{l+m+mi}{2}\PY{p}{,} \PY{l+m+mi}{5}\PY{p}{,} \PY{l+m+mi}{0}\PY{p}{,} \PY{l+m+mi}{0}\PY{p}{]}\PY{p}{,}
                \PY{p}{[}\PY{l+m+mi}{7}\PY{p}{,} \PY{l+m+mi}{5}\PY{p}{,} \PY{l+m+mi}{0}\PY{p}{,} \PY{l+m+mi}{0} \PY{p}{,}\PY{l+m+mi}{0}\PY{p}{]}\PY{p}{]}\PY{p}{)}
\PY{n+nb}{print}\PY{p}{(}\PY{l+s+s2}{\PYZdq{}}\PY{l+s+s2}{softmax(x) = }\PY{l+s+s2}{\PYZdq{}} \PY{o}{+} \PY{n+nb}{str}\PY{p}{(}\PY{n}{softmax}\PY{p}{(}\PY{n}{t\PYZus{}x}\PY{p}{)}\PY{p}{)}\PY{p}{)}

\PY{n}{softmax\PYZus{}test}\PY{p}{(}\PY{n}{softmax}\PY{p}{)}
\end{Verbatim}
\end{tcolorbox}

    \begin{Verbatim}[commandchars=\\\{\}]
softmax(x) = [[9.80897665e-01 8.94462891e-04 1.79657674e-02 1.21052389e-04
  1.21052389e-04]
 [8.78679856e-01 1.18916387e-01 8.01252314e-04 8.01252314e-04
  8.01252314e-04]]
\textcolor{ansi-green-intense}{ All tests passed.}
    \end{Verbatim}

    \hypertarget{notes}{%
\paragraph{Notes}\label{notes}}

\begin{itemize}
\tightlist
\item
  If you print the shapes of x\_exp, x\_sum and s above and rerun the
  assessment cell, you will see that x\_sum is of shape (2,1) while
  x\_exp and s are of shape (2,5). \textbf{x\_exp/x\_sum} works due to
  python broadcasting.
\end{itemize}

Congratulations! You now have a pretty good understanding of python
numpy and have implemented a few useful functions that you will be using
in deep learning.

    What you need to remember:

\begin{itemize}
\tightlist
\item
  np.exp(x) works for any np.array x and applies the exponential
  function to every coordinate
\item
  the sigmoid function and its gradient
\item
  image2vector is commonly used in deep learning
\item
  np.reshape is widely used. In the future, you'll see that keeping your
  matrix/vector dimensions straight will go toward eliminating a lot of
  bugs.
\item
  numpy has efficient built-in functions
\item
  broadcasting is extremely useful
\end{itemize}

    \#\# 2 - Vectorization

In deep learning, you deal with very large datasets. Hence, a
non-computationally-optimal function can become a huge bottleneck in
your algorithm and can result in a model that takes ages to run. To make
sure that your code is computationally efficient, you will use
vectorization. For example, try to tell the difference between the
following implementations of the dot/outer/elementwise product.

    \begin{tcolorbox}[breakable, size=fbox, boxrule=1pt, pad at break*=1mm,colback=cellbackground, colframe=cellborder]
\prompt{In}{incolor}{23}{\boxspacing}
\begin{Verbatim}[commandchars=\\\{\}]
\PY{k+kn}{import} \PY{n+nn}{time}

\PY{n}{x1} \PY{o}{=} \PY{p}{[}\PY{l+m+mi}{9}\PY{p}{,} \PY{l+m+mi}{2}\PY{p}{,} \PY{l+m+mi}{5}\PY{p}{,} \PY{l+m+mi}{0}\PY{p}{,} \PY{l+m+mi}{0}\PY{p}{,} \PY{l+m+mi}{7}\PY{p}{,} \PY{l+m+mi}{5}\PY{p}{,} \PY{l+m+mi}{0}\PY{p}{,} \PY{l+m+mi}{0}\PY{p}{,} \PY{l+m+mi}{0}\PY{p}{,} \PY{l+m+mi}{9}\PY{p}{,} \PY{l+m+mi}{2}\PY{p}{,} \PY{l+m+mi}{5}\PY{p}{,} \PY{l+m+mi}{0}\PY{p}{,} \PY{l+m+mi}{0}\PY{p}{]}
\PY{n}{x2} \PY{o}{=} \PY{p}{[}\PY{l+m+mi}{9}\PY{p}{,} \PY{l+m+mi}{2}\PY{p}{,} \PY{l+m+mi}{2}\PY{p}{,} \PY{l+m+mi}{9}\PY{p}{,} \PY{l+m+mi}{0}\PY{p}{,} \PY{l+m+mi}{9}\PY{p}{,} \PY{l+m+mi}{2}\PY{p}{,} \PY{l+m+mi}{5}\PY{p}{,} \PY{l+m+mi}{0}\PY{p}{,} \PY{l+m+mi}{0}\PY{p}{,} \PY{l+m+mi}{9}\PY{p}{,} \PY{l+m+mi}{2}\PY{p}{,} \PY{l+m+mi}{5}\PY{p}{,} \PY{l+m+mi}{0}\PY{p}{,} \PY{l+m+mi}{0}\PY{p}{]}

\PY{c+c1}{\PYZsh{}\PYZsh{}\PYZsh{} CLASSIC DOT PRODUCT OF VECTORS IMPLEMENTATION \PYZsh{}\PYZsh{}\PYZsh{}}
\PY{n}{tic} \PY{o}{=} \PY{n}{time}\PY{o}{.}\PY{n}{process\PYZus{}time}\PY{p}{(}\PY{p}{)}
\PY{n}{dot} \PY{o}{=} \PY{l+m+mi}{0}

\PY{k}{for} \PY{n}{i} \PY{o+ow}{in} \PY{n+nb}{range}\PY{p}{(}\PY{n+nb}{len}\PY{p}{(}\PY{n}{x1}\PY{p}{)}\PY{p}{)}\PY{p}{:}
    \PY{n}{dot} \PY{o}{+}\PY{o}{=} \PY{n}{x1}\PY{p}{[}\PY{n}{i}\PY{p}{]} \PY{o}{*} \PY{n}{x2}\PY{p}{[}\PY{n}{i}\PY{p}{]}
\PY{n}{toc} \PY{o}{=} \PY{n}{time}\PY{o}{.}\PY{n}{process\PYZus{}time}\PY{p}{(}\PY{p}{)}
\PY{n+nb}{print} \PY{p}{(}\PY{l+s+s2}{\PYZdq{}}\PY{l+s+s2}{dot = }\PY{l+s+s2}{\PYZdq{}} \PY{o}{+} \PY{n+nb}{str}\PY{p}{(}\PY{n}{dot}\PY{p}{)} \PY{o}{+} \PY{l+s+s2}{\PYZdq{}}\PY{l+s+se}{\PYZbs{}n}\PY{l+s+s2}{ \PYZhy{}\PYZhy{}\PYZhy{}\PYZhy{}\PYZhy{} Computation time = }\PY{l+s+s2}{\PYZdq{}} \PY{o}{+} \PY{n+nb}{str}\PY{p}{(}\PY{l+m+mi}{1000} \PY{o}{*} \PY{p}{(}\PY{n}{toc} \PY{o}{\PYZhy{}} \PY{n}{tic}\PY{p}{)}\PY{p}{)} \PY{o}{+} \PY{l+s+s2}{\PYZdq{}}\PY{l+s+s2}{ms}\PY{l+s+s2}{\PYZdq{}}\PY{p}{)}

\PY{c+c1}{\PYZsh{}\PYZsh{}\PYZsh{} CLASSIC OUTER PRODUCT IMPLEMENTATION \PYZsh{}\PYZsh{}\PYZsh{}}
\PY{n}{tic} \PY{o}{=} \PY{n}{time}\PY{o}{.}\PY{n}{process\PYZus{}time}\PY{p}{(}\PY{p}{)}
\PY{n}{outer} \PY{o}{=} \PY{n}{np}\PY{o}{.}\PY{n}{zeros}\PY{p}{(}\PY{p}{(}\PY{n+nb}{len}\PY{p}{(}\PY{n}{x1}\PY{p}{)}\PY{p}{,} \PY{n+nb}{len}\PY{p}{(}\PY{n}{x2}\PY{p}{)}\PY{p}{)}\PY{p}{)} \PY{c+c1}{\PYZsh{} we create a len(x1)*len(x2) matrix with only zeros}

\PY{k}{for} \PY{n}{i} \PY{o+ow}{in} \PY{n+nb}{range}\PY{p}{(}\PY{n+nb}{len}\PY{p}{(}\PY{n}{x1}\PY{p}{)}\PY{p}{)}\PY{p}{:}
    \PY{k}{for} \PY{n}{j} \PY{o+ow}{in} \PY{n+nb}{range}\PY{p}{(}\PY{n+nb}{len}\PY{p}{(}\PY{n}{x2}\PY{p}{)}\PY{p}{)}\PY{p}{:}
        \PY{n}{outer}\PY{p}{[}\PY{n}{i}\PY{p}{,}\PY{n}{j}\PY{p}{]} \PY{o}{=} \PY{n}{x1}\PY{p}{[}\PY{n}{i}\PY{p}{]} \PY{o}{*} \PY{n}{x2}\PY{p}{[}\PY{n}{j}\PY{p}{]}
\PY{n}{toc} \PY{o}{=} \PY{n}{time}\PY{o}{.}\PY{n}{process\PYZus{}time}\PY{p}{(}\PY{p}{)}
\PY{n+nb}{print} \PY{p}{(}\PY{l+s+s2}{\PYZdq{}}\PY{l+s+s2}{outer = }\PY{l+s+s2}{\PYZdq{}} \PY{o}{+} \PY{n+nb}{str}\PY{p}{(}\PY{n}{outer}\PY{p}{)} \PY{o}{+} \PY{l+s+s2}{\PYZdq{}}\PY{l+s+se}{\PYZbs{}n}\PY{l+s+s2}{ \PYZhy{}\PYZhy{}\PYZhy{}\PYZhy{}\PYZhy{} Computation time = }\PY{l+s+s2}{\PYZdq{}} \PY{o}{+} \PY{n+nb}{str}\PY{p}{(}\PY{l+m+mi}{1000} \PY{o}{*} \PY{p}{(}\PY{n}{toc} \PY{o}{\PYZhy{}} \PY{n}{tic}\PY{p}{)}\PY{p}{)} \PY{o}{+} \PY{l+s+s2}{\PYZdq{}}\PY{l+s+s2}{ms}\PY{l+s+s2}{\PYZdq{}}\PY{p}{)}

\PY{c+c1}{\PYZsh{}\PYZsh{}\PYZsh{} CLASSIC ELEMENTWISE IMPLEMENTATION \PYZsh{}\PYZsh{}\PYZsh{}}
\PY{n}{tic} \PY{o}{=} \PY{n}{time}\PY{o}{.}\PY{n}{process\PYZus{}time}\PY{p}{(}\PY{p}{)}
\PY{n}{mul} \PY{o}{=} \PY{n}{np}\PY{o}{.}\PY{n}{zeros}\PY{p}{(}\PY{n+nb}{len}\PY{p}{(}\PY{n}{x1}\PY{p}{)}\PY{p}{)}

\PY{k}{for} \PY{n}{i} \PY{o+ow}{in} \PY{n+nb}{range}\PY{p}{(}\PY{n+nb}{len}\PY{p}{(}\PY{n}{x1}\PY{p}{)}\PY{p}{)}\PY{p}{:}
    \PY{n}{mul}\PY{p}{[}\PY{n}{i}\PY{p}{]} \PY{o}{=} \PY{n}{x1}\PY{p}{[}\PY{n}{i}\PY{p}{]} \PY{o}{*} \PY{n}{x2}\PY{p}{[}\PY{n}{i}\PY{p}{]}
\PY{n}{toc} \PY{o}{=} \PY{n}{time}\PY{o}{.}\PY{n}{process\PYZus{}time}\PY{p}{(}\PY{p}{)}
\PY{n+nb}{print} \PY{p}{(}\PY{l+s+s2}{\PYZdq{}}\PY{l+s+s2}{elementwise multiplication = }\PY{l+s+s2}{\PYZdq{}} \PY{o}{+} \PY{n+nb}{str}\PY{p}{(}\PY{n}{mul}\PY{p}{)} \PY{o}{+} \PY{l+s+s2}{\PYZdq{}}\PY{l+s+se}{\PYZbs{}n}\PY{l+s+s2}{ \PYZhy{}\PYZhy{}\PYZhy{}\PYZhy{}\PYZhy{} Computation time = }\PY{l+s+s2}{\PYZdq{}} \PY{o}{+} \PY{n+nb}{str}\PY{p}{(}\PY{l+m+mi}{1000} \PY{o}{*} \PY{p}{(}\PY{n}{toc} \PY{o}{\PYZhy{}} \PY{n}{tic}\PY{p}{)}\PY{p}{)} \PY{o}{+} \PY{l+s+s2}{\PYZdq{}}\PY{l+s+s2}{ms}\PY{l+s+s2}{\PYZdq{}}\PY{p}{)}

\PY{c+c1}{\PYZsh{}\PYZsh{}\PYZsh{} CLASSIC GENERAL DOT PRODUCT IMPLEMENTATION \PYZsh{}\PYZsh{}\PYZsh{}}
\PY{n}{W} \PY{o}{=} \PY{n}{np}\PY{o}{.}\PY{n}{random}\PY{o}{.}\PY{n}{rand}\PY{p}{(}\PY{l+m+mi}{3}\PY{p}{,}\PY{n+nb}{len}\PY{p}{(}\PY{n}{x1}\PY{p}{)}\PY{p}{)} \PY{c+c1}{\PYZsh{} Random 3*len(x1) numpy array}
\PY{n}{tic} \PY{o}{=} \PY{n}{time}\PY{o}{.}\PY{n}{process\PYZus{}time}\PY{p}{(}\PY{p}{)}
\PY{n}{gdot} \PY{o}{=} \PY{n}{np}\PY{o}{.}\PY{n}{zeros}\PY{p}{(}\PY{n}{W}\PY{o}{.}\PY{n}{shape}\PY{p}{[}\PY{l+m+mi}{0}\PY{p}{]}\PY{p}{)}

\PY{k}{for} \PY{n}{i} \PY{o+ow}{in} \PY{n+nb}{range}\PY{p}{(}\PY{n}{W}\PY{o}{.}\PY{n}{shape}\PY{p}{[}\PY{l+m+mi}{0}\PY{p}{]}\PY{p}{)}\PY{p}{:}
    \PY{k}{for} \PY{n}{j} \PY{o+ow}{in} \PY{n+nb}{range}\PY{p}{(}\PY{n+nb}{len}\PY{p}{(}\PY{n}{x1}\PY{p}{)}\PY{p}{)}\PY{p}{:}
        \PY{n}{gdot}\PY{p}{[}\PY{n}{i}\PY{p}{]} \PY{o}{+}\PY{o}{=} \PY{n}{W}\PY{p}{[}\PY{n}{i}\PY{p}{,}\PY{n}{j}\PY{p}{]} \PY{o}{*} \PY{n}{x1}\PY{p}{[}\PY{n}{j}\PY{p}{]}
\PY{n}{toc} \PY{o}{=} \PY{n}{time}\PY{o}{.}\PY{n}{process\PYZus{}time}\PY{p}{(}\PY{p}{)}
\PY{n+nb}{print} \PY{p}{(}\PY{l+s+s2}{\PYZdq{}}\PY{l+s+s2}{gdot = }\PY{l+s+s2}{\PYZdq{}} \PY{o}{+} \PY{n+nb}{str}\PY{p}{(}\PY{n}{gdot}\PY{p}{)} \PY{o}{+} \PY{l+s+s2}{\PYZdq{}}\PY{l+s+se}{\PYZbs{}n}\PY{l+s+s2}{ \PYZhy{}\PYZhy{}\PYZhy{}\PYZhy{}\PYZhy{} Computation time = }\PY{l+s+s2}{\PYZdq{}} \PY{o}{+} \PY{n+nb}{str}\PY{p}{(}\PY{l+m+mi}{1000} \PY{o}{*} \PY{p}{(}\PY{n}{toc} \PY{o}{\PYZhy{}} \PY{n}{tic}\PY{p}{)}\PY{p}{)} \PY{o}{+} \PY{l+s+s2}{\PYZdq{}}\PY{l+s+s2}{ms}\PY{l+s+s2}{\PYZdq{}}\PY{p}{)}
\end{Verbatim}
\end{tcolorbox}

    \begin{Verbatim}[commandchars=\\\{\}]
dot = 278
 ----- Computation time = 0.09055099999999427ms
outer = [[81. 18. 18. 81.  0. 81. 18. 45.  0.  0. 81. 18. 45.  0.  0.]
 [18.  4.  4. 18.  0. 18.  4. 10.  0.  0. 18.  4. 10.  0.  0.]
 [45. 10. 10. 45.  0. 45. 10. 25.  0.  0. 45. 10. 25.  0.  0.]
 [ 0.  0.  0.  0.  0.  0.  0.  0.  0.  0.  0.  0.  0.  0.  0.]
 [ 0.  0.  0.  0.  0.  0.  0.  0.  0.  0.  0.  0.  0.  0.  0.]
 [63. 14. 14. 63.  0. 63. 14. 35.  0.  0. 63. 14. 35.  0.  0.]
 [45. 10. 10. 45.  0. 45. 10. 25.  0.  0. 45. 10. 25.  0.  0.]
 [ 0.  0.  0.  0.  0.  0.  0.  0.  0.  0.  0.  0.  0.  0.  0.]
 [ 0.  0.  0.  0.  0.  0.  0.  0.  0.  0.  0.  0.  0.  0.  0.]
 [ 0.  0.  0.  0.  0.  0.  0.  0.  0.  0.  0.  0.  0.  0.  0.]
 [81. 18. 18. 81.  0. 81. 18. 45.  0.  0. 81. 18. 45.  0.  0.]
 [18.  4.  4. 18.  0. 18.  4. 10.  0.  0. 18.  4. 10.  0.  0.]
 [45. 10. 10. 45.  0. 45. 10. 25.  0.  0. 45. 10. 25.  0.  0.]
 [ 0.  0.  0.  0.  0.  0.  0.  0.  0.  0.  0.  0.  0.  0.  0.]
 [ 0.  0.  0.  0.  0.  0.  0.  0.  0.  0.  0.  0.  0.  0.  0.]]
 ----- Computation time = 0.22034899999989754ms
elementwise multiplication = [81.  4. 10.  0.  0. 63. 10.  0.  0.  0. 81.  4.
25.  0.  0.]
 ----- Computation time = 0.11060199999990417ms
gdot = [18.18824772 26.11195249 19.08893009]
 ----- Computation time = 0.18324199999986135ms
    \end{Verbatim}

    \begin{tcolorbox}[breakable, size=fbox, boxrule=1pt, pad at break*=1mm,colback=cellbackground, colframe=cellborder]
\prompt{In}{incolor}{24}{\boxspacing}
\begin{Verbatim}[commandchars=\\\{\}]
\PY{n}{x1} \PY{o}{=} \PY{p}{[}\PY{l+m+mi}{9}\PY{p}{,} \PY{l+m+mi}{2}\PY{p}{,} \PY{l+m+mi}{5}\PY{p}{,} \PY{l+m+mi}{0}\PY{p}{,} \PY{l+m+mi}{0}\PY{p}{,} \PY{l+m+mi}{7}\PY{p}{,} \PY{l+m+mi}{5}\PY{p}{,} \PY{l+m+mi}{0}\PY{p}{,} \PY{l+m+mi}{0}\PY{p}{,} \PY{l+m+mi}{0}\PY{p}{,} \PY{l+m+mi}{9}\PY{p}{,} \PY{l+m+mi}{2}\PY{p}{,} \PY{l+m+mi}{5}\PY{p}{,} \PY{l+m+mi}{0}\PY{p}{,} \PY{l+m+mi}{0}\PY{p}{]}
\PY{n}{x2} \PY{o}{=} \PY{p}{[}\PY{l+m+mi}{9}\PY{p}{,} \PY{l+m+mi}{2}\PY{p}{,} \PY{l+m+mi}{2}\PY{p}{,} \PY{l+m+mi}{9}\PY{p}{,} \PY{l+m+mi}{0}\PY{p}{,} \PY{l+m+mi}{9}\PY{p}{,} \PY{l+m+mi}{2}\PY{p}{,} \PY{l+m+mi}{5}\PY{p}{,} \PY{l+m+mi}{0}\PY{p}{,} \PY{l+m+mi}{0}\PY{p}{,} \PY{l+m+mi}{9}\PY{p}{,} \PY{l+m+mi}{2}\PY{p}{,} \PY{l+m+mi}{5}\PY{p}{,} \PY{l+m+mi}{0}\PY{p}{,} \PY{l+m+mi}{0}\PY{p}{]}

\PY{c+c1}{\PYZsh{}\PYZsh{}\PYZsh{} VECTORIZED DOT PRODUCT OF VECTORS \PYZsh{}\PYZsh{}\PYZsh{}}
\PY{n}{tic} \PY{o}{=} \PY{n}{time}\PY{o}{.}\PY{n}{process\PYZus{}time}\PY{p}{(}\PY{p}{)}
\PY{n}{dot} \PY{o}{=} \PY{n}{np}\PY{o}{.}\PY{n}{dot}\PY{p}{(}\PY{n}{x1}\PY{p}{,}\PY{n}{x2}\PY{p}{)}
\PY{n}{toc} \PY{o}{=} \PY{n}{time}\PY{o}{.}\PY{n}{process\PYZus{}time}\PY{p}{(}\PY{p}{)}
\PY{n+nb}{print} \PY{p}{(}\PY{l+s+s2}{\PYZdq{}}\PY{l+s+s2}{dot = }\PY{l+s+s2}{\PYZdq{}} \PY{o}{+} \PY{n+nb}{str}\PY{p}{(}\PY{n}{dot}\PY{p}{)} \PY{o}{+} \PY{l+s+s2}{\PYZdq{}}\PY{l+s+se}{\PYZbs{}n}\PY{l+s+s2}{ \PYZhy{}\PYZhy{}\PYZhy{}\PYZhy{}\PYZhy{} Computation time = }\PY{l+s+s2}{\PYZdq{}} \PY{o}{+} \PY{n+nb}{str}\PY{p}{(}\PY{l+m+mi}{1000} \PY{o}{*} \PY{p}{(}\PY{n}{toc} \PY{o}{\PYZhy{}} \PY{n}{tic}\PY{p}{)}\PY{p}{)} \PY{o}{+} \PY{l+s+s2}{\PYZdq{}}\PY{l+s+s2}{ms}\PY{l+s+s2}{\PYZdq{}}\PY{p}{)}

\PY{c+c1}{\PYZsh{}\PYZsh{}\PYZsh{} VECTORIZED OUTER PRODUCT \PYZsh{}\PYZsh{}\PYZsh{}}
\PY{n}{tic} \PY{o}{=} \PY{n}{time}\PY{o}{.}\PY{n}{process\PYZus{}time}\PY{p}{(}\PY{p}{)}
\PY{n}{outer} \PY{o}{=} \PY{n}{np}\PY{o}{.}\PY{n}{outer}\PY{p}{(}\PY{n}{x1}\PY{p}{,}\PY{n}{x2}\PY{p}{)}
\PY{n}{toc} \PY{o}{=} \PY{n}{time}\PY{o}{.}\PY{n}{process\PYZus{}time}\PY{p}{(}\PY{p}{)}
\PY{n+nb}{print} \PY{p}{(}\PY{l+s+s2}{\PYZdq{}}\PY{l+s+s2}{outer = }\PY{l+s+s2}{\PYZdq{}} \PY{o}{+} \PY{n+nb}{str}\PY{p}{(}\PY{n}{outer}\PY{p}{)} \PY{o}{+} \PY{l+s+s2}{\PYZdq{}}\PY{l+s+se}{\PYZbs{}n}\PY{l+s+s2}{ \PYZhy{}\PYZhy{}\PYZhy{}\PYZhy{}\PYZhy{} Computation time = }\PY{l+s+s2}{\PYZdq{}} \PY{o}{+} \PY{n+nb}{str}\PY{p}{(}\PY{l+m+mi}{1000} \PY{o}{*} \PY{p}{(}\PY{n}{toc} \PY{o}{\PYZhy{}} \PY{n}{tic}\PY{p}{)}\PY{p}{)} \PY{o}{+} \PY{l+s+s2}{\PYZdq{}}\PY{l+s+s2}{ms}\PY{l+s+s2}{\PYZdq{}}\PY{p}{)}

\PY{c+c1}{\PYZsh{}\PYZsh{}\PYZsh{} VECTORIZED ELEMENTWISE MULTIPLICATION \PYZsh{}\PYZsh{}\PYZsh{}}
\PY{n}{tic} \PY{o}{=} \PY{n}{time}\PY{o}{.}\PY{n}{process\PYZus{}time}\PY{p}{(}\PY{p}{)}
\PY{n}{mul} \PY{o}{=} \PY{n}{np}\PY{o}{.}\PY{n}{multiply}\PY{p}{(}\PY{n}{x1}\PY{p}{,}\PY{n}{x2}\PY{p}{)}
\PY{n}{toc} \PY{o}{=} \PY{n}{time}\PY{o}{.}\PY{n}{process\PYZus{}time}\PY{p}{(}\PY{p}{)}
\PY{n+nb}{print} \PY{p}{(}\PY{l+s+s2}{\PYZdq{}}\PY{l+s+s2}{elementwise multiplication = }\PY{l+s+s2}{\PYZdq{}} \PY{o}{+} \PY{n+nb}{str}\PY{p}{(}\PY{n}{mul}\PY{p}{)} \PY{o}{+} \PY{l+s+s2}{\PYZdq{}}\PY{l+s+se}{\PYZbs{}n}\PY{l+s+s2}{ \PYZhy{}\PYZhy{}\PYZhy{}\PYZhy{}\PYZhy{} Computation time = }\PY{l+s+s2}{\PYZdq{}} \PY{o}{+} \PY{n+nb}{str}\PY{p}{(}\PY{l+m+mi}{1000}\PY{o}{*}\PY{p}{(}\PY{n}{toc} \PY{o}{\PYZhy{}} \PY{n}{tic}\PY{p}{)}\PY{p}{)} \PY{o}{+} \PY{l+s+s2}{\PYZdq{}}\PY{l+s+s2}{ms}\PY{l+s+s2}{\PYZdq{}}\PY{p}{)}

\PY{c+c1}{\PYZsh{}\PYZsh{}\PYZsh{} VECTORIZED GENERAL DOT PRODUCT \PYZsh{}\PYZsh{}\PYZsh{}}
\PY{n}{tic} \PY{o}{=} \PY{n}{time}\PY{o}{.}\PY{n}{process\PYZus{}time}\PY{p}{(}\PY{p}{)}
\PY{n}{dot} \PY{o}{=} \PY{n}{np}\PY{o}{.}\PY{n}{dot}\PY{p}{(}\PY{n}{W}\PY{p}{,}\PY{n}{x1}\PY{p}{)}
\PY{n}{toc} \PY{o}{=} \PY{n}{time}\PY{o}{.}\PY{n}{process\PYZus{}time}\PY{p}{(}\PY{p}{)}
\PY{n+nb}{print} \PY{p}{(}\PY{l+s+s2}{\PYZdq{}}\PY{l+s+s2}{gdot = }\PY{l+s+s2}{\PYZdq{}} \PY{o}{+} \PY{n+nb}{str}\PY{p}{(}\PY{n}{dot}\PY{p}{)} \PY{o}{+} \PY{l+s+s2}{\PYZdq{}}\PY{l+s+se}{\PYZbs{}n}\PY{l+s+s2}{ \PYZhy{}\PYZhy{}\PYZhy{}\PYZhy{}\PYZhy{} Computation time = }\PY{l+s+s2}{\PYZdq{}} \PY{o}{+} \PY{n+nb}{str}\PY{p}{(}\PY{l+m+mi}{1000} \PY{o}{*} \PY{p}{(}\PY{n}{toc} \PY{o}{\PYZhy{}} \PY{n}{tic}\PY{p}{)}\PY{p}{)} \PY{o}{+} \PY{l+s+s2}{\PYZdq{}}\PY{l+s+s2}{ms}\PY{l+s+s2}{\PYZdq{}}\PY{p}{)}
\end{Verbatim}
\end{tcolorbox}

    \begin{Verbatim}[commandchars=\\\{\}]
dot = 278
 ----- Computation time = 0.2827009999999408ms
outer = [[81 18 18 81  0 81 18 45  0  0 81 18 45  0  0]
 [18  4  4 18  0 18  4 10  0  0 18  4 10  0  0]
 [45 10 10 45  0 45 10 25  0  0 45 10 25  0  0]
 [ 0  0  0  0  0  0  0  0  0  0  0  0  0  0  0]
 [ 0  0  0  0  0  0  0  0  0  0  0  0  0  0  0]
 [63 14 14 63  0 63 14 35  0  0 63 14 35  0  0]
 [45 10 10 45  0 45 10 25  0  0 45 10 25  0  0]
 [ 0  0  0  0  0  0  0  0  0  0  0  0  0  0  0]
 [ 0  0  0  0  0  0  0  0  0  0  0  0  0  0  0]
 [ 0  0  0  0  0  0  0  0  0  0  0  0  0  0  0]
 [81 18 18 81  0 81 18 45  0  0 81 18 45  0  0]
 [18  4  4 18  0 18  4 10  0  0 18  4 10  0  0]
 [45 10 10 45  0 45 10 25  0  0 45 10 25  0  0]
 [ 0  0  0  0  0  0  0  0  0  0  0  0  0  0  0]
 [ 0  0  0  0  0  0  0  0  0  0  0  0  0  0  0]]
 ----- Computation time = 0.564795999999923ms
elementwise multiplication = [81  4 10  0  0 63 10  0  0  0 81  4 25  0  0]
 ----- Computation time = 0.1275889999998725ms
gdot = [18.18824772 26.11195249 19.08893009]
 ----- Computation time = 0.37402700000011ms
    \end{Verbatim}

    As you may have noticed, the vectorized implementation is much cleaner
and more efficient. For bigger vectors/matrices, the differences in
running time become even bigger.

\textbf{Note} that \texttt{np.dot()} performs a matrix-matrix or
matrix-vector multiplication. This is different from
\texttt{np.multiply()} and the \texttt{*} operator (which is equivalent
to \texttt{.*} in Matlab/Octave), which performs an element-wise
multiplication.

    \#\#\# 2.1 Implement the L1 and L2 loss functions

\#\#\# Exercise 8 - L1 Implement the numpy vectorized version of the L1
loss. You may find the function abs(x) (absolute value of x) useful.

\textbf{Reminder}: - The loss is used to evaluate the performance of
your model. The bigger your loss is, the more different your predictions
(\$ \hat{y} \() are from the true values (\)y\$). In deep learning, you
use optimization algorithms like Gradient Descent to train your model
and to minimize the cost. - L1 loss is defined as:
\[\begin{align*} & L_1(\hat{y}, y) = \sum_{i=0}^{m-1}|y^{(i)} - \hat{y}^{(i)}| \end{align*}\tag{6}\]

    \begin{tcolorbox}[breakable, size=fbox, boxrule=1pt, pad at break*=1mm,colback=cellbackground, colframe=cellborder]
\prompt{In}{incolor}{25}{\boxspacing}
\begin{Verbatim}[commandchars=\\\{\}]
\PY{c+c1}{\PYZsh{} GRADED FUNCTION: L1}

\PY{k}{def} \PY{n+nf}{L1}\PY{p}{(}\PY{n}{yhat}\PY{p}{,} \PY{n}{y}\PY{p}{)}\PY{p}{:}
    \PY{l+s+sd}{\PYZdq{}\PYZdq{}\PYZdq{}}
\PY{l+s+sd}{    Arguments:}
\PY{l+s+sd}{    yhat \PYZhy{}\PYZhy{} vector of size m (predicted labels)}
\PY{l+s+sd}{    y \PYZhy{}\PYZhy{} vector of size m (true labels)}
\PY{l+s+sd}{    }
\PY{l+s+sd}{    Returns:}
\PY{l+s+sd}{    loss \PYZhy{}\PYZhy{} the value of the L1 loss function defined above}
\PY{l+s+sd}{    \PYZdq{}\PYZdq{}\PYZdq{}}
    
    \PY{c+c1}{\PYZsh{}(≈ 1 line of code)}
    \PY{c+c1}{\PYZsh{} loss = }
    \PY{c+c1}{\PYZsh{} YOUR CODE STARTS HERE}
    \PY{n}{loss} \PY{o}{=} \PY{n}{np}\PY{o}{.}\PY{n}{sum}\PY{p}{(}\PY{n}{np}\PY{o}{.}\PY{n}{abs}\PY{p}{(}\PY{n}{y} \PY{o}{\PYZhy{}} \PY{n}{yhat}\PY{p}{)}\PY{p}{)}
    
    \PY{c+c1}{\PYZsh{} YOUR CODE ENDS HERE}
    
    \PY{k}{return} \PY{n}{loss}
\end{Verbatim}
\end{tcolorbox}

    \begin{tcolorbox}[breakable, size=fbox, boxrule=1pt, pad at break*=1mm,colback=cellbackground, colframe=cellborder]
\prompt{In}{incolor}{26}{\boxspacing}
\begin{Verbatim}[commandchars=\\\{\}]
\PY{n}{yhat} \PY{o}{=} \PY{n}{np}\PY{o}{.}\PY{n}{array}\PY{p}{(}\PY{p}{[}\PY{o}{.}\PY{l+m+mi}{9}\PY{p}{,} \PY{l+m+mf}{0.2}\PY{p}{,} \PY{l+m+mf}{0.1}\PY{p}{,} \PY{o}{.}\PY{l+m+mi}{4}\PY{p}{,} \PY{o}{.}\PY{l+m+mi}{9}\PY{p}{]}\PY{p}{)}
\PY{n}{y} \PY{o}{=} \PY{n}{np}\PY{o}{.}\PY{n}{array}\PY{p}{(}\PY{p}{[}\PY{l+m+mi}{1}\PY{p}{,} \PY{l+m+mi}{0}\PY{p}{,} \PY{l+m+mi}{0}\PY{p}{,} \PY{l+m+mi}{1}\PY{p}{,} \PY{l+m+mi}{1}\PY{p}{]}\PY{p}{)}
\PY{n+nb}{print}\PY{p}{(}\PY{l+s+s2}{\PYZdq{}}\PY{l+s+s2}{L1 = }\PY{l+s+s2}{\PYZdq{}} \PY{o}{+} \PY{n+nb}{str}\PY{p}{(}\PY{n}{L1}\PY{p}{(}\PY{n}{yhat}\PY{p}{,} \PY{n}{y}\PY{p}{)}\PY{p}{)}\PY{p}{)}

\PY{n}{L1\PYZus{}test}\PY{p}{(}\PY{n}{L1}\PY{p}{)}
\end{Verbatim}
\end{tcolorbox}

    \begin{Verbatim}[commandchars=\\\{\}]
L1 = 1.1
\textcolor{ansi-green-intense}{ All tests passed.}
    \end{Verbatim}

    \#\#\# Exercise 9 - L2 Implement the numpy vectorized version of the L2
loss. There are several way of implementing the L2 loss but you may find
the function np.dot() useful. As a reminder, if
\(x = [x_1, x_2, ..., x_n]\), then \texttt{np.dot(x,x)} =
\(\sum_{j=0}^n x_j^{2}\).

\begin{itemize}
\tightlist
\item
  L2 loss is defined as
  \[\begin{align*} & L_2(\hat{y},y) = \sum_{i=0}^{m-1}(y^{(i)} - \hat{y}^{(i)})^2 \end{align*}\tag{7}\]
\end{itemize}

    \begin{tcolorbox}[breakable, size=fbox, boxrule=1pt, pad at break*=1mm,colback=cellbackground, colframe=cellborder]
\prompt{In}{incolor}{27}{\boxspacing}
\begin{Verbatim}[commandchars=\\\{\}]
\PY{c+c1}{\PYZsh{} GRADED FUNCTION: L2}

\PY{k}{def} \PY{n+nf}{L2}\PY{p}{(}\PY{n}{yhat}\PY{p}{,} \PY{n}{y}\PY{p}{)}\PY{p}{:}
    \PY{l+s+sd}{\PYZdq{}\PYZdq{}\PYZdq{}}
\PY{l+s+sd}{    Arguments:}
\PY{l+s+sd}{    yhat \PYZhy{}\PYZhy{} vector of size m (predicted labels)}
\PY{l+s+sd}{    y \PYZhy{}\PYZhy{} vector of size m (true labels)}
\PY{l+s+sd}{    }
\PY{l+s+sd}{    Returns:}
\PY{l+s+sd}{    loss \PYZhy{}\PYZhy{} the value of the L2 loss function defined above}
\PY{l+s+sd}{    \PYZdq{}\PYZdq{}\PYZdq{}}
    
    \PY{c+c1}{\PYZsh{}(≈ 1 line of code)}
    \PY{c+c1}{\PYZsh{} loss = ...}
    \PY{c+c1}{\PYZsh{} YOUR CODE STARTS HERE}
    \PY{n}{loss} \PY{o}{=} \PY{n}{np}\PY{o}{.}\PY{n}{dot}\PY{p}{(}\PY{n}{y}\PY{o}{\PYZhy{}}\PY{n}{yhat}\PY{p}{,} \PY{n}{y}\PY{o}{\PYZhy{}}\PY{n}{yhat}\PY{p}{)}
    
    \PY{c+c1}{\PYZsh{} YOUR CODE ENDS HERE}
    
    \PY{k}{return} \PY{n}{loss}
\end{Verbatim}
\end{tcolorbox}

    \begin{tcolorbox}[breakable, size=fbox, boxrule=1pt, pad at break*=1mm,colback=cellbackground, colframe=cellborder]
\prompt{In}{incolor}{28}{\boxspacing}
\begin{Verbatim}[commandchars=\\\{\}]
\PY{n}{yhat} \PY{o}{=} \PY{n}{np}\PY{o}{.}\PY{n}{array}\PY{p}{(}\PY{p}{[}\PY{o}{.}\PY{l+m+mi}{9}\PY{p}{,} \PY{l+m+mf}{0.2}\PY{p}{,} \PY{l+m+mf}{0.1}\PY{p}{,} \PY{o}{.}\PY{l+m+mi}{4}\PY{p}{,} \PY{o}{.}\PY{l+m+mi}{9}\PY{p}{]}\PY{p}{)}
\PY{n}{y} \PY{o}{=} \PY{n}{np}\PY{o}{.}\PY{n}{array}\PY{p}{(}\PY{p}{[}\PY{l+m+mi}{1}\PY{p}{,} \PY{l+m+mi}{0}\PY{p}{,} \PY{l+m+mi}{0}\PY{p}{,} \PY{l+m+mi}{1}\PY{p}{,} \PY{l+m+mi}{1}\PY{p}{]}\PY{p}{)}

\PY{n+nb}{print}\PY{p}{(}\PY{l+s+s2}{\PYZdq{}}\PY{l+s+s2}{L2 = }\PY{l+s+s2}{\PYZdq{}} \PY{o}{+} \PY{n+nb}{str}\PY{p}{(}\PY{n}{L2}\PY{p}{(}\PY{n}{yhat}\PY{p}{,} \PY{n}{y}\PY{p}{)}\PY{p}{)}\PY{p}{)}

\PY{n}{L2\PYZus{}test}\PY{p}{(}\PY{n}{L2}\PY{p}{)}
\end{Verbatim}
\end{tcolorbox}

    \begin{Verbatim}[commandchars=\\\{\}]
L2 = 0.43
\textcolor{ansi-green-intense}{ All tests passed.}
    \end{Verbatim}

    Congratulations on completing this assignment. We hope that this little
warm-up exercise helps you in the future assignments, which will be more
exciting and interesting!

    What to remember:

\begin{itemize}
\tightlist
\item
  Vectorization is very important in deep learning. It provides
  computational efficiency and clarity.
\item
  You have reviewed the L1 and L2 loss.
\item
  You are familiar with many numpy functions such as np.sum, np.dot,
  np.multiply, np.maximum, etc\ldots{}
\end{itemize}


    % Add a bibliography block to the postdoc
    
    
    
\end{document}
