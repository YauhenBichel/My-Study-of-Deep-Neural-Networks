\documentclass[11pt]{article}

    \usepackage[breakable]{tcolorbox}
    \usepackage{parskip} % Stop auto-indenting (to mimic markdown behaviour)
    
    \usepackage{iftex}
    \ifPDFTeX
    	\usepackage[T1]{fontenc}
    	\usepackage{mathpazo}
    \else
    	\usepackage{fontspec}
    \fi

    % Basic figure setup, for now with no caption control since it's done
    % automatically by Pandoc (which extracts ![](path) syntax from Markdown).
    \usepackage{graphicx}
    % Maintain compatibility with old templates. Remove in nbconvert 6.0
    \let\Oldincludegraphics\includegraphics
    % Ensure that by default, figures have no caption (until we provide a
    % proper Figure object with a Caption API and a way to capture that
    % in the conversion process - todo).
    \usepackage{caption}
    \DeclareCaptionFormat{nocaption}{}
    \captionsetup{format=nocaption,aboveskip=0pt,belowskip=0pt}

    \usepackage[Export]{adjustbox} % Used to constrain images to a maximum size
    \adjustboxset{max size={0.9\linewidth}{0.9\paperheight}}
    \usepackage{float}
    \floatplacement{figure}{H} % forces figures to be placed at the correct location
    \usepackage{xcolor} % Allow colors to be defined
    \usepackage{enumerate} % Needed for markdown enumerations to work
    \usepackage{geometry} % Used to adjust the document margins
    \usepackage{amsmath} % Equations
    \usepackage{amssymb} % Equations
    \usepackage{textcomp} % defines textquotesingle
    % Hack from http://tex.stackexchange.com/a/47451/13684:
    \AtBeginDocument{%
        \def\PYZsq{\textquotesingle}% Upright quotes in Pygmentized code
    }
    \usepackage{upquote} % Upright quotes for verbatim code
    \usepackage{eurosym} % defines \euro
    \usepackage[mathletters]{ucs} % Extended unicode (utf-8) support
    \usepackage{fancyvrb} % verbatim replacement that allows latex
    \usepackage{grffile} % extends the file name processing of package graphics 
                         % to support a larger range
    \makeatletter % fix for grffile with XeLaTeX
    \def\Gread@@xetex#1{%
      \IfFileExists{"\Gin@base".bb}%
      {\Gread@eps{\Gin@base.bb}}%
      {\Gread@@xetex@aux#1}%
    }
    \makeatother

    % The hyperref package gives us a pdf with properly built
    % internal navigation ('pdf bookmarks' for the table of contents,
    % internal cross-reference links, web links for URLs, etc.)
    \usepackage{hyperref}
    % The default LaTeX title has an obnoxious amount of whitespace. By default,
    % titling removes some of it. It also provides customization options.
    \usepackage{titling}
    \usepackage{longtable} % longtable support required by pandoc >1.10
    \usepackage{booktabs}  % table support for pandoc > 1.12.2
    \usepackage[inline]{enumitem} % IRkernel/repr support (it uses the enumerate* environment)
    \usepackage[normalem]{ulem} % ulem is needed to support strikethroughs (\sout)
                                % normalem makes italics be italics, not underlines
    \usepackage{mathrsfs}
    

    
    % Colors for the hyperref package
    \definecolor{urlcolor}{rgb}{0,.145,.698}
    \definecolor{linkcolor}{rgb}{.71,0.21,0.01}
    \definecolor{citecolor}{rgb}{.12,.54,.11}

    % ANSI colors
    \definecolor{ansi-black}{HTML}{3E424D}
    \definecolor{ansi-black-intense}{HTML}{282C36}
    \definecolor{ansi-red}{HTML}{E75C58}
    \definecolor{ansi-red-intense}{HTML}{B22B31}
    \definecolor{ansi-green}{HTML}{00A250}
    \definecolor{ansi-green-intense}{HTML}{007427}
    \definecolor{ansi-yellow}{HTML}{DDB62B}
    \definecolor{ansi-yellow-intense}{HTML}{B27D12}
    \definecolor{ansi-blue}{HTML}{208FFB}
    \definecolor{ansi-blue-intense}{HTML}{0065CA}
    \definecolor{ansi-magenta}{HTML}{D160C4}
    \definecolor{ansi-magenta-intense}{HTML}{A03196}
    \definecolor{ansi-cyan}{HTML}{60C6C8}
    \definecolor{ansi-cyan-intense}{HTML}{258F8F}
    \definecolor{ansi-white}{HTML}{C5C1B4}
    \definecolor{ansi-white-intense}{HTML}{A1A6B2}
    \definecolor{ansi-default-inverse-fg}{HTML}{FFFFFF}
    \definecolor{ansi-default-inverse-bg}{HTML}{000000}

    % commands and environments needed by pandoc snippets
    % extracted from the output of `pandoc -s`
    \providecommand{\tightlist}{%
      \setlength{\itemsep}{0pt}\setlength{\parskip}{0pt}}
    \DefineVerbatimEnvironment{Highlighting}{Verbatim}{commandchars=\\\{\}}
    % Add ',fontsize=\small' for more characters per line
    \newenvironment{Shaded}{}{}
    \newcommand{\KeywordTok}[1]{\textcolor[rgb]{0.00,0.44,0.13}{\textbf{{#1}}}}
    \newcommand{\DataTypeTok}[1]{\textcolor[rgb]{0.56,0.13,0.00}{{#1}}}
    \newcommand{\DecValTok}[1]{\textcolor[rgb]{0.25,0.63,0.44}{{#1}}}
    \newcommand{\BaseNTok}[1]{\textcolor[rgb]{0.25,0.63,0.44}{{#1}}}
    \newcommand{\FloatTok}[1]{\textcolor[rgb]{0.25,0.63,0.44}{{#1}}}
    \newcommand{\CharTok}[1]{\textcolor[rgb]{0.25,0.44,0.63}{{#1}}}
    \newcommand{\StringTok}[1]{\textcolor[rgb]{0.25,0.44,0.63}{{#1}}}
    \newcommand{\CommentTok}[1]{\textcolor[rgb]{0.38,0.63,0.69}{\textit{{#1}}}}
    \newcommand{\OtherTok}[1]{\textcolor[rgb]{0.00,0.44,0.13}{{#1}}}
    \newcommand{\AlertTok}[1]{\textcolor[rgb]{1.00,0.00,0.00}{\textbf{{#1}}}}
    \newcommand{\FunctionTok}[1]{\textcolor[rgb]{0.02,0.16,0.49}{{#1}}}
    \newcommand{\RegionMarkerTok}[1]{{#1}}
    \newcommand{\ErrorTok}[1]{\textcolor[rgb]{1.00,0.00,0.00}{\textbf{{#1}}}}
    \newcommand{\NormalTok}[1]{{#1}}
    
    % Additional commands for more recent versions of Pandoc
    \newcommand{\ConstantTok}[1]{\textcolor[rgb]{0.53,0.00,0.00}{{#1}}}
    \newcommand{\SpecialCharTok}[1]{\textcolor[rgb]{0.25,0.44,0.63}{{#1}}}
    \newcommand{\VerbatimStringTok}[1]{\textcolor[rgb]{0.25,0.44,0.63}{{#1}}}
    \newcommand{\SpecialStringTok}[1]{\textcolor[rgb]{0.73,0.40,0.53}{{#1}}}
    \newcommand{\ImportTok}[1]{{#1}}
    \newcommand{\DocumentationTok}[1]{\textcolor[rgb]{0.73,0.13,0.13}{\textit{{#1}}}}
    \newcommand{\AnnotationTok}[1]{\textcolor[rgb]{0.38,0.63,0.69}{\textbf{\textit{{#1}}}}}
    \newcommand{\CommentVarTok}[1]{\textcolor[rgb]{0.38,0.63,0.69}{\textbf{\textit{{#1}}}}}
    \newcommand{\VariableTok}[1]{\textcolor[rgb]{0.10,0.09,0.49}{{#1}}}
    \newcommand{\ControlFlowTok}[1]{\textcolor[rgb]{0.00,0.44,0.13}{\textbf{{#1}}}}
    \newcommand{\OperatorTok}[1]{\textcolor[rgb]{0.40,0.40,0.40}{{#1}}}
    \newcommand{\BuiltInTok}[1]{{#1}}
    \newcommand{\ExtensionTok}[1]{{#1}}
    \newcommand{\PreprocessorTok}[1]{\textcolor[rgb]{0.74,0.48,0.00}{{#1}}}
    \newcommand{\AttributeTok}[1]{\textcolor[rgb]{0.49,0.56,0.16}{{#1}}}
    \newcommand{\InformationTok}[1]{\textcolor[rgb]{0.38,0.63,0.69}{\textbf{\textit{{#1}}}}}
    \newcommand{\WarningTok}[1]{\textcolor[rgb]{0.38,0.63,0.69}{\textbf{\textit{{#1}}}}}
    
    
    % Define a nice break command that doesn't care if a line doesn't already
    % exist.
    \def\br{\hspace*{\fill} \\* }
    % Math Jax compatibility definitions
    \def\gt{>}
    \def\lt{<}
    \let\Oldtex\TeX
    \let\Oldlatex\LaTeX
    \renewcommand{\TeX}{\textrm{\Oldtex}}
    \renewcommand{\LaTeX}{\textrm{\Oldlatex}}
    % Document parameters
    % Document title
    \title{Optimization\_methods}
    
    
    
    
    
% Pygments definitions
\makeatletter
\def\PY@reset{\let\PY@it=\relax \let\PY@bf=\relax%
    \let\PY@ul=\relax \let\PY@tc=\relax%
    \let\PY@bc=\relax \let\PY@ff=\relax}
\def\PY@tok#1{\csname PY@tok@#1\endcsname}
\def\PY@toks#1+{\ifx\relax#1\empty\else%
    \PY@tok{#1}\expandafter\PY@toks\fi}
\def\PY@do#1{\PY@bc{\PY@tc{\PY@ul{%
    \PY@it{\PY@bf{\PY@ff{#1}}}}}}}
\def\PY#1#2{\PY@reset\PY@toks#1+\relax+\PY@do{#2}}

\expandafter\def\csname PY@tok@w\endcsname{\def\PY@tc##1{\textcolor[rgb]{0.73,0.73,0.73}{##1}}}
\expandafter\def\csname PY@tok@c\endcsname{\let\PY@it=\textit\def\PY@tc##1{\textcolor[rgb]{0.25,0.50,0.50}{##1}}}
\expandafter\def\csname PY@tok@cp\endcsname{\def\PY@tc##1{\textcolor[rgb]{0.74,0.48,0.00}{##1}}}
\expandafter\def\csname PY@tok@k\endcsname{\let\PY@bf=\textbf\def\PY@tc##1{\textcolor[rgb]{0.00,0.50,0.00}{##1}}}
\expandafter\def\csname PY@tok@kp\endcsname{\def\PY@tc##1{\textcolor[rgb]{0.00,0.50,0.00}{##1}}}
\expandafter\def\csname PY@tok@kt\endcsname{\def\PY@tc##1{\textcolor[rgb]{0.69,0.00,0.25}{##1}}}
\expandafter\def\csname PY@tok@o\endcsname{\def\PY@tc##1{\textcolor[rgb]{0.40,0.40,0.40}{##1}}}
\expandafter\def\csname PY@tok@ow\endcsname{\let\PY@bf=\textbf\def\PY@tc##1{\textcolor[rgb]{0.67,0.13,1.00}{##1}}}
\expandafter\def\csname PY@tok@nb\endcsname{\def\PY@tc##1{\textcolor[rgb]{0.00,0.50,0.00}{##1}}}
\expandafter\def\csname PY@tok@nf\endcsname{\def\PY@tc##1{\textcolor[rgb]{0.00,0.00,1.00}{##1}}}
\expandafter\def\csname PY@tok@nc\endcsname{\let\PY@bf=\textbf\def\PY@tc##1{\textcolor[rgb]{0.00,0.00,1.00}{##1}}}
\expandafter\def\csname PY@tok@nn\endcsname{\let\PY@bf=\textbf\def\PY@tc##1{\textcolor[rgb]{0.00,0.00,1.00}{##1}}}
\expandafter\def\csname PY@tok@ne\endcsname{\let\PY@bf=\textbf\def\PY@tc##1{\textcolor[rgb]{0.82,0.25,0.23}{##1}}}
\expandafter\def\csname PY@tok@nv\endcsname{\def\PY@tc##1{\textcolor[rgb]{0.10,0.09,0.49}{##1}}}
\expandafter\def\csname PY@tok@no\endcsname{\def\PY@tc##1{\textcolor[rgb]{0.53,0.00,0.00}{##1}}}
\expandafter\def\csname PY@tok@nl\endcsname{\def\PY@tc##1{\textcolor[rgb]{0.63,0.63,0.00}{##1}}}
\expandafter\def\csname PY@tok@ni\endcsname{\let\PY@bf=\textbf\def\PY@tc##1{\textcolor[rgb]{0.60,0.60,0.60}{##1}}}
\expandafter\def\csname PY@tok@na\endcsname{\def\PY@tc##1{\textcolor[rgb]{0.49,0.56,0.16}{##1}}}
\expandafter\def\csname PY@tok@nt\endcsname{\let\PY@bf=\textbf\def\PY@tc##1{\textcolor[rgb]{0.00,0.50,0.00}{##1}}}
\expandafter\def\csname PY@tok@nd\endcsname{\def\PY@tc##1{\textcolor[rgb]{0.67,0.13,1.00}{##1}}}
\expandafter\def\csname PY@tok@s\endcsname{\def\PY@tc##1{\textcolor[rgb]{0.73,0.13,0.13}{##1}}}
\expandafter\def\csname PY@tok@sd\endcsname{\let\PY@it=\textit\def\PY@tc##1{\textcolor[rgb]{0.73,0.13,0.13}{##1}}}
\expandafter\def\csname PY@tok@si\endcsname{\let\PY@bf=\textbf\def\PY@tc##1{\textcolor[rgb]{0.73,0.40,0.53}{##1}}}
\expandafter\def\csname PY@tok@se\endcsname{\let\PY@bf=\textbf\def\PY@tc##1{\textcolor[rgb]{0.73,0.40,0.13}{##1}}}
\expandafter\def\csname PY@tok@sr\endcsname{\def\PY@tc##1{\textcolor[rgb]{0.73,0.40,0.53}{##1}}}
\expandafter\def\csname PY@tok@ss\endcsname{\def\PY@tc##1{\textcolor[rgb]{0.10,0.09,0.49}{##1}}}
\expandafter\def\csname PY@tok@sx\endcsname{\def\PY@tc##1{\textcolor[rgb]{0.00,0.50,0.00}{##1}}}
\expandafter\def\csname PY@tok@m\endcsname{\def\PY@tc##1{\textcolor[rgb]{0.40,0.40,0.40}{##1}}}
\expandafter\def\csname PY@tok@gh\endcsname{\let\PY@bf=\textbf\def\PY@tc##1{\textcolor[rgb]{0.00,0.00,0.50}{##1}}}
\expandafter\def\csname PY@tok@gu\endcsname{\let\PY@bf=\textbf\def\PY@tc##1{\textcolor[rgb]{0.50,0.00,0.50}{##1}}}
\expandafter\def\csname PY@tok@gd\endcsname{\def\PY@tc##1{\textcolor[rgb]{0.63,0.00,0.00}{##1}}}
\expandafter\def\csname PY@tok@gi\endcsname{\def\PY@tc##1{\textcolor[rgb]{0.00,0.63,0.00}{##1}}}
\expandafter\def\csname PY@tok@gr\endcsname{\def\PY@tc##1{\textcolor[rgb]{1.00,0.00,0.00}{##1}}}
\expandafter\def\csname PY@tok@ge\endcsname{\let\PY@it=\textit}
\expandafter\def\csname PY@tok@gs\endcsname{\let\PY@bf=\textbf}
\expandafter\def\csname PY@tok@gp\endcsname{\let\PY@bf=\textbf\def\PY@tc##1{\textcolor[rgb]{0.00,0.00,0.50}{##1}}}
\expandafter\def\csname PY@tok@go\endcsname{\def\PY@tc##1{\textcolor[rgb]{0.53,0.53,0.53}{##1}}}
\expandafter\def\csname PY@tok@gt\endcsname{\def\PY@tc##1{\textcolor[rgb]{0.00,0.27,0.87}{##1}}}
\expandafter\def\csname PY@tok@err\endcsname{\def\PY@bc##1{\setlength{\fboxsep}{0pt}\fcolorbox[rgb]{1.00,0.00,0.00}{1,1,1}{\strut ##1}}}
\expandafter\def\csname PY@tok@kc\endcsname{\let\PY@bf=\textbf\def\PY@tc##1{\textcolor[rgb]{0.00,0.50,0.00}{##1}}}
\expandafter\def\csname PY@tok@kd\endcsname{\let\PY@bf=\textbf\def\PY@tc##1{\textcolor[rgb]{0.00,0.50,0.00}{##1}}}
\expandafter\def\csname PY@tok@kn\endcsname{\let\PY@bf=\textbf\def\PY@tc##1{\textcolor[rgb]{0.00,0.50,0.00}{##1}}}
\expandafter\def\csname PY@tok@kr\endcsname{\let\PY@bf=\textbf\def\PY@tc##1{\textcolor[rgb]{0.00,0.50,0.00}{##1}}}
\expandafter\def\csname PY@tok@bp\endcsname{\def\PY@tc##1{\textcolor[rgb]{0.00,0.50,0.00}{##1}}}
\expandafter\def\csname PY@tok@fm\endcsname{\def\PY@tc##1{\textcolor[rgb]{0.00,0.00,1.00}{##1}}}
\expandafter\def\csname PY@tok@vc\endcsname{\def\PY@tc##1{\textcolor[rgb]{0.10,0.09,0.49}{##1}}}
\expandafter\def\csname PY@tok@vg\endcsname{\def\PY@tc##1{\textcolor[rgb]{0.10,0.09,0.49}{##1}}}
\expandafter\def\csname PY@tok@vi\endcsname{\def\PY@tc##1{\textcolor[rgb]{0.10,0.09,0.49}{##1}}}
\expandafter\def\csname PY@tok@vm\endcsname{\def\PY@tc##1{\textcolor[rgb]{0.10,0.09,0.49}{##1}}}
\expandafter\def\csname PY@tok@sa\endcsname{\def\PY@tc##1{\textcolor[rgb]{0.73,0.13,0.13}{##1}}}
\expandafter\def\csname PY@tok@sb\endcsname{\def\PY@tc##1{\textcolor[rgb]{0.73,0.13,0.13}{##1}}}
\expandafter\def\csname PY@tok@sc\endcsname{\def\PY@tc##1{\textcolor[rgb]{0.73,0.13,0.13}{##1}}}
\expandafter\def\csname PY@tok@dl\endcsname{\def\PY@tc##1{\textcolor[rgb]{0.73,0.13,0.13}{##1}}}
\expandafter\def\csname PY@tok@s2\endcsname{\def\PY@tc##1{\textcolor[rgb]{0.73,0.13,0.13}{##1}}}
\expandafter\def\csname PY@tok@sh\endcsname{\def\PY@tc##1{\textcolor[rgb]{0.73,0.13,0.13}{##1}}}
\expandafter\def\csname PY@tok@s1\endcsname{\def\PY@tc##1{\textcolor[rgb]{0.73,0.13,0.13}{##1}}}
\expandafter\def\csname PY@tok@mb\endcsname{\def\PY@tc##1{\textcolor[rgb]{0.40,0.40,0.40}{##1}}}
\expandafter\def\csname PY@tok@mf\endcsname{\def\PY@tc##1{\textcolor[rgb]{0.40,0.40,0.40}{##1}}}
\expandafter\def\csname PY@tok@mh\endcsname{\def\PY@tc##1{\textcolor[rgb]{0.40,0.40,0.40}{##1}}}
\expandafter\def\csname PY@tok@mi\endcsname{\def\PY@tc##1{\textcolor[rgb]{0.40,0.40,0.40}{##1}}}
\expandafter\def\csname PY@tok@il\endcsname{\def\PY@tc##1{\textcolor[rgb]{0.40,0.40,0.40}{##1}}}
\expandafter\def\csname PY@tok@mo\endcsname{\def\PY@tc##1{\textcolor[rgb]{0.40,0.40,0.40}{##1}}}
\expandafter\def\csname PY@tok@ch\endcsname{\let\PY@it=\textit\def\PY@tc##1{\textcolor[rgb]{0.25,0.50,0.50}{##1}}}
\expandafter\def\csname PY@tok@cm\endcsname{\let\PY@it=\textit\def\PY@tc##1{\textcolor[rgb]{0.25,0.50,0.50}{##1}}}
\expandafter\def\csname PY@tok@cpf\endcsname{\let\PY@it=\textit\def\PY@tc##1{\textcolor[rgb]{0.25,0.50,0.50}{##1}}}
\expandafter\def\csname PY@tok@c1\endcsname{\let\PY@it=\textit\def\PY@tc##1{\textcolor[rgb]{0.25,0.50,0.50}{##1}}}
\expandafter\def\csname PY@tok@cs\endcsname{\let\PY@it=\textit\def\PY@tc##1{\textcolor[rgb]{0.25,0.50,0.50}{##1}}}

\def\PYZbs{\char`\\}
\def\PYZus{\char`\_}
\def\PYZob{\char`\{}
\def\PYZcb{\char`\}}
\def\PYZca{\char`\^}
\def\PYZam{\char`\&}
\def\PYZlt{\char`\<}
\def\PYZgt{\char`\>}
\def\PYZsh{\char`\#}
\def\PYZpc{\char`\%}
\def\PYZdl{\char`\$}
\def\PYZhy{\char`\-}
\def\PYZsq{\char`\'}
\def\PYZdq{\char`\"}
\def\PYZti{\char`\~}
% for compatibility with earlier versions
\def\PYZat{@}
\def\PYZlb{[}
\def\PYZrb{]}
\makeatother


    % For linebreaks inside Verbatim environment from package fancyvrb. 
    \makeatletter
        \newbox\Wrappedcontinuationbox 
        \newbox\Wrappedvisiblespacebox 
        \newcommand*\Wrappedvisiblespace {\textcolor{red}{\textvisiblespace}} 
        \newcommand*\Wrappedcontinuationsymbol {\textcolor{red}{\llap{\tiny$\m@th\hookrightarrow$}}} 
        \newcommand*\Wrappedcontinuationindent {3ex } 
        \newcommand*\Wrappedafterbreak {\kern\Wrappedcontinuationindent\copy\Wrappedcontinuationbox} 
        % Take advantage of the already applied Pygments mark-up to insert 
        % potential linebreaks for TeX processing. 
        %        {, <, #, %, $, ' and ": go to next line. 
        %        _, }, ^, &, >, - and ~: stay at end of broken line. 
        % Use of \textquotesingle for straight quote. 
        \newcommand*\Wrappedbreaksatspecials {% 
            \def\PYGZus{\discretionary{\char`\_}{\Wrappedafterbreak}{\char`\_}}% 
            \def\PYGZob{\discretionary{}{\Wrappedafterbreak\char`\{}{\char`\{}}% 
            \def\PYGZcb{\discretionary{\char`\}}{\Wrappedafterbreak}{\char`\}}}% 
            \def\PYGZca{\discretionary{\char`\^}{\Wrappedafterbreak}{\char`\^}}% 
            \def\PYGZam{\discretionary{\char`\&}{\Wrappedafterbreak}{\char`\&}}% 
            \def\PYGZlt{\discretionary{}{\Wrappedafterbreak\char`\<}{\char`\<}}% 
            \def\PYGZgt{\discretionary{\char`\>}{\Wrappedafterbreak}{\char`\>}}% 
            \def\PYGZsh{\discretionary{}{\Wrappedafterbreak\char`\#}{\char`\#}}% 
            \def\PYGZpc{\discretionary{}{\Wrappedafterbreak\char`\%}{\char`\%}}% 
            \def\PYGZdl{\discretionary{}{\Wrappedafterbreak\char`\$}{\char`\$}}% 
            \def\PYGZhy{\discretionary{\char`\-}{\Wrappedafterbreak}{\char`\-}}% 
            \def\PYGZsq{\discretionary{}{\Wrappedafterbreak\textquotesingle}{\textquotesingle}}% 
            \def\PYGZdq{\discretionary{}{\Wrappedafterbreak\char`\"}{\char`\"}}% 
            \def\PYGZti{\discretionary{\char`\~}{\Wrappedafterbreak}{\char`\~}}% 
        } 
        % Some characters . , ; ? ! / are not pygmentized. 
        % This macro makes them "active" and they will insert potential linebreaks 
        \newcommand*\Wrappedbreaksatpunct {% 
            \lccode`\~`\.\lowercase{\def~}{\discretionary{\hbox{\char`\.}}{\Wrappedafterbreak}{\hbox{\char`\.}}}% 
            \lccode`\~`\,\lowercase{\def~}{\discretionary{\hbox{\char`\,}}{\Wrappedafterbreak}{\hbox{\char`\,}}}% 
            \lccode`\~`\;\lowercase{\def~}{\discretionary{\hbox{\char`\;}}{\Wrappedafterbreak}{\hbox{\char`\;}}}% 
            \lccode`\~`\:\lowercase{\def~}{\discretionary{\hbox{\char`\:}}{\Wrappedafterbreak}{\hbox{\char`\:}}}% 
            \lccode`\~`\?\lowercase{\def~}{\discretionary{\hbox{\char`\?}}{\Wrappedafterbreak}{\hbox{\char`\?}}}% 
            \lccode`\~`\!\lowercase{\def~}{\discretionary{\hbox{\char`\!}}{\Wrappedafterbreak}{\hbox{\char`\!}}}% 
            \lccode`\~`\/\lowercase{\def~}{\discretionary{\hbox{\char`\/}}{\Wrappedafterbreak}{\hbox{\char`\/}}}% 
            \catcode`\.\active
            \catcode`\,\active 
            \catcode`\;\active
            \catcode`\:\active
            \catcode`\?\active
            \catcode`\!\active
            \catcode`\/\active 
            \lccode`\~`\~ 	
        }
    \makeatother

    \let\OriginalVerbatim=\Verbatim
    \makeatletter
    \renewcommand{\Verbatim}[1][1]{%
        %\parskip\z@skip
        \sbox\Wrappedcontinuationbox {\Wrappedcontinuationsymbol}%
        \sbox\Wrappedvisiblespacebox {\FV@SetupFont\Wrappedvisiblespace}%
        \def\FancyVerbFormatLine ##1{\hsize\linewidth
            \vtop{\raggedright\hyphenpenalty\z@\exhyphenpenalty\z@
                \doublehyphendemerits\z@\finalhyphendemerits\z@
                \strut ##1\strut}%
        }%
        % If the linebreak is at a space, the latter will be displayed as visible
        % space at end of first line, and a continuation symbol starts next line.
        % Stretch/shrink are however usually zero for typewriter font.
        \def\FV@Space {%
            \nobreak\hskip\z@ plus\fontdimen3\font minus\fontdimen4\font
            \discretionary{\copy\Wrappedvisiblespacebox}{\Wrappedafterbreak}
            {\kern\fontdimen2\font}%
        }%
        
        % Allow breaks at special characters using \PYG... macros.
        \Wrappedbreaksatspecials
        % Breaks at punctuation characters . , ; ? ! and / need catcode=\active 	
        \OriginalVerbatim[#1,codes*=\Wrappedbreaksatpunct]%
    }
    \makeatother

    % Exact colors from NB
    \definecolor{incolor}{HTML}{303F9F}
    \definecolor{outcolor}{HTML}{D84315}
    \definecolor{cellborder}{HTML}{CFCFCF}
    \definecolor{cellbackground}{HTML}{F7F7F7}
    
    % prompt
    \makeatletter
    \newcommand{\boxspacing}{\kern\kvtcb@left@rule\kern\kvtcb@boxsep}
    \makeatother
    \newcommand{\prompt}[4]{
        \ttfamily\llap{{\color{#2}[#3]:\hspace{3pt}#4}}\vspace{-\baselineskip}
    }
    

    
    % Prevent overflowing lines due to hard-to-break entities
    \sloppy 
    % Setup hyperref package
    \hypersetup{
      breaklinks=true,  % so long urls are correctly broken across lines
      colorlinks=true,
      urlcolor=urlcolor,
      linkcolor=linkcolor,
      citecolor=citecolor,
      }
    % Slightly bigger margins than the latex defaults
    
    \geometry{verbose,tmargin=1in,bmargin=1in,lmargin=1in,rmargin=1in}
    
    

\begin{document}
    
    \maketitle
    
    

    
    \hypertarget{optimization-methods}{%
\section{Optimization Methods}\label{optimization-methods}}

Until now, you've always used Gradient Descent to update the parameters
and minimize the cost. In this notebook, you'll gain skills with some
more advanced optimization methods that can speed up learning and
perhaps even get you to a better final value for the cost function.
Having a good optimization algorithm can be the difference between
waiting days vs.~just a few hours to get a good result.

By the end of this notebook, you'll be able to:

\begin{itemize}
\tightlist
\item
  Apply optimization methods such as (Stochastic) Gradient Descent,
  Momentum, RMSProp and Adam
\item
  Use random minibatches to accelerate convergence and improve
  optimization
\end{itemize}

Gradient descent goes ``downhill'' on a cost function \(J\). Think of it
as trying to do this:

Figure 1 : Minimizing the cost is like finding the lowest point in a
hilly landscape At each step of the training, you update your parameters
following a certain direction to try to get to the lowest possible
point.

\textbf{Notations}: As usual, \$\frac{\partial J}{\partial a } = \$
\texttt{da} for any variable \texttt{a}.

Let's get started!

    \hypertarget{table-of-contents}{%
\subsection{Table of Contents}\label{table-of-contents}}

\begin{itemize}
\tightlist
\item
  Section \ref{1}
\item
  Section \ref{2}

  \begin{itemize}
  \tightlist
  \item
    Section \ref{ex-1}
  \end{itemize}
\item
  Section \ref{3}

  \begin{itemize}
  \tightlist
  \item
    Section \ref{ex-2}
  \end{itemize}
\item
  Section \ref{4}

  \begin{itemize}
  \tightlist
  \item
    Section \ref{ex-3}
  \item
    Section \ref{ex-4}
  \end{itemize}
\item
  Section \ref{5}

  \begin{itemize}
  \tightlist
  \item
    Section \ref{ex-5}
  \item
    Section \ref{ex-6}
  \end{itemize}
\item
  Section \ref{6}

  \begin{itemize}
  \tightlist
  \item
    Section \ref{6-1}
  \item
    Section \ref{6-2}
  \item
    Section \ref{6-3}
  \item
    Section \ref{6-4}
  \end{itemize}
\item
  Section \ref{7}

  \begin{itemize}
  \tightlist
  \item
    Section \ref{7-1}

    \begin{itemize}
    \tightlist
    \item
      Section \ref{ex-7}
    \end{itemize}
  \item
    Section \ref{7-2}

    \begin{itemize}
    \tightlist
    \item
      Section \ref{ex-8}
    \end{itemize}
  \item
    Section \ref{7-3}

    \begin{itemize}
    \tightlist
    \item
      Section \ref{7-3-1}
    \item
      Section \ref{7-3-2}
    \item
      Section \ref{7-3-3}
    \end{itemize}
  \item
    Section \ref{7-4}
  \end{itemize}
\end{itemize}

    \#\# 1- Packages

    \begin{tcolorbox}[breakable, size=fbox, boxrule=1pt, pad at break*=1mm,colback=cellbackground, colframe=cellborder]
\prompt{In}{incolor}{1}{\boxspacing}
\begin{Verbatim}[commandchars=\\\{\}]
\PY{k+kn}{import} \PY{n+nn}{numpy} \PY{k}{as} \PY{n+nn}{np}
\PY{k+kn}{import} \PY{n+nn}{matplotlib}\PY{n+nn}{.}\PY{n+nn}{pyplot} \PY{k}{as} \PY{n+nn}{plt}
\PY{k+kn}{import} \PY{n+nn}{scipy}\PY{n+nn}{.}\PY{n+nn}{io}
\PY{k+kn}{import} \PY{n+nn}{math}
\PY{k+kn}{import} \PY{n+nn}{sklearn}
\PY{k+kn}{import} \PY{n+nn}{sklearn}\PY{n+nn}{.}\PY{n+nn}{datasets}

\PY{k+kn}{from} \PY{n+nn}{opt\PYZus{}utils\PYZus{}v1a} \PY{k+kn}{import} \PY{n}{load\PYZus{}params\PYZus{}and\PYZus{}grads}\PY{p}{,} \PY{n}{initialize\PYZus{}parameters}\PY{p}{,} \PY{n}{forward\PYZus{}propagation}\PY{p}{,} \PY{n}{backward\PYZus{}propagation}
\PY{k+kn}{from} \PY{n+nn}{opt\PYZus{}utils\PYZus{}v1a} \PY{k+kn}{import} \PY{n}{compute\PYZus{}cost}\PY{p}{,} \PY{n}{predict}\PY{p}{,} \PY{n}{predict\PYZus{}dec}\PY{p}{,} \PY{n}{plot\PYZus{}decision\PYZus{}boundary}\PY{p}{,} \PY{n}{load\PYZus{}dataset}
\PY{k+kn}{from} \PY{n+nn}{copy} \PY{k+kn}{import} \PY{n}{deepcopy}
\PY{k+kn}{from} \PY{n+nn}{testCases} \PY{k+kn}{import} \PY{o}{*}
\PY{k+kn}{from} \PY{n+nn}{public\PYZus{}tests} \PY{k+kn}{import} \PY{o}{*}

\PY{o}{\PYZpc{}}\PY{k}{matplotlib} inline
\PY{n}{plt}\PY{o}{.}\PY{n}{rcParams}\PY{p}{[}\PY{l+s+s1}{\PYZsq{}}\PY{l+s+s1}{figure.figsize}\PY{l+s+s1}{\PYZsq{}}\PY{p}{]} \PY{o}{=} \PY{p}{(}\PY{l+m+mf}{7.0}\PY{p}{,} \PY{l+m+mf}{4.0}\PY{p}{)} \PY{c+c1}{\PYZsh{} set default size of plots}
\PY{n}{plt}\PY{o}{.}\PY{n}{rcParams}\PY{p}{[}\PY{l+s+s1}{\PYZsq{}}\PY{l+s+s1}{image.interpolation}\PY{l+s+s1}{\PYZsq{}}\PY{p}{]} \PY{o}{=} \PY{l+s+s1}{\PYZsq{}}\PY{l+s+s1}{nearest}\PY{l+s+s1}{\PYZsq{}}
\PY{n}{plt}\PY{o}{.}\PY{n}{rcParams}\PY{p}{[}\PY{l+s+s1}{\PYZsq{}}\PY{l+s+s1}{image.cmap}\PY{l+s+s1}{\PYZsq{}}\PY{p}{]} \PY{o}{=} \PY{l+s+s1}{\PYZsq{}}\PY{l+s+s1}{gray}\PY{l+s+s1}{\PYZsq{}}

\PY{o}{\PYZpc{}}\PY{k}{load\PYZus{}ext} autoreload
\PY{o}{\PYZpc{}}\PY{k}{autoreload} 2
\end{Verbatim}
\end{tcolorbox}

    \#\# 2 - Gradient Descent

A simple optimization method in machine learning is gradient descent
(GD). When you take gradient steps with respect to all \(m\) examples on
each step, it is also called Batch Gradient Descent.

\#\#\# Exercise 1 - update\_parameters\_with\_gd

Implement the gradient descent update rule. The gradient descent rule
is, for \(l = 1, ..., L\):
\[ W^{[l]} = W^{[l]} - \alpha \text{ } dW^{[l]} \tag{1}\]
\[ b^{[l]} = b^{[l]} - \alpha \text{ } db^{[l]} \tag{2}\]

where L is the number of layers and \(\alpha\) is the learning rate. All
parameters should be stored in the \texttt{parameters} dictionary. Note
that the iterator \texttt{l} starts at 1 in the \texttt{for} loop as the
first parameters are \(W^{[1]}\) and \(b^{[1]}\).

    \begin{tcolorbox}[breakable, size=fbox, boxrule=1pt, pad at break*=1mm,colback=cellbackground, colframe=cellborder]
\prompt{In}{incolor}{ }{\boxspacing}
\begin{Verbatim}[commandchars=\\\{\}]
\PY{c+c1}{\PYZsh{} GRADED FUNCTION: update\PYZus{}parameters\PYZus{}with\PYZus{}gd}

\PY{k}{def} \PY{n+nf}{update\PYZus{}parameters\PYZus{}with\PYZus{}gd}\PY{p}{(}\PY{n}{parameters}\PY{p}{,} \PY{n}{grads}\PY{p}{,} \PY{n}{learning\PYZus{}rate}\PY{p}{)}\PY{p}{:}
    \PY{l+s+sd}{\PYZdq{}\PYZdq{}\PYZdq{}}
\PY{l+s+sd}{    Update parameters using one step of gradient descent}
\PY{l+s+sd}{    }
\PY{l+s+sd}{    Arguments:}
\PY{l+s+sd}{    parameters \PYZhy{}\PYZhy{} python dictionary containing your parameters to be updated:}
\PY{l+s+sd}{                    parameters[\PYZsq{}W\PYZsq{} + str(l)] = Wl}
\PY{l+s+sd}{                    parameters[\PYZsq{}b\PYZsq{} + str(l)] = bl}
\PY{l+s+sd}{    grads \PYZhy{}\PYZhy{} python dictionary containing your gradients to update each parameters:}
\PY{l+s+sd}{                    grads[\PYZsq{}dW\PYZsq{} + str(l)] = dWl}
\PY{l+s+sd}{                    grads[\PYZsq{}db\PYZsq{} + str(l)] = dbl}
\PY{l+s+sd}{    learning\PYZus{}rate \PYZhy{}\PYZhy{} the learning rate, scalar.}
\PY{l+s+sd}{    }
\PY{l+s+sd}{    Returns:}
\PY{l+s+sd}{    parameters \PYZhy{}\PYZhy{} python dictionary containing your updated parameters }
\PY{l+s+sd}{    \PYZdq{}\PYZdq{}\PYZdq{}}
    \PY{n}{L} \PY{o}{=} \PY{n+nb}{len}\PY{p}{(}\PY{n}{parameters}\PY{p}{)} \PY{o}{/}\PY{o}{/} \PY{l+m+mi}{2} \PY{c+c1}{\PYZsh{} number of layers in the neural networks}

    \PY{c+c1}{\PYZsh{} Update rule for each parameter}
    \PY{k}{for} \PY{n}{l} \PY{o+ow}{in} \PY{n+nb}{range}\PY{p}{(}\PY{l+m+mi}{1}\PY{p}{,} \PY{n}{L} \PY{o}{+} \PY{l+m+mi}{1}\PY{p}{)}\PY{p}{:}
        \PY{c+c1}{\PYZsh{} (approx. 2 lines)}
        \PY{c+c1}{\PYZsh{} parameters[\PYZdq{}W\PYZdq{} + str(l)] =  }
        \PY{c+c1}{\PYZsh{} parameters[\PYZdq{}b\PYZdq{} + str(l)] = }
        \PY{c+c1}{\PYZsh{} YOUR CODE STARTS HERE}
        
        
        \PY{c+c1}{\PYZsh{} YOUR CODE ENDS HERE}
    \PY{k}{return} \PY{n}{parameters}
\end{Verbatim}
\end{tcolorbox}

    \begin{tcolorbox}[breakable, size=fbox, boxrule=1pt, pad at break*=1mm,colback=cellbackground, colframe=cellborder]
\prompt{In}{incolor}{ }{\boxspacing}
\begin{Verbatim}[commandchars=\\\{\}]
\PY{n}{parameters}\PY{p}{,} \PY{n}{grads}\PY{p}{,} \PY{n}{learning\PYZus{}rate} \PY{o}{=} \PY{n}{update\PYZus{}parameters\PYZus{}with\PYZus{}gd\PYZus{}test\PYZus{}case}\PY{p}{(}\PY{p}{)}
\PY{n}{learning\PYZus{}rate} \PY{o}{=} \PY{l+m+mf}{0.01}
\PY{n}{parameters} \PY{o}{=} \PY{n}{update\PYZus{}parameters\PYZus{}with\PYZus{}gd}\PY{p}{(}\PY{n}{parameters}\PY{p}{,} \PY{n}{grads}\PY{p}{,} \PY{n}{learning\PYZus{}rate}\PY{p}{)}

\PY{n+nb}{print}\PY{p}{(}\PY{l+s+s2}{\PYZdq{}}\PY{l+s+s2}{W1 =}\PY{l+s+se}{\PYZbs{}n}\PY{l+s+s2}{\PYZdq{}} \PY{o}{+} \PY{n+nb}{str}\PY{p}{(}\PY{n}{parameters}\PY{p}{[}\PY{l+s+s2}{\PYZdq{}}\PY{l+s+s2}{W1}\PY{l+s+s2}{\PYZdq{}}\PY{p}{]}\PY{p}{)}\PY{p}{)}
\PY{n+nb}{print}\PY{p}{(}\PY{l+s+s2}{\PYZdq{}}\PY{l+s+s2}{b1 =}\PY{l+s+se}{\PYZbs{}n}\PY{l+s+s2}{\PYZdq{}} \PY{o}{+} \PY{n+nb}{str}\PY{p}{(}\PY{n}{parameters}\PY{p}{[}\PY{l+s+s2}{\PYZdq{}}\PY{l+s+s2}{b1}\PY{l+s+s2}{\PYZdq{}}\PY{p}{]}\PY{p}{)}\PY{p}{)}
\PY{n+nb}{print}\PY{p}{(}\PY{l+s+s2}{\PYZdq{}}\PY{l+s+s2}{W2 =}\PY{l+s+se}{\PYZbs{}n}\PY{l+s+s2}{\PYZdq{}} \PY{o}{+} \PY{n+nb}{str}\PY{p}{(}\PY{n}{parameters}\PY{p}{[}\PY{l+s+s2}{\PYZdq{}}\PY{l+s+s2}{W2}\PY{l+s+s2}{\PYZdq{}}\PY{p}{]}\PY{p}{)}\PY{p}{)}
\PY{n+nb}{print}\PY{p}{(}\PY{l+s+s2}{\PYZdq{}}\PY{l+s+s2}{b2 =}\PY{l+s+se}{\PYZbs{}n}\PY{l+s+s2}{\PYZdq{}} \PY{o}{+} \PY{n+nb}{str}\PY{p}{(}\PY{n}{parameters}\PY{p}{[}\PY{l+s+s2}{\PYZdq{}}\PY{l+s+s2}{b2}\PY{l+s+s2}{\PYZdq{}}\PY{p}{]}\PY{p}{)}\PY{p}{)}

\PY{n}{update\PYZus{}parameters\PYZus{}with\PYZus{}gd\PYZus{}test}\PY{p}{(}\PY{n}{update\PYZus{}parameters\PYZus{}with\PYZus{}gd}\PY{p}{)}
\end{Verbatim}
\end{tcolorbox}

    A variant of this is Stochastic Gradient Descent (SGD), which is
equivalent to mini-batch gradient descent, where each mini-batch has
just 1 example. The update rule that you have just implemented does not
change. What changes is that you would be computing gradients on just
one training example at a time, rather than on the whole training set.
The code examples below illustrate the difference between stochastic
gradient descent and (batch) gradient descent.

\begin{itemize}
\tightlist
\item
  \textbf{(Batch) Gradient Descent}:
\end{itemize}

\begin{Shaded}
\begin{Highlighting}[]
\NormalTok{X }\OperatorTok{=}\NormalTok{ data\_input}
\NormalTok{Y }\OperatorTok{=}\NormalTok{ labels}
\NormalTok{parameters }\OperatorTok{=}\NormalTok{ initialize\_parameters(layers\_dims)}
\ControlFlowTok{for}\NormalTok{ i }\KeywordTok{in} \BuiltInTok{range}\NormalTok{(}\DecValTok{0}\NormalTok{, num\_iterations):}
    \CommentTok{\# Forward propagation}
\NormalTok{    a, caches }\OperatorTok{=}\NormalTok{ forward\_propagation(X, parameters)}
    \CommentTok{\# Compute cost.}
\NormalTok{    cost }\OperatorTok{+=}\NormalTok{ compute\_cost(a, Y)}
    \CommentTok{\# Backward propagation.}
\NormalTok{    grads }\OperatorTok{=}\NormalTok{ backward\_propagation(a, caches, parameters)}
    \CommentTok{\# Update parameters.}
\NormalTok{    parameters }\OperatorTok{=}\NormalTok{ update\_parameters(parameters, grads)}
        
\end{Highlighting}
\end{Shaded}

\begin{itemize}
\tightlist
\item
  \textbf{Stochastic Gradient Descent}:
\end{itemize}

\begin{Shaded}
\begin{Highlighting}[]
\NormalTok{X }\OperatorTok{=}\NormalTok{ data\_input}
\NormalTok{Y }\OperatorTok{=}\NormalTok{ labels}
\NormalTok{parameters }\OperatorTok{=}\NormalTok{ initialize\_parameters(layers\_dims)}
\ControlFlowTok{for}\NormalTok{ i }\KeywordTok{in} \BuiltInTok{range}\NormalTok{(}\DecValTok{0}\NormalTok{, num\_iterations):}
    \ControlFlowTok{for}\NormalTok{ j }\KeywordTok{in} \BuiltInTok{range}\NormalTok{(}\DecValTok{0}\NormalTok{, m):}
        \CommentTok{\# Forward propagation}
\NormalTok{        a, caches }\OperatorTok{=}\NormalTok{ forward\_propagation(X[:,j], parameters)}
        \CommentTok{\# Compute cost}
\NormalTok{        cost }\OperatorTok{+=}\NormalTok{ compute\_cost(a, Y[:,j])}
        \CommentTok{\# Backward propagation}
\NormalTok{        grads }\OperatorTok{=}\NormalTok{ backward\_propagation(a, caches, parameters)}
        \CommentTok{\# Update parameters.}
\NormalTok{        parameters }\OperatorTok{=}\NormalTok{ update\_parameters(parameters, grads)}
\end{Highlighting}
\end{Shaded}

    In Stochastic Gradient Descent, you use only 1 training example before
updating the gradients. When the training set is large, SGD can be
faster. But the parameters will ``oscillate'' toward the minimum rather
than converge smoothly. Here's what that looks like:

Figure 1 : SGD vs GD ``+'' denotes a minimum of the cost. SGD leads to
many oscillations to reach convergence, but each step is a lot faster to
compute for SGD than it is for GD, as it uses only one training example
(vs.~the whole batch for GD).

\textbf{Note} also that implementing SGD requires 3 for-loops in total:
1. Over the number of iterations 2. Over the \(m\) training examples 3.
Over the layers (to update all parameters, from \((W^{[1]},b^{[1]})\) to
\((W^{[L]},b^{[L]})\))

In practice, you'll often get faster results if you don't use the entire
training set, or just one training example, to perform each update.
Mini-batch gradient descent uses an intermediate number of examples for
each step. With mini-batch gradient descent, you loop over the
mini-batches instead of looping over individual training examples.

Figure 2 : SGD vs Mini-Batch GD ``+'' denotes a minimum of the cost.
Using mini-batches in your optimization algorithm often leads to faster
optimization.

    \#\# 3 - Mini-Batch Gradient Descent

Now you'll build some mini-batches from the training set (X, Y).

There are two steps: - \textbf{Shuffle}: Create a shuffled version of
the training set (X, Y) as shown below. Each column of X and Y
represents a training example. Note that the random shuffling is done
synchronously between X and Y. Such that after the shuffling the
\(i^{th}\) column of X is the example corresponding to the \(i^{th}\)
label in Y. The shuffling step ensures that examples will be split
randomly into different mini-batches.

\begin{itemize}
\tightlist
\item
  \textbf{Partition}: Partition the shuffled (X, Y) into mini-batches of
  size \texttt{mini\_batch\_size} (here 64). Note that the number of
  training examples is not always divisible by
  \texttt{mini\_batch\_size}. The last mini batch might be smaller, but
  you don't need to worry about this. When the final mini-batch is
  smaller than the full \texttt{mini\_batch\_size}, it will look like
  this:
\end{itemize}

\#\#\# Exercise 2 - random\_mini\_batches

Implement \texttt{random\_mini\_batches}. The shuffling part has already
been coded for you! To help with the partitioning step, you've been
provided the following code that selects the indexes for the \(1^{st}\)
and \(2^{nd}\) mini-batches:

\begin{Shaded}
\begin{Highlighting}[]
\NormalTok{first\_mini\_batch\_X }\OperatorTok{=}\NormalTok{ shuffled\_X[:, }\DecValTok{0}\NormalTok{ : mini\_batch\_size]}
\NormalTok{second\_mini\_batch\_X }\OperatorTok{=}\NormalTok{ shuffled\_X[:, mini\_batch\_size : }\DecValTok{2} \OperatorTok{*}\NormalTok{ mini\_batch\_size]}
\NormalTok{...}
\end{Highlighting}
\end{Shaded}

Note that the last mini-batch might end up smaller than
\texttt{mini\_batch\_size=64}. Let \(\lfloor s \rfloor\) represents
\(s\) rounded down to the nearest integer (this is
\texttt{math.floor(s)} in Python). If the total number of examples is
not a multiple of \texttt{mini\_batch\_size=64} then there will be
\(\left\lfloor \frac{m}{mini\_batch\_size}\right\rfloor\) mini-batches
with a full 64 examples, and the number of examples in the final
mini-batch will be
\(\left(m-mini_\_batch_\_size \times \left\lfloor \frac{m}{mini\_batch\_size}\right\rfloor\right)\).

\textbf{Hint:}

\[mini\_batch\_X = shuffled\_X[:, i : j]\]

Think of a way in which you can use the for loop variable \texttt{k}
help you increment \texttt{i} and \texttt{j} in multiples of
mini\_batch\_size.

As an example, if you want to increment in multiples of 3, you could the
following:

\begin{Shaded}
\begin{Highlighting}[]
\NormalTok{n }\OperatorTok{=} \DecValTok{3}
\ControlFlowTok{for}\NormalTok{ k }\KeywordTok{in}\NormalTok{ (}\DecValTok{0}\NormalTok{ , }\DecValTok{5}\NormalTok{):}
    \BuiltInTok{print}\NormalTok{(k }\OperatorTok{*}\NormalTok{ n)}
\end{Highlighting}
\end{Shaded}

    \begin{tcolorbox}[breakable, size=fbox, boxrule=1pt, pad at break*=1mm,colback=cellbackground, colframe=cellborder]
\prompt{In}{incolor}{ }{\boxspacing}
\begin{Verbatim}[commandchars=\\\{\}]
\PY{c+c1}{\PYZsh{} GRADED FUNCTION: random\PYZus{}mini\PYZus{}batches}

\PY{k}{def} \PY{n+nf}{random\PYZus{}mini\PYZus{}batches}\PY{p}{(}\PY{n}{X}\PY{p}{,} \PY{n}{Y}\PY{p}{,} \PY{n}{mini\PYZus{}batch\PYZus{}size} \PY{o}{=} \PY{l+m+mi}{64}\PY{p}{,} \PY{n}{seed} \PY{o}{=} \PY{l+m+mi}{0}\PY{p}{)}\PY{p}{:}
    \PY{l+s+sd}{\PYZdq{}\PYZdq{}\PYZdq{}}
\PY{l+s+sd}{    Creates a list of random minibatches from (X, Y)}
\PY{l+s+sd}{    }
\PY{l+s+sd}{    Arguments:}
\PY{l+s+sd}{    X \PYZhy{}\PYZhy{} input data, of shape (input size, number of examples)}
\PY{l+s+sd}{    Y \PYZhy{}\PYZhy{} true \PYZdq{}label\PYZdq{} vector (1 for blue dot / 0 for red dot), of shape (1, number of examples)}
\PY{l+s+sd}{    mini\PYZus{}batch\PYZus{}size \PYZhy{}\PYZhy{} size of the mini\PYZhy{}batches, integer}
\PY{l+s+sd}{    }
\PY{l+s+sd}{    Returns:}
\PY{l+s+sd}{    mini\PYZus{}batches \PYZhy{}\PYZhy{} list of synchronous (mini\PYZus{}batch\PYZus{}X, mini\PYZus{}batch\PYZus{}Y)}
\PY{l+s+sd}{    \PYZdq{}\PYZdq{}\PYZdq{}}
    
    \PY{n}{np}\PY{o}{.}\PY{n}{random}\PY{o}{.}\PY{n}{seed}\PY{p}{(}\PY{n}{seed}\PY{p}{)}            \PY{c+c1}{\PYZsh{} To make your \PYZdq{}random\PYZdq{} minibatches the same as ours}
    \PY{n}{m} \PY{o}{=} \PY{n}{X}\PY{o}{.}\PY{n}{shape}\PY{p}{[}\PY{l+m+mi}{1}\PY{p}{]}                  \PY{c+c1}{\PYZsh{} number of training examples}
    \PY{n}{mini\PYZus{}batches} \PY{o}{=} \PY{p}{[}\PY{p}{]}
        
    \PY{c+c1}{\PYZsh{} Step 1: Shuffle (X, Y)}
    \PY{n}{permutation} \PY{o}{=} \PY{n+nb}{list}\PY{p}{(}\PY{n}{np}\PY{o}{.}\PY{n}{random}\PY{o}{.}\PY{n}{permutation}\PY{p}{(}\PY{n}{m}\PY{p}{)}\PY{p}{)}
    \PY{n}{shuffled\PYZus{}X} \PY{o}{=} \PY{n}{X}\PY{p}{[}\PY{p}{:}\PY{p}{,} \PY{n}{permutation}\PY{p}{]}
    \PY{n}{shuffled\PYZus{}Y} \PY{o}{=} \PY{n}{Y}\PY{p}{[}\PY{p}{:}\PY{p}{,} \PY{n}{permutation}\PY{p}{]}\PY{o}{.}\PY{n}{reshape}\PY{p}{(}\PY{p}{(}\PY{l+m+mi}{1}\PY{p}{,} \PY{n}{m}\PY{p}{)}\PY{p}{)}
    
    \PY{n}{inc} \PY{o}{=} \PY{n}{mini\PYZus{}batch\PYZus{}size}

    \PY{c+c1}{\PYZsh{} Step 2 \PYZhy{} Partition (shuffled\PYZus{}X, shuffled\PYZus{}Y).}
    \PY{c+c1}{\PYZsh{} Cases with a complete mini batch size only i.e each of 64 examples.}
    \PY{n}{num\PYZus{}complete\PYZus{}minibatches} \PY{o}{=} \PY{n}{math}\PY{o}{.}\PY{n}{floor}\PY{p}{(}\PY{n}{m} \PY{o}{/} \PY{n}{mini\PYZus{}batch\PYZus{}size}\PY{p}{)} \PY{c+c1}{\PYZsh{} number of mini batches of size mini\PYZus{}batch\PYZus{}size in your partitionning}
    \PY{k}{for} \PY{n}{k} \PY{o+ow}{in} \PY{n+nb}{range}\PY{p}{(}\PY{l+m+mi}{0}\PY{p}{,} \PY{n}{num\PYZus{}complete\PYZus{}minibatches}\PY{p}{)}\PY{p}{:}
        \PY{c+c1}{\PYZsh{} (approx. 2 lines)}
        \PY{c+c1}{\PYZsh{} mini\PYZus{}batch\PYZus{}X =  }
        \PY{c+c1}{\PYZsh{} mini\PYZus{}batch\PYZus{}Y =}
        \PY{c+c1}{\PYZsh{} YOUR CODE STARTS HERE}
        
        
        \PY{c+c1}{\PYZsh{} YOUR CODE ENDS HERE}
        \PY{n}{mini\PYZus{}batch} \PY{o}{=} \PY{p}{(}\PY{n}{mini\PYZus{}batch\PYZus{}X}\PY{p}{,} \PY{n}{mini\PYZus{}batch\PYZus{}Y}\PY{p}{)}
        \PY{n}{mini\PYZus{}batches}\PY{o}{.}\PY{n}{append}\PY{p}{(}\PY{n}{mini\PYZus{}batch}\PY{p}{)}
    
    \PY{c+c1}{\PYZsh{} For handling the end case (last mini\PYZhy{}batch \PYZlt{} mini\PYZus{}batch\PYZus{}size i.e less than 64)}
    \PY{k}{if} \PY{n}{m} \PY{o}{\PYZpc{}} \PY{n}{mini\PYZus{}batch\PYZus{}size} \PY{o}{!=} \PY{l+m+mi}{0}\PY{p}{:}
        \PY{c+c1}{\PYZsh{}(approx. 2 lines)}
        \PY{c+c1}{\PYZsh{} mini\PYZus{}batch\PYZus{}X =}
        \PY{c+c1}{\PYZsh{} mini\PYZus{}batch\PYZus{}Y =}
        \PY{c+c1}{\PYZsh{} YOUR CODE STARTS HERE}
        
        
        \PY{c+c1}{\PYZsh{} YOUR CODE ENDS HERE}
        \PY{n}{mini\PYZus{}batch} \PY{o}{=} \PY{p}{(}\PY{n}{mini\PYZus{}batch\PYZus{}X}\PY{p}{,} \PY{n}{mini\PYZus{}batch\PYZus{}Y}\PY{p}{)}
        \PY{n}{mini\PYZus{}batches}\PY{o}{.}\PY{n}{append}\PY{p}{(}\PY{n}{mini\PYZus{}batch}\PY{p}{)}
    
    \PY{k}{return} \PY{n}{mini\PYZus{}batches}
\end{Verbatim}
\end{tcolorbox}

    \begin{tcolorbox}[breakable, size=fbox, boxrule=1pt, pad at break*=1mm,colback=cellbackground, colframe=cellborder]
\prompt{In}{incolor}{ }{\boxspacing}
\begin{Verbatim}[commandchars=\\\{\}]
\PY{n}{np}\PY{o}{.}\PY{n}{random}\PY{o}{.}\PY{n}{seed}\PY{p}{(}\PY{l+m+mi}{1}\PY{p}{)}
\PY{n}{mini\PYZus{}batch\PYZus{}size} \PY{o}{=} \PY{l+m+mi}{64}
\PY{n}{nx} \PY{o}{=} \PY{l+m+mi}{12288}
\PY{n}{m} \PY{o}{=} \PY{l+m+mi}{148}
\PY{n}{X} \PY{o}{=} \PY{n}{np}\PY{o}{.}\PY{n}{array}\PY{p}{(}\PY{p}{[}\PY{n}{x} \PY{k}{for} \PY{n}{x} \PY{o+ow}{in} \PY{n+nb}{range}\PY{p}{(}\PY{n}{nx} \PY{o}{*} \PY{n}{m}\PY{p}{)}\PY{p}{]}\PY{p}{)}\PY{o}{.}\PY{n}{reshape}\PY{p}{(}\PY{p}{(}\PY{n}{m}\PY{p}{,} \PY{n}{nx}\PY{p}{)}\PY{p}{)}\PY{o}{.}\PY{n}{T}
\PY{n}{Y} \PY{o}{=} \PY{n}{np}\PY{o}{.}\PY{n}{random}\PY{o}{.}\PY{n}{randn}\PY{p}{(}\PY{l+m+mi}{1}\PY{p}{,} \PY{n}{m}\PY{p}{)} \PY{o}{\PYZlt{}} \PY{l+m+mf}{0.5}

\PY{n}{mini\PYZus{}batches} \PY{o}{=} \PY{n}{random\PYZus{}mini\PYZus{}batches}\PY{p}{(}\PY{n}{X}\PY{p}{,} \PY{n}{Y}\PY{p}{,} \PY{n}{mini\PYZus{}batch\PYZus{}size}\PY{p}{)}
\PY{n}{n\PYZus{}batches} \PY{o}{=} \PY{n+nb}{len}\PY{p}{(}\PY{n}{mini\PYZus{}batches}\PY{p}{)}

\PY{k}{assert} \PY{n}{n\PYZus{}batches} \PY{o}{==} \PY{n}{math}\PY{o}{.}\PY{n}{ceil}\PY{p}{(}\PY{n}{m} \PY{o}{/} \PY{n}{mini\PYZus{}batch\PYZus{}size}\PY{p}{)}\PY{p}{,} \PY{l+s+sa}{f}\PY{l+s+s2}{\PYZdq{}}\PY{l+s+s2}{Wrong number of mini batches. }\PY{l+s+si}{\PYZob{}}\PY{n}{n\PYZus{}batches}\PY{l+s+si}{\PYZcb{}}\PY{l+s+s2}{ != }\PY{l+s+si}{\PYZob{}}\PY{n}{math}\PY{o}{.}\PY{n}{ceil}\PY{p}{(}\PY{n}{m} \PY{o}{/} \PY{n}{mini\PYZus{}batch\PYZus{}size}\PY{p}{)}\PY{l+s+si}{\PYZcb{}}\PY{l+s+s2}{\PYZdq{}}
\PY{k}{for} \PY{n}{k} \PY{o+ow}{in} \PY{n+nb}{range}\PY{p}{(}\PY{n}{n\PYZus{}batches} \PY{o}{\PYZhy{}} \PY{l+m+mi}{1}\PY{p}{)}\PY{p}{:}
    \PY{k}{assert} \PY{n}{mini\PYZus{}batches}\PY{p}{[}\PY{n}{k}\PY{p}{]}\PY{p}{[}\PY{l+m+mi}{0}\PY{p}{]}\PY{o}{.}\PY{n}{shape} \PY{o}{==} \PY{p}{(}\PY{n}{nx}\PY{p}{,} \PY{n}{mini\PYZus{}batch\PYZus{}size}\PY{p}{)}\PY{p}{,} \PY{l+s+sa}{f}\PY{l+s+s2}{\PYZdq{}}\PY{l+s+s2}{Wrong shape in }\PY{l+s+si}{\PYZob{}}\PY{n}{k}\PY{l+s+si}{\PYZcb{}}\PY{l+s+s2}{ mini batch for X}\PY{l+s+s2}{\PYZdq{}}
    \PY{k}{assert} \PY{n}{mini\PYZus{}batches}\PY{p}{[}\PY{n}{k}\PY{p}{]}\PY{p}{[}\PY{l+m+mi}{1}\PY{p}{]}\PY{o}{.}\PY{n}{shape} \PY{o}{==} \PY{p}{(}\PY{l+m+mi}{1}\PY{p}{,} \PY{n}{mini\PYZus{}batch\PYZus{}size}\PY{p}{)}\PY{p}{,} \PY{l+s+sa}{f}\PY{l+s+s2}{\PYZdq{}}\PY{l+s+s2}{Wrong shape in }\PY{l+s+si}{\PYZob{}}\PY{n}{k}\PY{l+s+si}{\PYZcb{}}\PY{l+s+s2}{ mini batch for Y}\PY{l+s+s2}{\PYZdq{}}
    \PY{k}{assert} \PY{n}{np}\PY{o}{.}\PY{n}{sum}\PY{p}{(}\PY{n}{np}\PY{o}{.}\PY{n}{sum}\PY{p}{(}\PY{n}{mini\PYZus{}batches}\PY{p}{[}\PY{n}{k}\PY{p}{]}\PY{p}{[}\PY{l+m+mi}{0}\PY{p}{]} \PY{o}{\PYZhy{}} \PY{n}{mini\PYZus{}batches}\PY{p}{[}\PY{n}{k}\PY{p}{]}\PY{p}{[}\PY{l+m+mi}{0}\PY{p}{]}\PY{p}{[}\PY{l+m+mi}{0}\PY{p}{]}\PY{p}{,} \PY{n}{axis}\PY{o}{=}\PY{l+m+mi}{0}\PY{p}{)}\PY{p}{)} \PY{o}{==} \PY{p}{(}\PY{p}{(}\PY{n}{nx} \PY{o}{*} \PY{p}{(}\PY{n}{nx} \PY{o}{\PYZhy{}} \PY{l+m+mi}{1}\PY{p}{)} \PY{o}{/} \PY{l+m+mi}{2} \PY{p}{)} \PY{o}{*} \PY{n}{mini\PYZus{}batch\PYZus{}size}\PY{p}{)}\PY{p}{,} \PY{l+s+s2}{\PYZdq{}}\PY{l+s+s2}{Wrong values. It happens if the order of X rows(features) changes}\PY{l+s+s2}{\PYZdq{}}
\PY{k}{if} \PY{p}{(} \PY{n}{m} \PY{o}{\PYZpc{}} \PY{n}{mini\PYZus{}batch\PYZus{}size} \PY{o}{\PYZgt{}} \PY{l+m+mi}{0}\PY{p}{)}\PY{p}{:}
    \PY{k}{assert} \PY{n}{mini\PYZus{}batches}\PY{p}{[}\PY{n}{n\PYZus{}batches} \PY{o}{\PYZhy{}} \PY{l+m+mi}{1}\PY{p}{]}\PY{p}{[}\PY{l+m+mi}{0}\PY{p}{]}\PY{o}{.}\PY{n}{shape} \PY{o}{==} \PY{p}{(}\PY{n}{nx}\PY{p}{,} \PY{n}{m} \PY{o}{\PYZpc{}} \PY{n}{mini\PYZus{}batch\PYZus{}size}\PY{p}{)}\PY{p}{,} \PY{l+s+sa}{f}\PY{l+s+s2}{\PYZdq{}}\PY{l+s+s2}{Wrong shape in the last minibatch. }\PY{l+s+si}{\PYZob{}}\PY{n}{mini\PYZus{}batches}\PY{p}{[}\PY{n}{n\PYZus{}batches} \PY{o}{\PYZhy{}} \PY{l+m+mi}{1}\PY{p}{]}\PY{p}{[}\PY{l+m+mi}{0}\PY{p}{]}\PY{o}{.}\PY{n}{shape}\PY{l+s+si}{\PYZcb{}}\PY{l+s+s2}{ != }\PY{l+s+si}{\PYZob{}}\PY{p}{(}\PY{n}{nx}\PY{p}{,} \PY{n}{m} \PY{o}{\PYZpc{}} \PY{n}{mini\PYZus{}batch\PYZus{}size}\PY{p}{)}\PY{l+s+si}{\PYZcb{}}\PY{l+s+s2}{\PYZdq{}}

\PY{k}{assert} \PY{n}{np}\PY{o}{.}\PY{n}{allclose}\PY{p}{(}\PY{n}{mini\PYZus{}batches}\PY{p}{[}\PY{l+m+mi}{0}\PY{p}{]}\PY{p}{[}\PY{l+m+mi}{0}\PY{p}{]}\PY{p}{[}\PY{l+m+mi}{0}\PY{p}{]}\PY{p}{[}\PY{l+m+mi}{0}\PY{p}{:}\PY{l+m+mi}{3}\PY{p}{]}\PY{p}{,} \PY{p}{[}\PY{l+m+mi}{294912}\PY{p}{,}  \PY{l+m+mi}{86016}\PY{p}{,} \PY{l+m+mi}{454656}\PY{p}{]}\PY{p}{)}\PY{p}{,} \PY{l+s+s2}{\PYZdq{}}\PY{l+s+s2}{Wrong values. Check the indexes used to form the mini batches}\PY{l+s+s2}{\PYZdq{}}
\PY{k}{assert} \PY{n}{np}\PY{o}{.}\PY{n}{allclose}\PY{p}{(}\PY{n}{mini\PYZus{}batches}\PY{p}{[}\PY{o}{\PYZhy{}}\PY{l+m+mi}{1}\PY{p}{]}\PY{p}{[}\PY{l+m+mi}{0}\PY{p}{]}\PY{p}{[}\PY{o}{\PYZhy{}}\PY{l+m+mi}{1}\PY{p}{]}\PY{p}{[}\PY{l+m+mi}{0}\PY{p}{:}\PY{l+m+mi}{3}\PY{p}{]}\PY{p}{,} \PY{p}{[}\PY{l+m+mi}{1425407}\PY{p}{,} \PY{l+m+mi}{1769471}\PY{p}{,} \PY{l+m+mi}{897023}\PY{p}{]}\PY{p}{)}\PY{p}{,} \PY{l+s+s2}{\PYZdq{}}\PY{l+s+s2}{Wrong values. Check the indexes used to form the mini batches}\PY{l+s+s2}{\PYZdq{}}

\PY{n+nb}{print}\PY{p}{(}\PY{l+s+s2}{\PYZdq{}}\PY{l+s+se}{\PYZbs{}033}\PY{l+s+s2}{[92mAll test passed!}\PY{l+s+s2}{\PYZdq{}}\PY{p}{)}
\end{Verbatim}
\end{tcolorbox}

    \begin{tcolorbox}[breakable, size=fbox, boxrule=1pt, pad at break*=1mm,colback=cellbackground, colframe=cellborder]
\prompt{In}{incolor}{ }{\boxspacing}
\begin{Verbatim}[commandchars=\\\{\}]
\PY{n}{t\PYZus{}X}\PY{p}{,} \PY{n}{t\PYZus{}Y}\PY{p}{,} \PY{n}{mini\PYZus{}batch\PYZus{}size} \PY{o}{=} \PY{n}{random\PYZus{}mini\PYZus{}batches\PYZus{}test\PYZus{}case}\PY{p}{(}\PY{p}{)}
\PY{n}{mini\PYZus{}batches} \PY{o}{=} \PY{n}{random\PYZus{}mini\PYZus{}batches}\PY{p}{(}\PY{n}{t\PYZus{}X}\PY{p}{,} \PY{n}{t\PYZus{}Y}\PY{p}{,} \PY{n}{mini\PYZus{}batch\PYZus{}size}\PY{p}{)}

\PY{n+nb}{print} \PY{p}{(}\PY{l+s+s2}{\PYZdq{}}\PY{l+s+s2}{shape of the 1st mini\PYZus{}batch\PYZus{}X: }\PY{l+s+s2}{\PYZdq{}} \PY{o}{+} \PY{n+nb}{str}\PY{p}{(}\PY{n}{mini\PYZus{}batches}\PY{p}{[}\PY{l+m+mi}{0}\PY{p}{]}\PY{p}{[}\PY{l+m+mi}{0}\PY{p}{]}\PY{o}{.}\PY{n}{shape}\PY{p}{)}\PY{p}{)}
\PY{n+nb}{print} \PY{p}{(}\PY{l+s+s2}{\PYZdq{}}\PY{l+s+s2}{shape of the 2nd mini\PYZus{}batch\PYZus{}X: }\PY{l+s+s2}{\PYZdq{}} \PY{o}{+} \PY{n+nb}{str}\PY{p}{(}\PY{n}{mini\PYZus{}batches}\PY{p}{[}\PY{l+m+mi}{1}\PY{p}{]}\PY{p}{[}\PY{l+m+mi}{0}\PY{p}{]}\PY{o}{.}\PY{n}{shape}\PY{p}{)}\PY{p}{)}
\PY{n+nb}{print} \PY{p}{(}\PY{l+s+s2}{\PYZdq{}}\PY{l+s+s2}{shape of the 3rd mini\PYZus{}batch\PYZus{}X: }\PY{l+s+s2}{\PYZdq{}} \PY{o}{+} \PY{n+nb}{str}\PY{p}{(}\PY{n}{mini\PYZus{}batches}\PY{p}{[}\PY{l+m+mi}{2}\PY{p}{]}\PY{p}{[}\PY{l+m+mi}{0}\PY{p}{]}\PY{o}{.}\PY{n}{shape}\PY{p}{)}\PY{p}{)}
\PY{n+nb}{print} \PY{p}{(}\PY{l+s+s2}{\PYZdq{}}\PY{l+s+s2}{shape of the 1st mini\PYZus{}batch\PYZus{}Y: }\PY{l+s+s2}{\PYZdq{}} \PY{o}{+} \PY{n+nb}{str}\PY{p}{(}\PY{n}{mini\PYZus{}batches}\PY{p}{[}\PY{l+m+mi}{0}\PY{p}{]}\PY{p}{[}\PY{l+m+mi}{1}\PY{p}{]}\PY{o}{.}\PY{n}{shape}\PY{p}{)}\PY{p}{)}
\PY{n+nb}{print} \PY{p}{(}\PY{l+s+s2}{\PYZdq{}}\PY{l+s+s2}{shape of the 2nd mini\PYZus{}batch\PYZus{}Y: }\PY{l+s+s2}{\PYZdq{}} \PY{o}{+} \PY{n+nb}{str}\PY{p}{(}\PY{n}{mini\PYZus{}batches}\PY{p}{[}\PY{l+m+mi}{1}\PY{p}{]}\PY{p}{[}\PY{l+m+mi}{1}\PY{p}{]}\PY{o}{.}\PY{n}{shape}\PY{p}{)}\PY{p}{)} 
\PY{n+nb}{print} \PY{p}{(}\PY{l+s+s2}{\PYZdq{}}\PY{l+s+s2}{shape of the 3rd mini\PYZus{}batch\PYZus{}Y: }\PY{l+s+s2}{\PYZdq{}} \PY{o}{+} \PY{n+nb}{str}\PY{p}{(}\PY{n}{mini\PYZus{}batches}\PY{p}{[}\PY{l+m+mi}{2}\PY{p}{]}\PY{p}{[}\PY{l+m+mi}{1}\PY{p}{]}\PY{o}{.}\PY{n}{shape}\PY{p}{)}\PY{p}{)}
\PY{n+nb}{print} \PY{p}{(}\PY{l+s+s2}{\PYZdq{}}\PY{l+s+s2}{mini batch sanity check: }\PY{l+s+s2}{\PYZdq{}} \PY{o}{+} \PY{n+nb}{str}\PY{p}{(}\PY{n}{mini\PYZus{}batches}\PY{p}{[}\PY{l+m+mi}{0}\PY{p}{]}\PY{p}{[}\PY{l+m+mi}{0}\PY{p}{]}\PY{p}{[}\PY{l+m+mi}{0}\PY{p}{]}\PY{p}{[}\PY{l+m+mi}{0}\PY{p}{:}\PY{l+m+mi}{3}\PY{p}{]}\PY{p}{)}\PY{p}{)}

\PY{n}{random\PYZus{}mini\PYZus{}batches\PYZus{}test}\PY{p}{(}\PY{n}{random\PYZus{}mini\PYZus{}batches}\PY{p}{)}
\end{Verbatim}
\end{tcolorbox}

    \textbf{What you should remember}: - Shuffling and Partitioning are the
two steps required to build mini-batches - Powers of two are often
chosen to be the mini-batch size, e.g., 16, 32, 64, 128.

    \#\# 4 - Momentum

Because mini-batch gradient descent makes a parameter update after
seeing just a subset of examples, the direction of the update has some
variance, and so the path taken by mini-batch gradient descent will
``oscillate'' toward convergence. Using momentum can reduce these
oscillations.

Momentum takes into account the past gradients to smooth out the update.
The `direction' of the previous gradients is stored in the variable
\(v\). Formally, this will be the exponentially weighted average of the
gradient on previous steps. You can also think of \(v\) as the
``velocity'' of a ball rolling downhill, building up speed (and
momentum) according to the direction of the gradient/slope of the hill.

Figure 3 : The red arrows show the direction taken by one step of
mini-batch gradient descent with momentum. The blue points show the
direction of the gradient (with respect to the current mini-batch) on
each step. Rather than just following the gradient, the gradient is
allowed to influence \(v\) and then take a step in the direction of
\(v\).

    ~\\
\#\#\# Exercise 3 - initialize\_velocity Initialize the velocity. The
velocity, \(v\), is a python dictionary that needs to be initialized
with arrays of zeros. Its keys are the same as those in the
\texttt{grads} dictionary, that is: for \(l =1,...,L\):

\begin{Shaded}
\begin{Highlighting}[]
\NormalTok{v[}\StringTok{"dW"} \OperatorTok{+} \BuiltInTok{str}\NormalTok{(l)] }\OperatorTok{=}\NormalTok{ ... }\CommentTok{\#(numpy array of zeros with the same shape as parameters["W" + str(l)])}
\NormalTok{v[}\StringTok{"db"} \OperatorTok{+} \BuiltInTok{str}\NormalTok{(l)] }\OperatorTok{=}\NormalTok{ ... }\CommentTok{\#(numpy array of zeros with the same shape as parameters["b" + str(l)])}
\end{Highlighting}
\end{Shaded}

\textbf{Note} that the iterator l starts at 1 in the for loop as the
first parameters are v{[}``dW1''{]} and v{[}``db1''{]} (that's a ``one''
on the superscript).

    \begin{tcolorbox}[breakable, size=fbox, boxrule=1pt, pad at break*=1mm,colback=cellbackground, colframe=cellborder]
\prompt{In}{incolor}{ }{\boxspacing}
\begin{Verbatim}[commandchars=\\\{\}]
\PY{c+c1}{\PYZsh{} GRADED FUNCTION: initialize\PYZus{}velocity}

\PY{k}{def} \PY{n+nf}{initialize\PYZus{}velocity}\PY{p}{(}\PY{n}{parameters}\PY{p}{)}\PY{p}{:}
    \PY{l+s+sd}{\PYZdq{}\PYZdq{}\PYZdq{}}
\PY{l+s+sd}{    Initializes the velocity as a python dictionary with:}
\PY{l+s+sd}{                \PYZhy{} keys: \PYZdq{}dW1\PYZdq{}, \PYZdq{}db1\PYZdq{}, ..., \PYZdq{}dWL\PYZdq{}, \PYZdq{}dbL\PYZdq{} }
\PY{l+s+sd}{                \PYZhy{} values: numpy arrays of zeros of the same shape as the corresponding gradients/parameters.}
\PY{l+s+sd}{    Arguments:}
\PY{l+s+sd}{    parameters \PYZhy{}\PYZhy{} python dictionary containing your parameters.}
\PY{l+s+sd}{                    parameters[\PYZsq{}W\PYZsq{} + str(l)] = Wl}
\PY{l+s+sd}{                    parameters[\PYZsq{}b\PYZsq{} + str(l)] = bl}
\PY{l+s+sd}{    }
\PY{l+s+sd}{    Returns:}
\PY{l+s+sd}{    v \PYZhy{}\PYZhy{} python dictionary containing the current velocity.}
\PY{l+s+sd}{                    v[\PYZsq{}dW\PYZsq{} + str(l)] = velocity of dWl}
\PY{l+s+sd}{                    v[\PYZsq{}db\PYZsq{} + str(l)] = velocity of dbl}
\PY{l+s+sd}{    \PYZdq{}\PYZdq{}\PYZdq{}}
    
    \PY{n}{L} \PY{o}{=} \PY{n+nb}{len}\PY{p}{(}\PY{n}{parameters}\PY{p}{)} \PY{o}{/}\PY{o}{/} \PY{l+m+mi}{2} \PY{c+c1}{\PYZsh{} number of layers in the neural networks}
    \PY{n}{v} \PY{o}{=} \PY{p}{\PYZob{}}\PY{p}{\PYZcb{}}
    
    \PY{c+c1}{\PYZsh{} Initialize velocity}
    \PY{k}{for} \PY{n}{l} \PY{o+ow}{in} \PY{n+nb}{range}\PY{p}{(}\PY{l+m+mi}{1}\PY{p}{,} \PY{n}{L} \PY{o}{+} \PY{l+m+mi}{1}\PY{p}{)}\PY{p}{:}
        \PY{c+c1}{\PYZsh{} (approx. 2 lines)}
        \PY{c+c1}{\PYZsh{} v[\PYZdq{}dW\PYZdq{} + str(l)] =}
        \PY{c+c1}{\PYZsh{} v[\PYZdq{}db\PYZdq{} + str(l)] =}
        \PY{c+c1}{\PYZsh{} YOUR CODE STARTS HERE}
        
        
        \PY{c+c1}{\PYZsh{} YOUR CODE ENDS HERE}
        
    \PY{k}{return} \PY{n}{v}
\end{Verbatim}
\end{tcolorbox}

    \begin{tcolorbox}[breakable, size=fbox, boxrule=1pt, pad at break*=1mm,colback=cellbackground, colframe=cellborder]
\prompt{In}{incolor}{ }{\boxspacing}
\begin{Verbatim}[commandchars=\\\{\}]
\PY{n}{parameters} \PY{o}{=} \PY{n}{initialize\PYZus{}velocity\PYZus{}test\PYZus{}case}\PY{p}{(}\PY{p}{)}

\PY{n}{v} \PY{o}{=} \PY{n}{initialize\PYZus{}velocity}\PY{p}{(}\PY{n}{parameters}\PY{p}{)}
\PY{n+nb}{print}\PY{p}{(}\PY{l+s+s2}{\PYZdq{}}\PY{l+s+s2}{v[}\PY{l+s+se}{\PYZbs{}\PYZdq{}}\PY{l+s+s2}{dW1}\PY{l+s+se}{\PYZbs{}\PYZdq{}}\PY{l+s+s2}{] =}\PY{l+s+se}{\PYZbs{}n}\PY{l+s+s2}{\PYZdq{}} \PY{o}{+} \PY{n+nb}{str}\PY{p}{(}\PY{n}{v}\PY{p}{[}\PY{l+s+s2}{\PYZdq{}}\PY{l+s+s2}{dW1}\PY{l+s+s2}{\PYZdq{}}\PY{p}{]}\PY{p}{)}\PY{p}{)}
\PY{n+nb}{print}\PY{p}{(}\PY{l+s+s2}{\PYZdq{}}\PY{l+s+s2}{v[}\PY{l+s+se}{\PYZbs{}\PYZdq{}}\PY{l+s+s2}{db1}\PY{l+s+se}{\PYZbs{}\PYZdq{}}\PY{l+s+s2}{] =}\PY{l+s+se}{\PYZbs{}n}\PY{l+s+s2}{\PYZdq{}} \PY{o}{+} \PY{n+nb}{str}\PY{p}{(}\PY{n}{v}\PY{p}{[}\PY{l+s+s2}{\PYZdq{}}\PY{l+s+s2}{db1}\PY{l+s+s2}{\PYZdq{}}\PY{p}{]}\PY{p}{)}\PY{p}{)}
\PY{n+nb}{print}\PY{p}{(}\PY{l+s+s2}{\PYZdq{}}\PY{l+s+s2}{v[}\PY{l+s+se}{\PYZbs{}\PYZdq{}}\PY{l+s+s2}{dW2}\PY{l+s+se}{\PYZbs{}\PYZdq{}}\PY{l+s+s2}{] =}\PY{l+s+se}{\PYZbs{}n}\PY{l+s+s2}{\PYZdq{}} \PY{o}{+} \PY{n+nb}{str}\PY{p}{(}\PY{n}{v}\PY{p}{[}\PY{l+s+s2}{\PYZdq{}}\PY{l+s+s2}{dW2}\PY{l+s+s2}{\PYZdq{}}\PY{p}{]}\PY{p}{)}\PY{p}{)}
\PY{n+nb}{print}\PY{p}{(}\PY{l+s+s2}{\PYZdq{}}\PY{l+s+s2}{v[}\PY{l+s+se}{\PYZbs{}\PYZdq{}}\PY{l+s+s2}{db2}\PY{l+s+se}{\PYZbs{}\PYZdq{}}\PY{l+s+s2}{] =}\PY{l+s+se}{\PYZbs{}n}\PY{l+s+s2}{\PYZdq{}} \PY{o}{+} \PY{n+nb}{str}\PY{p}{(}\PY{n}{v}\PY{p}{[}\PY{l+s+s2}{\PYZdq{}}\PY{l+s+s2}{db2}\PY{l+s+s2}{\PYZdq{}}\PY{p}{]}\PY{p}{)}\PY{p}{)}

\PY{n}{initialize\PYZus{}velocity\PYZus{}test}\PY{p}{(}\PY{n}{initialize\PYZus{}velocity}\PY{p}{)}
\end{Verbatim}
\end{tcolorbox}

    ~\\
\#\#\# Exercise 4 - update\_parameters\_with\_momentum

Now, implement the parameters update with momentum. The momentum update
rule is, for \(l = 1, ..., L\):

\[ \begin{cases}
v_{dW^{[l]}} = \beta v_{dW^{[l]}} + (1 - \beta) dW^{[l]} \\
W^{[l]} = W^{[l]} - \alpha v_{dW^{[l]}}
\end{cases}\tag{3}\]

\[\begin{cases}
v_{db^{[l]}} = \beta v_{db^{[l]}} + (1 - \beta) db^{[l]} \\
b^{[l]} = b^{[l]} - \alpha v_{db^{[l]}} 
\end{cases}\tag{4}\]

where L is the number of layers, \(\beta\) is the momentum and
\(\alpha\) is the learning rate. All parameters should be stored in the
\texttt{parameters} dictionary. Note that the iterator \texttt{l} starts
at 1 in the \texttt{for} loop as the first parameters are \(W^{[1]}\)
and \(b^{[1]}\) (that's a ``one'' on the superscript).

    \begin{tcolorbox}[breakable, size=fbox, boxrule=1pt, pad at break*=1mm,colback=cellbackground, colframe=cellborder]
\prompt{In}{incolor}{ }{\boxspacing}
\begin{Verbatim}[commandchars=\\\{\}]
\PY{c+c1}{\PYZsh{} GRADED FUNCTION: update\PYZus{}parameters\PYZus{}with\PYZus{}momentum}

\PY{k}{def} \PY{n+nf}{update\PYZus{}parameters\PYZus{}with\PYZus{}momentum}\PY{p}{(}\PY{n}{parameters}\PY{p}{,} \PY{n}{grads}\PY{p}{,} \PY{n}{v}\PY{p}{,} \PY{n}{beta}\PY{p}{,} \PY{n}{learning\PYZus{}rate}\PY{p}{)}\PY{p}{:}
    \PY{l+s+sd}{\PYZdq{}\PYZdq{}\PYZdq{}}
\PY{l+s+sd}{    Update parameters using Momentum}
\PY{l+s+sd}{    }
\PY{l+s+sd}{    Arguments:}
\PY{l+s+sd}{    parameters \PYZhy{}\PYZhy{} python dictionary containing your parameters:}
\PY{l+s+sd}{                    parameters[\PYZsq{}W\PYZsq{} + str(l)] = Wl}
\PY{l+s+sd}{                    parameters[\PYZsq{}b\PYZsq{} + str(l)] = bl}
\PY{l+s+sd}{    grads \PYZhy{}\PYZhy{} python dictionary containing your gradients for each parameters:}
\PY{l+s+sd}{                    grads[\PYZsq{}dW\PYZsq{} + str(l)] = dWl}
\PY{l+s+sd}{                    grads[\PYZsq{}db\PYZsq{} + str(l)] = dbl}
\PY{l+s+sd}{    v \PYZhy{}\PYZhy{} python dictionary containing the current velocity:}
\PY{l+s+sd}{                    v[\PYZsq{}dW\PYZsq{} + str(l)] = ...}
\PY{l+s+sd}{                    v[\PYZsq{}db\PYZsq{} + str(l)] = ...}
\PY{l+s+sd}{    beta \PYZhy{}\PYZhy{} the momentum hyperparameter, scalar}
\PY{l+s+sd}{    learning\PYZus{}rate \PYZhy{}\PYZhy{} the learning rate, scalar}
\PY{l+s+sd}{    }
\PY{l+s+sd}{    Returns:}
\PY{l+s+sd}{    parameters \PYZhy{}\PYZhy{} python dictionary containing your updated parameters }
\PY{l+s+sd}{    v \PYZhy{}\PYZhy{} python dictionary containing your updated velocities}
\PY{l+s+sd}{    \PYZdq{}\PYZdq{}\PYZdq{}}

    \PY{n}{L} \PY{o}{=} \PY{n+nb}{len}\PY{p}{(}\PY{n}{parameters}\PY{p}{)} \PY{o}{/}\PY{o}{/} \PY{l+m+mi}{2} \PY{c+c1}{\PYZsh{} number of layers in the neural networks}
    
    \PY{c+c1}{\PYZsh{} Momentum update for each parameter}
    \PY{k}{for} \PY{n}{l} \PY{o+ow}{in} \PY{n+nb}{range}\PY{p}{(}\PY{l+m+mi}{1}\PY{p}{,} \PY{n}{L} \PY{o}{+} \PY{l+m+mi}{1}\PY{p}{)}\PY{p}{:}
        
        \PY{c+c1}{\PYZsh{} (approx. 4 lines)}
        \PY{c+c1}{\PYZsh{} compute velocities}
        \PY{c+c1}{\PYZsh{} v[\PYZdq{}dW\PYZdq{} + str(l)] = ...}
        \PY{c+c1}{\PYZsh{} v[\PYZdq{}db\PYZdq{} + str(l)] = ...}
        \PY{c+c1}{\PYZsh{} update parameters}
        \PY{c+c1}{\PYZsh{} parameters[\PYZdq{}W\PYZdq{} + str(l)] = ...}
        \PY{c+c1}{\PYZsh{} parameters[\PYZdq{}b\PYZdq{} + str(l)] = ...}
        \PY{c+c1}{\PYZsh{} YOUR CODE STARTS HERE}
        
        
        \PY{c+c1}{\PYZsh{} YOUR CODE ENDS HERE}
        
    \PY{k}{return} \PY{n}{parameters}\PY{p}{,} \PY{n}{v}
\end{Verbatim}
\end{tcolorbox}

    \begin{tcolorbox}[breakable, size=fbox, boxrule=1pt, pad at break*=1mm,colback=cellbackground, colframe=cellborder]
\prompt{In}{incolor}{ }{\boxspacing}
\begin{Verbatim}[commandchars=\\\{\}]
\PY{n}{parameters}\PY{p}{,} \PY{n}{grads}\PY{p}{,} \PY{n}{v} \PY{o}{=} \PY{n}{update\PYZus{}parameters\PYZus{}with\PYZus{}momentum\PYZus{}test\PYZus{}case}\PY{p}{(}\PY{p}{)}

\PY{n}{parameters}\PY{p}{,} \PY{n}{v} \PY{o}{=} \PY{n}{update\PYZus{}parameters\PYZus{}with\PYZus{}momentum}\PY{p}{(}\PY{n}{parameters}\PY{p}{,} \PY{n}{grads}\PY{p}{,} \PY{n}{v}\PY{p}{,} \PY{n}{beta} \PY{o}{=} \PY{l+m+mf}{0.9}\PY{p}{,} \PY{n}{learning\PYZus{}rate} \PY{o}{=} \PY{l+m+mf}{0.01}\PY{p}{)}
\PY{n+nb}{print}\PY{p}{(}\PY{l+s+s2}{\PYZdq{}}\PY{l+s+s2}{W1 = }\PY{l+s+se}{\PYZbs{}n}\PY{l+s+s2}{\PYZdq{}} \PY{o}{+} \PY{n+nb}{str}\PY{p}{(}\PY{n}{parameters}\PY{p}{[}\PY{l+s+s2}{\PYZdq{}}\PY{l+s+s2}{W1}\PY{l+s+s2}{\PYZdq{}}\PY{p}{]}\PY{p}{)}\PY{p}{)}
\PY{n+nb}{print}\PY{p}{(}\PY{l+s+s2}{\PYZdq{}}\PY{l+s+s2}{b1 = }\PY{l+s+se}{\PYZbs{}n}\PY{l+s+s2}{\PYZdq{}} \PY{o}{+} \PY{n+nb}{str}\PY{p}{(}\PY{n}{parameters}\PY{p}{[}\PY{l+s+s2}{\PYZdq{}}\PY{l+s+s2}{b1}\PY{l+s+s2}{\PYZdq{}}\PY{p}{]}\PY{p}{)}\PY{p}{)}
\PY{n+nb}{print}\PY{p}{(}\PY{l+s+s2}{\PYZdq{}}\PY{l+s+s2}{W2 = }\PY{l+s+se}{\PYZbs{}n}\PY{l+s+s2}{\PYZdq{}} \PY{o}{+} \PY{n+nb}{str}\PY{p}{(}\PY{n}{parameters}\PY{p}{[}\PY{l+s+s2}{\PYZdq{}}\PY{l+s+s2}{W2}\PY{l+s+s2}{\PYZdq{}}\PY{p}{]}\PY{p}{)}\PY{p}{)}
\PY{n+nb}{print}\PY{p}{(}\PY{l+s+s2}{\PYZdq{}}\PY{l+s+s2}{b2 = }\PY{l+s+se}{\PYZbs{}n}\PY{l+s+s2}{\PYZdq{}} \PY{o}{+} \PY{n+nb}{str}\PY{p}{(}\PY{n}{parameters}\PY{p}{[}\PY{l+s+s2}{\PYZdq{}}\PY{l+s+s2}{b2}\PY{l+s+s2}{\PYZdq{}}\PY{p}{]}\PY{p}{)}\PY{p}{)}
\PY{n+nb}{print}\PY{p}{(}\PY{l+s+s2}{\PYZdq{}}\PY{l+s+s2}{v[}\PY{l+s+se}{\PYZbs{}\PYZdq{}}\PY{l+s+s2}{dW1}\PY{l+s+se}{\PYZbs{}\PYZdq{}}\PY{l+s+s2}{] = }\PY{l+s+se}{\PYZbs{}n}\PY{l+s+s2}{\PYZdq{}} \PY{o}{+} \PY{n+nb}{str}\PY{p}{(}\PY{n}{v}\PY{p}{[}\PY{l+s+s2}{\PYZdq{}}\PY{l+s+s2}{dW1}\PY{l+s+s2}{\PYZdq{}}\PY{p}{]}\PY{p}{)}\PY{p}{)}
\PY{n+nb}{print}\PY{p}{(}\PY{l+s+s2}{\PYZdq{}}\PY{l+s+s2}{v[}\PY{l+s+se}{\PYZbs{}\PYZdq{}}\PY{l+s+s2}{db1}\PY{l+s+se}{\PYZbs{}\PYZdq{}}\PY{l+s+s2}{] = }\PY{l+s+se}{\PYZbs{}n}\PY{l+s+s2}{\PYZdq{}} \PY{o}{+} \PY{n+nb}{str}\PY{p}{(}\PY{n}{v}\PY{p}{[}\PY{l+s+s2}{\PYZdq{}}\PY{l+s+s2}{db1}\PY{l+s+s2}{\PYZdq{}}\PY{p}{]}\PY{p}{)}\PY{p}{)}
\PY{n+nb}{print}\PY{p}{(}\PY{l+s+s2}{\PYZdq{}}\PY{l+s+s2}{v[}\PY{l+s+se}{\PYZbs{}\PYZdq{}}\PY{l+s+s2}{dW2}\PY{l+s+se}{\PYZbs{}\PYZdq{}}\PY{l+s+s2}{] = }\PY{l+s+se}{\PYZbs{}n}\PY{l+s+s2}{\PYZdq{}} \PY{o}{+} \PY{n+nb}{str}\PY{p}{(}\PY{n}{v}\PY{p}{[}\PY{l+s+s2}{\PYZdq{}}\PY{l+s+s2}{dW2}\PY{l+s+s2}{\PYZdq{}}\PY{p}{]}\PY{p}{)}\PY{p}{)}
\PY{n+nb}{print}\PY{p}{(}\PY{l+s+s2}{\PYZdq{}}\PY{l+s+s2}{v[}\PY{l+s+se}{\PYZbs{}\PYZdq{}}\PY{l+s+s2}{db2}\PY{l+s+se}{\PYZbs{}\PYZdq{}}\PY{l+s+s2}{] = v}\PY{l+s+s2}{\PYZdq{}} \PY{o}{+} \PY{n+nb}{str}\PY{p}{(}\PY{n}{v}\PY{p}{[}\PY{l+s+s2}{\PYZdq{}}\PY{l+s+s2}{db2}\PY{l+s+s2}{\PYZdq{}}\PY{p}{]}\PY{p}{)}\PY{p}{)}

\PY{n}{update\PYZus{}parameters\PYZus{}with\PYZus{}momentum\PYZus{}test}\PY{p}{(}\PY{n}{update\PYZus{}parameters\PYZus{}with\PYZus{}momentum}\PY{p}{)}
\end{Verbatim}
\end{tcolorbox}

    \textbf{Note that}: - The velocity is initialized with zeros. So the
algorithm will take a few iterations to ``build up'' velocity and start
to take bigger steps. - If \(\beta = 0\), then this just becomes
standard gradient descent without momentum.

\textbf{How do you choose \(\beta\)?}

\begin{itemize}
\tightlist
\item
  The larger the momentum \(\beta\) is, the smoother the update, because
  it takes the past gradients into account more. But if \(\beta\) is too
  big, it could also smooth out the updates too much.
\item
  Common values for \(\beta\) range from 0.8 to 0.999. If you don't feel
  inclined to tune this, \(\beta = 0.9\) is often a reasonable default.
\item
  Tuning the optimal \(\beta\) for your model might require trying
  several values to see what works best in terms of reducing the value
  of the cost function \(J\).
\end{itemize}

    \textbf{What you should remember}: - Momentum takes past gradients into
account to smooth out the steps of gradient descent. It can be applied
with batch gradient descent, mini-batch gradient descent or stochastic
gradient descent. - You have to tune a momentum hyperparameter \(\beta\)
and a learning rate \(\alpha\).

    ~\\
\#\# 5 - Adam

Adam is one of the most effective optimization algorithms for training
neural networks. It combines ideas from RMSProp (described in lecture)
and Momentum.

\textbf{How does Adam work?} 1. It calculates an exponentially weighted
average of past gradients, and stores it in variables \(v\) (before bias
correction) and \(v^{corrected}\) (with bias correction). 2. It
calculates an exponentially weighted average of the squares of the past
gradients, and stores it in variables \(s\) (before bias correction) and
\(s^{corrected}\) (with bias correction). 3. It updates parameters in a
direction based on combining information from ``1'' and ``2''.

The update rule is, for \(l = 1, ..., L\):

\[\begin{cases}
v_{dW^{[l]}} = \beta_1 v_{dW^{[l]}} + (1 - \beta_1) \frac{\partial \mathcal{J} }{ \partial W^{[l]} } \\
v^{corrected}_{dW^{[l]}} = \frac{v_{dW^{[l]}}}{1 - (\beta_1)^t} \\
s_{dW^{[l]}} = \beta_2 s_{dW^{[l]}} + (1 - \beta_2) (\frac{\partial \mathcal{J} }{\partial W^{[l]} })^2 \\
s^{corrected}_{dW^{[l]}} = \frac{s_{dW^{[l]}}}{1 - (\beta_2)^t} \\
W^{[l]} = W^{[l]} - \alpha \frac{v^{corrected}_{dW^{[l]}}}{\sqrt{s^{corrected}_{dW^{[l]}}} + \varepsilon}
\end{cases}\] where: - t counts the number of steps taken of Adam - L is
the number of layers - \(\beta_1\) and \(\beta_2\) are hyperparameters
that control the two exponentially weighted averages. - \(\alpha\) is
the learning rate - \(\varepsilon\) is a very small number to avoid
dividing by zero

As usual, all parameters are stored in the \texttt{parameters}
dictionary

    ~\\
\#\#\# Exercise 5 - initialize\_adam

Initialize the Adam variables \(v, s\) which keep track of the past
information.

\textbf{Instruction}: The variables \(v, s\) are python dictionaries
that need to be initialized with arrays of zeros. Their keys are the
same as for \texttt{grads}, that is: for \(l = 1, ..., L\):

\begin{Shaded}
\begin{Highlighting}[]
\NormalTok{v[}\StringTok{"dW"} \OperatorTok{+} \BuiltInTok{str}\NormalTok{(l)] }\OperatorTok{=}\NormalTok{ ... }\CommentTok{\#(numpy array of zeros with the same shape as parameters["W" + str(l)])}
\NormalTok{v[}\StringTok{"db"} \OperatorTok{+} \BuiltInTok{str}\NormalTok{(l)] }\OperatorTok{=}\NormalTok{ ... }\CommentTok{\#(numpy array of zeros with the same shape as parameters["b" + str(l)])}
\NormalTok{s[}\StringTok{"dW"} \OperatorTok{+} \BuiltInTok{str}\NormalTok{(l)] }\OperatorTok{=}\NormalTok{ ... }\CommentTok{\#(numpy array of zeros with the same shape as parameters["W" + str(l)])}
\NormalTok{s[}\StringTok{"db"} \OperatorTok{+} \BuiltInTok{str}\NormalTok{(l)] }\OperatorTok{=}\NormalTok{ ... }\CommentTok{\#(numpy array of zeros with the same shape as parameters["b" + str(l)])}
\end{Highlighting}
\end{Shaded}

    \begin{tcolorbox}[breakable, size=fbox, boxrule=1pt, pad at break*=1mm,colback=cellbackground, colframe=cellborder]
\prompt{In}{incolor}{ }{\boxspacing}
\begin{Verbatim}[commandchars=\\\{\}]
\PY{c+c1}{\PYZsh{} GRADED FUNCTION: initialize\PYZus{}adam}

\PY{k}{def} \PY{n+nf}{initialize\PYZus{}adam}\PY{p}{(}\PY{n}{parameters}\PY{p}{)} \PY{p}{:}
    \PY{l+s+sd}{\PYZdq{}\PYZdq{}\PYZdq{}}
\PY{l+s+sd}{    Initializes v and s as two python dictionaries with:}
\PY{l+s+sd}{                \PYZhy{} keys: \PYZdq{}dW1\PYZdq{}, \PYZdq{}db1\PYZdq{}, ..., \PYZdq{}dWL\PYZdq{}, \PYZdq{}dbL\PYZdq{} }
\PY{l+s+sd}{                \PYZhy{} values: numpy arrays of zeros of the same shape as the corresponding gradients/parameters.}
\PY{l+s+sd}{    }
\PY{l+s+sd}{    Arguments:}
\PY{l+s+sd}{    parameters \PYZhy{}\PYZhy{} python dictionary containing your parameters.}
\PY{l+s+sd}{                    parameters[\PYZdq{}W\PYZdq{} + str(l)] = Wl}
\PY{l+s+sd}{                    parameters[\PYZdq{}b\PYZdq{} + str(l)] = bl}
\PY{l+s+sd}{    }
\PY{l+s+sd}{    Returns: }
\PY{l+s+sd}{    v \PYZhy{}\PYZhy{} python dictionary that will contain the exponentially weighted average of the gradient. Initialized with zeros.}
\PY{l+s+sd}{                    v[\PYZdq{}dW\PYZdq{} + str(l)] = ...}
\PY{l+s+sd}{                    v[\PYZdq{}db\PYZdq{} + str(l)] = ...}
\PY{l+s+sd}{    s \PYZhy{}\PYZhy{} python dictionary that will contain the exponentially weighted average of the squared gradient. Initialized with zeros.}
\PY{l+s+sd}{                    s[\PYZdq{}dW\PYZdq{} + str(l)] = ...}
\PY{l+s+sd}{                    s[\PYZdq{}db\PYZdq{} + str(l)] = ...}

\PY{l+s+sd}{    \PYZdq{}\PYZdq{}\PYZdq{}}
    
    \PY{n}{L} \PY{o}{=} \PY{n+nb}{len}\PY{p}{(}\PY{n}{parameters}\PY{p}{)} \PY{o}{/}\PY{o}{/} \PY{l+m+mi}{2} \PY{c+c1}{\PYZsh{} number of layers in the neural networks}
    \PY{n}{v} \PY{o}{=} \PY{p}{\PYZob{}}\PY{p}{\PYZcb{}}
    \PY{n}{s} \PY{o}{=} \PY{p}{\PYZob{}}\PY{p}{\PYZcb{}}
    
    \PY{c+c1}{\PYZsh{} Initialize v, s. Input: \PYZdq{}parameters\PYZdq{}. Outputs: \PYZdq{}v, s\PYZdq{}.}
    \PY{k}{for} \PY{n}{l} \PY{o+ow}{in} \PY{n+nb}{range}\PY{p}{(}\PY{l+m+mi}{1}\PY{p}{,} \PY{n}{L} \PY{o}{+} \PY{l+m+mi}{1}\PY{p}{)}\PY{p}{:}
    \PY{c+c1}{\PYZsh{} (approx. 4 lines)}
        \PY{c+c1}{\PYZsh{} v[\PYZdq{}dW\PYZdq{} + str(l)] = ...}
        \PY{c+c1}{\PYZsh{} v[\PYZdq{}db\PYZdq{} + str(l)] = ...}
        \PY{c+c1}{\PYZsh{} s[\PYZdq{}dW\PYZdq{} + str(l)] = ...}
        \PY{c+c1}{\PYZsh{} s[\PYZdq{}db\PYZdq{} + str(l)] = ...}
    \PY{c+c1}{\PYZsh{} YOUR CODE STARTS HERE}
    
    
    \PY{c+c1}{\PYZsh{} YOUR CODE ENDS HERE}
    
    \PY{k}{return} \PY{n}{v}\PY{p}{,} \PY{n}{s}
\end{Verbatim}
\end{tcolorbox}

    \begin{tcolorbox}[breakable, size=fbox, boxrule=1pt, pad at break*=1mm,colback=cellbackground, colframe=cellborder]
\prompt{In}{incolor}{ }{\boxspacing}
\begin{Verbatim}[commandchars=\\\{\}]
\PY{n}{parameters} \PY{o}{=} \PY{n}{initialize\PYZus{}adam\PYZus{}test\PYZus{}case}\PY{p}{(}\PY{p}{)}

\PY{n}{v}\PY{p}{,} \PY{n}{s} \PY{o}{=} \PY{n}{initialize\PYZus{}adam}\PY{p}{(}\PY{n}{parameters}\PY{p}{)}
\PY{n+nb}{print}\PY{p}{(}\PY{l+s+s2}{\PYZdq{}}\PY{l+s+s2}{v[}\PY{l+s+se}{\PYZbs{}\PYZdq{}}\PY{l+s+s2}{dW1}\PY{l+s+se}{\PYZbs{}\PYZdq{}}\PY{l+s+s2}{] = }\PY{l+s+se}{\PYZbs{}n}\PY{l+s+s2}{\PYZdq{}} \PY{o}{+} \PY{n+nb}{str}\PY{p}{(}\PY{n}{v}\PY{p}{[}\PY{l+s+s2}{\PYZdq{}}\PY{l+s+s2}{dW1}\PY{l+s+s2}{\PYZdq{}}\PY{p}{]}\PY{p}{)}\PY{p}{)}
\PY{n+nb}{print}\PY{p}{(}\PY{l+s+s2}{\PYZdq{}}\PY{l+s+s2}{v[}\PY{l+s+se}{\PYZbs{}\PYZdq{}}\PY{l+s+s2}{db1}\PY{l+s+se}{\PYZbs{}\PYZdq{}}\PY{l+s+s2}{] = }\PY{l+s+se}{\PYZbs{}n}\PY{l+s+s2}{\PYZdq{}} \PY{o}{+} \PY{n+nb}{str}\PY{p}{(}\PY{n}{v}\PY{p}{[}\PY{l+s+s2}{\PYZdq{}}\PY{l+s+s2}{db1}\PY{l+s+s2}{\PYZdq{}}\PY{p}{]}\PY{p}{)}\PY{p}{)}
\PY{n+nb}{print}\PY{p}{(}\PY{l+s+s2}{\PYZdq{}}\PY{l+s+s2}{v[}\PY{l+s+se}{\PYZbs{}\PYZdq{}}\PY{l+s+s2}{dW2}\PY{l+s+se}{\PYZbs{}\PYZdq{}}\PY{l+s+s2}{] = }\PY{l+s+se}{\PYZbs{}n}\PY{l+s+s2}{\PYZdq{}} \PY{o}{+} \PY{n+nb}{str}\PY{p}{(}\PY{n}{v}\PY{p}{[}\PY{l+s+s2}{\PYZdq{}}\PY{l+s+s2}{dW2}\PY{l+s+s2}{\PYZdq{}}\PY{p}{]}\PY{p}{)}\PY{p}{)}
\PY{n+nb}{print}\PY{p}{(}\PY{l+s+s2}{\PYZdq{}}\PY{l+s+s2}{v[}\PY{l+s+se}{\PYZbs{}\PYZdq{}}\PY{l+s+s2}{db2}\PY{l+s+se}{\PYZbs{}\PYZdq{}}\PY{l+s+s2}{] = }\PY{l+s+se}{\PYZbs{}n}\PY{l+s+s2}{\PYZdq{}} \PY{o}{+} \PY{n+nb}{str}\PY{p}{(}\PY{n}{v}\PY{p}{[}\PY{l+s+s2}{\PYZdq{}}\PY{l+s+s2}{db2}\PY{l+s+s2}{\PYZdq{}}\PY{p}{]}\PY{p}{)}\PY{p}{)}
\PY{n+nb}{print}\PY{p}{(}\PY{l+s+s2}{\PYZdq{}}\PY{l+s+s2}{s[}\PY{l+s+se}{\PYZbs{}\PYZdq{}}\PY{l+s+s2}{dW1}\PY{l+s+se}{\PYZbs{}\PYZdq{}}\PY{l+s+s2}{] = }\PY{l+s+se}{\PYZbs{}n}\PY{l+s+s2}{\PYZdq{}} \PY{o}{+} \PY{n+nb}{str}\PY{p}{(}\PY{n}{s}\PY{p}{[}\PY{l+s+s2}{\PYZdq{}}\PY{l+s+s2}{dW1}\PY{l+s+s2}{\PYZdq{}}\PY{p}{]}\PY{p}{)}\PY{p}{)}
\PY{n+nb}{print}\PY{p}{(}\PY{l+s+s2}{\PYZdq{}}\PY{l+s+s2}{s[}\PY{l+s+se}{\PYZbs{}\PYZdq{}}\PY{l+s+s2}{db1}\PY{l+s+se}{\PYZbs{}\PYZdq{}}\PY{l+s+s2}{] = }\PY{l+s+se}{\PYZbs{}n}\PY{l+s+s2}{\PYZdq{}} \PY{o}{+} \PY{n+nb}{str}\PY{p}{(}\PY{n}{s}\PY{p}{[}\PY{l+s+s2}{\PYZdq{}}\PY{l+s+s2}{db1}\PY{l+s+s2}{\PYZdq{}}\PY{p}{]}\PY{p}{)}\PY{p}{)}
\PY{n+nb}{print}\PY{p}{(}\PY{l+s+s2}{\PYZdq{}}\PY{l+s+s2}{s[}\PY{l+s+se}{\PYZbs{}\PYZdq{}}\PY{l+s+s2}{dW2}\PY{l+s+se}{\PYZbs{}\PYZdq{}}\PY{l+s+s2}{] = }\PY{l+s+se}{\PYZbs{}n}\PY{l+s+s2}{\PYZdq{}} \PY{o}{+} \PY{n+nb}{str}\PY{p}{(}\PY{n}{s}\PY{p}{[}\PY{l+s+s2}{\PYZdq{}}\PY{l+s+s2}{dW2}\PY{l+s+s2}{\PYZdq{}}\PY{p}{]}\PY{p}{)}\PY{p}{)}
\PY{n+nb}{print}\PY{p}{(}\PY{l+s+s2}{\PYZdq{}}\PY{l+s+s2}{s[}\PY{l+s+se}{\PYZbs{}\PYZdq{}}\PY{l+s+s2}{db2}\PY{l+s+se}{\PYZbs{}\PYZdq{}}\PY{l+s+s2}{] = }\PY{l+s+se}{\PYZbs{}n}\PY{l+s+s2}{\PYZdq{}} \PY{o}{+} \PY{n+nb}{str}\PY{p}{(}\PY{n}{s}\PY{p}{[}\PY{l+s+s2}{\PYZdq{}}\PY{l+s+s2}{db2}\PY{l+s+s2}{\PYZdq{}}\PY{p}{]}\PY{p}{)}\PY{p}{)}

\PY{n}{initialize\PYZus{}adam\PYZus{}test}\PY{p}{(}\PY{n}{initialize\PYZus{}adam}\PY{p}{)}
\end{Verbatim}
\end{tcolorbox}

    ~\\
\#\#\# Exercise 6 - update\_parameters\_with\_adam

Now, implement the parameters update with Adam. Recall the general
update rule is, for \(l = 1, ..., L\):

\[\begin{cases}
v_{dW^{[l]}} = \beta_1 v_{dW^{[l]}} + (1 - \beta_1) \frac{\partial \mathcal{J} }{ \partial W^{[l]} } \\
v^{corrected}_{dW^{[l]}} = \frac{v_{dW^{[l]}}}{1 - (\beta_1)^t} \\
s_{dW^{[l]}} = \beta_2 s_{dW^{[l]}} + (1 - \beta_2) (\frac{\partial \mathcal{J} }{\partial W^{[l]} })^2 \\
s^{corrected}_{dW^{[l]}} = \frac{s_{dW^{[l]}}}{1 - (\beta_2)^t} \\
W^{[l]} = W^{[l]} - \alpha \frac{v^{corrected}_{dW^{[l]}}}{\sqrt{s^{corrected}_{dW^{[l]}}} + \varepsilon}
\end{cases}\]

\textbf{Note} that the iterator \texttt{l} starts at 1 in the
\texttt{for} loop as the first parameters are \(W^{[1]}\) and
\(b^{[1]}\).

    \begin{tcolorbox}[breakable, size=fbox, boxrule=1pt, pad at break*=1mm,colback=cellbackground, colframe=cellborder]
\prompt{In}{incolor}{ }{\boxspacing}
\begin{Verbatim}[commandchars=\\\{\}]
\PY{c+c1}{\PYZsh{} GRADED FUNCTION: update\PYZus{}parameters\PYZus{}with\PYZus{}adam}

\PY{k}{def} \PY{n+nf}{update\PYZus{}parameters\PYZus{}with\PYZus{}adam}\PY{p}{(}\PY{n}{parameters}\PY{p}{,} \PY{n}{grads}\PY{p}{,} \PY{n}{v}\PY{p}{,} \PY{n}{s}\PY{p}{,} \PY{n}{t}\PY{p}{,} \PY{n}{learning\PYZus{}rate} \PY{o}{=} \PY{l+m+mf}{0.01}\PY{p}{,}
                                \PY{n}{beta1} \PY{o}{=} \PY{l+m+mf}{0.9}\PY{p}{,} \PY{n}{beta2} \PY{o}{=} \PY{l+m+mf}{0.999}\PY{p}{,}  \PY{n}{epsilon} \PY{o}{=} \PY{l+m+mf}{1e\PYZhy{}8}\PY{p}{)}\PY{p}{:}
    \PY{l+s+sd}{\PYZdq{}\PYZdq{}\PYZdq{}}
\PY{l+s+sd}{    Update parameters using Adam}
\PY{l+s+sd}{    }
\PY{l+s+sd}{    Arguments:}
\PY{l+s+sd}{    parameters \PYZhy{}\PYZhy{} python dictionary containing your parameters:}
\PY{l+s+sd}{                    parameters[\PYZsq{}W\PYZsq{} + str(l)] = Wl}
\PY{l+s+sd}{                    parameters[\PYZsq{}b\PYZsq{} + str(l)] = bl}
\PY{l+s+sd}{    grads \PYZhy{}\PYZhy{} python dictionary containing your gradients for each parameters:}
\PY{l+s+sd}{                    grads[\PYZsq{}dW\PYZsq{} + str(l)] = dWl}
\PY{l+s+sd}{                    grads[\PYZsq{}db\PYZsq{} + str(l)] = dbl}
\PY{l+s+sd}{    v \PYZhy{}\PYZhy{} Adam variable, moving average of the first gradient, python dictionary}
\PY{l+s+sd}{    s \PYZhy{}\PYZhy{} Adam variable, moving average of the squared gradient, python dictionary}
\PY{l+s+sd}{    t \PYZhy{}\PYZhy{} Adam variable, counts the number of taken steps}
\PY{l+s+sd}{    learning\PYZus{}rate \PYZhy{}\PYZhy{} the learning rate, scalar.}
\PY{l+s+sd}{    beta1 \PYZhy{}\PYZhy{} Exponential decay hyperparameter for the first moment estimates }
\PY{l+s+sd}{    beta2 \PYZhy{}\PYZhy{} Exponential decay hyperparameter for the second moment estimates }
\PY{l+s+sd}{    epsilon \PYZhy{}\PYZhy{} hyperparameter preventing division by zero in Adam updates}

\PY{l+s+sd}{    Returns:}
\PY{l+s+sd}{    parameters \PYZhy{}\PYZhy{} python dictionary containing your updated parameters }
\PY{l+s+sd}{    v \PYZhy{}\PYZhy{} Adam variable, moving average of the first gradient, python dictionary}
\PY{l+s+sd}{    s \PYZhy{}\PYZhy{} Adam variable, moving average of the squared gradient, python dictionary}
\PY{l+s+sd}{    \PYZdq{}\PYZdq{}\PYZdq{}}
    
    \PY{n}{L} \PY{o}{=} \PY{n+nb}{len}\PY{p}{(}\PY{n}{parameters}\PY{p}{)} \PY{o}{/}\PY{o}{/} \PY{l+m+mi}{2}                 \PY{c+c1}{\PYZsh{} number of layers in the neural networks}
    \PY{n}{v\PYZus{}corrected} \PY{o}{=} \PY{p}{\PYZob{}}\PY{p}{\PYZcb{}}                         \PY{c+c1}{\PYZsh{} Initializing first moment estimate, python dictionary}
    \PY{n}{s\PYZus{}corrected} \PY{o}{=} \PY{p}{\PYZob{}}\PY{p}{\PYZcb{}}                         \PY{c+c1}{\PYZsh{} Initializing second moment estimate, python dictionary}
    
    \PY{c+c1}{\PYZsh{} Perform Adam update on all parameters}
    \PY{k}{for} \PY{n}{l} \PY{o+ow}{in} \PY{n+nb}{range}\PY{p}{(}\PY{l+m+mi}{1}\PY{p}{,} \PY{n}{L} \PY{o}{+} \PY{l+m+mi}{1}\PY{p}{)}\PY{p}{:}
        \PY{c+c1}{\PYZsh{} Moving average of the gradients. Inputs: \PYZdq{}v, grads, beta1\PYZdq{}. Output: \PYZdq{}v\PYZdq{}.}
        \PY{c+c1}{\PYZsh{} (approx. 2 lines)}
        \PY{c+c1}{\PYZsh{} v[\PYZdq{}dW\PYZdq{} + str(l)] = ...}
        \PY{c+c1}{\PYZsh{} v[\PYZdq{}db\PYZdq{} + str(l)] = ...}
        \PY{c+c1}{\PYZsh{} YOUR CODE STARTS HERE}
        
        
        \PY{c+c1}{\PYZsh{} YOUR CODE ENDS HERE}

        \PY{c+c1}{\PYZsh{} Compute bias\PYZhy{}corrected first moment estimate. Inputs: \PYZdq{}v, beta1, t\PYZdq{}. Output: \PYZdq{}v\PYZus{}corrected\PYZdq{}.}
        \PY{c+c1}{\PYZsh{} (approx. 2 lines)}
        \PY{c+c1}{\PYZsh{} v\PYZus{}corrected[\PYZdq{}dW\PYZdq{} + str(l)] = ...}
        \PY{c+c1}{\PYZsh{} v\PYZus{}corrected[\PYZdq{}db\PYZdq{} + str(l)] = ...}
        \PY{c+c1}{\PYZsh{} YOUR CODE STARTS HERE}
        
        
        \PY{c+c1}{\PYZsh{} YOUR CODE ENDS HERE}

        \PY{c+c1}{\PYZsh{} Moving average of the squared gradients. Inputs: \PYZdq{}s, grads, beta2\PYZdq{}. Output: \PYZdq{}s\PYZdq{}.}
        \PY{c+c1}{\PYZsh{}(approx. 2 lines)}
        \PY{c+c1}{\PYZsh{} s[\PYZdq{}dW\PYZdq{} + str(l)] = ...}
        \PY{c+c1}{\PYZsh{} s[\PYZdq{}db\PYZdq{} + str(l)] = ...}
        \PY{c+c1}{\PYZsh{} YOUR CODE STARTS HERE}
        
        
        \PY{c+c1}{\PYZsh{} YOUR CODE ENDS HERE}

        \PY{c+c1}{\PYZsh{} Compute bias\PYZhy{}corrected second raw moment estimate. Inputs: \PYZdq{}s, beta2, t\PYZdq{}. Output: \PYZdq{}s\PYZus{}corrected\PYZdq{}.}
        \PY{c+c1}{\PYZsh{} (approx. 2 lines)}
        \PY{c+c1}{\PYZsh{} s\PYZus{}corrected[\PYZdq{}dW\PYZdq{} + str(l)] = ...}
        \PY{c+c1}{\PYZsh{} s\PYZus{}corrected[\PYZdq{}db\PYZdq{} + str(l)] = ...}
        \PY{c+c1}{\PYZsh{} YOUR CODE STARTS HERE}
        
        
        \PY{c+c1}{\PYZsh{} YOUR CODE ENDS HERE}

        \PY{c+c1}{\PYZsh{} Update parameters. Inputs: \PYZdq{}parameters, learning\PYZus{}rate, v\PYZus{}corrected, s\PYZus{}corrected, epsilon\PYZdq{}. Output: \PYZdq{}parameters\PYZdq{}.}
        \PY{c+c1}{\PYZsh{} (approx. 2 lines)}
        \PY{c+c1}{\PYZsh{} parameters[\PYZdq{}W\PYZdq{} + str(l)] = ...}
        \PY{c+c1}{\PYZsh{} parameters[\PYZdq{}b\PYZdq{} + str(l)] = ...}
        \PY{c+c1}{\PYZsh{} YOUR CODE STARTS HERE}
        
        
        \PY{c+c1}{\PYZsh{} YOUR CODE ENDS HERE}

    \PY{k}{return} \PY{n}{parameters}\PY{p}{,} \PY{n}{v}\PY{p}{,} \PY{n}{s}\PY{p}{,} \PY{n}{v\PYZus{}corrected}\PY{p}{,} \PY{n}{s\PYZus{}corrected}
\end{Verbatim}
\end{tcolorbox}

    \begin{tcolorbox}[breakable, size=fbox, boxrule=1pt, pad at break*=1mm,colback=cellbackground, colframe=cellborder]
\prompt{In}{incolor}{ }{\boxspacing}
\begin{Verbatim}[commandchars=\\\{\}]
\PY{n}{parametersi}\PY{p}{,} \PY{n}{grads}\PY{p}{,} \PY{n}{vi}\PY{p}{,} \PY{n}{si} \PY{o}{=} \PY{n}{update\PYZus{}parameters\PYZus{}with\PYZus{}adam\PYZus{}test\PYZus{}case}\PY{p}{(}\PY{p}{)}

\PY{n}{t} \PY{o}{=} \PY{l+m+mi}{2}
\PY{n}{learning\PYZus{}rate} \PY{o}{=} \PY{l+m+mf}{0.02}
\PY{n}{beta1} \PY{o}{=} \PY{l+m+mf}{0.8}
\PY{n}{beta2} \PY{o}{=} \PY{l+m+mf}{0.888}
\PY{n}{epsilon} \PY{o}{=} \PY{l+m+mf}{1e\PYZhy{}2}

\PY{n}{parameters}\PY{p}{,} \PY{n}{v}\PY{p}{,} \PY{n}{s}\PY{p}{,} \PY{n}{vc}\PY{p}{,} \PY{n}{sc}  \PY{o}{=} \PY{n}{update\PYZus{}parameters\PYZus{}with\PYZus{}adam}\PY{p}{(}\PY{n}{parametersi}\PY{p}{,} \PY{n}{grads}\PY{p}{,} \PY{n}{vi}\PY{p}{,} \PY{n}{si}\PY{p}{,} \PY{n}{t}\PY{p}{,} \PY{n}{learning\PYZus{}rate}\PY{p}{,} \PY{n}{beta1}\PY{p}{,} \PY{n}{beta2}\PY{p}{,} \PY{n}{epsilon}\PY{p}{)}
\PY{n+nb}{print}\PY{p}{(}\PY{l+s+sa}{f}\PY{l+s+s2}{\PYZdq{}}\PY{l+s+s2}{W1 = }\PY{l+s+se}{\PYZbs{}n}\PY{l+s+si}{\PYZob{}}\PY{n}{parameters}\PY{p}{[}\PY{l+s+s1}{\PYZsq{}}\PY{l+s+s1}{W1}\PY{l+s+s1}{\PYZsq{}}\PY{p}{]}\PY{l+s+si}{\PYZcb{}}\PY{l+s+s2}{\PYZdq{}}\PY{p}{)}
\PY{n+nb}{print}\PY{p}{(}\PY{l+s+sa}{f}\PY{l+s+s2}{\PYZdq{}}\PY{l+s+s2}{W2 = }\PY{l+s+se}{\PYZbs{}n}\PY{l+s+si}{\PYZob{}}\PY{n}{parameters}\PY{p}{[}\PY{l+s+s1}{\PYZsq{}}\PY{l+s+s1}{W2}\PY{l+s+s1}{\PYZsq{}}\PY{p}{]}\PY{l+s+si}{\PYZcb{}}\PY{l+s+s2}{\PYZdq{}}\PY{p}{)}
\PY{n+nb}{print}\PY{p}{(}\PY{l+s+sa}{f}\PY{l+s+s2}{\PYZdq{}}\PY{l+s+s2}{b1 = }\PY{l+s+se}{\PYZbs{}n}\PY{l+s+si}{\PYZob{}}\PY{n}{parameters}\PY{p}{[}\PY{l+s+s1}{\PYZsq{}}\PY{l+s+s1}{b1}\PY{l+s+s1}{\PYZsq{}}\PY{p}{]}\PY{l+s+si}{\PYZcb{}}\PY{l+s+s2}{\PYZdq{}}\PY{p}{)}
\PY{n+nb}{print}\PY{p}{(}\PY{l+s+sa}{f}\PY{l+s+s2}{\PYZdq{}}\PY{l+s+s2}{b2 = }\PY{l+s+se}{\PYZbs{}n}\PY{l+s+si}{\PYZob{}}\PY{n}{parameters}\PY{p}{[}\PY{l+s+s1}{\PYZsq{}}\PY{l+s+s1}{b2}\PY{l+s+s1}{\PYZsq{}}\PY{p}{]}\PY{l+s+si}{\PYZcb{}}\PY{l+s+s2}{\PYZdq{}}\PY{p}{)}

\PY{n}{update\PYZus{}parameters\PYZus{}with\PYZus{}adam\PYZus{}test}\PY{p}{(}\PY{n}{update\PYZus{}parameters\PYZus{}with\PYZus{}adam}\PY{p}{)}
\end{Verbatim}
\end{tcolorbox}

    \textbf{Expected values:}

\begin{verbatim}
W1 = 
[[ 1.63942428 -0.6268425  -0.54320974]
 [-1.08782943  0.85036983 -2.2865723 ]]
W2 = 
[[ 0.33356139 -0.26425199  1.47707772]
 [-2.04538458 -0.30744933 -0.36903141]
 [ 1.14873036 -1.09256871 -0.15734651]]
b1 = 
[[ 1.75854357]
 [-0.74616067]]
b2 = 
[[-0.89228024]
 [ 0.02707193]
 [ 0.56782561]]
\end{verbatim}

    You now have three working optimization algorithms (mini-batch gradient
descent, Momentum, Adam). Let's implement a model with each of these
optimizers and observe the difference.

    ~\\
\#\# 6 - Model with different Optimization algorithms

Below, you'll use the following ``moons'' dataset to test the different
optimization methods. (The dataset is named ``moons'' because the data
from each of the two classes looks a bit like a crescent-shaped moon.)

    \begin{tcolorbox}[breakable, size=fbox, boxrule=1pt, pad at break*=1mm,colback=cellbackground, colframe=cellborder]
\prompt{In}{incolor}{ }{\boxspacing}
\begin{Verbatim}[commandchars=\\\{\}]
\PY{n}{train\PYZus{}X}\PY{p}{,} \PY{n}{train\PYZus{}Y} \PY{o}{=} \PY{n}{load\PYZus{}dataset}\PY{p}{(}\PY{p}{)}
\end{Verbatim}
\end{tcolorbox}

    A 3-layer neural network has already been implemented for you! You'll
train it with: - Mini-batch \textbf{Gradient Descent}: it will call your
function: - \texttt{update\_parameters\_with\_gd()} - Mini-batch
\textbf{Momentum}: it will call your functions: -
\texttt{initialize\_velocity()} and
\texttt{update\_parameters\_with\_momentum()} - Mini-batch
\textbf{Adam}: it will call your functions: -
\texttt{initialize\_adam()} and
\texttt{update\_parameters\_with\_adam()}

    \begin{tcolorbox}[breakable, size=fbox, boxrule=1pt, pad at break*=1mm,colback=cellbackground, colframe=cellborder]
\prompt{In}{incolor}{ }{\boxspacing}
\begin{Verbatim}[commandchars=\\\{\}]
\PY{k}{def} \PY{n+nf}{model}\PY{p}{(}\PY{n}{X}\PY{p}{,} \PY{n}{Y}\PY{p}{,} \PY{n}{layers\PYZus{}dims}\PY{p}{,} \PY{n}{optimizer}\PY{p}{,} \PY{n}{learning\PYZus{}rate} \PY{o}{=} \PY{l+m+mf}{0.0007}\PY{p}{,} \PY{n}{mini\PYZus{}batch\PYZus{}size} \PY{o}{=} \PY{l+m+mi}{64}\PY{p}{,} \PY{n}{beta} \PY{o}{=} \PY{l+m+mf}{0.9}\PY{p}{,}
          \PY{n}{beta1} \PY{o}{=} \PY{l+m+mf}{0.9}\PY{p}{,} \PY{n}{beta2} \PY{o}{=} \PY{l+m+mf}{0.999}\PY{p}{,}  \PY{n}{epsilon} \PY{o}{=} \PY{l+m+mf}{1e\PYZhy{}8}\PY{p}{,} \PY{n}{num\PYZus{}epochs} \PY{o}{=} \PY{l+m+mi}{5000}\PY{p}{,} \PY{n}{print\PYZus{}cost} \PY{o}{=} \PY{k+kc}{True}\PY{p}{)}\PY{p}{:}
    \PY{l+s+sd}{\PYZdq{}\PYZdq{}\PYZdq{}}
\PY{l+s+sd}{    3\PYZhy{}layer neural network model which can be run in different optimizer modes.}
\PY{l+s+sd}{    }
\PY{l+s+sd}{    Arguments:}
\PY{l+s+sd}{    X \PYZhy{}\PYZhy{} input data, of shape (2, number of examples)}
\PY{l+s+sd}{    Y \PYZhy{}\PYZhy{} true \PYZdq{}label\PYZdq{} vector (1 for blue dot / 0 for red dot), of shape (1, number of examples)}
\PY{l+s+sd}{    layers\PYZus{}dims \PYZhy{}\PYZhy{} python list, containing the size of each layer}
\PY{l+s+sd}{    learning\PYZus{}rate \PYZhy{}\PYZhy{} the learning rate, scalar.}
\PY{l+s+sd}{    mini\PYZus{}batch\PYZus{}size \PYZhy{}\PYZhy{} the size of a mini batch}
\PY{l+s+sd}{    beta \PYZhy{}\PYZhy{} Momentum hyperparameter}
\PY{l+s+sd}{    beta1 \PYZhy{}\PYZhy{} Exponential decay hyperparameter for the past gradients estimates }
\PY{l+s+sd}{    beta2 \PYZhy{}\PYZhy{} Exponential decay hyperparameter for the past squared gradients estimates }
\PY{l+s+sd}{    epsilon \PYZhy{}\PYZhy{} hyperparameter preventing division by zero in Adam updates}
\PY{l+s+sd}{    num\PYZus{}epochs \PYZhy{}\PYZhy{} number of epochs}
\PY{l+s+sd}{    print\PYZus{}cost \PYZhy{}\PYZhy{} True to print the cost every 1000 epochs}

\PY{l+s+sd}{    Returns:}
\PY{l+s+sd}{    parameters \PYZhy{}\PYZhy{} python dictionary containing your updated parameters }
\PY{l+s+sd}{    \PYZdq{}\PYZdq{}\PYZdq{}}

    \PY{n}{L} \PY{o}{=} \PY{n+nb}{len}\PY{p}{(}\PY{n}{layers\PYZus{}dims}\PY{p}{)}             \PY{c+c1}{\PYZsh{} number of layers in the neural networks}
    \PY{n}{costs} \PY{o}{=} \PY{p}{[}\PY{p}{]}                       \PY{c+c1}{\PYZsh{} to keep track of the cost}
    \PY{n}{t} \PY{o}{=} \PY{l+m+mi}{0}                            \PY{c+c1}{\PYZsh{} initializing the counter required for Adam update}
    \PY{n}{seed} \PY{o}{=} \PY{l+m+mi}{10}                        \PY{c+c1}{\PYZsh{} For grading purposes, so that your \PYZdq{}random\PYZdq{} minibatches are the same as ours}
    \PY{n}{m} \PY{o}{=} \PY{n}{X}\PY{o}{.}\PY{n}{shape}\PY{p}{[}\PY{l+m+mi}{1}\PY{p}{]}                   \PY{c+c1}{\PYZsh{} number of training examples}
    
    \PY{c+c1}{\PYZsh{} Initialize parameters}
    \PY{n}{parameters} \PY{o}{=} \PY{n}{initialize\PYZus{}parameters}\PY{p}{(}\PY{n}{layers\PYZus{}dims}\PY{p}{)}

    \PY{c+c1}{\PYZsh{} Initialize the optimizer}
    \PY{k}{if} \PY{n}{optimizer} \PY{o}{==} \PY{l+s+s2}{\PYZdq{}}\PY{l+s+s2}{gd}\PY{l+s+s2}{\PYZdq{}}\PY{p}{:}
        \PY{k}{pass} \PY{c+c1}{\PYZsh{} no initialization required for gradient descent}
    \PY{k}{elif} \PY{n}{optimizer} \PY{o}{==} \PY{l+s+s2}{\PYZdq{}}\PY{l+s+s2}{momentum}\PY{l+s+s2}{\PYZdq{}}\PY{p}{:}
        \PY{n}{v} \PY{o}{=} \PY{n}{initialize\PYZus{}velocity}\PY{p}{(}\PY{n}{parameters}\PY{p}{)}
    \PY{k}{elif} \PY{n}{optimizer} \PY{o}{==} \PY{l+s+s2}{\PYZdq{}}\PY{l+s+s2}{adam}\PY{l+s+s2}{\PYZdq{}}\PY{p}{:}
        \PY{n}{v}\PY{p}{,} \PY{n}{s} \PY{o}{=} \PY{n}{initialize\PYZus{}adam}\PY{p}{(}\PY{n}{parameters}\PY{p}{)}
    
    \PY{c+c1}{\PYZsh{} Optimization loop}
    \PY{k}{for} \PY{n}{i} \PY{o+ow}{in} \PY{n+nb}{range}\PY{p}{(}\PY{n}{num\PYZus{}epochs}\PY{p}{)}\PY{p}{:}
        
        \PY{c+c1}{\PYZsh{} Define the random minibatches. We increment the seed to reshuffle differently the dataset after each epoch}
        \PY{n}{seed} \PY{o}{=} \PY{n}{seed} \PY{o}{+} \PY{l+m+mi}{1}
        \PY{n}{minibatches} \PY{o}{=} \PY{n}{random\PYZus{}mini\PYZus{}batches}\PY{p}{(}\PY{n}{X}\PY{p}{,} \PY{n}{Y}\PY{p}{,} \PY{n}{mini\PYZus{}batch\PYZus{}size}\PY{p}{,} \PY{n}{seed}\PY{p}{)}
        \PY{n}{cost\PYZus{}total} \PY{o}{=} \PY{l+m+mi}{0}
        
        \PY{k}{for} \PY{n}{minibatch} \PY{o+ow}{in} \PY{n}{minibatches}\PY{p}{:}

            \PY{c+c1}{\PYZsh{} Select a minibatch}
            \PY{p}{(}\PY{n}{minibatch\PYZus{}X}\PY{p}{,} \PY{n}{minibatch\PYZus{}Y}\PY{p}{)} \PY{o}{=} \PY{n}{minibatch}

            \PY{c+c1}{\PYZsh{} Forward propagation}
            \PY{n}{a3}\PY{p}{,} \PY{n}{caches} \PY{o}{=} \PY{n}{forward\PYZus{}propagation}\PY{p}{(}\PY{n}{minibatch\PYZus{}X}\PY{p}{,} \PY{n}{parameters}\PY{p}{)}

            \PY{c+c1}{\PYZsh{} Compute cost and add to the cost total}
            \PY{n}{cost\PYZus{}total} \PY{o}{+}\PY{o}{=} \PY{n}{compute\PYZus{}cost}\PY{p}{(}\PY{n}{a3}\PY{p}{,} \PY{n}{minibatch\PYZus{}Y}\PY{p}{)}

            \PY{c+c1}{\PYZsh{} Backward propagation}
            \PY{n}{grads} \PY{o}{=} \PY{n}{backward\PYZus{}propagation}\PY{p}{(}\PY{n}{minibatch\PYZus{}X}\PY{p}{,} \PY{n}{minibatch\PYZus{}Y}\PY{p}{,} \PY{n}{caches}\PY{p}{)}

            \PY{c+c1}{\PYZsh{} Update parameters}
            \PY{k}{if} \PY{n}{optimizer} \PY{o}{==} \PY{l+s+s2}{\PYZdq{}}\PY{l+s+s2}{gd}\PY{l+s+s2}{\PYZdq{}}\PY{p}{:}
                \PY{n}{parameters} \PY{o}{=} \PY{n}{update\PYZus{}parameters\PYZus{}with\PYZus{}gd}\PY{p}{(}\PY{n}{parameters}\PY{p}{,} \PY{n}{grads}\PY{p}{,} \PY{n}{learning\PYZus{}rate}\PY{p}{)}
            \PY{k}{elif} \PY{n}{optimizer} \PY{o}{==} \PY{l+s+s2}{\PYZdq{}}\PY{l+s+s2}{momentum}\PY{l+s+s2}{\PYZdq{}}\PY{p}{:}
                \PY{n}{parameters}\PY{p}{,} \PY{n}{v} \PY{o}{=} \PY{n}{update\PYZus{}parameters\PYZus{}with\PYZus{}momentum}\PY{p}{(}\PY{n}{parameters}\PY{p}{,} \PY{n}{grads}\PY{p}{,} \PY{n}{v}\PY{p}{,} \PY{n}{beta}\PY{p}{,} \PY{n}{learning\PYZus{}rate}\PY{p}{)}
            \PY{k}{elif} \PY{n}{optimizer} \PY{o}{==} \PY{l+s+s2}{\PYZdq{}}\PY{l+s+s2}{adam}\PY{l+s+s2}{\PYZdq{}}\PY{p}{:}
                \PY{n}{t} \PY{o}{=} \PY{n}{t} \PY{o}{+} \PY{l+m+mi}{1} \PY{c+c1}{\PYZsh{} Adam counter}
                \PY{n}{parameters}\PY{p}{,} \PY{n}{v}\PY{p}{,} \PY{n}{s}\PY{p}{,} \PY{n}{\PYZus{}}\PY{p}{,} \PY{n}{\PYZus{}} \PY{o}{=} \PY{n}{update\PYZus{}parameters\PYZus{}with\PYZus{}adam}\PY{p}{(}\PY{n}{parameters}\PY{p}{,} \PY{n}{grads}\PY{p}{,} \PY{n}{v}\PY{p}{,} \PY{n}{s}\PY{p}{,}
                                                               \PY{n}{t}\PY{p}{,} \PY{n}{learning\PYZus{}rate}\PY{p}{,} \PY{n}{beta1}\PY{p}{,} \PY{n}{beta2}\PY{p}{,}  \PY{n}{epsilon}\PY{p}{)}
        \PY{n}{cost\PYZus{}avg} \PY{o}{=} \PY{n}{cost\PYZus{}total} \PY{o}{/} \PY{n}{m}
        
        \PY{c+c1}{\PYZsh{} Print the cost every 1000 epoch}
        \PY{k}{if} \PY{n}{print\PYZus{}cost} \PY{o+ow}{and} \PY{n}{i} \PY{o}{\PYZpc{}} \PY{l+m+mi}{1000} \PY{o}{==} \PY{l+m+mi}{0}\PY{p}{:}
            \PY{n+nb}{print} \PY{p}{(}\PY{l+s+s2}{\PYZdq{}}\PY{l+s+s2}{Cost after epoch }\PY{l+s+si}{\PYZpc{}i}\PY{l+s+s2}{: }\PY{l+s+si}{\PYZpc{}f}\PY{l+s+s2}{\PYZdq{}} \PY{o}{\PYZpc{}}\PY{p}{(}\PY{n}{i}\PY{p}{,} \PY{n}{cost\PYZus{}avg}\PY{p}{)}\PY{p}{)}
        \PY{k}{if} \PY{n}{print\PYZus{}cost} \PY{o+ow}{and} \PY{n}{i} \PY{o}{\PYZpc{}} \PY{l+m+mi}{100} \PY{o}{==} \PY{l+m+mi}{0}\PY{p}{:}
            \PY{n}{costs}\PY{o}{.}\PY{n}{append}\PY{p}{(}\PY{n}{cost\PYZus{}avg}\PY{p}{)}
                
    \PY{c+c1}{\PYZsh{} plot the cost}
    \PY{n}{plt}\PY{o}{.}\PY{n}{plot}\PY{p}{(}\PY{n}{costs}\PY{p}{)}
    \PY{n}{plt}\PY{o}{.}\PY{n}{ylabel}\PY{p}{(}\PY{l+s+s1}{\PYZsq{}}\PY{l+s+s1}{cost}\PY{l+s+s1}{\PYZsq{}}\PY{p}{)}
    \PY{n}{plt}\PY{o}{.}\PY{n}{xlabel}\PY{p}{(}\PY{l+s+s1}{\PYZsq{}}\PY{l+s+s1}{epochs (per 100)}\PY{l+s+s1}{\PYZsq{}}\PY{p}{)}
    \PY{n}{plt}\PY{o}{.}\PY{n}{title}\PY{p}{(}\PY{l+s+s2}{\PYZdq{}}\PY{l+s+s2}{Learning rate = }\PY{l+s+s2}{\PYZdq{}} \PY{o}{+} \PY{n+nb}{str}\PY{p}{(}\PY{n}{learning\PYZus{}rate}\PY{p}{)}\PY{p}{)}
    \PY{n}{plt}\PY{o}{.}\PY{n}{show}\PY{p}{(}\PY{p}{)}

    \PY{k}{return} \PY{n}{parameters}
\end{Verbatim}
\end{tcolorbox}

    Now, run this 3 layer neural network with each of the 3 optimization
methods.

~\\
\#\#\# 6.1 - Mini-Batch Gradient Descent

Run the following code to see how the model does with mini-batch
gradient descent.

    \begin{tcolorbox}[breakable, size=fbox, boxrule=1pt, pad at break*=1mm,colback=cellbackground, colframe=cellborder]
\prompt{In}{incolor}{ }{\boxspacing}
\begin{Verbatim}[commandchars=\\\{\}]
\PY{c+c1}{\PYZsh{} train 3\PYZhy{}layer model}
\PY{n}{layers\PYZus{}dims} \PY{o}{=} \PY{p}{[}\PY{n}{train\PYZus{}X}\PY{o}{.}\PY{n}{shape}\PY{p}{[}\PY{l+m+mi}{0}\PY{p}{]}\PY{p}{,} \PY{l+m+mi}{5}\PY{p}{,} \PY{l+m+mi}{2}\PY{p}{,} \PY{l+m+mi}{1}\PY{p}{]}
\PY{n}{parameters} \PY{o}{=} \PY{n}{model}\PY{p}{(}\PY{n}{train\PYZus{}X}\PY{p}{,} \PY{n}{train\PYZus{}Y}\PY{p}{,} \PY{n}{layers\PYZus{}dims}\PY{p}{,} \PY{n}{optimizer} \PY{o}{=} \PY{l+s+s2}{\PYZdq{}}\PY{l+s+s2}{gd}\PY{l+s+s2}{\PYZdq{}}\PY{p}{)}

\PY{c+c1}{\PYZsh{} Predict}
\PY{n}{predictions} \PY{o}{=} \PY{n}{predict}\PY{p}{(}\PY{n}{train\PYZus{}X}\PY{p}{,} \PY{n}{train\PYZus{}Y}\PY{p}{,} \PY{n}{parameters}\PY{p}{)}

\PY{c+c1}{\PYZsh{} Plot decision boundary}
\PY{n}{plt}\PY{o}{.}\PY{n}{title}\PY{p}{(}\PY{l+s+s2}{\PYZdq{}}\PY{l+s+s2}{Model with Gradient Descent optimization}\PY{l+s+s2}{\PYZdq{}}\PY{p}{)}
\PY{n}{axes} \PY{o}{=} \PY{n}{plt}\PY{o}{.}\PY{n}{gca}\PY{p}{(}\PY{p}{)}
\PY{n}{axes}\PY{o}{.}\PY{n}{set\PYZus{}xlim}\PY{p}{(}\PY{p}{[}\PY{o}{\PYZhy{}}\PY{l+m+mf}{1.5}\PY{p}{,}\PY{l+m+mf}{2.5}\PY{p}{]}\PY{p}{)}
\PY{n}{axes}\PY{o}{.}\PY{n}{set\PYZus{}ylim}\PY{p}{(}\PY{p}{[}\PY{o}{\PYZhy{}}\PY{l+m+mi}{1}\PY{p}{,}\PY{l+m+mf}{1.5}\PY{p}{]}\PY{p}{)}
\PY{n}{plot\PYZus{}decision\PYZus{}boundary}\PY{p}{(}\PY{k}{lambda} \PY{n}{x}\PY{p}{:} \PY{n}{predict\PYZus{}dec}\PY{p}{(}\PY{n}{parameters}\PY{p}{,} \PY{n}{x}\PY{o}{.}\PY{n}{T}\PY{p}{)}\PY{p}{,} \PY{n}{train\PYZus{}X}\PY{p}{,} \PY{n}{train\PYZus{}Y}\PY{p}{)}
\end{Verbatim}
\end{tcolorbox}

    ~\\
\#\#\# 6.2 - Mini-Batch Gradient Descent with Momentum

Next, run the following code to see how the model does with momentum.
Because this example is relatively simple, the gains from using momemtum
are small - but for more complex problems you might see bigger gains.

    \begin{tcolorbox}[breakable, size=fbox, boxrule=1pt, pad at break*=1mm,colback=cellbackground, colframe=cellborder]
\prompt{In}{incolor}{ }{\boxspacing}
\begin{Verbatim}[commandchars=\\\{\}]
\PY{c+c1}{\PYZsh{} train 3\PYZhy{}layer model}
\PY{n}{layers\PYZus{}dims} \PY{o}{=} \PY{p}{[}\PY{n}{train\PYZus{}X}\PY{o}{.}\PY{n}{shape}\PY{p}{[}\PY{l+m+mi}{0}\PY{p}{]}\PY{p}{,} \PY{l+m+mi}{5}\PY{p}{,} \PY{l+m+mi}{2}\PY{p}{,} \PY{l+m+mi}{1}\PY{p}{]}
\PY{n}{parameters} \PY{o}{=} \PY{n}{model}\PY{p}{(}\PY{n}{train\PYZus{}X}\PY{p}{,} \PY{n}{train\PYZus{}Y}\PY{p}{,} \PY{n}{layers\PYZus{}dims}\PY{p}{,} \PY{n}{beta} \PY{o}{=} \PY{l+m+mf}{0.9}\PY{p}{,} \PY{n}{optimizer} \PY{o}{=} \PY{l+s+s2}{\PYZdq{}}\PY{l+s+s2}{momentum}\PY{l+s+s2}{\PYZdq{}}\PY{p}{)}

\PY{c+c1}{\PYZsh{} Predict}
\PY{n}{predictions} \PY{o}{=} \PY{n}{predict}\PY{p}{(}\PY{n}{train\PYZus{}X}\PY{p}{,} \PY{n}{train\PYZus{}Y}\PY{p}{,} \PY{n}{parameters}\PY{p}{)}

\PY{c+c1}{\PYZsh{} Plot decision boundary}
\PY{n}{plt}\PY{o}{.}\PY{n}{title}\PY{p}{(}\PY{l+s+s2}{\PYZdq{}}\PY{l+s+s2}{Model with Momentum optimization}\PY{l+s+s2}{\PYZdq{}}\PY{p}{)}
\PY{n}{axes} \PY{o}{=} \PY{n}{plt}\PY{o}{.}\PY{n}{gca}\PY{p}{(}\PY{p}{)}
\PY{n}{axes}\PY{o}{.}\PY{n}{set\PYZus{}xlim}\PY{p}{(}\PY{p}{[}\PY{o}{\PYZhy{}}\PY{l+m+mf}{1.5}\PY{p}{,}\PY{l+m+mf}{2.5}\PY{p}{]}\PY{p}{)}
\PY{n}{axes}\PY{o}{.}\PY{n}{set\PYZus{}ylim}\PY{p}{(}\PY{p}{[}\PY{o}{\PYZhy{}}\PY{l+m+mi}{1}\PY{p}{,}\PY{l+m+mf}{1.5}\PY{p}{]}\PY{p}{)}
\PY{n}{plot\PYZus{}decision\PYZus{}boundary}\PY{p}{(}\PY{k}{lambda} \PY{n}{x}\PY{p}{:} \PY{n}{predict\PYZus{}dec}\PY{p}{(}\PY{n}{parameters}\PY{p}{,} \PY{n}{x}\PY{o}{.}\PY{n}{T}\PY{p}{)}\PY{p}{,} \PY{n}{train\PYZus{}X}\PY{p}{,} \PY{n}{train\PYZus{}Y}\PY{p}{)}
\end{Verbatim}
\end{tcolorbox}

    ~\\
\#\#\# 6.3 - Mini-Batch with Adam

Finally, run the following code to see how the model does with Adam.

    \begin{tcolorbox}[breakable, size=fbox, boxrule=1pt, pad at break*=1mm,colback=cellbackground, colframe=cellborder]
\prompt{In}{incolor}{ }{\boxspacing}
\begin{Verbatim}[commandchars=\\\{\}]
\PY{c+c1}{\PYZsh{} train 3\PYZhy{}layer model}
\PY{n}{layers\PYZus{}dims} \PY{o}{=} \PY{p}{[}\PY{n}{train\PYZus{}X}\PY{o}{.}\PY{n}{shape}\PY{p}{[}\PY{l+m+mi}{0}\PY{p}{]}\PY{p}{,} \PY{l+m+mi}{5}\PY{p}{,} \PY{l+m+mi}{2}\PY{p}{,} \PY{l+m+mi}{1}\PY{p}{]}
\PY{n}{parameters} \PY{o}{=} \PY{n}{model}\PY{p}{(}\PY{n}{train\PYZus{}X}\PY{p}{,} \PY{n}{train\PYZus{}Y}\PY{p}{,} \PY{n}{layers\PYZus{}dims}\PY{p}{,} \PY{n}{optimizer} \PY{o}{=} \PY{l+s+s2}{\PYZdq{}}\PY{l+s+s2}{adam}\PY{l+s+s2}{\PYZdq{}}\PY{p}{)}

\PY{c+c1}{\PYZsh{} Predict}
\PY{n}{predictions} \PY{o}{=} \PY{n}{predict}\PY{p}{(}\PY{n}{train\PYZus{}X}\PY{p}{,} \PY{n}{train\PYZus{}Y}\PY{p}{,} \PY{n}{parameters}\PY{p}{)}

\PY{c+c1}{\PYZsh{} Plot decision boundary}
\PY{n}{plt}\PY{o}{.}\PY{n}{title}\PY{p}{(}\PY{l+s+s2}{\PYZdq{}}\PY{l+s+s2}{Model with Adam optimization}\PY{l+s+s2}{\PYZdq{}}\PY{p}{)}
\PY{n}{axes} \PY{o}{=} \PY{n}{plt}\PY{o}{.}\PY{n}{gca}\PY{p}{(}\PY{p}{)}
\PY{n}{axes}\PY{o}{.}\PY{n}{set\PYZus{}xlim}\PY{p}{(}\PY{p}{[}\PY{o}{\PYZhy{}}\PY{l+m+mf}{1.5}\PY{p}{,}\PY{l+m+mf}{2.5}\PY{p}{]}\PY{p}{)}
\PY{n}{axes}\PY{o}{.}\PY{n}{set\PYZus{}ylim}\PY{p}{(}\PY{p}{[}\PY{o}{\PYZhy{}}\PY{l+m+mi}{1}\PY{p}{,}\PY{l+m+mf}{1.5}\PY{p}{]}\PY{p}{)}
\PY{n}{plot\PYZus{}decision\PYZus{}boundary}\PY{p}{(}\PY{k}{lambda} \PY{n}{x}\PY{p}{:} \PY{n}{predict\PYZus{}dec}\PY{p}{(}\PY{n}{parameters}\PY{p}{,} \PY{n}{x}\PY{o}{.}\PY{n}{T}\PY{p}{)}\PY{p}{,} \PY{n}{train\PYZus{}X}\PY{p}{,} \PY{n}{train\PYZus{}Y}\PY{p}{)}
\end{Verbatim}
\end{tcolorbox}

    ~\\
\#\#\# 6.4 - Summary

optimization method

accuracy

cost shape

\begin{verbatim}
<td>
    Gradient descent
    </td>
    <td>
    >71%
    </td>
    <td>
    smooth
    </td>
<tr>
    <td>
    Momentum
    </td>
    <td>
    >71%
    </td>
    <td>
    smooth
    </td>
</tr>
<tr>
    <td>
    Adam
    </td>
    <td>
    >94%
    </td>
    <td>
    smoother
    </td>
</tr>
\end{verbatim}

Momentum usually helps, but given the small learning rate and the
simplistic dataset, its impact is almost negligible.

On the other hand, Adam clearly outperforms mini-batch gradient descent
and Momentum. If you run the model for more epochs on this simple
dataset, all three methods will lead to very good results. However,
you've seen that Adam converges a lot faster.

Some advantages of Adam include:

\begin{itemize}
\tightlist
\item
  Relatively low memory requirements (though higher than gradient
  descent and gradient descent with momentum)
\item
  Usually works well even with little tuning of hyperparameters (except
  \(\alpha\))
\end{itemize}

    \textbf{References}:

\begin{itemize}
\tightlist
\item
  Adam paper: https://arxiv.org/pdf/1412.6980.pdf
\end{itemize}

    ~\\
\#\# 7 - Learning Rate Decay and Scheduling

Lastly, the learning rate is another hyperparameter that can help you
speed up learning.

During the first part of training, your model can get away with taking
large steps, but over time, using a fixed value for the learning rate
alpha can cause your model to get stuck in a wide oscillation that never
quite converges. But if you were to slowly reduce your learning rate
alpha over time, you could then take smaller, slower steps that bring
you closer to the minimum. This is the idea behind learning rate decay.

Learning rate decay can be achieved by using either adaptive methods or
pre-defined learning rate schedules.

Now, you'll apply scheduled learning rate decay to a 3-layer neural
network in three different optimizer modes and see how each one differs,
as well as the effect of scheduling at different epochs.

This model is essentially the same as the one you used before, except in
this one you'll be able to include learning rate decay. It includes two
new parameters, decay and decay\_rate.

    \begin{tcolorbox}[breakable, size=fbox, boxrule=1pt, pad at break*=1mm,colback=cellbackground, colframe=cellborder]
\prompt{In}{incolor}{ }{\boxspacing}
\begin{Verbatim}[commandchars=\\\{\}]
\PY{k}{def} \PY{n+nf}{model}\PY{p}{(}\PY{n}{X}\PY{p}{,} \PY{n}{Y}\PY{p}{,} \PY{n}{layers\PYZus{}dims}\PY{p}{,} \PY{n}{optimizer}\PY{p}{,} \PY{n}{learning\PYZus{}rate} \PY{o}{=} \PY{l+m+mf}{0.0007}\PY{p}{,} \PY{n}{mini\PYZus{}batch\PYZus{}size} \PY{o}{=} \PY{l+m+mi}{64}\PY{p}{,} \PY{n}{beta} \PY{o}{=} \PY{l+m+mf}{0.9}\PY{p}{,}
          \PY{n}{beta1} \PY{o}{=} \PY{l+m+mf}{0.9}\PY{p}{,} \PY{n}{beta2} \PY{o}{=} \PY{l+m+mf}{0.999}\PY{p}{,}  \PY{n}{epsilon} \PY{o}{=} \PY{l+m+mf}{1e\PYZhy{}8}\PY{p}{,} \PY{n}{num\PYZus{}epochs} \PY{o}{=} \PY{l+m+mi}{5000}\PY{p}{,} \PY{n}{print\PYZus{}cost} \PY{o}{=} \PY{k+kc}{True}\PY{p}{,} \PY{n}{decay}\PY{o}{=}\PY{k+kc}{None}\PY{p}{,} \PY{n}{decay\PYZus{}rate}\PY{o}{=}\PY{l+m+mi}{1}\PY{p}{)}\PY{p}{:}
    \PY{l+s+sd}{\PYZdq{}\PYZdq{}\PYZdq{}}
\PY{l+s+sd}{    3\PYZhy{}layer neural network model which can be run in different optimizer modes.}
\PY{l+s+sd}{    }
\PY{l+s+sd}{    Arguments:}
\PY{l+s+sd}{    X \PYZhy{}\PYZhy{} input data, of shape (2, number of examples)}
\PY{l+s+sd}{    Y \PYZhy{}\PYZhy{} true \PYZdq{}label\PYZdq{} vector (1 for blue dot / 0 for red dot), of shape (1, number of examples)}
\PY{l+s+sd}{    layers\PYZus{}dims \PYZhy{}\PYZhy{} python list, containing the size of each layer}
\PY{l+s+sd}{    learning\PYZus{}rate \PYZhy{}\PYZhy{} the learning rate, scalar.}
\PY{l+s+sd}{    mini\PYZus{}batch\PYZus{}size \PYZhy{}\PYZhy{} the size of a mini batch}
\PY{l+s+sd}{    beta \PYZhy{}\PYZhy{} Momentum hyperparameter}
\PY{l+s+sd}{    beta1 \PYZhy{}\PYZhy{} Exponential decay hyperparameter for the past gradients estimates }
\PY{l+s+sd}{    beta2 \PYZhy{}\PYZhy{} Exponential decay hyperparameter for the past squared gradients estimates }
\PY{l+s+sd}{    epsilon \PYZhy{}\PYZhy{} hyperparameter preventing division by zero in Adam updates}
\PY{l+s+sd}{    num\PYZus{}epochs \PYZhy{}\PYZhy{} number of epochs}
\PY{l+s+sd}{    print\PYZus{}cost \PYZhy{}\PYZhy{} True to print the cost every 1000 epochs}

\PY{l+s+sd}{    Returns:}
\PY{l+s+sd}{    parameters \PYZhy{}\PYZhy{} python dictionary containing your updated parameters }
\PY{l+s+sd}{    \PYZdq{}\PYZdq{}\PYZdq{}}

    \PY{n}{L} \PY{o}{=} \PY{n+nb}{len}\PY{p}{(}\PY{n}{layers\PYZus{}dims}\PY{p}{)}             \PY{c+c1}{\PYZsh{} number of layers in the neural networks}
    \PY{n}{costs} \PY{o}{=} \PY{p}{[}\PY{p}{]}                       \PY{c+c1}{\PYZsh{} to keep track of the cost}
    \PY{n}{t} \PY{o}{=} \PY{l+m+mi}{0}                            \PY{c+c1}{\PYZsh{} initializing the counter required for Adam update}
    \PY{n}{seed} \PY{o}{=} \PY{l+m+mi}{10}                        \PY{c+c1}{\PYZsh{} For grading purposes, so that your \PYZdq{}random\PYZdq{} minibatches are the same as ours}
    \PY{n}{m} \PY{o}{=} \PY{n}{X}\PY{o}{.}\PY{n}{shape}\PY{p}{[}\PY{l+m+mi}{1}\PY{p}{]}                   \PY{c+c1}{\PYZsh{} number of training examples}
    \PY{n}{lr\PYZus{}rates} \PY{o}{=} \PY{p}{[}\PY{p}{]}
    \PY{n}{learning\PYZus{}rate0} \PY{o}{=} \PY{n}{learning\PYZus{}rate}   \PY{c+c1}{\PYZsh{} the original learning rate}
    
    \PY{c+c1}{\PYZsh{} Initialize parameters}
    \PY{n}{parameters} \PY{o}{=} \PY{n}{initialize\PYZus{}parameters}\PY{p}{(}\PY{n}{layers\PYZus{}dims}\PY{p}{)}

    \PY{c+c1}{\PYZsh{} Initialize the optimizer}
    \PY{k}{if} \PY{n}{optimizer} \PY{o}{==} \PY{l+s+s2}{\PYZdq{}}\PY{l+s+s2}{gd}\PY{l+s+s2}{\PYZdq{}}\PY{p}{:}
        \PY{k}{pass} \PY{c+c1}{\PYZsh{} no initialization required for gradient descent}
    \PY{k}{elif} \PY{n}{optimizer} \PY{o}{==} \PY{l+s+s2}{\PYZdq{}}\PY{l+s+s2}{momentum}\PY{l+s+s2}{\PYZdq{}}\PY{p}{:}
        \PY{n}{v} \PY{o}{=} \PY{n}{initialize\PYZus{}velocity}\PY{p}{(}\PY{n}{parameters}\PY{p}{)}
    \PY{k}{elif} \PY{n}{optimizer} \PY{o}{==} \PY{l+s+s2}{\PYZdq{}}\PY{l+s+s2}{adam}\PY{l+s+s2}{\PYZdq{}}\PY{p}{:}
        \PY{n}{v}\PY{p}{,} \PY{n}{s} \PY{o}{=} \PY{n}{initialize\PYZus{}adam}\PY{p}{(}\PY{n}{parameters}\PY{p}{)}
    
    \PY{c+c1}{\PYZsh{} Optimization loop}
    \PY{k}{for} \PY{n}{i} \PY{o+ow}{in} \PY{n+nb}{range}\PY{p}{(}\PY{n}{num\PYZus{}epochs}\PY{p}{)}\PY{p}{:}
        
        \PY{c+c1}{\PYZsh{} Define the random minibatches. We increment the seed to reshuffle differently the dataset after each epoch}
        \PY{n}{seed} \PY{o}{=} \PY{n}{seed} \PY{o}{+} \PY{l+m+mi}{1}
        \PY{n}{minibatches} \PY{o}{=} \PY{n}{random\PYZus{}mini\PYZus{}batches}\PY{p}{(}\PY{n}{X}\PY{p}{,} \PY{n}{Y}\PY{p}{,} \PY{n}{mini\PYZus{}batch\PYZus{}size}\PY{p}{,} \PY{n}{seed}\PY{p}{)}
        \PY{n}{cost\PYZus{}total} \PY{o}{=} \PY{l+m+mi}{0}
        
        \PY{k}{for} \PY{n}{minibatch} \PY{o+ow}{in} \PY{n}{minibatches}\PY{p}{:}

            \PY{c+c1}{\PYZsh{} Select a minibatch}
            \PY{p}{(}\PY{n}{minibatch\PYZus{}X}\PY{p}{,} \PY{n}{minibatch\PYZus{}Y}\PY{p}{)} \PY{o}{=} \PY{n}{minibatch}

            \PY{c+c1}{\PYZsh{} Forward propagation}
            \PY{n}{a3}\PY{p}{,} \PY{n}{caches} \PY{o}{=} \PY{n}{forward\PYZus{}propagation}\PY{p}{(}\PY{n}{minibatch\PYZus{}X}\PY{p}{,} \PY{n}{parameters}\PY{p}{)}

            \PY{c+c1}{\PYZsh{} Compute cost and add to the cost total}
            \PY{n}{cost\PYZus{}total} \PY{o}{+}\PY{o}{=} \PY{n}{compute\PYZus{}cost}\PY{p}{(}\PY{n}{a3}\PY{p}{,} \PY{n}{minibatch\PYZus{}Y}\PY{p}{)}

            \PY{c+c1}{\PYZsh{} Backward propagation}
            \PY{n}{grads} \PY{o}{=} \PY{n}{backward\PYZus{}propagation}\PY{p}{(}\PY{n}{minibatch\PYZus{}X}\PY{p}{,} \PY{n}{minibatch\PYZus{}Y}\PY{p}{,} \PY{n}{caches}\PY{p}{)}

            \PY{c+c1}{\PYZsh{} Update parameters}
            \PY{k}{if} \PY{n}{optimizer} \PY{o}{==} \PY{l+s+s2}{\PYZdq{}}\PY{l+s+s2}{gd}\PY{l+s+s2}{\PYZdq{}}\PY{p}{:}
                \PY{n}{parameters} \PY{o}{=} \PY{n}{update\PYZus{}parameters\PYZus{}with\PYZus{}gd}\PY{p}{(}\PY{n}{parameters}\PY{p}{,} \PY{n}{grads}\PY{p}{,} \PY{n}{learning\PYZus{}rate}\PY{p}{)}
            \PY{k}{elif} \PY{n}{optimizer} \PY{o}{==} \PY{l+s+s2}{\PYZdq{}}\PY{l+s+s2}{momentum}\PY{l+s+s2}{\PYZdq{}}\PY{p}{:}
                \PY{n}{parameters}\PY{p}{,} \PY{n}{v} \PY{o}{=} \PY{n}{update\PYZus{}parameters\PYZus{}with\PYZus{}momentum}\PY{p}{(}\PY{n}{parameters}\PY{p}{,} \PY{n}{grads}\PY{p}{,} \PY{n}{v}\PY{p}{,} \PY{n}{beta}\PY{p}{,} \PY{n}{learning\PYZus{}rate}\PY{p}{)}
            \PY{k}{elif} \PY{n}{optimizer} \PY{o}{==} \PY{l+s+s2}{\PYZdq{}}\PY{l+s+s2}{adam}\PY{l+s+s2}{\PYZdq{}}\PY{p}{:}
                \PY{n}{t} \PY{o}{=} \PY{n}{t} \PY{o}{+} \PY{l+m+mi}{1} \PY{c+c1}{\PYZsh{} Adam counter}
                \PY{n}{parameters}\PY{p}{,} \PY{n}{v}\PY{p}{,} \PY{n}{s}\PY{p}{,} \PY{n}{\PYZus{}}\PY{p}{,} \PY{n}{\PYZus{}} \PY{o}{=} \PY{n}{update\PYZus{}parameters\PYZus{}with\PYZus{}adam}\PY{p}{(}\PY{n}{parameters}\PY{p}{,} \PY{n}{grads}\PY{p}{,} \PY{n}{v}\PY{p}{,} \PY{n}{s}\PY{p}{,}
                                                               \PY{n}{t}\PY{p}{,} \PY{n}{learning\PYZus{}rate}\PY{p}{,} \PY{n}{beta1}\PY{p}{,} \PY{n}{beta2}\PY{p}{,}  \PY{n}{epsilon}\PY{p}{)}
        \PY{n}{cost\PYZus{}avg} \PY{o}{=} \PY{n}{cost\PYZus{}total} \PY{o}{/} \PY{n}{m}
        \PY{k}{if} \PY{n}{decay}\PY{p}{:}
            \PY{n}{learning\PYZus{}rate} \PY{o}{=} \PY{n}{decay}\PY{p}{(}\PY{n}{learning\PYZus{}rate0}\PY{p}{,} \PY{n}{i}\PY{p}{,} \PY{n}{decay\PYZus{}rate}\PY{p}{)}
        \PY{c+c1}{\PYZsh{} Print the cost every 1000 epoch}
        \PY{k}{if} \PY{n}{print\PYZus{}cost} \PY{o+ow}{and} \PY{n}{i} \PY{o}{\PYZpc{}} \PY{l+m+mi}{1000} \PY{o}{==} \PY{l+m+mi}{0}\PY{p}{:}
            \PY{n+nb}{print} \PY{p}{(}\PY{l+s+s2}{\PYZdq{}}\PY{l+s+s2}{Cost after epoch }\PY{l+s+si}{\PYZpc{}i}\PY{l+s+s2}{: }\PY{l+s+si}{\PYZpc{}f}\PY{l+s+s2}{\PYZdq{}} \PY{o}{\PYZpc{}}\PY{p}{(}\PY{n}{i}\PY{p}{,} \PY{n}{cost\PYZus{}avg}\PY{p}{)}\PY{p}{)}
            \PY{k}{if} \PY{n}{decay}\PY{p}{:}
                \PY{n+nb}{print}\PY{p}{(}\PY{l+s+s2}{\PYZdq{}}\PY{l+s+s2}{learning rate after epoch }\PY{l+s+si}{\PYZpc{}i}\PY{l+s+s2}{: }\PY{l+s+si}{\PYZpc{}f}\PY{l+s+s2}{\PYZdq{}}\PY{o}{\PYZpc{}}\PY{p}{(}\PY{n}{i}\PY{p}{,} \PY{n}{learning\PYZus{}rate}\PY{p}{)}\PY{p}{)}
        \PY{k}{if} \PY{n}{print\PYZus{}cost} \PY{o+ow}{and} \PY{n}{i} \PY{o}{\PYZpc{}} \PY{l+m+mi}{100} \PY{o}{==} \PY{l+m+mi}{0}\PY{p}{:}
            \PY{n}{costs}\PY{o}{.}\PY{n}{append}\PY{p}{(}\PY{n}{cost\PYZus{}avg}\PY{p}{)}
                
    \PY{c+c1}{\PYZsh{} plot the cost}
    \PY{n}{plt}\PY{o}{.}\PY{n}{plot}\PY{p}{(}\PY{n}{costs}\PY{p}{)}
    \PY{n}{plt}\PY{o}{.}\PY{n}{ylabel}\PY{p}{(}\PY{l+s+s1}{\PYZsq{}}\PY{l+s+s1}{cost}\PY{l+s+s1}{\PYZsq{}}\PY{p}{)}
    \PY{n}{plt}\PY{o}{.}\PY{n}{xlabel}\PY{p}{(}\PY{l+s+s1}{\PYZsq{}}\PY{l+s+s1}{epochs (per 100)}\PY{l+s+s1}{\PYZsq{}}\PY{p}{)}
    \PY{n}{plt}\PY{o}{.}\PY{n}{title}\PY{p}{(}\PY{l+s+s2}{\PYZdq{}}\PY{l+s+s2}{Learning rate = }\PY{l+s+s2}{\PYZdq{}} \PY{o}{+} \PY{n+nb}{str}\PY{p}{(}\PY{n}{learning\PYZus{}rate}\PY{p}{)}\PY{p}{)}
    \PY{n}{plt}\PY{o}{.}\PY{n}{show}\PY{p}{(}\PY{p}{)}

    \PY{k}{return} \PY{n}{parameters}
\end{Verbatim}
\end{tcolorbox}

    ~\\
\#\#\# 7.1 - Decay on every iteration

For this portion of the assignment, you'll try one of the pre-defined
schedules for learning rate decay, called exponential learning rate
decay. It takes this mathematical form:

\[\alpha = \frac{1}{1 + decayRate \times epochNumber} \alpha_{0}\]

~\\
\#\#\# Exercise 7 - update\_lr

Calculate the new learning rate using exponential weight decay.

    \begin{tcolorbox}[breakable, size=fbox, boxrule=1pt, pad at break*=1mm,colback=cellbackground, colframe=cellborder]
\prompt{In}{incolor}{ }{\boxspacing}
\begin{Verbatim}[commandchars=\\\{\}]
\PY{c+c1}{\PYZsh{} GRADED FUNCTION: update\PYZus{}lr}

\PY{k}{def} \PY{n+nf}{update\PYZus{}lr}\PY{p}{(}\PY{n}{learning\PYZus{}rate0}\PY{p}{,} \PY{n}{epoch\PYZus{}num}\PY{p}{,} \PY{n}{decay\PYZus{}rate}\PY{p}{)}\PY{p}{:}
    \PY{l+s+sd}{\PYZdq{}\PYZdq{}\PYZdq{}}
\PY{l+s+sd}{    Calculates updated the learning rate using exponential weight decay.}
\PY{l+s+sd}{    }
\PY{l+s+sd}{    Arguments:}
\PY{l+s+sd}{    learning\PYZus{}rate0 \PYZhy{}\PYZhy{} Original learning rate. Scalar}
\PY{l+s+sd}{    epoch\PYZus{}num \PYZhy{}\PYZhy{} Epoch number. Integer}
\PY{l+s+sd}{    decay\PYZus{}rate \PYZhy{}\PYZhy{} Decay rate. Scalar}

\PY{l+s+sd}{    Returns:}
\PY{l+s+sd}{    learning\PYZus{}rate \PYZhy{}\PYZhy{} Updated learning rate. Scalar }
\PY{l+s+sd}{    \PYZdq{}\PYZdq{}\PYZdq{}}
    \PY{c+c1}{\PYZsh{}(approx. 1 line)}
    \PY{c+c1}{\PYZsh{} learning\PYZus{}rate = }
    \PY{c+c1}{\PYZsh{} YOUR CODE STARTS HERE}
    
    
    \PY{c+c1}{\PYZsh{} YOUR CODE ENDS HERE}
    \PY{k}{return} \PY{n}{learning\PYZus{}rate}
\end{Verbatim}
\end{tcolorbox}

    \begin{tcolorbox}[breakable, size=fbox, boxrule=1pt, pad at break*=1mm,colback=cellbackground, colframe=cellborder]
\prompt{In}{incolor}{ }{\boxspacing}
\begin{Verbatim}[commandchars=\\\{\}]
\PY{n}{learning\PYZus{}rate} \PY{o}{=} \PY{l+m+mf}{0.5}
\PY{n+nb}{print}\PY{p}{(}\PY{l+s+s2}{\PYZdq{}}\PY{l+s+s2}{Original learning rate: }\PY{l+s+s2}{\PYZdq{}}\PY{p}{,} \PY{n}{learning\PYZus{}rate}\PY{p}{)}
\PY{n}{epoch\PYZus{}num} \PY{o}{=} \PY{l+m+mi}{2}
\PY{n}{decay\PYZus{}rate} \PY{o}{=} \PY{l+m+mi}{1}
\PY{n}{learning\PYZus{}rate\PYZus{}2} \PY{o}{=} \PY{n}{update\PYZus{}lr}\PY{p}{(}\PY{n}{learning\PYZus{}rate}\PY{p}{,} \PY{n}{epoch\PYZus{}num}\PY{p}{,} \PY{n}{decay\PYZus{}rate}\PY{p}{)}

\PY{n+nb}{print}\PY{p}{(}\PY{l+s+s2}{\PYZdq{}}\PY{l+s+s2}{Updated learning rate: }\PY{l+s+s2}{\PYZdq{}}\PY{p}{,} \PY{n}{learning\PYZus{}rate\PYZus{}2}\PY{p}{)}

\PY{n}{update\PYZus{}lr\PYZus{}test}\PY{p}{(}\PY{n}{update\PYZus{}lr}\PY{p}{)}
\end{Verbatim}
\end{tcolorbox}

    \begin{tcolorbox}[breakable, size=fbox, boxrule=1pt, pad at break*=1mm,colback=cellbackground, colframe=cellborder]
\prompt{In}{incolor}{ }{\boxspacing}
\begin{Verbatim}[commandchars=\\\{\}]
\PY{c+c1}{\PYZsh{} train 3\PYZhy{}layer model}
\PY{n}{layers\PYZus{}dims} \PY{o}{=} \PY{p}{[}\PY{n}{train\PYZus{}X}\PY{o}{.}\PY{n}{shape}\PY{p}{[}\PY{l+m+mi}{0}\PY{p}{]}\PY{p}{,} \PY{l+m+mi}{5}\PY{p}{,} \PY{l+m+mi}{2}\PY{p}{,} \PY{l+m+mi}{1}\PY{p}{]}
\PY{n}{parameters} \PY{o}{=} \PY{n}{model}\PY{p}{(}\PY{n}{train\PYZus{}X}\PY{p}{,} \PY{n}{train\PYZus{}Y}\PY{p}{,} \PY{n}{layers\PYZus{}dims}\PY{p}{,} \PY{n}{optimizer} \PY{o}{=} \PY{l+s+s2}{\PYZdq{}}\PY{l+s+s2}{gd}\PY{l+s+s2}{\PYZdq{}}\PY{p}{,} \PY{n}{learning\PYZus{}rate} \PY{o}{=} \PY{l+m+mf}{0.1}\PY{p}{,} \PY{n}{num\PYZus{}epochs}\PY{o}{=}\PY{l+m+mi}{5000}\PY{p}{,} \PY{n}{decay}\PY{o}{=}\PY{n}{update\PYZus{}lr}\PY{p}{)}

\PY{c+c1}{\PYZsh{} Predict}
\PY{n}{predictions} \PY{o}{=} \PY{n}{predict}\PY{p}{(}\PY{n}{train\PYZus{}X}\PY{p}{,} \PY{n}{train\PYZus{}Y}\PY{p}{,} \PY{n}{parameters}\PY{p}{)}

\PY{c+c1}{\PYZsh{} Plot decision boundary}
\PY{n}{plt}\PY{o}{.}\PY{n}{title}\PY{p}{(}\PY{l+s+s2}{\PYZdq{}}\PY{l+s+s2}{Model with Gradient Descent optimization}\PY{l+s+s2}{\PYZdq{}}\PY{p}{)}
\PY{n}{axes} \PY{o}{=} \PY{n}{plt}\PY{o}{.}\PY{n}{gca}\PY{p}{(}\PY{p}{)}
\PY{n}{axes}\PY{o}{.}\PY{n}{set\PYZus{}xlim}\PY{p}{(}\PY{p}{[}\PY{o}{\PYZhy{}}\PY{l+m+mf}{1.5}\PY{p}{,}\PY{l+m+mf}{2.5}\PY{p}{]}\PY{p}{)}
\PY{n}{axes}\PY{o}{.}\PY{n}{set\PYZus{}ylim}\PY{p}{(}\PY{p}{[}\PY{o}{\PYZhy{}}\PY{l+m+mi}{1}\PY{p}{,}\PY{l+m+mf}{1.5}\PY{p}{]}\PY{p}{)}
\PY{n}{plot\PYZus{}decision\PYZus{}boundary}\PY{p}{(}\PY{k}{lambda} \PY{n}{x}\PY{p}{:} \PY{n}{predict\PYZus{}dec}\PY{p}{(}\PY{n}{parameters}\PY{p}{,} \PY{n}{x}\PY{o}{.}\PY{n}{T}\PY{p}{)}\PY{p}{,} \PY{n}{train\PYZus{}X}\PY{p}{,} \PY{n}{train\PYZus{}Y}\PY{p}{)}
\end{Verbatim}
\end{tcolorbox}

    Notice that if you set the decay to occur at every iteration, the
learning rate goes to zero too quickly - even if you start with a higher
learning rate.

Epoch Number

Learning Rate

Cost

0

0.100000

0.701091

1000

0.000100

0.661884

2000

0.000050

0.658620

3000

0.000033

0.656765

4000

0.000025

0.655486

5000

0.000020

0.654514

When you're training for a few epoch this doesn't cause a lot of
troubles, but when the number of epochs is large the optimization
algorithm will stop updating. One common fix to this issue is to decay
the learning rate every few steps. This is called fixed interval
scheduling.

    \#\#\# 7.2 - Fixed Interval Scheduling

You can help prevent the learning rate speeding to zero too quickly by
scheduling the exponential learning rate decay at a fixed time interval,
for example 1000. You can either number the intervals, or divide the
epoch by the time interval, which is the size of window with the
constant learning rate.

    \#\#\# Exercise 8 - schedule\_lr\_decay

Calculate the new learning rate using exponential weight decay with
fixed interval scheduling.

\textbf{Instructions}: Implement the learning rate scheduling such that
it only changes when the epochNum is a multiple of the timeInterval.

\textbf{Note:} The fraction in the denominator uses the floor operation.

\[\alpha = \frac{1}{1 + decayRate \times \lfloor\frac{epochNum}{timeInterval}\rfloor} \alpha_{0}\]

\textbf{Hint:}
\href{https://numpy.org/doc/stable/reference/generated/numpy.floor.html}{numpy.floor}

    \begin{tcolorbox}[breakable, size=fbox, boxrule=1pt, pad at break*=1mm,colback=cellbackground, colframe=cellborder]
\prompt{In}{incolor}{ }{\boxspacing}
\begin{Verbatim}[commandchars=\\\{\}]
\PY{c+c1}{\PYZsh{} GRADED FUNCTION: schedule\PYZus{}lr\PYZus{}decay}

\PY{k}{def} \PY{n+nf}{schedule\PYZus{}lr\PYZus{}decay}\PY{p}{(}\PY{n}{learning\PYZus{}rate0}\PY{p}{,} \PY{n}{epoch\PYZus{}num}\PY{p}{,} \PY{n}{decay\PYZus{}rate}\PY{p}{,} \PY{n}{time\PYZus{}interval}\PY{o}{=}\PY{l+m+mi}{1000}\PY{p}{)}\PY{p}{:}
    \PY{l+s+sd}{\PYZdq{}\PYZdq{}\PYZdq{}}
\PY{l+s+sd}{    Calculates updated the learning rate using exponential weight decay.}
\PY{l+s+sd}{    }
\PY{l+s+sd}{    Arguments:}
\PY{l+s+sd}{    learning\PYZus{}rate0 \PYZhy{}\PYZhy{} Original learning rate. Scalar}
\PY{l+s+sd}{    epoch\PYZus{}num \PYZhy{}\PYZhy{} Epoch number. Integer.}
\PY{l+s+sd}{    decay\PYZus{}rate \PYZhy{}\PYZhy{} Decay rate. Scalar.}
\PY{l+s+sd}{    time\PYZus{}interval \PYZhy{}\PYZhy{} Number of epochs where you update the learning rate.}

\PY{l+s+sd}{    Returns:}
\PY{l+s+sd}{    learning\PYZus{}rate \PYZhy{}\PYZhy{} Updated learning rate. Scalar }
\PY{l+s+sd}{    \PYZdq{}\PYZdq{}\PYZdq{}}
    \PY{c+c1}{\PYZsh{} (approx. 1 lines)}
    \PY{c+c1}{\PYZsh{} learning\PYZus{}rate = ...}
    \PY{c+c1}{\PYZsh{} YOUR CODE STARTS HERE}
    
    
    \PY{c+c1}{\PYZsh{} YOUR CODE ENDS HERE}
    \PY{k}{return} \PY{n}{learning\PYZus{}rate}
\end{Verbatim}
\end{tcolorbox}

    \begin{tcolorbox}[breakable, size=fbox, boxrule=1pt, pad at break*=1mm,colback=cellbackground, colframe=cellborder]
\prompt{In}{incolor}{ }{\boxspacing}
\begin{Verbatim}[commandchars=\\\{\}]
\PY{n}{learning\PYZus{}rate} \PY{o}{=} \PY{l+m+mf}{0.5}
\PY{n+nb}{print}\PY{p}{(}\PY{l+s+s2}{\PYZdq{}}\PY{l+s+s2}{Original learning rate: }\PY{l+s+s2}{\PYZdq{}}\PY{p}{,} \PY{n}{learning\PYZus{}rate}\PY{p}{)}

\PY{n}{epoch\PYZus{}num\PYZus{}1} \PY{o}{=} \PY{l+m+mi}{10}
\PY{n}{epoch\PYZus{}num\PYZus{}2} \PY{o}{=} \PY{l+m+mi}{100}
\PY{n}{decay\PYZus{}rate} \PY{o}{=} \PY{l+m+mf}{0.3}
\PY{n}{time\PYZus{}interval} \PY{o}{=} \PY{l+m+mi}{100}
\PY{n}{learning\PYZus{}rate\PYZus{}1} \PY{o}{=} \PY{n}{schedule\PYZus{}lr\PYZus{}decay}\PY{p}{(}\PY{n}{learning\PYZus{}rate}\PY{p}{,} \PY{n}{epoch\PYZus{}num\PYZus{}1}\PY{p}{,} \PY{n}{decay\PYZus{}rate}\PY{p}{,} \PY{n}{time\PYZus{}interval}\PY{p}{)}
\PY{n}{learning\PYZus{}rate\PYZus{}2} \PY{o}{=} \PY{n}{schedule\PYZus{}lr\PYZus{}decay}\PY{p}{(}\PY{n}{learning\PYZus{}rate}\PY{p}{,} \PY{n}{epoch\PYZus{}num\PYZus{}2}\PY{p}{,} \PY{n}{decay\PYZus{}rate}\PY{p}{,} \PY{n}{time\PYZus{}interval}\PY{p}{)}
\PY{n+nb}{print}\PY{p}{(}\PY{l+s+s2}{\PYZdq{}}\PY{l+s+s2}{Updated learning rate after }\PY{l+s+si}{\PYZob{}\PYZcb{}}\PY{l+s+s2}{ epochs: }\PY{l+s+s2}{\PYZdq{}}\PY{o}{.}\PY{n}{format}\PY{p}{(}\PY{n}{epoch\PYZus{}num\PYZus{}1}\PY{p}{)}\PY{p}{,} \PY{n}{learning\PYZus{}rate\PYZus{}1}\PY{p}{)}
\PY{n+nb}{print}\PY{p}{(}\PY{l+s+s2}{\PYZdq{}}\PY{l+s+s2}{Updated learning rate after }\PY{l+s+si}{\PYZob{}\PYZcb{}}\PY{l+s+s2}{ epochs: }\PY{l+s+s2}{\PYZdq{}}\PY{o}{.}\PY{n}{format}\PY{p}{(}\PY{n}{epoch\PYZus{}num\PYZus{}2}\PY{p}{)}\PY{p}{,} \PY{n}{learning\PYZus{}rate\PYZus{}2}\PY{p}{)}

\PY{n}{schedule\PYZus{}lr\PYZus{}decay\PYZus{}test}\PY{p}{(}\PY{n}{schedule\PYZus{}lr\PYZus{}decay}\PY{p}{)}
\end{Verbatim}
\end{tcolorbox}

    \textbf{Expected output}

\begin{verbatim}
Original learning rate:  0.5
Updated learning rate after 10 epochs:  0.5
Updated learning rate after 100 epochs:  0.3846153846153846
\end{verbatim}

    \#\#\# 7.3 - Using Learning Rate Decay for each Optimization Method

Below, you'll use the following ``moons'' dataset to test the different
optimization methods. (The dataset is named ``moons'' because the data
from each of the two classes looks a bit like a crescent-shaped moon.)

    \#\#\#\# 7.3.1 - Gradient Descent with Learning Rate Decay

Run the following code to see how the model does gradient descent and
weight decay.

    \begin{tcolorbox}[breakable, size=fbox, boxrule=1pt, pad at break*=1mm,colback=cellbackground, colframe=cellborder]
\prompt{In}{incolor}{ }{\boxspacing}
\begin{Verbatim}[commandchars=\\\{\}]
\PY{c+c1}{\PYZsh{} train 3\PYZhy{}layer model}
\PY{n}{layers\PYZus{}dims} \PY{o}{=} \PY{p}{[}\PY{n}{train\PYZus{}X}\PY{o}{.}\PY{n}{shape}\PY{p}{[}\PY{l+m+mi}{0}\PY{p}{]}\PY{p}{,} \PY{l+m+mi}{5}\PY{p}{,} \PY{l+m+mi}{2}\PY{p}{,} \PY{l+m+mi}{1}\PY{p}{]}
\PY{n}{parameters} \PY{o}{=} \PY{n}{model}\PY{p}{(}\PY{n}{train\PYZus{}X}\PY{p}{,} \PY{n}{train\PYZus{}Y}\PY{p}{,} \PY{n}{layers\PYZus{}dims}\PY{p}{,} \PY{n}{optimizer} \PY{o}{=} \PY{l+s+s2}{\PYZdq{}}\PY{l+s+s2}{gd}\PY{l+s+s2}{\PYZdq{}}\PY{p}{,} \PY{n}{learning\PYZus{}rate} \PY{o}{=} \PY{l+m+mf}{0.1}\PY{p}{,} \PY{n}{num\PYZus{}epochs}\PY{o}{=}\PY{l+m+mi}{5000}\PY{p}{,} \PY{n}{decay}\PY{o}{=}\PY{n}{schedule\PYZus{}lr\PYZus{}decay}\PY{p}{)}

\PY{c+c1}{\PYZsh{} Predict}
\PY{n}{predictions} \PY{o}{=} \PY{n}{predict}\PY{p}{(}\PY{n}{train\PYZus{}X}\PY{p}{,} \PY{n}{train\PYZus{}Y}\PY{p}{,} \PY{n}{parameters}\PY{p}{)}

\PY{c+c1}{\PYZsh{} Plot decision boundary}
\PY{n}{plt}\PY{o}{.}\PY{n}{title}\PY{p}{(}\PY{l+s+s2}{\PYZdq{}}\PY{l+s+s2}{Model with Gradient Descent optimization}\PY{l+s+s2}{\PYZdq{}}\PY{p}{)}
\PY{n}{axes} \PY{o}{=} \PY{n}{plt}\PY{o}{.}\PY{n}{gca}\PY{p}{(}\PY{p}{)}
\PY{n}{axes}\PY{o}{.}\PY{n}{set\PYZus{}xlim}\PY{p}{(}\PY{p}{[}\PY{o}{\PYZhy{}}\PY{l+m+mf}{1.5}\PY{p}{,}\PY{l+m+mf}{2.5}\PY{p}{]}\PY{p}{)}
\PY{n}{axes}\PY{o}{.}\PY{n}{set\PYZus{}ylim}\PY{p}{(}\PY{p}{[}\PY{o}{\PYZhy{}}\PY{l+m+mi}{1}\PY{p}{,}\PY{l+m+mf}{1.5}\PY{p}{]}\PY{p}{)}
\PY{n}{plot\PYZus{}decision\PYZus{}boundary}\PY{p}{(}\PY{k}{lambda} \PY{n}{x}\PY{p}{:} \PY{n}{predict\PYZus{}dec}\PY{p}{(}\PY{n}{parameters}\PY{p}{,} \PY{n}{x}\PY{o}{.}\PY{n}{T}\PY{p}{)}\PY{p}{,} \PY{n}{train\PYZus{}X}\PY{p}{,} \PY{n}{train\PYZus{}Y}\PY{p}{)}
\end{Verbatim}
\end{tcolorbox}

    \#\#\#\# 7.3.2 - Gradient Descent with Momentum and Learning Rate Decay

Run the following code to see how the model does gradient descent with
momentum and weight decay.

    \begin{tcolorbox}[breakable, size=fbox, boxrule=1pt, pad at break*=1mm,colback=cellbackground, colframe=cellborder]
\prompt{In}{incolor}{ }{\boxspacing}
\begin{Verbatim}[commandchars=\\\{\}]
\PY{c+c1}{\PYZsh{} train 3\PYZhy{}layer model}
\PY{n}{layers\PYZus{}dims} \PY{o}{=} \PY{p}{[}\PY{n}{train\PYZus{}X}\PY{o}{.}\PY{n}{shape}\PY{p}{[}\PY{l+m+mi}{0}\PY{p}{]}\PY{p}{,} \PY{l+m+mi}{5}\PY{p}{,} \PY{l+m+mi}{2}\PY{p}{,} \PY{l+m+mi}{1}\PY{p}{]}
\PY{n}{parameters} \PY{o}{=} \PY{n}{model}\PY{p}{(}\PY{n}{train\PYZus{}X}\PY{p}{,} \PY{n}{train\PYZus{}Y}\PY{p}{,} \PY{n}{layers\PYZus{}dims}\PY{p}{,} \PY{n}{optimizer} \PY{o}{=} \PY{l+s+s2}{\PYZdq{}}\PY{l+s+s2}{momentum}\PY{l+s+s2}{\PYZdq{}}\PY{p}{,} \PY{n}{learning\PYZus{}rate} \PY{o}{=} \PY{l+m+mf}{0.1}\PY{p}{,} \PY{n}{num\PYZus{}epochs}\PY{o}{=}\PY{l+m+mi}{5000}\PY{p}{,} \PY{n}{decay}\PY{o}{=}\PY{n}{schedule\PYZus{}lr\PYZus{}decay}\PY{p}{)}

\PY{c+c1}{\PYZsh{} Predict}
\PY{n}{predictions} \PY{o}{=} \PY{n}{predict}\PY{p}{(}\PY{n}{train\PYZus{}X}\PY{p}{,} \PY{n}{train\PYZus{}Y}\PY{p}{,} \PY{n}{parameters}\PY{p}{)}

\PY{c+c1}{\PYZsh{} Plot decision boundary}
\PY{n}{plt}\PY{o}{.}\PY{n}{title}\PY{p}{(}\PY{l+s+s2}{\PYZdq{}}\PY{l+s+s2}{Model with Gradient Descent with momentum optimization}\PY{l+s+s2}{\PYZdq{}}\PY{p}{)}
\PY{n}{axes} \PY{o}{=} \PY{n}{plt}\PY{o}{.}\PY{n}{gca}\PY{p}{(}\PY{p}{)}
\PY{n}{axes}\PY{o}{.}\PY{n}{set\PYZus{}xlim}\PY{p}{(}\PY{p}{[}\PY{o}{\PYZhy{}}\PY{l+m+mf}{1.5}\PY{p}{,}\PY{l+m+mf}{2.5}\PY{p}{]}\PY{p}{)}
\PY{n}{axes}\PY{o}{.}\PY{n}{set\PYZus{}ylim}\PY{p}{(}\PY{p}{[}\PY{o}{\PYZhy{}}\PY{l+m+mi}{1}\PY{p}{,}\PY{l+m+mf}{1.5}\PY{p}{]}\PY{p}{)}
\PY{n}{plot\PYZus{}decision\PYZus{}boundary}\PY{p}{(}\PY{k}{lambda} \PY{n}{x}\PY{p}{:} \PY{n}{predict\PYZus{}dec}\PY{p}{(}\PY{n}{parameters}\PY{p}{,} \PY{n}{x}\PY{o}{.}\PY{n}{T}\PY{p}{)}\PY{p}{,} \PY{n}{train\PYZus{}X}\PY{p}{,} \PY{n}{train\PYZus{}Y}\PY{p}{)}
\end{Verbatim}
\end{tcolorbox}

    \#\#\#\# 7.3.3 - Adam with Learning Rate Decay

Run the following code to see how the model does Adam and weight decay.

    \begin{tcolorbox}[breakable, size=fbox, boxrule=1pt, pad at break*=1mm,colback=cellbackground, colframe=cellborder]
\prompt{In}{incolor}{ }{\boxspacing}
\begin{Verbatim}[commandchars=\\\{\}]
\PY{c+c1}{\PYZsh{} train 3\PYZhy{}layer model}
\PY{n}{layers\PYZus{}dims} \PY{o}{=} \PY{p}{[}\PY{n}{train\PYZus{}X}\PY{o}{.}\PY{n}{shape}\PY{p}{[}\PY{l+m+mi}{0}\PY{p}{]}\PY{p}{,} \PY{l+m+mi}{5}\PY{p}{,} \PY{l+m+mi}{2}\PY{p}{,} \PY{l+m+mi}{1}\PY{p}{]}
\PY{n}{parameters} \PY{o}{=} \PY{n}{model}\PY{p}{(}\PY{n}{train\PYZus{}X}\PY{p}{,} \PY{n}{train\PYZus{}Y}\PY{p}{,} \PY{n}{layers\PYZus{}dims}\PY{p}{,} \PY{n}{optimizer} \PY{o}{=} \PY{l+s+s2}{\PYZdq{}}\PY{l+s+s2}{adam}\PY{l+s+s2}{\PYZdq{}}\PY{p}{,} \PY{n}{learning\PYZus{}rate} \PY{o}{=} \PY{l+m+mf}{0.01}\PY{p}{,} \PY{n}{num\PYZus{}epochs}\PY{o}{=}\PY{l+m+mi}{5000}\PY{p}{,} \PY{n}{decay}\PY{o}{=}\PY{n}{schedule\PYZus{}lr\PYZus{}decay}\PY{p}{)}

\PY{c+c1}{\PYZsh{} Predict}
\PY{n}{predictions} \PY{o}{=} \PY{n}{predict}\PY{p}{(}\PY{n}{train\PYZus{}X}\PY{p}{,} \PY{n}{train\PYZus{}Y}\PY{p}{,} \PY{n}{parameters}\PY{p}{)}

\PY{c+c1}{\PYZsh{} Plot decision boundary}
\PY{n}{plt}\PY{o}{.}\PY{n}{title}\PY{p}{(}\PY{l+s+s2}{\PYZdq{}}\PY{l+s+s2}{Model with Adam optimization}\PY{l+s+s2}{\PYZdq{}}\PY{p}{)}
\PY{n}{axes} \PY{o}{=} \PY{n}{plt}\PY{o}{.}\PY{n}{gca}\PY{p}{(}\PY{p}{)}
\PY{n}{axes}\PY{o}{.}\PY{n}{set\PYZus{}xlim}\PY{p}{(}\PY{p}{[}\PY{o}{\PYZhy{}}\PY{l+m+mf}{1.5}\PY{p}{,}\PY{l+m+mf}{2.5}\PY{p}{]}\PY{p}{)}
\PY{n}{axes}\PY{o}{.}\PY{n}{set\PYZus{}ylim}\PY{p}{(}\PY{p}{[}\PY{o}{\PYZhy{}}\PY{l+m+mi}{1}\PY{p}{,}\PY{l+m+mf}{1.5}\PY{p}{]}\PY{p}{)}
\PY{n}{plot\PYZus{}decision\PYZus{}boundary}\PY{p}{(}\PY{k}{lambda} \PY{n}{x}\PY{p}{:} \PY{n}{predict\PYZus{}dec}\PY{p}{(}\PY{n}{parameters}\PY{p}{,} \PY{n}{x}\PY{o}{.}\PY{n}{T}\PY{p}{)}\PY{p}{,} \PY{n}{train\PYZus{}X}\PY{p}{,} \PY{n}{train\PYZus{}Y}\PY{p}{)}
\end{Verbatim}
\end{tcolorbox}

    \#\#\# 7.4 - Achieving similar performance with different methods

With Mini-batch GD or Mini-batch GD with Momentum, the accuracy is
significantly lower than Adam, but when learning rate decay is added on
top, either can achieve performance at a speed and accuracy score that's
similar to Adam.

In the case of Adam, notice that the learning curve achieves a similar
accuracy but faster.

optimization method

accuracy

\begin{verbatim}
<td>
    Gradient descent
    </td>
    <td>
    >94.6%
    </td>
<tr>
    <td>
    Momentum
    </td>
    <td>
    >95.6%
    </td>
</tr>
<tr>
    <td>
    Adam
    </td>
    <td>
    94%
    </td>
</tr>
\end{verbatim}

    \textbf{Congratulations}! You've made it to the end of the Optimization
methods notebook. Here's a quick recap of everything you're now able to
do:

\begin{itemize}
\tightlist
\item
  Apply three different optimization methods to your models
\item
  Build mini-batches for your training set
\item
  Use learning rate decay scheduling to speed up your training
\end{itemize}

Great work!


    % Add a bibliography block to the postdoc
    
    
    
\end{document}
