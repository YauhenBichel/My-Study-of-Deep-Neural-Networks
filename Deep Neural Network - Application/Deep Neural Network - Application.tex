\documentclass[11pt]{article}

    \usepackage[breakable]{tcolorbox}
    \usepackage{parskip} % Stop auto-indenting (to mimic markdown behaviour)
    
    \usepackage{iftex}
    \ifPDFTeX
    	\usepackage[T1]{fontenc}
    	\usepackage{mathpazo}
    \else
    	\usepackage{fontspec}
    \fi

    % Basic figure setup, for now with no caption control since it's done
    % automatically by Pandoc (which extracts ![](path) syntax from Markdown).
    \usepackage{graphicx}
    % Maintain compatibility with old templates. Remove in nbconvert 6.0
    \let\Oldincludegraphics\includegraphics
    % Ensure that by default, figures have no caption (until we provide a
    % proper Figure object with a Caption API and a way to capture that
    % in the conversion process - todo).
    \usepackage{caption}
    \DeclareCaptionFormat{nocaption}{}
    \captionsetup{format=nocaption,aboveskip=0pt,belowskip=0pt}

    \usepackage[Export]{adjustbox} % Used to constrain images to a maximum size
    \adjustboxset{max size={0.9\linewidth}{0.9\paperheight}}
    \usepackage{float}
    \floatplacement{figure}{H} % forces figures to be placed at the correct location
    \usepackage{xcolor} % Allow colors to be defined
    \usepackage{enumerate} % Needed for markdown enumerations to work
    \usepackage{geometry} % Used to adjust the document margins
    \usepackage{amsmath} % Equations
    \usepackage{amssymb} % Equations
    \usepackage{textcomp} % defines textquotesingle
    % Hack from http://tex.stackexchange.com/a/47451/13684:
    \AtBeginDocument{%
        \def\PYZsq{\textquotesingle}% Upright quotes in Pygmentized code
    }
    \usepackage{upquote} % Upright quotes for verbatim code
    \usepackage{eurosym} % defines \euro
    \usepackage[mathletters]{ucs} % Extended unicode (utf-8) support
    \usepackage{fancyvrb} % verbatim replacement that allows latex
    \usepackage{grffile} % extends the file name processing of package graphics 
                         % to support a larger range
    \makeatletter % fix for grffile with XeLaTeX
    \def\Gread@@xetex#1{%
      \IfFileExists{"\Gin@base".bb}%
      {\Gread@eps{\Gin@base.bb}}%
      {\Gread@@xetex@aux#1}%
    }
    \makeatother

    % The hyperref package gives us a pdf with properly built
    % internal navigation ('pdf bookmarks' for the table of contents,
    % internal cross-reference links, web links for URLs, etc.)
    \usepackage{hyperref}
    % The default LaTeX title has an obnoxious amount of whitespace. By default,
    % titling removes some of it. It also provides customization options.
    \usepackage{titling}
    \usepackage{longtable} % longtable support required by pandoc >1.10
    \usepackage{booktabs}  % table support for pandoc > 1.12.2
    \usepackage[inline]{enumitem} % IRkernel/repr support (it uses the enumerate* environment)
    \usepackage[normalem]{ulem} % ulem is needed to support strikethroughs (\sout)
                                % normalem makes italics be italics, not underlines
    \usepackage{mathrsfs}
    

    
    % Colors for the hyperref package
    \definecolor{urlcolor}{rgb}{0,.145,.698}
    \definecolor{linkcolor}{rgb}{.71,0.21,0.01}
    \definecolor{citecolor}{rgb}{.12,.54,.11}

    % ANSI colors
    \definecolor{ansi-black}{HTML}{3E424D}
    \definecolor{ansi-black-intense}{HTML}{282C36}
    \definecolor{ansi-red}{HTML}{E75C58}
    \definecolor{ansi-red-intense}{HTML}{B22B31}
    \definecolor{ansi-green}{HTML}{00A250}
    \definecolor{ansi-green-intense}{HTML}{007427}
    \definecolor{ansi-yellow}{HTML}{DDB62B}
    \definecolor{ansi-yellow-intense}{HTML}{B27D12}
    \definecolor{ansi-blue}{HTML}{208FFB}
    \definecolor{ansi-blue-intense}{HTML}{0065CA}
    \definecolor{ansi-magenta}{HTML}{D160C4}
    \definecolor{ansi-magenta-intense}{HTML}{A03196}
    \definecolor{ansi-cyan}{HTML}{60C6C8}
    \definecolor{ansi-cyan-intense}{HTML}{258F8F}
    \definecolor{ansi-white}{HTML}{C5C1B4}
    \definecolor{ansi-white-intense}{HTML}{A1A6B2}
    \definecolor{ansi-default-inverse-fg}{HTML}{FFFFFF}
    \definecolor{ansi-default-inverse-bg}{HTML}{000000}

    % commands and environments needed by pandoc snippets
    % extracted from the output of `pandoc -s`
    \providecommand{\tightlist}{%
      \setlength{\itemsep}{0pt}\setlength{\parskip}{0pt}}
    \DefineVerbatimEnvironment{Highlighting}{Verbatim}{commandchars=\\\{\}}
    % Add ',fontsize=\small' for more characters per line
    \newenvironment{Shaded}{}{}
    \newcommand{\KeywordTok}[1]{\textcolor[rgb]{0.00,0.44,0.13}{\textbf{{#1}}}}
    \newcommand{\DataTypeTok}[1]{\textcolor[rgb]{0.56,0.13,0.00}{{#1}}}
    \newcommand{\DecValTok}[1]{\textcolor[rgb]{0.25,0.63,0.44}{{#1}}}
    \newcommand{\BaseNTok}[1]{\textcolor[rgb]{0.25,0.63,0.44}{{#1}}}
    \newcommand{\FloatTok}[1]{\textcolor[rgb]{0.25,0.63,0.44}{{#1}}}
    \newcommand{\CharTok}[1]{\textcolor[rgb]{0.25,0.44,0.63}{{#1}}}
    \newcommand{\StringTok}[1]{\textcolor[rgb]{0.25,0.44,0.63}{{#1}}}
    \newcommand{\CommentTok}[1]{\textcolor[rgb]{0.38,0.63,0.69}{\textit{{#1}}}}
    \newcommand{\OtherTok}[1]{\textcolor[rgb]{0.00,0.44,0.13}{{#1}}}
    \newcommand{\AlertTok}[1]{\textcolor[rgb]{1.00,0.00,0.00}{\textbf{{#1}}}}
    \newcommand{\FunctionTok}[1]{\textcolor[rgb]{0.02,0.16,0.49}{{#1}}}
    \newcommand{\RegionMarkerTok}[1]{{#1}}
    \newcommand{\ErrorTok}[1]{\textcolor[rgb]{1.00,0.00,0.00}{\textbf{{#1}}}}
    \newcommand{\NormalTok}[1]{{#1}}
    
    % Additional commands for more recent versions of Pandoc
    \newcommand{\ConstantTok}[1]{\textcolor[rgb]{0.53,0.00,0.00}{{#1}}}
    \newcommand{\SpecialCharTok}[1]{\textcolor[rgb]{0.25,0.44,0.63}{{#1}}}
    \newcommand{\VerbatimStringTok}[1]{\textcolor[rgb]{0.25,0.44,0.63}{{#1}}}
    \newcommand{\SpecialStringTok}[1]{\textcolor[rgb]{0.73,0.40,0.53}{{#1}}}
    \newcommand{\ImportTok}[1]{{#1}}
    \newcommand{\DocumentationTok}[1]{\textcolor[rgb]{0.73,0.13,0.13}{\textit{{#1}}}}
    \newcommand{\AnnotationTok}[1]{\textcolor[rgb]{0.38,0.63,0.69}{\textbf{\textit{{#1}}}}}
    \newcommand{\CommentVarTok}[1]{\textcolor[rgb]{0.38,0.63,0.69}{\textbf{\textit{{#1}}}}}
    \newcommand{\VariableTok}[1]{\textcolor[rgb]{0.10,0.09,0.49}{{#1}}}
    \newcommand{\ControlFlowTok}[1]{\textcolor[rgb]{0.00,0.44,0.13}{\textbf{{#1}}}}
    \newcommand{\OperatorTok}[1]{\textcolor[rgb]{0.40,0.40,0.40}{{#1}}}
    \newcommand{\BuiltInTok}[1]{{#1}}
    \newcommand{\ExtensionTok}[1]{{#1}}
    \newcommand{\PreprocessorTok}[1]{\textcolor[rgb]{0.74,0.48,0.00}{{#1}}}
    \newcommand{\AttributeTok}[1]{\textcolor[rgb]{0.49,0.56,0.16}{{#1}}}
    \newcommand{\InformationTok}[1]{\textcolor[rgb]{0.38,0.63,0.69}{\textbf{\textit{{#1}}}}}
    \newcommand{\WarningTok}[1]{\textcolor[rgb]{0.38,0.63,0.69}{\textbf{\textit{{#1}}}}}
    
    
    % Define a nice break command that doesn't care if a line doesn't already
    % exist.
    \def\br{\hspace*{\fill} \\* }
    % Math Jax compatibility definitions
    \def\gt{>}
    \def\lt{<}
    \let\Oldtex\TeX
    \let\Oldlatex\LaTeX
    \renewcommand{\TeX}{\textrm{\Oldtex}}
    \renewcommand{\LaTeX}{\textrm{\Oldlatex}}
    % Document parameters
    % Document title
    \title{Deep Neural Network - Application}
    
    
    
    
    
% Pygments definitions
\makeatletter
\def\PY@reset{\let\PY@it=\relax \let\PY@bf=\relax%
    \let\PY@ul=\relax \let\PY@tc=\relax%
    \let\PY@bc=\relax \let\PY@ff=\relax}
\def\PY@tok#1{\csname PY@tok@#1\endcsname}
\def\PY@toks#1+{\ifx\relax#1\empty\else%
    \PY@tok{#1}\expandafter\PY@toks\fi}
\def\PY@do#1{\PY@bc{\PY@tc{\PY@ul{%
    \PY@it{\PY@bf{\PY@ff{#1}}}}}}}
\def\PY#1#2{\PY@reset\PY@toks#1+\relax+\PY@do{#2}}

\expandafter\def\csname PY@tok@w\endcsname{\def\PY@tc##1{\textcolor[rgb]{0.73,0.73,0.73}{##1}}}
\expandafter\def\csname PY@tok@c\endcsname{\let\PY@it=\textit\def\PY@tc##1{\textcolor[rgb]{0.25,0.50,0.50}{##1}}}
\expandafter\def\csname PY@tok@cp\endcsname{\def\PY@tc##1{\textcolor[rgb]{0.74,0.48,0.00}{##1}}}
\expandafter\def\csname PY@tok@k\endcsname{\let\PY@bf=\textbf\def\PY@tc##1{\textcolor[rgb]{0.00,0.50,0.00}{##1}}}
\expandafter\def\csname PY@tok@kp\endcsname{\def\PY@tc##1{\textcolor[rgb]{0.00,0.50,0.00}{##1}}}
\expandafter\def\csname PY@tok@kt\endcsname{\def\PY@tc##1{\textcolor[rgb]{0.69,0.00,0.25}{##1}}}
\expandafter\def\csname PY@tok@o\endcsname{\def\PY@tc##1{\textcolor[rgb]{0.40,0.40,0.40}{##1}}}
\expandafter\def\csname PY@tok@ow\endcsname{\let\PY@bf=\textbf\def\PY@tc##1{\textcolor[rgb]{0.67,0.13,1.00}{##1}}}
\expandafter\def\csname PY@tok@nb\endcsname{\def\PY@tc##1{\textcolor[rgb]{0.00,0.50,0.00}{##1}}}
\expandafter\def\csname PY@tok@nf\endcsname{\def\PY@tc##1{\textcolor[rgb]{0.00,0.00,1.00}{##1}}}
\expandafter\def\csname PY@tok@nc\endcsname{\let\PY@bf=\textbf\def\PY@tc##1{\textcolor[rgb]{0.00,0.00,1.00}{##1}}}
\expandafter\def\csname PY@tok@nn\endcsname{\let\PY@bf=\textbf\def\PY@tc##1{\textcolor[rgb]{0.00,0.00,1.00}{##1}}}
\expandafter\def\csname PY@tok@ne\endcsname{\let\PY@bf=\textbf\def\PY@tc##1{\textcolor[rgb]{0.82,0.25,0.23}{##1}}}
\expandafter\def\csname PY@tok@nv\endcsname{\def\PY@tc##1{\textcolor[rgb]{0.10,0.09,0.49}{##1}}}
\expandafter\def\csname PY@tok@no\endcsname{\def\PY@tc##1{\textcolor[rgb]{0.53,0.00,0.00}{##1}}}
\expandafter\def\csname PY@tok@nl\endcsname{\def\PY@tc##1{\textcolor[rgb]{0.63,0.63,0.00}{##1}}}
\expandafter\def\csname PY@tok@ni\endcsname{\let\PY@bf=\textbf\def\PY@tc##1{\textcolor[rgb]{0.60,0.60,0.60}{##1}}}
\expandafter\def\csname PY@tok@na\endcsname{\def\PY@tc##1{\textcolor[rgb]{0.49,0.56,0.16}{##1}}}
\expandafter\def\csname PY@tok@nt\endcsname{\let\PY@bf=\textbf\def\PY@tc##1{\textcolor[rgb]{0.00,0.50,0.00}{##1}}}
\expandafter\def\csname PY@tok@nd\endcsname{\def\PY@tc##1{\textcolor[rgb]{0.67,0.13,1.00}{##1}}}
\expandafter\def\csname PY@tok@s\endcsname{\def\PY@tc##1{\textcolor[rgb]{0.73,0.13,0.13}{##1}}}
\expandafter\def\csname PY@tok@sd\endcsname{\let\PY@it=\textit\def\PY@tc##1{\textcolor[rgb]{0.73,0.13,0.13}{##1}}}
\expandafter\def\csname PY@tok@si\endcsname{\let\PY@bf=\textbf\def\PY@tc##1{\textcolor[rgb]{0.73,0.40,0.53}{##1}}}
\expandafter\def\csname PY@tok@se\endcsname{\let\PY@bf=\textbf\def\PY@tc##1{\textcolor[rgb]{0.73,0.40,0.13}{##1}}}
\expandafter\def\csname PY@tok@sr\endcsname{\def\PY@tc##1{\textcolor[rgb]{0.73,0.40,0.53}{##1}}}
\expandafter\def\csname PY@tok@ss\endcsname{\def\PY@tc##1{\textcolor[rgb]{0.10,0.09,0.49}{##1}}}
\expandafter\def\csname PY@tok@sx\endcsname{\def\PY@tc##1{\textcolor[rgb]{0.00,0.50,0.00}{##1}}}
\expandafter\def\csname PY@tok@m\endcsname{\def\PY@tc##1{\textcolor[rgb]{0.40,0.40,0.40}{##1}}}
\expandafter\def\csname PY@tok@gh\endcsname{\let\PY@bf=\textbf\def\PY@tc##1{\textcolor[rgb]{0.00,0.00,0.50}{##1}}}
\expandafter\def\csname PY@tok@gu\endcsname{\let\PY@bf=\textbf\def\PY@tc##1{\textcolor[rgb]{0.50,0.00,0.50}{##1}}}
\expandafter\def\csname PY@tok@gd\endcsname{\def\PY@tc##1{\textcolor[rgb]{0.63,0.00,0.00}{##1}}}
\expandafter\def\csname PY@tok@gi\endcsname{\def\PY@tc##1{\textcolor[rgb]{0.00,0.63,0.00}{##1}}}
\expandafter\def\csname PY@tok@gr\endcsname{\def\PY@tc##1{\textcolor[rgb]{1.00,0.00,0.00}{##1}}}
\expandafter\def\csname PY@tok@ge\endcsname{\let\PY@it=\textit}
\expandafter\def\csname PY@tok@gs\endcsname{\let\PY@bf=\textbf}
\expandafter\def\csname PY@tok@gp\endcsname{\let\PY@bf=\textbf\def\PY@tc##1{\textcolor[rgb]{0.00,0.00,0.50}{##1}}}
\expandafter\def\csname PY@tok@go\endcsname{\def\PY@tc##1{\textcolor[rgb]{0.53,0.53,0.53}{##1}}}
\expandafter\def\csname PY@tok@gt\endcsname{\def\PY@tc##1{\textcolor[rgb]{0.00,0.27,0.87}{##1}}}
\expandafter\def\csname PY@tok@err\endcsname{\def\PY@bc##1{\setlength{\fboxsep}{0pt}\fcolorbox[rgb]{1.00,0.00,0.00}{1,1,1}{\strut ##1}}}
\expandafter\def\csname PY@tok@kc\endcsname{\let\PY@bf=\textbf\def\PY@tc##1{\textcolor[rgb]{0.00,0.50,0.00}{##1}}}
\expandafter\def\csname PY@tok@kd\endcsname{\let\PY@bf=\textbf\def\PY@tc##1{\textcolor[rgb]{0.00,0.50,0.00}{##1}}}
\expandafter\def\csname PY@tok@kn\endcsname{\let\PY@bf=\textbf\def\PY@tc##1{\textcolor[rgb]{0.00,0.50,0.00}{##1}}}
\expandafter\def\csname PY@tok@kr\endcsname{\let\PY@bf=\textbf\def\PY@tc##1{\textcolor[rgb]{0.00,0.50,0.00}{##1}}}
\expandafter\def\csname PY@tok@bp\endcsname{\def\PY@tc##1{\textcolor[rgb]{0.00,0.50,0.00}{##1}}}
\expandafter\def\csname PY@tok@fm\endcsname{\def\PY@tc##1{\textcolor[rgb]{0.00,0.00,1.00}{##1}}}
\expandafter\def\csname PY@tok@vc\endcsname{\def\PY@tc##1{\textcolor[rgb]{0.10,0.09,0.49}{##1}}}
\expandafter\def\csname PY@tok@vg\endcsname{\def\PY@tc##1{\textcolor[rgb]{0.10,0.09,0.49}{##1}}}
\expandafter\def\csname PY@tok@vi\endcsname{\def\PY@tc##1{\textcolor[rgb]{0.10,0.09,0.49}{##1}}}
\expandafter\def\csname PY@tok@vm\endcsname{\def\PY@tc##1{\textcolor[rgb]{0.10,0.09,0.49}{##1}}}
\expandafter\def\csname PY@tok@sa\endcsname{\def\PY@tc##1{\textcolor[rgb]{0.73,0.13,0.13}{##1}}}
\expandafter\def\csname PY@tok@sb\endcsname{\def\PY@tc##1{\textcolor[rgb]{0.73,0.13,0.13}{##1}}}
\expandafter\def\csname PY@tok@sc\endcsname{\def\PY@tc##1{\textcolor[rgb]{0.73,0.13,0.13}{##1}}}
\expandafter\def\csname PY@tok@dl\endcsname{\def\PY@tc##1{\textcolor[rgb]{0.73,0.13,0.13}{##1}}}
\expandafter\def\csname PY@tok@s2\endcsname{\def\PY@tc##1{\textcolor[rgb]{0.73,0.13,0.13}{##1}}}
\expandafter\def\csname PY@tok@sh\endcsname{\def\PY@tc##1{\textcolor[rgb]{0.73,0.13,0.13}{##1}}}
\expandafter\def\csname PY@tok@s1\endcsname{\def\PY@tc##1{\textcolor[rgb]{0.73,0.13,0.13}{##1}}}
\expandafter\def\csname PY@tok@mb\endcsname{\def\PY@tc##1{\textcolor[rgb]{0.40,0.40,0.40}{##1}}}
\expandafter\def\csname PY@tok@mf\endcsname{\def\PY@tc##1{\textcolor[rgb]{0.40,0.40,0.40}{##1}}}
\expandafter\def\csname PY@tok@mh\endcsname{\def\PY@tc##1{\textcolor[rgb]{0.40,0.40,0.40}{##1}}}
\expandafter\def\csname PY@tok@mi\endcsname{\def\PY@tc##1{\textcolor[rgb]{0.40,0.40,0.40}{##1}}}
\expandafter\def\csname PY@tok@il\endcsname{\def\PY@tc##1{\textcolor[rgb]{0.40,0.40,0.40}{##1}}}
\expandafter\def\csname PY@tok@mo\endcsname{\def\PY@tc##1{\textcolor[rgb]{0.40,0.40,0.40}{##1}}}
\expandafter\def\csname PY@tok@ch\endcsname{\let\PY@it=\textit\def\PY@tc##1{\textcolor[rgb]{0.25,0.50,0.50}{##1}}}
\expandafter\def\csname PY@tok@cm\endcsname{\let\PY@it=\textit\def\PY@tc##1{\textcolor[rgb]{0.25,0.50,0.50}{##1}}}
\expandafter\def\csname PY@tok@cpf\endcsname{\let\PY@it=\textit\def\PY@tc##1{\textcolor[rgb]{0.25,0.50,0.50}{##1}}}
\expandafter\def\csname PY@tok@c1\endcsname{\let\PY@it=\textit\def\PY@tc##1{\textcolor[rgb]{0.25,0.50,0.50}{##1}}}
\expandafter\def\csname PY@tok@cs\endcsname{\let\PY@it=\textit\def\PY@tc##1{\textcolor[rgb]{0.25,0.50,0.50}{##1}}}

\def\PYZbs{\char`\\}
\def\PYZus{\char`\_}
\def\PYZob{\char`\{}
\def\PYZcb{\char`\}}
\def\PYZca{\char`\^}
\def\PYZam{\char`\&}
\def\PYZlt{\char`\<}
\def\PYZgt{\char`\>}
\def\PYZsh{\char`\#}
\def\PYZpc{\char`\%}
\def\PYZdl{\char`\$}
\def\PYZhy{\char`\-}
\def\PYZsq{\char`\'}
\def\PYZdq{\char`\"}
\def\PYZti{\char`\~}
% for compatibility with earlier versions
\def\PYZat{@}
\def\PYZlb{[}
\def\PYZrb{]}
\makeatother


    % For linebreaks inside Verbatim environment from package fancyvrb. 
    \makeatletter
        \newbox\Wrappedcontinuationbox 
        \newbox\Wrappedvisiblespacebox 
        \newcommand*\Wrappedvisiblespace {\textcolor{red}{\textvisiblespace}} 
        \newcommand*\Wrappedcontinuationsymbol {\textcolor{red}{\llap{\tiny$\m@th\hookrightarrow$}}} 
        \newcommand*\Wrappedcontinuationindent {3ex } 
        \newcommand*\Wrappedafterbreak {\kern\Wrappedcontinuationindent\copy\Wrappedcontinuationbox} 
        % Take advantage of the already applied Pygments mark-up to insert 
        % potential linebreaks for TeX processing. 
        %        {, <, #, %, $, ' and ": go to next line. 
        %        _, }, ^, &, >, - and ~: stay at end of broken line. 
        % Use of \textquotesingle for straight quote. 
        \newcommand*\Wrappedbreaksatspecials {% 
            \def\PYGZus{\discretionary{\char`\_}{\Wrappedafterbreak}{\char`\_}}% 
            \def\PYGZob{\discretionary{}{\Wrappedafterbreak\char`\{}{\char`\{}}% 
            \def\PYGZcb{\discretionary{\char`\}}{\Wrappedafterbreak}{\char`\}}}% 
            \def\PYGZca{\discretionary{\char`\^}{\Wrappedafterbreak}{\char`\^}}% 
            \def\PYGZam{\discretionary{\char`\&}{\Wrappedafterbreak}{\char`\&}}% 
            \def\PYGZlt{\discretionary{}{\Wrappedafterbreak\char`\<}{\char`\<}}% 
            \def\PYGZgt{\discretionary{\char`\>}{\Wrappedafterbreak}{\char`\>}}% 
            \def\PYGZsh{\discretionary{}{\Wrappedafterbreak\char`\#}{\char`\#}}% 
            \def\PYGZpc{\discretionary{}{\Wrappedafterbreak\char`\%}{\char`\%}}% 
            \def\PYGZdl{\discretionary{}{\Wrappedafterbreak\char`\$}{\char`\$}}% 
            \def\PYGZhy{\discretionary{\char`\-}{\Wrappedafterbreak}{\char`\-}}% 
            \def\PYGZsq{\discretionary{}{\Wrappedafterbreak\textquotesingle}{\textquotesingle}}% 
            \def\PYGZdq{\discretionary{}{\Wrappedafterbreak\char`\"}{\char`\"}}% 
            \def\PYGZti{\discretionary{\char`\~}{\Wrappedafterbreak}{\char`\~}}% 
        } 
        % Some characters . , ; ? ! / are not pygmentized. 
        % This macro makes them "active" and they will insert potential linebreaks 
        \newcommand*\Wrappedbreaksatpunct {% 
            \lccode`\~`\.\lowercase{\def~}{\discretionary{\hbox{\char`\.}}{\Wrappedafterbreak}{\hbox{\char`\.}}}% 
            \lccode`\~`\,\lowercase{\def~}{\discretionary{\hbox{\char`\,}}{\Wrappedafterbreak}{\hbox{\char`\,}}}% 
            \lccode`\~`\;\lowercase{\def~}{\discretionary{\hbox{\char`\;}}{\Wrappedafterbreak}{\hbox{\char`\;}}}% 
            \lccode`\~`\:\lowercase{\def~}{\discretionary{\hbox{\char`\:}}{\Wrappedafterbreak}{\hbox{\char`\:}}}% 
            \lccode`\~`\?\lowercase{\def~}{\discretionary{\hbox{\char`\?}}{\Wrappedafterbreak}{\hbox{\char`\?}}}% 
            \lccode`\~`\!\lowercase{\def~}{\discretionary{\hbox{\char`\!}}{\Wrappedafterbreak}{\hbox{\char`\!}}}% 
            \lccode`\~`\/\lowercase{\def~}{\discretionary{\hbox{\char`\/}}{\Wrappedafterbreak}{\hbox{\char`\/}}}% 
            \catcode`\.\active
            \catcode`\,\active 
            \catcode`\;\active
            \catcode`\:\active
            \catcode`\?\active
            \catcode`\!\active
            \catcode`\/\active 
            \lccode`\~`\~ 	
        }
    \makeatother

    \let\OriginalVerbatim=\Verbatim
    \makeatletter
    \renewcommand{\Verbatim}[1][1]{%
        %\parskip\z@skip
        \sbox\Wrappedcontinuationbox {\Wrappedcontinuationsymbol}%
        \sbox\Wrappedvisiblespacebox {\FV@SetupFont\Wrappedvisiblespace}%
        \def\FancyVerbFormatLine ##1{\hsize\linewidth
            \vtop{\raggedright\hyphenpenalty\z@\exhyphenpenalty\z@
                \doublehyphendemerits\z@\finalhyphendemerits\z@
                \strut ##1\strut}%
        }%
        % If the linebreak is at a space, the latter will be displayed as visible
        % space at end of first line, and a continuation symbol starts next line.
        % Stretch/shrink are however usually zero for typewriter font.
        \def\FV@Space {%
            \nobreak\hskip\z@ plus\fontdimen3\font minus\fontdimen4\font
            \discretionary{\copy\Wrappedvisiblespacebox}{\Wrappedafterbreak}
            {\kern\fontdimen2\font}%
        }%
        
        % Allow breaks at special characters using \PYG... macros.
        \Wrappedbreaksatspecials
        % Breaks at punctuation characters . , ; ? ! and / need catcode=\active 	
        \OriginalVerbatim[#1,codes*=\Wrappedbreaksatpunct]%
    }
    \makeatother

    % Exact colors from NB
    \definecolor{incolor}{HTML}{303F9F}
    \definecolor{outcolor}{HTML}{D84315}
    \definecolor{cellborder}{HTML}{CFCFCF}
    \definecolor{cellbackground}{HTML}{F7F7F7}
    
    % prompt
    \makeatletter
    \newcommand{\boxspacing}{\kern\kvtcb@left@rule\kern\kvtcb@boxsep}
    \makeatother
    \newcommand{\prompt}[4]{
        \ttfamily\llap{{\color{#2}[#3]:\hspace{3pt}#4}}\vspace{-\baselineskip}
    }
    

    
    % Prevent overflowing lines due to hard-to-break entities
    \sloppy 
    % Setup hyperref package
    \hypersetup{
      breaklinks=true,  % so long urls are correctly broken across lines
      colorlinks=true,
      urlcolor=urlcolor,
      linkcolor=linkcolor,
      citecolor=citecolor,
      }
    % Slightly bigger margins than the latex defaults
    
    \geometry{verbose,tmargin=1in,bmargin=1in,lmargin=1in,rmargin=1in}
    
    

\begin{document}
    
    \maketitle
    
    

    
    \hypertarget{deep-neural-network-for-image-classification-application}{%
\section{Deep Neural Network for Image Classification:
Application}\label{deep-neural-network-for-image-classification-application}}

By the time you complete this notebook, you will have finished the last
programming assignment of Week 4, and also the last programming
assignment of Course 1! Go you!

To build your cat/not-a-cat classifier, you'll use the functions from
the previous assignment to build a deep network. Hopefully, you'll see
an improvement in accuracy over your previous logistic regression
implementation.

\textbf{After this assignment you will be able to:}

\begin{itemize}
\tightlist
\item
  Build and train a deep L-layer neural network, and apply it to
  supervised learning
\end{itemize}

Let's get started!

    \hypertarget{table-of-contents}{%
\subsection{Table of Contents}\label{table-of-contents}}

\begin{itemize}
\tightlist
\item
  Section \ref{1}
\item
  Section \ref{2}
\item
  Section \ref{3}

  \begin{itemize}
  \tightlist
  \item
    Section \ref{3-1}
  \item
    Section \ref{3-2}
  \item
    Section \ref{3-3}
  \end{itemize}
\item
  Section \ref{4}

  \begin{itemize}
  \tightlist
  \item
    Section \ref{ex-1}
  \item
    Section \ref{4-1}
  \end{itemize}
\item
  Section \ref{5}

  \begin{itemize}
  \tightlist
  \item
    Section \ref{ex-2}
  \item
    Section \ref{5-1}
  \end{itemize}
\item
  Section \ref{6}
\item
  Section \ref{7}
\end{itemize}

    \#\# 1 - Packages

    Begin by importing all the packages you'll need during this assignment.

\begin{itemize}
\tightlist
\item
  \href{https://www.numpy.org/}{numpy} is the fundamental package for
  scientific computing with Python.
\item
  \href{http://matplotlib.org}{matplotlib} is a library to plot graphs
  in Python.
\item
  \href{http://www.h5py.org}{h5py} is a common package to interact with
  a dataset that is stored on an H5 file.
\item
  \href{http://www.pythonware.com/products/pil/}{PIL} and
  \href{https://www.scipy.org/}{scipy} are used here to test your model
  with your own picture at the end.
\item
  \texttt{dnn\_app\_utils} provides the functions implemented in the
  ``Building your Deep Neural Network: Step by Step'' assignment to this
  notebook.
\item
  \texttt{np.random.seed(1)} is used to keep all the random function
  calls consistent. It helps grade your work - so please don't change
  it!
\end{itemize}

    \begin{tcolorbox}[breakable, size=fbox, boxrule=1pt, pad at break*=1mm,colback=cellbackground, colframe=cellborder]
\prompt{In}{incolor}{2}{\boxspacing}
\begin{Verbatim}[commandchars=\\\{\}]
\PY{k+kn}{import} \PY{n+nn}{time}
\PY{k+kn}{import} \PY{n+nn}{numpy} \PY{k}{as} \PY{n+nn}{np}
\PY{k+kn}{import} \PY{n+nn}{h5py}
\PY{k+kn}{import} \PY{n+nn}{matplotlib}\PY{n+nn}{.}\PY{n+nn}{pyplot} \PY{k}{as} \PY{n+nn}{plt}
\PY{k+kn}{import} \PY{n+nn}{scipy}
\PY{k+kn}{from} \PY{n+nn}{PIL} \PY{k+kn}{import} \PY{n}{Image}
\PY{k+kn}{from} \PY{n+nn}{scipy} \PY{k+kn}{import} \PY{n}{ndimage}
\PY{k+kn}{from} \PY{n+nn}{dnn\PYZus{}app\PYZus{}utils\PYZus{}v3} \PY{k+kn}{import} \PY{o}{*}
\PY{k+kn}{from} \PY{n+nn}{public\PYZus{}tests} \PY{k+kn}{import} \PY{o}{*}

\PY{o}{\PYZpc{}}\PY{k}{matplotlib} inline
\PY{n}{plt}\PY{o}{.}\PY{n}{rcParams}\PY{p}{[}\PY{l+s+s1}{\PYZsq{}}\PY{l+s+s1}{figure.figsize}\PY{l+s+s1}{\PYZsq{}}\PY{p}{]} \PY{o}{=} \PY{p}{(}\PY{l+m+mf}{5.0}\PY{p}{,} \PY{l+m+mf}{4.0}\PY{p}{)} \PY{c+c1}{\PYZsh{} set default size of plots}
\PY{n}{plt}\PY{o}{.}\PY{n}{rcParams}\PY{p}{[}\PY{l+s+s1}{\PYZsq{}}\PY{l+s+s1}{image.interpolation}\PY{l+s+s1}{\PYZsq{}}\PY{p}{]} \PY{o}{=} \PY{l+s+s1}{\PYZsq{}}\PY{l+s+s1}{nearest}\PY{l+s+s1}{\PYZsq{}}
\PY{n}{plt}\PY{o}{.}\PY{n}{rcParams}\PY{p}{[}\PY{l+s+s1}{\PYZsq{}}\PY{l+s+s1}{image.cmap}\PY{l+s+s1}{\PYZsq{}}\PY{p}{]} \PY{o}{=} \PY{l+s+s1}{\PYZsq{}}\PY{l+s+s1}{gray}\PY{l+s+s1}{\PYZsq{}}

\PY{o}{\PYZpc{}}\PY{k}{load\PYZus{}ext} autoreload
\PY{o}{\PYZpc{}}\PY{k}{autoreload} 2

\PY{n}{np}\PY{o}{.}\PY{n}{random}\PY{o}{.}\PY{n}{seed}\PY{p}{(}\PY{l+m+mi}{1}\PY{p}{)}
\end{Verbatim}
\end{tcolorbox}

    \begin{Verbatim}[commandchars=\\\{\}]
The autoreload extension is already loaded. To reload it, use:
  \%reload\_ext autoreload
    \end{Verbatim}

    \#\# 2 - Load and Process the Dataset

You'll be using the same ``Cat vs non-Cat'' dataset as in ``Logistic
Regression as a Neural Network'' (Assignment 2). The model you built
back then had 70\% test accuracy on classifying cat vs non-cat images.
Hopefully, your new model will perform even better!

\textbf{Problem Statement}: You are given a dataset (``data.h5'')
containing: - a training set of \texttt{m\_train} images labelled as cat
(1) or non-cat (0) - a test set of \texttt{m\_test} images labelled as
cat and non-cat - each image is of shape (num\_px, num\_px, 3) where 3
is for the 3 channels (RGB).

Let's get more familiar with the dataset. Load the data by running the
cell below.

    \begin{tcolorbox}[breakable, size=fbox, boxrule=1pt, pad at break*=1mm,colback=cellbackground, colframe=cellborder]
\prompt{In}{incolor}{3}{\boxspacing}
\begin{Verbatim}[commandchars=\\\{\}]
\PY{n}{train\PYZus{}x\PYZus{}orig}\PY{p}{,} \PY{n}{train\PYZus{}y}\PY{p}{,} \PY{n}{test\PYZus{}x\PYZus{}orig}\PY{p}{,} \PY{n}{test\PYZus{}y}\PY{p}{,} \PY{n}{classes} \PY{o}{=} \PY{n}{load\PYZus{}data}\PY{p}{(}\PY{p}{)}
\end{Verbatim}
\end{tcolorbox}

    The following code will show you an image in the dataset. Feel free to
change the index and re-run the cell multiple times to check out other
images.

    \begin{tcolorbox}[breakable, size=fbox, boxrule=1pt, pad at break*=1mm,colback=cellbackground, colframe=cellborder]
\prompt{In}{incolor}{4}{\boxspacing}
\begin{Verbatim}[commandchars=\\\{\}]
\PY{c+c1}{\PYZsh{} Example of a picture}
\PY{n}{index} \PY{o}{=} \PY{l+m+mi}{10}
\PY{n}{plt}\PY{o}{.}\PY{n}{imshow}\PY{p}{(}\PY{n}{train\PYZus{}x\PYZus{}orig}\PY{p}{[}\PY{n}{index}\PY{p}{]}\PY{p}{)}
\PY{n+nb}{print} \PY{p}{(}\PY{l+s+s2}{\PYZdq{}}\PY{l+s+s2}{y = }\PY{l+s+s2}{\PYZdq{}} \PY{o}{+} \PY{n+nb}{str}\PY{p}{(}\PY{n}{train\PYZus{}y}\PY{p}{[}\PY{l+m+mi}{0}\PY{p}{,}\PY{n}{index}\PY{p}{]}\PY{p}{)} \PY{o}{+} \PY{l+s+s2}{\PYZdq{}}\PY{l+s+s2}{. It}\PY{l+s+s2}{\PYZsq{}}\PY{l+s+s2}{s a }\PY{l+s+s2}{\PYZdq{}} \PY{o}{+} \PY{n}{classes}\PY{p}{[}\PY{n}{train\PYZus{}y}\PY{p}{[}\PY{l+m+mi}{0}\PY{p}{,}\PY{n}{index}\PY{p}{]}\PY{p}{]}\PY{o}{.}\PY{n}{decode}\PY{p}{(}\PY{l+s+s2}{\PYZdq{}}\PY{l+s+s2}{utf\PYZhy{}8}\PY{l+s+s2}{\PYZdq{}}\PY{p}{)} \PY{o}{+}  \PY{l+s+s2}{\PYZdq{}}\PY{l+s+s2}{ picture.}\PY{l+s+s2}{\PYZdq{}}\PY{p}{)}
\end{Verbatim}
\end{tcolorbox}

    \begin{Verbatim}[commandchars=\\\{\}]
y = 0. It's a non-cat picture.
    \end{Verbatim}

    \begin{center}
    \adjustimage{max size={0.9\linewidth}{0.9\paperheight}}{output_8_1.png}
    \end{center}
    { \hspace*{\fill} \\}
    
    \begin{tcolorbox}[breakable, size=fbox, boxrule=1pt, pad at break*=1mm,colback=cellbackground, colframe=cellborder]
\prompt{In}{incolor}{5}{\boxspacing}
\begin{Verbatim}[commandchars=\\\{\}]
\PY{c+c1}{\PYZsh{} Explore your dataset }
\PY{n}{m\PYZus{}train} \PY{o}{=} \PY{n}{train\PYZus{}x\PYZus{}orig}\PY{o}{.}\PY{n}{shape}\PY{p}{[}\PY{l+m+mi}{0}\PY{p}{]}
\PY{n}{num\PYZus{}px} \PY{o}{=} \PY{n}{train\PYZus{}x\PYZus{}orig}\PY{o}{.}\PY{n}{shape}\PY{p}{[}\PY{l+m+mi}{1}\PY{p}{]}
\PY{n}{m\PYZus{}test} \PY{o}{=} \PY{n}{test\PYZus{}x\PYZus{}orig}\PY{o}{.}\PY{n}{shape}\PY{p}{[}\PY{l+m+mi}{0}\PY{p}{]}

\PY{n+nb}{print} \PY{p}{(}\PY{l+s+s2}{\PYZdq{}}\PY{l+s+s2}{Number of training examples: }\PY{l+s+s2}{\PYZdq{}} \PY{o}{+} \PY{n+nb}{str}\PY{p}{(}\PY{n}{m\PYZus{}train}\PY{p}{)}\PY{p}{)}
\PY{n+nb}{print} \PY{p}{(}\PY{l+s+s2}{\PYZdq{}}\PY{l+s+s2}{Number of testing examples: }\PY{l+s+s2}{\PYZdq{}} \PY{o}{+} \PY{n+nb}{str}\PY{p}{(}\PY{n}{m\PYZus{}test}\PY{p}{)}\PY{p}{)}
\PY{n+nb}{print} \PY{p}{(}\PY{l+s+s2}{\PYZdq{}}\PY{l+s+s2}{Each image is of size: (}\PY{l+s+s2}{\PYZdq{}} \PY{o}{+} \PY{n+nb}{str}\PY{p}{(}\PY{n}{num\PYZus{}px}\PY{p}{)} \PY{o}{+} \PY{l+s+s2}{\PYZdq{}}\PY{l+s+s2}{, }\PY{l+s+s2}{\PYZdq{}} \PY{o}{+} \PY{n+nb}{str}\PY{p}{(}\PY{n}{num\PYZus{}px}\PY{p}{)} \PY{o}{+} \PY{l+s+s2}{\PYZdq{}}\PY{l+s+s2}{, 3)}\PY{l+s+s2}{\PYZdq{}}\PY{p}{)}
\PY{n+nb}{print} \PY{p}{(}\PY{l+s+s2}{\PYZdq{}}\PY{l+s+s2}{train\PYZus{}x\PYZus{}orig shape: }\PY{l+s+s2}{\PYZdq{}} \PY{o}{+} \PY{n+nb}{str}\PY{p}{(}\PY{n}{train\PYZus{}x\PYZus{}orig}\PY{o}{.}\PY{n}{shape}\PY{p}{)}\PY{p}{)}
\PY{n+nb}{print} \PY{p}{(}\PY{l+s+s2}{\PYZdq{}}\PY{l+s+s2}{train\PYZus{}y shape: }\PY{l+s+s2}{\PYZdq{}} \PY{o}{+} \PY{n+nb}{str}\PY{p}{(}\PY{n}{train\PYZus{}y}\PY{o}{.}\PY{n}{shape}\PY{p}{)}\PY{p}{)}
\PY{n+nb}{print} \PY{p}{(}\PY{l+s+s2}{\PYZdq{}}\PY{l+s+s2}{test\PYZus{}x\PYZus{}orig shape: }\PY{l+s+s2}{\PYZdq{}} \PY{o}{+} \PY{n+nb}{str}\PY{p}{(}\PY{n}{test\PYZus{}x\PYZus{}orig}\PY{o}{.}\PY{n}{shape}\PY{p}{)}\PY{p}{)}
\PY{n+nb}{print} \PY{p}{(}\PY{l+s+s2}{\PYZdq{}}\PY{l+s+s2}{test\PYZus{}y shape: }\PY{l+s+s2}{\PYZdq{}} \PY{o}{+} \PY{n+nb}{str}\PY{p}{(}\PY{n}{test\PYZus{}y}\PY{o}{.}\PY{n}{shape}\PY{p}{)}\PY{p}{)}
\end{Verbatim}
\end{tcolorbox}

    \begin{Verbatim}[commandchars=\\\{\}]
Number of training examples: 209
Number of testing examples: 50
Each image is of size: (64, 64, 3)
train\_x\_orig shape: (209, 64, 64, 3)
train\_y shape: (1, 209)
test\_x\_orig shape: (50, 64, 64, 3)
test\_y shape: (1, 50)
    \end{Verbatim}

    As usual, you reshape and standardize the images before feeding them to
the network. The code is given in the cell below.

Figure 1: Image to vector conversion.

    \begin{tcolorbox}[breakable, size=fbox, boxrule=1pt, pad at break*=1mm,colback=cellbackground, colframe=cellborder]
\prompt{In}{incolor}{6}{\boxspacing}
\begin{Verbatim}[commandchars=\\\{\}]
\PY{c+c1}{\PYZsh{} Reshape the training and test examples }
\PY{n}{train\PYZus{}x\PYZus{}flatten} \PY{o}{=} \PY{n}{train\PYZus{}x\PYZus{}orig}\PY{o}{.}\PY{n}{reshape}\PY{p}{(}\PY{n}{train\PYZus{}x\PYZus{}orig}\PY{o}{.}\PY{n}{shape}\PY{p}{[}\PY{l+m+mi}{0}\PY{p}{]}\PY{p}{,} \PY{o}{\PYZhy{}}\PY{l+m+mi}{1}\PY{p}{)}\PY{o}{.}\PY{n}{T}   \PY{c+c1}{\PYZsh{} The \PYZdq{}\PYZhy{}1\PYZdq{} makes reshape flatten the remaining dimensions}
\PY{n}{test\PYZus{}x\PYZus{}flatten} \PY{o}{=} \PY{n}{test\PYZus{}x\PYZus{}orig}\PY{o}{.}\PY{n}{reshape}\PY{p}{(}\PY{n}{test\PYZus{}x\PYZus{}orig}\PY{o}{.}\PY{n}{shape}\PY{p}{[}\PY{l+m+mi}{0}\PY{p}{]}\PY{p}{,} \PY{o}{\PYZhy{}}\PY{l+m+mi}{1}\PY{p}{)}\PY{o}{.}\PY{n}{T}

\PY{c+c1}{\PYZsh{} Standardize data to have feature values between 0 and 1.}
\PY{n}{train\PYZus{}x} \PY{o}{=} \PY{n}{train\PYZus{}x\PYZus{}flatten}\PY{o}{/}\PY{l+m+mf}{255.}
\PY{n}{test\PYZus{}x} \PY{o}{=} \PY{n}{test\PYZus{}x\PYZus{}flatten}\PY{o}{/}\PY{l+m+mf}{255.}

\PY{n+nb}{print} \PY{p}{(}\PY{l+s+s2}{\PYZdq{}}\PY{l+s+s2}{train\PYZus{}x}\PY{l+s+s2}{\PYZsq{}}\PY{l+s+s2}{s shape: }\PY{l+s+s2}{\PYZdq{}} \PY{o}{+} \PY{n+nb}{str}\PY{p}{(}\PY{n}{train\PYZus{}x}\PY{o}{.}\PY{n}{shape}\PY{p}{)}\PY{p}{)}
\PY{n+nb}{print} \PY{p}{(}\PY{l+s+s2}{\PYZdq{}}\PY{l+s+s2}{test\PYZus{}x}\PY{l+s+s2}{\PYZsq{}}\PY{l+s+s2}{s shape: }\PY{l+s+s2}{\PYZdq{}} \PY{o}{+} \PY{n+nb}{str}\PY{p}{(}\PY{n}{test\PYZus{}x}\PY{o}{.}\PY{n}{shape}\PY{p}{)}\PY{p}{)}
\end{Verbatim}
\end{tcolorbox}

    \begin{Verbatim}[commandchars=\\\{\}]
train\_x's shape: (12288, 209)
test\_x's shape: (12288, 50)
    \end{Verbatim}

    \textbf{Note}: \(12,288\) equals \(64 \times 64 \times 3\), which is the
size of one reshaped image vector.

    \#\# 3 - Model Architecture

    \#\#\# 3.1 - 2-layer Neural Network

Now that you're familiar with the dataset, it's time to build a deep
neural network to distinguish cat images from non-cat images!

You're going to build two different models:

\begin{itemize}
\tightlist
\item
  A 2-layer neural network
\item
  An L-layer deep neural network
\end{itemize}

Then, you'll compare the performance of these models, and try out some
different values for \(L\).

Let's look at the two architectures:

Figure 2: 2-layer neural network. The model can be summarized as: INPUT
-\textgreater{} LINEAR -\textgreater{} RELU -\textgreater{} LINEAR
-\textgreater{} SIGMOID -\textgreater{} OUTPUT.

Detailed Architecture of Figure 2: - The input is a (64,64,3) image
which is flattened to a vector of size \((12288,1)\). - The
corresponding vector: \([x_0,x_1,...,x_{12287}]^T\) is then multiplied
by the weight matrix \(W^{[1]}\) of size \((n^{[1]}, 12288)\). - Then,
add a bias term and take its relu to get the following vector:
\([a_0^{[1]}, a_1^{[1]},..., a_{n^{[1]}-1}^{[1]}]^T\). - Repeat the same
process. - Multiply the resulting vector by \(W^{[2]}\) and add the
intercept (bias). - Finally, take the sigmoid of the result. If it's
greater than 0.5, classify it as a cat.

\#\#\# 3.2 - L-layer Deep Neural Network

It's pretty difficult to represent an L-layer deep neural network using
the above representation. However, here is a simplified network
representation:

Figure 3: L-layer neural network. The model can be summarized as:
{[}LINEAR -\textgreater{} RELU{]} \(\times\) (L-1) -\textgreater{}
LINEAR -\textgreater{} SIGMOID

Detailed Architecture of Figure 3: - The input is a (64,64,3) image
which is flattened to a vector of size (12288,1). - The corresponding
vector: \([x_0,x_1,...,x_{12287}]^T\) is then multiplied by the weight
matrix \(W^{[1]}\) and then you add the intercept \(b^{[1]}\). The
result is called the linear unit. - Next, take the relu of the linear
unit. This process could be repeated several times for each
\((W^{[l]}, b^{[l]})\) depending on the model architecture. - Finally,
take the sigmoid of the final linear unit. If it is greater than 0.5,
classify it as a cat.

\#\#\# 3.3 - General Methodology

As usual, you'll follow the Deep Learning methodology to build the
model:

\begin{enumerate}
\def\labelenumi{\arabic{enumi}.}
\tightlist
\item
  Initialize parameters / Define hyperparameters
\item
  Loop for num\_iterations:

  \begin{enumerate}
  \def\labelenumii{\alph{enumii}.}
  \tightlist
  \item
    Forward propagation
  \item
    Compute cost function
  \item
    Backward propagation
  \item
    Update parameters (using parameters, and grads from backprop)
  \end{enumerate}
\item
  Use trained parameters to predict labels
\end{enumerate}

Now go ahead and implement those two models!

    \#\# 4 - Two-layer Neural Network

\#\#\# Exercise 1 - two\_layer\_model

Use the helper functions you have implemented in the previous assignment
to build a 2-layer neural network with the following structure:
\emph{LINEAR -\textgreater{} RELU -\textgreater{} LINEAR -\textgreater{}
SIGMOID}. The functions and their inputs are:

\begin{Shaded}
\begin{Highlighting}[]
\KeywordTok{def}\NormalTok{ initialize\_parameters(n\_x, n\_h, n\_y):}
\NormalTok{    ...}
    \ControlFlowTok{return}\NormalTok{ parameters }
\KeywordTok{def}\NormalTok{ linear\_activation\_forward(A\_prev, W, b, activation):}
\NormalTok{    ...}
    \ControlFlowTok{return}\NormalTok{ A, cache}
\KeywordTok{def}\NormalTok{ compute\_cost(AL, Y):}
\NormalTok{    ...}
    \ControlFlowTok{return}\NormalTok{ cost}
\KeywordTok{def}\NormalTok{ linear\_activation\_backward(dA, cache, activation):}
\NormalTok{    ...}
    \ControlFlowTok{return}\NormalTok{ dA\_prev, dW, db}
\KeywordTok{def}\NormalTok{ update\_parameters(parameters, grads, learning\_rate):}
\NormalTok{    ...}
    \ControlFlowTok{return}\NormalTok{ parameters}
\end{Highlighting}
\end{Shaded}

    \begin{tcolorbox}[breakable, size=fbox, boxrule=1pt, pad at break*=1mm,colback=cellbackground, colframe=cellborder]
\prompt{In}{incolor}{7}{\boxspacing}
\begin{Verbatim}[commandchars=\\\{\}]
\PY{c+c1}{\PYZsh{}\PYZsh{}\PYZsh{} CONSTANTS DEFINING THE MODEL \PYZsh{}\PYZsh{}\PYZsh{}\PYZsh{}}
\PY{n}{n\PYZus{}x} \PY{o}{=} \PY{l+m+mi}{12288}     \PY{c+c1}{\PYZsh{} num\PYZus{}px * num\PYZus{}px * 3}
\PY{n}{n\PYZus{}h} \PY{o}{=} \PY{l+m+mi}{7}
\PY{n}{n\PYZus{}y} \PY{o}{=} \PY{l+m+mi}{1}
\PY{n}{layers\PYZus{}dims} \PY{o}{=} \PY{p}{(}\PY{n}{n\PYZus{}x}\PY{p}{,} \PY{n}{n\PYZus{}h}\PY{p}{,} \PY{n}{n\PYZus{}y}\PY{p}{)}
\PY{n}{learning\PYZus{}rate} \PY{o}{=} \PY{l+m+mf}{0.0075}
\end{Verbatim}
\end{tcolorbox}

    \begin{tcolorbox}[breakable, size=fbox, boxrule=1pt, pad at break*=1mm,colback=cellbackground, colframe=cellborder]
\prompt{In}{incolor}{13}{\boxspacing}
\begin{Verbatim}[commandchars=\\\{\}]
\PY{c+c1}{\PYZsh{} GRADED FUNCTION: two\PYZus{}layer\PYZus{}model}

\PY{k}{def} \PY{n+nf}{two\PYZus{}layer\PYZus{}model}\PY{p}{(}\PY{n}{X}\PY{p}{,} \PY{n}{Y}\PY{p}{,} \PY{n}{layers\PYZus{}dims}\PY{p}{,} \PY{n}{learning\PYZus{}rate} \PY{o}{=} \PY{l+m+mf}{0.0075}\PY{p}{,} \PY{n}{num\PYZus{}iterations} \PY{o}{=} \PY{l+m+mi}{3000}\PY{p}{,} \PY{n}{print\PYZus{}cost}\PY{o}{=}\PY{k+kc}{False}\PY{p}{)}\PY{p}{:}
    \PY{l+s+sd}{\PYZdq{}\PYZdq{}\PYZdq{}}
\PY{l+s+sd}{    Implements a two\PYZhy{}layer neural network: LINEAR\PYZhy{}\PYZgt{}RELU\PYZhy{}\PYZgt{}LINEAR\PYZhy{}\PYZgt{}SIGMOID.}
\PY{l+s+sd}{    }
\PY{l+s+sd}{    Arguments:}
\PY{l+s+sd}{    X \PYZhy{}\PYZhy{} input data, of shape (n\PYZus{}x, number of examples)}
\PY{l+s+sd}{    Y \PYZhy{}\PYZhy{} true \PYZdq{}label\PYZdq{} vector (containing 1 if cat, 0 if non\PYZhy{}cat), of shape (1, number of examples)}
\PY{l+s+sd}{    layers\PYZus{}dims \PYZhy{}\PYZhy{} dimensions of the layers (n\PYZus{}x, n\PYZus{}h, n\PYZus{}y)}
\PY{l+s+sd}{    num\PYZus{}iterations \PYZhy{}\PYZhy{} number of iterations of the optimization loop}
\PY{l+s+sd}{    learning\PYZus{}rate \PYZhy{}\PYZhy{} learning rate of the gradient descent update rule}
\PY{l+s+sd}{    print\PYZus{}cost \PYZhy{}\PYZhy{} If set to True, this will print the cost every 100 iterations }
\PY{l+s+sd}{    }
\PY{l+s+sd}{    Returns:}
\PY{l+s+sd}{    parameters \PYZhy{}\PYZhy{} a dictionary containing W1, W2, b1, and b2}
\PY{l+s+sd}{    \PYZdq{}\PYZdq{}\PYZdq{}}
    
    \PY{n}{np}\PY{o}{.}\PY{n}{random}\PY{o}{.}\PY{n}{seed}\PY{p}{(}\PY{l+m+mi}{1}\PY{p}{)}
    \PY{n}{grads} \PY{o}{=} \PY{p}{\PYZob{}}\PY{p}{\PYZcb{}}
    \PY{n}{costs} \PY{o}{=} \PY{p}{[}\PY{p}{]}                              \PY{c+c1}{\PYZsh{} to keep track of the cost}
    \PY{n}{m} \PY{o}{=} \PY{n}{X}\PY{o}{.}\PY{n}{shape}\PY{p}{[}\PY{l+m+mi}{1}\PY{p}{]}                           \PY{c+c1}{\PYZsh{} number of examples}
    \PY{p}{(}\PY{n}{n\PYZus{}x}\PY{p}{,} \PY{n}{n\PYZus{}h}\PY{p}{,} \PY{n}{n\PYZus{}y}\PY{p}{)} \PY{o}{=} \PY{n}{layers\PYZus{}dims}
    
    \PY{c+c1}{\PYZsh{} Initialize parameters dictionary, by calling one of the functions you\PYZsq{}d previously implemented}
    \PY{c+c1}{\PYZsh{}(≈ 1 line of code)}
    \PY{c+c1}{\PYZsh{} parameters = ...}
    \PY{c+c1}{\PYZsh{} YOUR CODE STARTS HERE}
    \PY{n}{parameters} \PY{o}{=} \PY{n}{initialize\PYZus{}parameters}\PY{p}{(}\PY{n}{n\PYZus{}x}\PY{p}{,} \PY{n}{n\PYZus{}h}\PY{p}{,} \PY{n}{n\PYZus{}y}\PY{p}{)}
    
    \PY{c+c1}{\PYZsh{} YOUR CODE ENDS HERE}
    
    \PY{c+c1}{\PYZsh{} Get W1, b1, W2 and b2 from the dictionary parameters.}
    \PY{n}{W1} \PY{o}{=} \PY{n}{parameters}\PY{p}{[}\PY{l+s+s2}{\PYZdq{}}\PY{l+s+s2}{W1}\PY{l+s+s2}{\PYZdq{}}\PY{p}{]}
    \PY{n}{b1} \PY{o}{=} \PY{n}{parameters}\PY{p}{[}\PY{l+s+s2}{\PYZdq{}}\PY{l+s+s2}{b1}\PY{l+s+s2}{\PYZdq{}}\PY{p}{]}
    \PY{n}{W2} \PY{o}{=} \PY{n}{parameters}\PY{p}{[}\PY{l+s+s2}{\PYZdq{}}\PY{l+s+s2}{W2}\PY{l+s+s2}{\PYZdq{}}\PY{p}{]}
    \PY{n}{b2} \PY{o}{=} \PY{n}{parameters}\PY{p}{[}\PY{l+s+s2}{\PYZdq{}}\PY{l+s+s2}{b2}\PY{l+s+s2}{\PYZdq{}}\PY{p}{]}
    
    \PY{c+c1}{\PYZsh{} Loop (gradient descent)}

    \PY{k}{for} \PY{n}{i} \PY{o+ow}{in} \PY{n+nb}{range}\PY{p}{(}\PY{l+m+mi}{0}\PY{p}{,} \PY{n}{num\PYZus{}iterations}\PY{p}{)}\PY{p}{:}

        \PY{c+c1}{\PYZsh{} Forward propagation: LINEAR \PYZhy{}\PYZgt{} RELU \PYZhy{}\PYZgt{} LINEAR \PYZhy{}\PYZgt{} SIGMOID. Inputs: \PYZdq{}X, W1, b1, W2, b2\PYZdq{}. Output: \PYZdq{}A1, cache1, A2, cache2\PYZdq{}.}
        \PY{c+c1}{\PYZsh{}(≈ 2 lines of code)}
        \PY{c+c1}{\PYZsh{} A1, cache1 = ...}
        \PY{c+c1}{\PYZsh{} A2, cache2 = ...}
        \PY{c+c1}{\PYZsh{} YOUR CODE STARTS HERE}
        \PY{n}{A1}\PY{p}{,} \PY{n}{cache1} \PY{o}{=} \PY{n}{linear\PYZus{}activation\PYZus{}forward}\PY{p}{(}\PY{n}{X}\PY{p}{,} \PY{n}{W1}\PY{p}{,} \PY{n}{b1}\PY{p}{,} \PY{n}{activation} \PY{o}{=} \PY{l+s+s2}{\PYZdq{}}\PY{l+s+s2}{relu}\PY{l+s+s2}{\PYZdq{}}\PY{p}{)}
        \PY{n}{A2}\PY{p}{,} \PY{n}{cache2} \PY{o}{=} \PY{n}{linear\PYZus{}activation\PYZus{}forward}\PY{p}{(}\PY{n}{A1}\PY{p}{,} \PY{n}{W2}\PY{p}{,} \PY{n}{b2}\PY{p}{,} \PY{n}{activation} \PY{o}{=} \PY{l+s+s2}{\PYZdq{}}\PY{l+s+s2}{sigmoid}\PY{l+s+s2}{\PYZdq{}}\PY{p}{)}
        
        \PY{c+c1}{\PYZsh{} YOUR CODE ENDS HERE}
        
        \PY{c+c1}{\PYZsh{} Compute cost}
        \PY{c+c1}{\PYZsh{}(≈ 1 line of code)}
        \PY{c+c1}{\PYZsh{} cost = ...}
        \PY{c+c1}{\PYZsh{} YOUR CODE STARTS HERE}
        \PY{n}{cost} \PY{o}{=} \PY{n}{compute\PYZus{}cost}\PY{p}{(}\PY{n}{A2}\PY{p}{,} \PY{n}{Y}\PY{p}{)}
        
        \PY{c+c1}{\PYZsh{} YOUR CODE ENDS HERE}
        
        \PY{c+c1}{\PYZsh{} Initializing backward propagation}
        \PY{n}{dA2} \PY{o}{=} \PY{o}{\PYZhy{}} \PY{p}{(}\PY{n}{np}\PY{o}{.}\PY{n}{divide}\PY{p}{(}\PY{n}{Y}\PY{p}{,} \PY{n}{A2}\PY{p}{)} \PY{o}{\PYZhy{}} \PY{n}{np}\PY{o}{.}\PY{n}{divide}\PY{p}{(}\PY{l+m+mi}{1} \PY{o}{\PYZhy{}} \PY{n}{Y}\PY{p}{,} \PY{l+m+mi}{1} \PY{o}{\PYZhy{}} \PY{n}{A2}\PY{p}{)}\PY{p}{)}
        
        \PY{c+c1}{\PYZsh{} Backward propagation. Inputs: \PYZdq{}dA2, cache2, cache1\PYZdq{}. Outputs: \PYZdq{}dA1, dW2, db2; also dA0 (not used), dW1, db1\PYZdq{}.}
        \PY{c+c1}{\PYZsh{}(≈ 2 lines of code)}
        \PY{c+c1}{\PYZsh{} dA1, dW2, db2 = ...}
        \PY{c+c1}{\PYZsh{} dA0, dW1, db1 = ...}
        \PY{c+c1}{\PYZsh{} YOUR CODE STARTS HERE}
        \PY{n}{dA1}\PY{p}{,} \PY{n}{dW2}\PY{p}{,} \PY{n}{db2} \PY{o}{=} \PY{n}{linear\PYZus{}activation\PYZus{}backward}\PY{p}{(}\PY{n}{dA2}\PY{p}{,} \PY{n}{cache2}\PY{p}{,} \PY{n}{activation} \PY{o}{=} \PY{l+s+s2}{\PYZdq{}}\PY{l+s+s2}{sigmoid}\PY{l+s+s2}{\PYZdq{}}\PY{p}{)}
        \PY{n}{dA0}\PY{p}{,} \PY{n}{dW1}\PY{p}{,} \PY{n}{db1} \PY{o}{=} \PY{n}{linear\PYZus{}activation\PYZus{}backward}\PY{p}{(}\PY{n}{dA1}\PY{p}{,} \PY{n}{cache1}\PY{p}{,} \PY{n}{activation} \PY{o}{=} \PY{l+s+s2}{\PYZdq{}}\PY{l+s+s2}{relu}\PY{l+s+s2}{\PYZdq{}}\PY{p}{)}
        
        \PY{c+c1}{\PYZsh{} YOUR CODE ENDS HERE}
        
        \PY{c+c1}{\PYZsh{} Set grads[\PYZsq{}dWl\PYZsq{}] to dW1, grads[\PYZsq{}db1\PYZsq{}] to db1, grads[\PYZsq{}dW2\PYZsq{}] to dW2, grads[\PYZsq{}db2\PYZsq{}] to db2}
        \PY{n}{grads}\PY{p}{[}\PY{l+s+s1}{\PYZsq{}}\PY{l+s+s1}{dW1}\PY{l+s+s1}{\PYZsq{}}\PY{p}{]} \PY{o}{=} \PY{n}{dW1}
        \PY{n}{grads}\PY{p}{[}\PY{l+s+s1}{\PYZsq{}}\PY{l+s+s1}{db1}\PY{l+s+s1}{\PYZsq{}}\PY{p}{]} \PY{o}{=} \PY{n}{db1}
        \PY{n}{grads}\PY{p}{[}\PY{l+s+s1}{\PYZsq{}}\PY{l+s+s1}{dW2}\PY{l+s+s1}{\PYZsq{}}\PY{p}{]} \PY{o}{=} \PY{n}{dW2}
        \PY{n}{grads}\PY{p}{[}\PY{l+s+s1}{\PYZsq{}}\PY{l+s+s1}{db2}\PY{l+s+s1}{\PYZsq{}}\PY{p}{]} \PY{o}{=} \PY{n}{db2}
        
        \PY{c+c1}{\PYZsh{} Update parameters.}
        \PY{c+c1}{\PYZsh{}(approx. 1 line of code)}
        \PY{c+c1}{\PYZsh{} parameters = ...}
        \PY{c+c1}{\PYZsh{} YOUR CODE STARTS HERE}
        \PY{n}{parameters} \PY{o}{=} \PY{n}{update\PYZus{}parameters}\PY{p}{(}\PY{n}{parameters}\PY{p}{,} \PY{n}{grads}\PY{p}{,} \PY{n}{learning\PYZus{}rate}\PY{p}{)}
        
        \PY{c+c1}{\PYZsh{} YOUR CODE ENDS HERE}

        \PY{c+c1}{\PYZsh{} Retrieve W1, b1, W2, b2 from parameters}
        \PY{n}{W1} \PY{o}{=} \PY{n}{parameters}\PY{p}{[}\PY{l+s+s2}{\PYZdq{}}\PY{l+s+s2}{W1}\PY{l+s+s2}{\PYZdq{}}\PY{p}{]}
        \PY{n}{b1} \PY{o}{=} \PY{n}{parameters}\PY{p}{[}\PY{l+s+s2}{\PYZdq{}}\PY{l+s+s2}{b1}\PY{l+s+s2}{\PYZdq{}}\PY{p}{]}
        \PY{n}{W2} \PY{o}{=} \PY{n}{parameters}\PY{p}{[}\PY{l+s+s2}{\PYZdq{}}\PY{l+s+s2}{W2}\PY{l+s+s2}{\PYZdq{}}\PY{p}{]}
        \PY{n}{b2} \PY{o}{=} \PY{n}{parameters}\PY{p}{[}\PY{l+s+s2}{\PYZdq{}}\PY{l+s+s2}{b2}\PY{l+s+s2}{\PYZdq{}}\PY{p}{]}
        
        \PY{c+c1}{\PYZsh{} Print the cost every 100 iterations}
        \PY{k}{if} \PY{n}{print\PYZus{}cost} \PY{o+ow}{and} \PY{n}{i} \PY{o}{\PYZpc{}} \PY{l+m+mi}{100} \PY{o}{==} \PY{l+m+mi}{0} \PY{o+ow}{or} \PY{n}{i} \PY{o}{==} \PY{n}{num\PYZus{}iterations} \PY{o}{\PYZhy{}} \PY{l+m+mi}{1}\PY{p}{:}
            \PY{n+nb}{print}\PY{p}{(}\PY{l+s+s2}{\PYZdq{}}\PY{l+s+s2}{Cost after iteration }\PY{l+s+si}{\PYZob{}\PYZcb{}}\PY{l+s+s2}{: }\PY{l+s+si}{\PYZob{}\PYZcb{}}\PY{l+s+s2}{\PYZdq{}}\PY{o}{.}\PY{n}{format}\PY{p}{(}\PY{n}{i}\PY{p}{,} \PY{n}{np}\PY{o}{.}\PY{n}{squeeze}\PY{p}{(}\PY{n}{cost}\PY{p}{)}\PY{p}{)}\PY{p}{)}
        \PY{k}{if} \PY{n}{i} \PY{o}{\PYZpc{}} \PY{l+m+mi}{100} \PY{o}{==} \PY{l+m+mi}{0} \PY{o+ow}{or} \PY{n}{i} \PY{o}{==} \PY{n}{num\PYZus{}iterations}\PY{p}{:}
            \PY{n}{costs}\PY{o}{.}\PY{n}{append}\PY{p}{(}\PY{n}{cost}\PY{p}{)}

    \PY{k}{return} \PY{n}{parameters}\PY{p}{,} \PY{n}{costs}

\PY{k}{def} \PY{n+nf}{plot\PYZus{}costs}\PY{p}{(}\PY{n}{costs}\PY{p}{,} \PY{n}{learning\PYZus{}rate}\PY{o}{=}\PY{l+m+mf}{0.0075}\PY{p}{)}\PY{p}{:}
    \PY{n}{plt}\PY{o}{.}\PY{n}{plot}\PY{p}{(}\PY{n}{np}\PY{o}{.}\PY{n}{squeeze}\PY{p}{(}\PY{n}{costs}\PY{p}{)}\PY{p}{)}
    \PY{n}{plt}\PY{o}{.}\PY{n}{ylabel}\PY{p}{(}\PY{l+s+s1}{\PYZsq{}}\PY{l+s+s1}{cost}\PY{l+s+s1}{\PYZsq{}}\PY{p}{)}
    \PY{n}{plt}\PY{o}{.}\PY{n}{xlabel}\PY{p}{(}\PY{l+s+s1}{\PYZsq{}}\PY{l+s+s1}{iterations (per hundreds)}\PY{l+s+s1}{\PYZsq{}}\PY{p}{)}
    \PY{n}{plt}\PY{o}{.}\PY{n}{title}\PY{p}{(}\PY{l+s+s2}{\PYZdq{}}\PY{l+s+s2}{Learning rate =}\PY{l+s+s2}{\PYZdq{}} \PY{o}{+} \PY{n+nb}{str}\PY{p}{(}\PY{n}{learning\PYZus{}rate}\PY{p}{)}\PY{p}{)}
    \PY{n}{plt}\PY{o}{.}\PY{n}{show}\PY{p}{(}\PY{p}{)}
\end{Verbatim}
\end{tcolorbox}

    \begin{tcolorbox}[breakable, size=fbox, boxrule=1pt, pad at break*=1mm,colback=cellbackground, colframe=cellborder]
\prompt{In}{incolor}{14}{\boxspacing}
\begin{Verbatim}[commandchars=\\\{\}]
\PY{n}{parameters}\PY{p}{,} \PY{n}{costs} \PY{o}{=} \PY{n}{two\PYZus{}layer\PYZus{}model}\PY{p}{(}\PY{n}{train\PYZus{}x}\PY{p}{,} \PY{n}{train\PYZus{}y}\PY{p}{,} \PY{n}{layers\PYZus{}dims} \PY{o}{=} \PY{p}{(}\PY{n}{n\PYZus{}x}\PY{p}{,} \PY{n}{n\PYZus{}h}\PY{p}{,} \PY{n}{n\PYZus{}y}\PY{p}{)}\PY{p}{,} \PY{n}{num\PYZus{}iterations} \PY{o}{=} \PY{l+m+mi}{2}\PY{p}{,} \PY{n}{print\PYZus{}cost}\PY{o}{=}\PY{k+kc}{False}\PY{p}{)}

\PY{n+nb}{print}\PY{p}{(}\PY{l+s+s2}{\PYZdq{}}\PY{l+s+s2}{Cost after first iteration: }\PY{l+s+s2}{\PYZdq{}} \PY{o}{+} \PY{n+nb}{str}\PY{p}{(}\PY{n}{costs}\PY{p}{[}\PY{l+m+mi}{0}\PY{p}{]}\PY{p}{)}\PY{p}{)}

\PY{n}{two\PYZus{}layer\PYZus{}model\PYZus{}test}\PY{p}{(}\PY{n}{two\PYZus{}layer\PYZus{}model}\PY{p}{)}
\end{Verbatim}
\end{tcolorbox}

    \begin{Verbatim}[commandchars=\\\{\}]
Cost after iteration 1: 0.6926114346158595
Cost after first iteration: 0.693049735659989
Cost after iteration 1: 0.6915746967050506
Cost after iteration 1: 0.6915746967050506
Cost after iteration 1: 0.6915746967050506
Cost after iteration 2: 0.6524135179683452
\textcolor{ansi-green-intense}{ All tests passed.}
    \end{Verbatim}

    \textbf{Expected output:}

\begin{verbatim}
cost after iteration 1 must be around 0.69
\end{verbatim}

    \#\#\# 4.1 - Train the model

If your code passed the previous cell, run the cell below to train your
parameters.

\begin{itemize}
\item
  The cost should decrease on every iteration.
\item
  It may take up to 5 minutes to run 2500 iterations.
\end{itemize}

    \begin{tcolorbox}[breakable, size=fbox, boxrule=1pt, pad at break*=1mm,colback=cellbackground, colframe=cellborder]
\prompt{In}{incolor}{17}{\boxspacing}
\begin{Verbatim}[commandchars=\\\{\}]
\PY{n}{parameters}\PY{p}{,} \PY{n}{costs} \PY{o}{=} \PY{n}{two\PYZus{}layer\PYZus{}model}\PY{p}{(}\PY{n}{train\PYZus{}x}\PY{p}{,} \PY{n}{train\PYZus{}y}\PY{p}{,} \PY{n}{layers\PYZus{}dims} \PY{o}{=} \PY{p}{(}\PY{n}{n\PYZus{}x}\PY{p}{,} \PY{n}{n\PYZus{}h}\PY{p}{,} \PY{n}{n\PYZus{}y}\PY{p}{)}\PY{p}{,} \PY{n}{num\PYZus{}iterations} \PY{o}{=} \PY{l+m+mi}{2500}\PY{p}{,} \PY{n}{print\PYZus{}cost}\PY{o}{=}\PY{k+kc}{True}\PY{p}{)}
\PY{n}{plot\PYZus{}costs}\PY{p}{(}\PY{n}{costs}\PY{p}{,} \PY{n}{learning\PYZus{}rate}\PY{p}{)}
\end{Verbatim}
\end{tcolorbox}

    \begin{Verbatim}[commandchars=\\\{\}]
Cost after iteration 0: 0.693049735659989
Cost after iteration 100: 0.6464320953428849
Cost after iteration 200: 0.6325140647912677
Cost after iteration 300: 0.6015024920354665
Cost after iteration 400: 0.5601966311605747
Cost after iteration 500: 0.5158304772764729
Cost after iteration 600: 0.4754901313943325
Cost after iteration 700: 0.43391631512257495
Cost after iteration 800: 0.4007977536203886
Cost after iteration 900: 0.3580705011323798
Cost after iteration 1000: 0.3394281538366413
Cost after iteration 1100: 0.30527536361962654
Cost after iteration 1200: 0.2749137728213015
Cost after iteration 1300: 0.2468176821061484
Cost after iteration 1400: 0.19850735037466102
Cost after iteration 1500: 0.17448318112556638
Cost after iteration 1600: 0.1708076297809692
Cost after iteration 1700: 0.11306524562164715
Cost after iteration 1800: 0.09629426845937156
Cost after iteration 1900: 0.0834261795972687
Cost after iteration 2000: 0.07439078704319085
Cost after iteration 2100: 0.06630748132267933
Cost after iteration 2200: 0.05919329501038172
Cost after iteration 2300: 0.053361403485605606
Cost after iteration 2400: 0.04855478562877019
Cost after iteration 2499: 0.04421498215868956
    \end{Verbatim}

    \begin{center}
    \adjustimage{max size={0.9\linewidth}{0.9\paperheight}}{output_21_1.png}
    \end{center}
    { \hspace*{\fill} \\}
    
    \textbf{Expected Output}:

Cost after iteration 0

0.6930497356599888

Cost after iteration 100

0.6464320953428849

\ldots{}

\ldots{}

Cost after iteration 2499

0.04421498215868956

    \textbf{Nice!} You successfully trained the model. Good thing you built
a vectorized implementation! Otherwise it might have taken 10 times
longer to train this.

Now, you can use the trained parameters to classify images from the
dataset. To see your predictions on the training and test sets, run the
cell below.

    \begin{tcolorbox}[breakable, size=fbox, boxrule=1pt, pad at break*=1mm,colback=cellbackground, colframe=cellborder]
\prompt{In}{incolor}{18}{\boxspacing}
\begin{Verbatim}[commandchars=\\\{\}]
\PY{n}{predictions\PYZus{}train} \PY{o}{=} \PY{n}{predict}\PY{p}{(}\PY{n}{train\PYZus{}x}\PY{p}{,} \PY{n}{train\PYZus{}y}\PY{p}{,} \PY{n}{parameters}\PY{p}{)}
\end{Verbatim}
\end{tcolorbox}

    \begin{Verbatim}[commandchars=\\\{\}]
Accuracy: 0.9999999999999998
    \end{Verbatim}

    \textbf{Expected Output}:

Accuracy

0.9999999999999998

    \begin{tcolorbox}[breakable, size=fbox, boxrule=1pt, pad at break*=1mm,colback=cellbackground, colframe=cellborder]
\prompt{In}{incolor}{19}{\boxspacing}
\begin{Verbatim}[commandchars=\\\{\}]
\PY{n}{predictions\PYZus{}test} \PY{o}{=} \PY{n}{predict}\PY{p}{(}\PY{n}{test\PYZus{}x}\PY{p}{,} \PY{n}{test\PYZus{}y}\PY{p}{,} \PY{n}{parameters}\PY{p}{)}
\end{Verbatim}
\end{tcolorbox}

    \begin{Verbatim}[commandchars=\\\{\}]
Accuracy: 0.72
    \end{Verbatim}

    \textbf{Expected Output}:

Accuracy

0.72

    \hypertarget{congratulations-it-seems-that-your-2-layer-neural-network-has-better-performance-72-than-the-logistic-regression-implementation-70-assignment-week-2.-lets-see-if-you-can-do-even-better-with-an-l-layer-model.}{%
\subsubsection{\texorpdfstring{Congratulations! It seems that your
2-layer neural network has better performance (72\%) than the logistic
regression implementation (70\%, assignment week 2). Let's see if you
can do even better with an \(L\)-layer
model.}{Congratulations! It seems that your 2-layer neural network has better performance (72\%) than the logistic regression implementation (70\%, assignment week 2). Let's see if you can do even better with an L-layer model.}}\label{congratulations-it-seems-that-your-2-layer-neural-network-has-better-performance-72-than-the-logistic-regression-implementation-70-assignment-week-2.-lets-see-if-you-can-do-even-better-with-an-l-layer-model.}}

\textbf{Note}: You may notice that running the model on fewer iterations
(say 1500) gives better accuracy on the test set. This is called ``early
stopping'' and you'll hear more about it in the next course. Early
stopping is a way to prevent overfitting.

    \#\# 5 - L-layer Neural Network

\#\#\# Exercise 2 - L\_layer\_model

Use the helper functions you implemented previously to build an
\(L\)-layer neural network with the following structure: \emph{{[}LINEAR
-\textgreater{} RELU{]}\(\times\)(L-1) -\textgreater{} LINEAR
-\textgreater{} SIGMOID}. The functions and their inputs are:

\begin{Shaded}
\begin{Highlighting}[]
\KeywordTok{def}\NormalTok{ initialize\_parameters\_deep(layers\_dims):}
\NormalTok{    ...}
    \ControlFlowTok{return}\NormalTok{ parameters }
\KeywordTok{def}\NormalTok{ L\_model\_forward(X, parameters):}
\NormalTok{    ...}
    \ControlFlowTok{return}\NormalTok{ AL, caches}
\KeywordTok{def}\NormalTok{ compute\_cost(AL, Y):}
\NormalTok{    ...}
    \ControlFlowTok{return}\NormalTok{ cost}
\KeywordTok{def}\NormalTok{ L\_model\_backward(AL, Y, caches):}
\NormalTok{    ...}
    \ControlFlowTok{return}\NormalTok{ grads}
\KeywordTok{def}\NormalTok{ update\_parameters(parameters, grads, learning\_rate):}
\NormalTok{    ...}
    \ControlFlowTok{return}\NormalTok{ parameters}
\end{Highlighting}
\end{Shaded}

    \begin{tcolorbox}[breakable, size=fbox, boxrule=1pt, pad at break*=1mm,colback=cellbackground, colframe=cellborder]
\prompt{In}{incolor}{ }{\boxspacing}
\begin{Verbatim}[commandchars=\\\{\}]
\PY{c+c1}{\PYZsh{}\PYZsh{}\PYZsh{} CONSTANTS \PYZsh{}\PYZsh{}\PYZsh{}}
\PY{n}{layers\PYZus{}dims} \PY{o}{=} \PY{p}{[}\PY{l+m+mi}{12288}\PY{p}{,} \PY{l+m+mi}{20}\PY{p}{,} \PY{l+m+mi}{7}\PY{p}{,} \PY{l+m+mi}{5}\PY{p}{,} \PY{l+m+mi}{1}\PY{p}{]} \PY{c+c1}{\PYZsh{}  4\PYZhy{}layer model}
\end{Verbatim}
\end{tcolorbox}

    \begin{tcolorbox}[breakable, size=fbox, boxrule=1pt, pad at break*=1mm,colback=cellbackground, colframe=cellborder]
\prompt{In}{incolor}{20}{\boxspacing}
\begin{Verbatim}[commandchars=\\\{\}]
\PY{c+c1}{\PYZsh{} GRADED FUNCTION: L\PYZus{}layer\PYZus{}model}

\PY{k}{def} \PY{n+nf}{L\PYZus{}layer\PYZus{}model}\PY{p}{(}\PY{n}{X}\PY{p}{,} \PY{n}{Y}\PY{p}{,} \PY{n}{layers\PYZus{}dims}\PY{p}{,} \PY{n}{learning\PYZus{}rate} \PY{o}{=} \PY{l+m+mf}{0.0075}\PY{p}{,} \PY{n}{num\PYZus{}iterations} \PY{o}{=} \PY{l+m+mi}{3000}\PY{p}{,} \PY{n}{print\PYZus{}cost}\PY{o}{=}\PY{k+kc}{False}\PY{p}{)}\PY{p}{:}
    \PY{l+s+sd}{\PYZdq{}\PYZdq{}\PYZdq{}}
\PY{l+s+sd}{    Implements a L\PYZhy{}layer neural network: [LINEAR\PYZhy{}\PYZgt{}RELU]*(L\PYZhy{}1)\PYZhy{}\PYZgt{}LINEAR\PYZhy{}\PYZgt{}SIGMOID.}
\PY{l+s+sd}{    }
\PY{l+s+sd}{    Arguments:}
\PY{l+s+sd}{    X \PYZhy{}\PYZhy{} data, numpy array of shape (num\PYZus{}px * num\PYZus{}px * 3, number of examples)}
\PY{l+s+sd}{    Y \PYZhy{}\PYZhy{} true \PYZdq{}label\PYZdq{} vector (containing 0 if cat, 1 if non\PYZhy{}cat), of shape (1, number of examples)}
\PY{l+s+sd}{    layers\PYZus{}dims \PYZhy{}\PYZhy{} list containing the input size and each layer size, of length (number of layers + 1).}
\PY{l+s+sd}{    learning\PYZus{}rate \PYZhy{}\PYZhy{} learning rate of the gradient descent update rule}
\PY{l+s+sd}{    num\PYZus{}iterations \PYZhy{}\PYZhy{} number of iterations of the optimization loop}
\PY{l+s+sd}{    print\PYZus{}cost \PYZhy{}\PYZhy{} if True, it prints the cost every 100 steps}
\PY{l+s+sd}{    }
\PY{l+s+sd}{    Returns:}
\PY{l+s+sd}{    parameters \PYZhy{}\PYZhy{} parameters learnt by the model. They can then be used to predict.}
\PY{l+s+sd}{    \PYZdq{}\PYZdq{}\PYZdq{}}

    \PY{n}{np}\PY{o}{.}\PY{n}{random}\PY{o}{.}\PY{n}{seed}\PY{p}{(}\PY{l+m+mi}{1}\PY{p}{)}
    \PY{n}{costs} \PY{o}{=} \PY{p}{[}\PY{p}{]}                         \PY{c+c1}{\PYZsh{} keep track of cost}
    
    \PY{c+c1}{\PYZsh{} Parameters initialization.}
    \PY{c+c1}{\PYZsh{}(≈ 1 line of code)}
    \PY{c+c1}{\PYZsh{} parameters = ...}
    \PY{c+c1}{\PYZsh{} YOUR CODE STARTS HERE}
    \PY{n}{parameters} \PY{o}{=} \PY{n}{initialize\PYZus{}parameters\PYZus{}deep}\PY{p}{(}\PY{n}{layers\PYZus{}dims}\PY{p}{)}
    
    \PY{c+c1}{\PYZsh{} YOUR CODE ENDS HERE}
    
    \PY{c+c1}{\PYZsh{} Loop (gradient descent)}
    \PY{k}{for} \PY{n}{i} \PY{o+ow}{in} \PY{n+nb}{range}\PY{p}{(}\PY{l+m+mi}{0}\PY{p}{,} \PY{n}{num\PYZus{}iterations}\PY{p}{)}\PY{p}{:}

        \PY{c+c1}{\PYZsh{} Forward propagation: [LINEAR \PYZhy{}\PYZgt{} RELU]*(L\PYZhy{}1) \PYZhy{}\PYZgt{} LINEAR \PYZhy{}\PYZgt{} SIGMOID.}
        \PY{c+c1}{\PYZsh{}(≈ 1 line of code)}
        \PY{c+c1}{\PYZsh{} AL, caches = ...}
        \PY{c+c1}{\PYZsh{} YOUR CODE STARTS HERE}
        \PY{n}{AL}\PY{p}{,} \PY{n}{caches} \PY{o}{=} \PY{n}{L\PYZus{}model\PYZus{}forward}\PY{p}{(}\PY{n}{X}\PY{p}{,} \PY{n}{parameters}\PY{p}{)}
        
        \PY{c+c1}{\PYZsh{} YOUR CODE ENDS HERE}
        
        \PY{c+c1}{\PYZsh{} Compute cost.}
        \PY{c+c1}{\PYZsh{}(≈ 1 line of code)}
        \PY{c+c1}{\PYZsh{} cost = ...}
        \PY{c+c1}{\PYZsh{} YOUR CODE STARTS HERE}
        \PY{n}{cost} \PY{o}{=} \PY{n}{compute\PYZus{}cost}\PY{p}{(}\PY{n}{AL}\PY{p}{,} \PY{n}{Y}\PY{p}{)}
        
        \PY{c+c1}{\PYZsh{} YOUR CODE ENDS HERE}
    
        \PY{c+c1}{\PYZsh{} Backward propagation.}
        \PY{c+c1}{\PYZsh{}(≈ 1 line of code)}
        \PY{c+c1}{\PYZsh{} grads = ...    }
        \PY{c+c1}{\PYZsh{} YOUR CODE STARTS HERE}
        \PY{n}{grads} \PY{o}{=} \PY{n}{L\PYZus{}model\PYZus{}backward}\PY{p}{(}\PY{n}{AL}\PY{p}{,} \PY{n}{Y}\PY{p}{,} \PY{n}{caches}\PY{p}{)}
        
        \PY{c+c1}{\PYZsh{} YOUR CODE ENDS HERE}
 
        \PY{c+c1}{\PYZsh{} Update parameters.}
        \PY{c+c1}{\PYZsh{}(≈ 1 line of code)}
        \PY{c+c1}{\PYZsh{} parameters = ...}
        \PY{c+c1}{\PYZsh{} YOUR CODE STARTS HERE}
        \PY{n}{parameters} \PY{o}{=} \PY{n}{update\PYZus{}parameters}\PY{p}{(}\PY{n}{parameters}\PY{p}{,} \PY{n}{grads}\PY{p}{,} \PY{n}{learning\PYZus{}rate}\PY{p}{)}
        
        \PY{c+c1}{\PYZsh{} YOUR CODE ENDS HERE}
                
        \PY{c+c1}{\PYZsh{} Print the cost every 100 iterations}
        \PY{k}{if} \PY{n}{print\PYZus{}cost} \PY{o+ow}{and} \PY{n}{i} \PY{o}{\PYZpc{}} \PY{l+m+mi}{100} \PY{o}{==} \PY{l+m+mi}{0} \PY{o+ow}{or} \PY{n}{i} \PY{o}{==} \PY{n}{num\PYZus{}iterations} \PY{o}{\PYZhy{}} \PY{l+m+mi}{1}\PY{p}{:}
            \PY{n+nb}{print}\PY{p}{(}\PY{l+s+s2}{\PYZdq{}}\PY{l+s+s2}{Cost after iteration }\PY{l+s+si}{\PYZob{}\PYZcb{}}\PY{l+s+s2}{: }\PY{l+s+si}{\PYZob{}\PYZcb{}}\PY{l+s+s2}{\PYZdq{}}\PY{o}{.}\PY{n}{format}\PY{p}{(}\PY{n}{i}\PY{p}{,} \PY{n}{np}\PY{o}{.}\PY{n}{squeeze}\PY{p}{(}\PY{n}{cost}\PY{p}{)}\PY{p}{)}\PY{p}{)}
        \PY{k}{if} \PY{n}{i} \PY{o}{\PYZpc{}} \PY{l+m+mi}{100} \PY{o}{==} \PY{l+m+mi}{0} \PY{o+ow}{or} \PY{n}{i} \PY{o}{==} \PY{n}{num\PYZus{}iterations}\PY{p}{:}
            \PY{n}{costs}\PY{o}{.}\PY{n}{append}\PY{p}{(}\PY{n}{cost}\PY{p}{)}
    
    \PY{k}{return} \PY{n}{parameters}\PY{p}{,} \PY{n}{costs}
\end{Verbatim}
\end{tcolorbox}

    \begin{tcolorbox}[breakable, size=fbox, boxrule=1pt, pad at break*=1mm,colback=cellbackground, colframe=cellborder]
\prompt{In}{incolor}{21}{\boxspacing}
\begin{Verbatim}[commandchars=\\\{\}]
\PY{n}{parameters}\PY{p}{,} \PY{n}{costs} \PY{o}{=} \PY{n}{L\PYZus{}layer\PYZus{}model}\PY{p}{(}\PY{n}{train\PYZus{}x}\PY{p}{,} \PY{n}{train\PYZus{}y}\PY{p}{,} \PY{n}{layers\PYZus{}dims}\PY{p}{,} \PY{n}{num\PYZus{}iterations} \PY{o}{=} \PY{l+m+mi}{1}\PY{p}{,} \PY{n}{print\PYZus{}cost} \PY{o}{=} \PY{k+kc}{False}\PY{p}{)}

\PY{n+nb}{print}\PY{p}{(}\PY{l+s+s2}{\PYZdq{}}\PY{l+s+s2}{Cost after first iteration: }\PY{l+s+s2}{\PYZdq{}} \PY{o}{+} \PY{n+nb}{str}\PY{p}{(}\PY{n}{costs}\PY{p}{[}\PY{l+m+mi}{0}\PY{p}{]}\PY{p}{)}\PY{p}{)}

\PY{n}{L\PYZus{}layer\PYZus{}model\PYZus{}test}\PY{p}{(}\PY{n}{L\PYZus{}layer\PYZus{}model}\PY{p}{)}
\end{Verbatim}
\end{tcolorbox}

    \begin{Verbatim}[commandchars=\\\{\}]
Cost after iteration 0: 0.6950464961800915
Cost after first iteration: 0.6950464961800915
Cost after iteration 1: 0.7070709008912569
Cost after iteration 1: 0.7070709008912569
Cost after iteration 1: 0.7070709008912569
Cost after iteration 2: 0.7063462654190897
\textcolor{ansi-green-intense}{ All tests passed.}
    \end{Verbatim}

    \#\#\# 5.1 - Train the model

If your code passed the previous cell, run the cell below to train your
model as a 4-layer neural network.

\begin{itemize}
\item
  The cost should decrease on every iteration.
\item
  It may take up to 5 minutes to run 2500 iterations.
\end{itemize}

    \begin{tcolorbox}[breakable, size=fbox, boxrule=1pt, pad at break*=1mm,colback=cellbackground, colframe=cellborder]
\prompt{In}{incolor}{22}{\boxspacing}
\begin{Verbatim}[commandchars=\\\{\}]
\PY{n}{parameters}\PY{p}{,} \PY{n}{costs} \PY{o}{=} \PY{n}{L\PYZus{}layer\PYZus{}model}\PY{p}{(}\PY{n}{train\PYZus{}x}\PY{p}{,} \PY{n}{train\PYZus{}y}\PY{p}{,} \PY{n}{layers\PYZus{}dims}\PY{p}{,} \PY{n}{num\PYZus{}iterations} \PY{o}{=} \PY{l+m+mi}{2500}\PY{p}{,} \PY{n}{print\PYZus{}cost} \PY{o}{=} \PY{k+kc}{True}\PY{p}{)}
\end{Verbatim}
\end{tcolorbox}

    \begin{Verbatim}[commandchars=\\\{\}]
Cost after iteration 0: 0.6950464961800915
Cost after iteration 100: 0.5892596054583805
Cost after iteration 200: 0.5232609173622991
Cost after iteration 300: 0.4497686396221906
Cost after iteration 400: 0.4209002161883899
Cost after iteration 500: 0.37246403061745953
Cost after iteration 600: 0.3474205187020191
Cost after iteration 700: 0.31719191987370265
Cost after iteration 800: 0.2664377434774658
Cost after iteration 900: 0.21991432807842573
Cost after iteration 1000: 0.1435789889362377
Cost after iteration 1100: 0.4530921262322132
Cost after iteration 1200: 0.09499357670093511
Cost after iteration 1300: 0.08014128076781366
Cost after iteration 1400: 0.0694023400553646
Cost after iteration 1500: 0.060216640231745895
Cost after iteration 1600: 0.05327415758001879
Cost after iteration 1700: 0.04762903262098432
Cost after iteration 1800: 0.04297588879436867
Cost after iteration 1900: 0.03903607436513823
Cost after iteration 2000: 0.03568313638049028
Cost after iteration 2100: 0.032915263730546776
Cost after iteration 2200: 0.030472193059120623
Cost after iteration 2300: 0.028387859212946117
Cost after iteration 2400: 0.026615212372776077
Cost after iteration 2499: 0.024821292218353375
    \end{Verbatim}

    \textbf{Expected Output}:

Cost after iteration 0

0.771749

Cost after iteration 100

0.672053

\ldots{}

\ldots{}

Cost after iteration 2499

0.088439

    \begin{tcolorbox}[breakable, size=fbox, boxrule=1pt, pad at break*=1mm,colback=cellbackground, colframe=cellborder]
\prompt{In}{incolor}{23}{\boxspacing}
\begin{Verbatim}[commandchars=\\\{\}]
\PY{n}{pred\PYZus{}train} \PY{o}{=} \PY{n}{predict}\PY{p}{(}\PY{n}{train\PYZus{}x}\PY{p}{,} \PY{n}{train\PYZus{}y}\PY{p}{,} \PY{n}{parameters}\PY{p}{)}
\end{Verbatim}
\end{tcolorbox}

    \begin{Verbatim}[commandchars=\\\{\}]
Accuracy: 0.9999999999999998
    \end{Verbatim}

    \textbf{Expected Output}:

Train Accuracy

0.985645933014

    \begin{tcolorbox}[breakable, size=fbox, boxrule=1pt, pad at break*=1mm,colback=cellbackground, colframe=cellborder]
\prompt{In}{incolor}{24}{\boxspacing}
\begin{Verbatim}[commandchars=\\\{\}]
\PY{n}{pred\PYZus{}test} \PY{o}{=} \PY{n}{predict}\PY{p}{(}\PY{n}{test\PYZus{}x}\PY{p}{,} \PY{n}{test\PYZus{}y}\PY{p}{,} \PY{n}{parameters}\PY{p}{)}
\end{Verbatim}
\end{tcolorbox}

    \begin{Verbatim}[commandchars=\\\{\}]
Accuracy: 0.74
    \end{Verbatim}

    \textbf{Expected Output}:

Test Accuracy

0.8

    \hypertarget{congrats-it-seems-that-your-4-layer-neural-network-has-better-performance-80-than-your-2-layer-neural-network-72-on-the-same-test-set.}{%
\subsubsection{Congrats! It seems that your 4-layer neural network has
better performance (80\%) than your 2-layer neural network (72\%) on the
same test
set.}\label{congrats-it-seems-that-your-4-layer-neural-network-has-better-performance-80-than-your-2-layer-neural-network-72-on-the-same-test-set.}}

This is pretty good performance for this task. Nice job!

In the next course on ``Improving deep neural networks,'' you'll be able
to obtain even higher accuracy by systematically searching for better
hyperparameters: learning\_rate, layers\_dims, or num\_iterations, for
example.

    \#\# 6 - Results Analysis

First, take a look at some images the L-layer model labeled incorrectly.
This will show a few mislabeled images.

    \begin{tcolorbox}[breakable, size=fbox, boxrule=1pt, pad at break*=1mm,colback=cellbackground, colframe=cellborder]
\prompt{In}{incolor}{25}{\boxspacing}
\begin{Verbatim}[commandchars=\\\{\}]
\PY{n}{print\PYZus{}mislabeled\PYZus{}images}\PY{p}{(}\PY{n}{classes}\PY{p}{,} \PY{n}{test\PYZus{}x}\PY{p}{,} \PY{n}{test\PYZus{}y}\PY{p}{,} \PY{n}{pred\PYZus{}test}\PY{p}{)}
\end{Verbatim}
\end{tcolorbox}

    \begin{center}
    \adjustimage{max size={0.9\linewidth}{0.9\paperheight}}{output_42_0.png}
    \end{center}
    { \hspace*{\fill} \\}
    
    \textbf{A few types of images the model tends to do poorly on include:}
- Cat body in an unusual position - Cat appears against a background of
a similar color - Unusual cat color and species - Camera Angle -
Brightness of the picture - Scale variation (cat is very large or small
in image)

    \hypertarget{congratulations-on-finishing-this-assignment}{%
\subsubsection{Congratulations on finishing this
assignment!}\label{congratulations-on-finishing-this-assignment}}

You just built and trained a deep L-layer neural network, and applied it
in order to distinguish cats from non-cats, a very serious and important
task in deep learning. ;)

By now, you've also completed all the assignments for Course 1 in the
Deep Learning Specialization. Amazing work! If you'd like to test out
how closely you resemble a cat yourself, there's an optional ungraded
exercise below, where you can test your own image.

Great work and hope to see you in the next course!

    \#\# 7 - Test with your own image (optional/ungraded exercise) \#\#

From this point, if you so choose, you can use your own image to test
the output of your model. To do that follow these steps:

\begin{enumerate}
\def\labelenumi{\arabic{enumi}.}
\tightlist
\item
  Click on ``File'' in the upper bar of this notebook, then click
  ``Open'' to go on your Coursera Hub.
\item
  Add your image to this Jupyter Notebook's directory, in the ``images''
  folder
\item
  Change your image's name in the following code
\item
  Run the code and check if the algorithm is right (1 = cat, 0 =
  non-cat)!
\end{enumerate}

    \begin{tcolorbox}[breakable, size=fbox, boxrule=1pt, pad at break*=1mm,colback=cellbackground, colframe=cellborder]
\prompt{In}{incolor}{26}{\boxspacing}
\begin{Verbatim}[commandchars=\\\{\}]
\PY{c+c1}{\PYZsh{}\PYZsh{} START CODE HERE \PYZsh{}\PYZsh{}}
\PY{n}{my\PYZus{}image} \PY{o}{=} \PY{l+s+s2}{\PYZdq{}}\PY{l+s+s2}{my\PYZus{}image.jpg}\PY{l+s+s2}{\PYZdq{}} \PY{c+c1}{\PYZsh{} change this to the name of your image file }
\PY{n}{my\PYZus{}label\PYZus{}y} \PY{o}{=} \PY{p}{[}\PY{l+m+mi}{1}\PY{p}{]} \PY{c+c1}{\PYZsh{} the true class of your image (1 \PYZhy{}\PYZgt{} cat, 0 \PYZhy{}\PYZgt{} non\PYZhy{}cat)}
\PY{c+c1}{\PYZsh{}\PYZsh{} END CODE HERE \PYZsh{}\PYZsh{}}

\PY{n}{fname} \PY{o}{=} \PY{l+s+s2}{\PYZdq{}}\PY{l+s+s2}{images/}\PY{l+s+s2}{\PYZdq{}} \PY{o}{+} \PY{n}{my\PYZus{}image}
\PY{n}{image} \PY{o}{=} \PY{n}{np}\PY{o}{.}\PY{n}{array}\PY{p}{(}\PY{n}{Image}\PY{o}{.}\PY{n}{open}\PY{p}{(}\PY{n}{fname}\PY{p}{)}\PY{o}{.}\PY{n}{resize}\PY{p}{(}\PY{p}{(}\PY{n}{num\PYZus{}px}\PY{p}{,} \PY{n}{num\PYZus{}px}\PY{p}{)}\PY{p}{)}\PY{p}{)}
\PY{n}{plt}\PY{o}{.}\PY{n}{imshow}\PY{p}{(}\PY{n}{image}\PY{p}{)}
\PY{n}{image} \PY{o}{=} \PY{n}{image} \PY{o}{/} \PY{l+m+mf}{255.}
\PY{n}{image} \PY{o}{=} \PY{n}{image}\PY{o}{.}\PY{n}{reshape}\PY{p}{(}\PY{p}{(}\PY{l+m+mi}{1}\PY{p}{,} \PY{n}{num\PYZus{}px} \PY{o}{*} \PY{n}{num\PYZus{}px} \PY{o}{*} \PY{l+m+mi}{3}\PY{p}{)}\PY{p}{)}\PY{o}{.}\PY{n}{T}

\PY{n}{my\PYZus{}predicted\PYZus{}image} \PY{o}{=} \PY{n}{predict}\PY{p}{(}\PY{n}{image}\PY{p}{,} \PY{n}{my\PYZus{}label\PYZus{}y}\PY{p}{,} \PY{n}{parameters}\PY{p}{)}


\PY{n+nb}{print} \PY{p}{(}\PY{l+s+s2}{\PYZdq{}}\PY{l+s+s2}{y = }\PY{l+s+s2}{\PYZdq{}} \PY{o}{+} \PY{n+nb}{str}\PY{p}{(}\PY{n}{np}\PY{o}{.}\PY{n}{squeeze}\PY{p}{(}\PY{n}{my\PYZus{}predicted\PYZus{}image}\PY{p}{)}\PY{p}{)} \PY{o}{+} \PY{l+s+s2}{\PYZdq{}}\PY{l+s+s2}{, your L\PYZhy{}layer model predicts a }\PY{l+s+se}{\PYZbs{}\PYZdq{}}\PY{l+s+s2}{\PYZdq{}} \PY{o}{+} \PY{n}{classes}\PY{p}{[}\PY{n+nb}{int}\PY{p}{(}\PY{n}{np}\PY{o}{.}\PY{n}{squeeze}\PY{p}{(}\PY{n}{my\PYZus{}predicted\PYZus{}image}\PY{p}{)}\PY{p}{)}\PY{p}{,}\PY{p}{]}\PY{o}{.}\PY{n}{decode}\PY{p}{(}\PY{l+s+s2}{\PYZdq{}}\PY{l+s+s2}{utf\PYZhy{}8}\PY{l+s+s2}{\PYZdq{}}\PY{p}{)} \PY{o}{+}  \PY{l+s+s2}{\PYZdq{}}\PY{l+s+se}{\PYZbs{}\PYZdq{}}\PY{l+s+s2}{ picture.}\PY{l+s+s2}{\PYZdq{}}\PY{p}{)}
\end{Verbatim}
\end{tcolorbox}

    \begin{Verbatim}[commandchars=\\\{\}]
Accuracy: 1.0
y = 1.0, your L-layer model predicts a "cat" picture.
    \end{Verbatim}

    \begin{center}
    \adjustimage{max size={0.9\linewidth}{0.9\paperheight}}{output_46_1.png}
    \end{center}
    { \hspace*{\fill} \\}
    
    \textbf{References}:

\begin{itemize}
\tightlist
\item
  for auto-reloading external module:
  http://stackoverflow.com/questions/1907993/autoreload-of-modules-in-ipython
\end{itemize}


    % Add a bibliography block to the postdoc
    
    
    
\end{document}
